\documentclass[twoside]{article}

% Geometry of the page:
\usepackage{geometry}
\geometry{paperwidth=14.85cm, paperheight=21cm, inner=1.3cm, outer=1.7cm, tmargin=1cm, bmargin=1cm, includehead, includefoot}

% Main font:
\usepackage{luatextra}
\setmainfont{Arno Pro}

% Hyphenations etc.:
\usepackage[latin]{babel}

% Colors:
\usepackage{xcolor}
\definecolor{liturgicalred}{cmyk}{0.15,1,1,0}

% Entêtes et pieds de pages :
\usepackage{fancyhdr}
\pagestyle{fancy}
\fancyhead{}
\fancyhead[CE]{\fontsize{16}{16}\selectfont\textsc{De Imitatione Christi}}
\fancyhead[CO]{\fontsize{16}{16}\selectfont\textsc{Liber \leftmark, caput \rightmark}}
\fancyfoot[CE,CO]{\fontsize{16}{16}\selectfont\thepage}
\renewcommand{\headrule}{}
\renewcommand{\footrulewidth}{0pt}
\setlength{\parindent}{0cm}
\setlength{\headsep}{0.7cm} % Distance between header and body.
\setlength{\footskip}{1cm} % Distance between footer and body.

% Style de paragraphe TitreA :
\newenvironment{TitreA}[1]{
    \setlength{\parindent}{0cm}
    \setlength{\leftskip}{0cm}
    \fontsize{36}{42}\selectfont
    \setlength{\parskip}{-0.3cm}
    \begin{center}
        \MakeUppercase{#1}
    \end{center}
}

% Style de paragraphe TitreB :
\newenvironment{TitreB}[1]{
    \setlength{\parindent}{0cm}
    \setlength{\leftskip}{0cm}
    \setlength{\parskip}{-0.1cm}
    \fontsize{24}{36}\selectfont
    \begin{center}
        \textsc{#1}
    \end{center}
}

% Style de paragraphe TitreC :
\newenvironment{TitreC}[1]{
    \setlength{\parindent}{0cm}
    \setlength{\leftskip}{0cm}
    \setlength{\parskip}{-0.1cm}
    \fontsize{18}{24}\selectfont
    \begin{center}
        \textsc{#1}
    \end{center}
}

% Style de paragraphe TitreD :
\newenvironment{TitreD}[1]{
    \setlength{\parindent}{0cm}
    \setlength{\leftskip}{0cm}
    \setlength{\parskip}{-0.1cm}
    \fontsize{14}{18}\selectfont
    \begin{center}
        {\color{liturgicalred}\textsc{#1}}
    \end{center}
}

% Style de paragraphe Normal :
\newenvironment{Normal}[1]{
    \setlength{\parindent}{0cm}
    \setlength{\leftskip}{0cm}
    \setlength{\parskip}{0cm}
    \fontsize{16}{18}\selectfont
    #1\par
    \vspace{0.1cm}
}

\newcommand{\Verse}[1]{{\color{liturgicalred}{\textbf{#1}}}}


\begin{document}

\selectlanguage{latin}

\thispagestyle{empty}

\TitreA{DE IMITATIONE CHRISTI}
\TitreA{LIBRI QUATUOR}


\markboth{I}{}
\TitreB{LIBER I}
\TitreB{ADMONITIONES AD SPIRITUALEM VITAM UTILES}

\markright{I}
\TitreC{CAPUT I}
\TitreC{De Imitatione Christi et contemptu omnium vanitatum mundi}

1. Qui sequitur me, non ambulat in tenebris, dicit Dominus. Hæc sunt verba Christi, quibus admonemur, quatenus vitam eius et mores imitemur, si velimus veraciter illuminari et ab omni cæcitate cordis liberari. Summum igitur studium nostrum sit, in vita Iesu Christi meditari.
2. Doctrina Christi omnes doctrinas Sanctorum præcellit; et qui spiritum haberet, absconditum ibi manna inveniret. Sed contingit, quod multi ex frequenti auditu Evangelii parvum desiderium sentiunt, quia spiritum Christi non habent. Qui autem vult plene et sapide Christi verba intellegere, oportet, ut totam vitam suam illi studeat conformare.
3. Quid prodest tibi, alta de Trinitate disputare, si careas humilitate, unde displiceas Trinitati? Vere, alta verba non faciunt sanctum et iustum, sed virtuosa vita efficit Deo carum. Opto magis sentire compunctionem, quam scire eius definitionem. Si scires totam Bibliam exterius, et omnium philosophorum dicta: quid totum prodesset sine caritate Dei et gratia? Vanitas vanitatum, et omnia vanitas, præter amare Deum et illi soli servire. Ista est summa sapientia, per contemptum mundi tendere ad regna cælestia.
4. Vanitas igitur est, divitias perituras quærere et in illis sperare. Vanitas quoque est, honores ambire, et in altum statum se extollere. Vanitas est, carnis desideria sequi, et illud desiderare, unde postmodum graviter oportet puniri. Vanitas est, longam vitam optare, et de bona vita parum curare. Vanitas est, præsentem vitam solum attendere, et quæ futura sunt, non prævidere. Vanitas est, diligere, quod cum omni celeritate transit, et illic non festinare, ubi sempiternum gaudium manet.
5. Memento illius frequenter proverbii: Quia non satiatur oculus visu, nec auris impletur auditu. Stude ergo cor tuum ab amore visibilium abstrahere et ad invisibilia te transferre. Nam sequentes suam sensualitatem maculant conscientiam et perdunt Dei gratiam.

\markright{II}
\TitreC{CAPUT II}
De humili sentire sui ipsius

1. Omnis homo naturaliter scire desiderat; sed scientia sine timore Dei quid importat? Melior est profecto humilis rusticus, qui Deo servit, quam superbus philosophus, qui se neglecto cursum cæli considerat. Qui bene se ipsum cognoscit, sibi ipsi vilescit, nec laudibus delectatur humanis. Si scirem omnia, quæ in mundo sunt, et non essem in caritate: quid me iuvaret coram Deo, qui me iudicaturus est ex facto?
2. Quiesce a nimio sciendi desiderio, quia magna ibi invenitur distractio et deceptio. Scientes libenter volunt videri et dici sapientes. Multa sunt, quæ scire parum vel nihil animæ prosunt. Et valde insipiens est, qui aliquibus intendit quam his, quæ saluti suæ deserviunt. Multa verba non satiant animam; sed bona vita refrigerat mentem, et pura conscientia magnam ad Deum præstat confidentiam.
3. Ouanto plus et melius scis, tanto gravius inde iudicaberis, nisi sanctius vixeris. Noli ergo extolli de ulla arte vel scientia, sed potius time de data tibi notitia. Si tibi videtur, quod multa scis et satis bene intellegis: scito tamen, quia sunt multo plura, quæ nescis. Noli altum sapere, sed ignorantiam tuam magis fatere. Quid te vis alicui præferre, cum plures doctiores te inveniantur, et magis in lege periti? Si vis utiliter aliquid scire et discere: ama nesciri et pro nihilo reputari.
4. Hæc est altissima et utilissima lectio: sui ipsius vera cognitio et despectio. De se ipso nihil tenere, et de aliis semper bene et alte sentire, magna sapientia est et perfectio. Si videres alium aperte peccare, vel aliqua gravia perpetrare, non deberes te tamen meliorem æstimare; quia nescis, quam diu possis in bono stare. Omnes fragiles sumus, sed tu neminem fragiliorem te ipso tenebis.

\markright{III}
CAPUT III.
De doctrina veritatis

1. Felix, quem veritas per se docet, non per figuras et voces transeuntes, sed sicuti se habet. Nostra opinio et noster sensus sæpe nos fallit et modicum videt. Quid prodest magna cavillatio de occultis et obscuris rebus, de quibus nec arguemur in iudicio, quia ignoravimus? Grandis insipientia, quod neglectis utilibus et necessariis ultro intendimus curiosis et damnosis. Oculos habentes non videmus.
2. Et quid curæ nobis de generibus et speciebus? Cui æternum Verbum loquitur, a multis opinionibus expeditur. Ex uno verbo omnia, et unum loquuntur omnia: et hoc est principium, quod et loquitur nobis. Nemo sine illo intellegit aut recte iudicat. Cui omnia unum sunt, et omnia ad unum trahit, et omnia in uno videt, potest stabilis corde esse et in Deo pacificus permanere. O veritas Deus, fac me unum tecum in caritate perpetua. Tædet me sæpe, multa legere et audire: in te est totum, quod volo et desidero. Taceant omnes doctores, sileant universæ creaturæ in conspectu tuo: tu mihi loquere solus.
3. Quanto aliquis magis sibi unitus et interius simplificatus fuerit, tanto plura et altiora sine labore intellegit, quia desuper lumen intellegentiæ accipit. Purus, simplex et stabilis spiritus in multis operibus non dissipatur, quia omnia ad Dei honorem operatur, et in se otiosus ab omni propria exquisitione esse nititur. Quis te magis impedit et molestat quam tua immortificata affectio cordis? Bonus et devotus homo opera sua prius intus disponit, quæ foris agere debet. Nec illa trahunt eum ad desideria vitiosæ inclinationis, sed ipse inflectit ea ad arbitrium rectæ rationis. Quis habet fortius certamen, quam qui nititur vincere se ipsum? Et hoc deberet esse negotium nostrum: vincere videlicet se ipsum, et cotidie se ipso fortiorem fieri atque in melius aliquid proficere.
4. Omnis perfectio in hac vita quandam imperfectionem sibi habet annexam; et omnis speculatio nostra quadam caligine non caret. Humilis tui cognitio certior via est ad Deum, quam profunda scientiæ inquisitio. Non est culpanda scientia aut quælibet simplex rei notitia, quæ bona est in se considerata et a Deo ordinata; sed præferenda est semper bona conscientia et virtuosa vita. Quia vero plures magis student scire, quam bene vivere: ideo sæpe errant et pæne nullum vel modicum fructum ferunt.
5. O si tantam adhiberent diligentiam ad exstirpanda vitia et virtutes inserendas, sicuti ad movendas quæstiones, non fierent tanta mala et scandala in populo, nec tanta dissolutio in cœnobiis. Certe adveniente die iudicii non quæretur a nobis, quid legimus, sed quid fecimus; nec quam bene diximus, sed quam religiose viximus. Dic mihi: Ubi sunt modo omnes illi domini et magistri, quos bene novisti, dum adhuc viverent, et studiis florerent? Iam eorum præbendas alii possident; et nescio, utrum de iis recogitant. In vita sua aliquid esse videbantur, et modo de illis tacetur.
6. O quam cito transit gloria mundi! Utinam vita eorum scientiæ ipsorum concordasset! Tunc bene studuissent et legissent. Quam multi pereunt per vanam scientiam in sæculo, qui parum curant de Dei servitio! Et quia magis eligunt magni esse quam humiles, ideo evanescunt in cogitationibus suis. Vere magnus est, qui magnam habet caritatem. Vere magnus est: qui in se parvus est et pro nihilo omne culmen honoris ducit. Vere prudens est, qui omnia terrena arbitratur ut stercora, ut Christum lucrifaciat. Et vere bene doctus est, qui Dei voluntatem facit, et suam voluntatem relinquit.

\markright{IV}
CAPUT IV.
De providentia in agendis

1. Non est credendum omni verbo nec instinctui; sed caute et longanimiter res est secundum Deum ponderanda. Pro dolor, sæpe malum facilius quam bonum de alio creditur et dicitur, ita infirmi sumus. Sed perfecti viri non facile credunt omni enarranti, quia sciunt infirmitatem humanam ad malum proclivam et in verbis satis labilem.
2. Magna sapientia, non esse præcipitem in agendis, nec pertinaciter in propriis stare sensibus. Ad hanc etiam pertinet, non quibuslibet hominum verbis credere, nec audita vel credita mox ad aliorum aures effundere. Cum sapiente et conscientioso viro consilium habe; et quære potius a meliore instrui, quam tuas adinventiones sequi. Bona vita facit hominem sapientem secundum Deum et expertum in multis. Quanto quis in se humilior fuerit et Deo subiectior, tanto in omnibus erit sapientior et pacatior.

\markright{V}
CAPUT V.
De lectione Sanctarum Scripturarum

1. Veritas est in Scripturis Sanctis quærenda, non eloquentia. Omnis Scriptura Sacra eo spiritu debet legi, quo facta est. Quærere potius debemus utilitatem in Scripturis, quam subtilitatem sermonis. Ita libenter devotos et simplices libros legere debemus, sicut altos et profundos. Non te offendat auctoritas scribentis, utrum parvæ vel magnæ litteraturæ fuerit; sed amor puræ veritatis te trahat ad legendum. Non quæras, quis hoc dixerit; sed quid dicatur, attende.
2. Homines transeunt, sed veritas Domini manet in æternum. Sine personarum acceptione, variis modis loquitur nobis Deus. Curiositas nostra sæpe nos impedit in lectione Scripturarum, cum volumus intellegere et discutere, ubi simpliciter esset transeundum. Si vis profectum haurire, lege humiliter, simpliciter et fideliter; nec umquam velis habere nomen scientiæ. Interroga libenter et audi tacens verba sanctorum; nec displiceant tibi parabolæ seniorum: sine causa enim non proferuntur.


CAPUT VI.
De inordinatis affectionibus

1. Quandocumque homo aliquid inordinate appetit, statim in se inquietus fit. Superbus et avarus numquam quiescunt; pauper et humilis spiritu in multitudine pacis conversantur. Homo qui necdum perfecte in se mortuus est, cito temptatur et vincitur in parvis et vilibus rebus. Infirmus in spiritu et quodammodo adhuc carnalis et ad sensibilia inclinatus difficulter se potest a terrenis desideriis ex toto abstrahere. Et ideo sæpe habet tristitiam, cum se subtrahit; leviter etiam indignatur, si quis ei resistit.
2. Si autem prosecutus fuerit, quod concupiscit, statim ex reatu conscientiæ gravatur: quia secutus est passionem suam, quæ nihil iuvat ad pacem, quam quæsivit. Resistendo igitur passionibus invenitur pax vera cordis, non autem eis serviendo. Non est ergo pax in corde hominis carnalis, non in homine exterioribus dedito, sed in fervido et spirituali.


CAPUT VII.
De vana spe et elatione fugienda

1. Vanus est, qui spem suam ponit in hominibus aut in creaturis. Non te pudeat aliis servire amore Iesu Christi, et pauperem in hoc sæculo videri. Non stes super te ipsum, sed in Deo spem tuam constitue. Fac quod in te est, et Deus aderit bonæ voluntati tuæ. Non confidas in tua scientia vel astutia cuiuscumque viventis, sed magis in Dei gratia, qui adiuvat humiles, et de se præsumentes humiliat.
2. Ne glorieris in divitiis, si adsunt, nec in amicis, quia potentes sunt, sed in Deo, qui omnia præstat et se ipsum super omnia dare desiderat. Non te extollas de magnitudine vel pulchritudine corporis, quæ modica infirmitate corrumpitur et defœdatur. Non placeas tibi ipsi de habilitate aut ingenio tuo, ne displiceas Deo, cuius est totum quidquid boni naturaliter habueris.
3. Non te reputes aliis meliorem, ne forte coram Deo deterior habearis, qui scit, quid est in homine. Non superbias operibus bonis, quia aliter sunt iudicia Dei quam hominum, cui sæpe displicet, quod hominibus placet. Si aliquid boni habueris, crede de aliis meliora, ut humilitatem conserves. Non nocet, si omnibus te supponas; nocet autem plurimum, si vel uni te præponas. Iugis pax cum humili; in corde autem superbi zelus et indignatio frequens.


CAPUT VIII.
De cavenda nimia familiaritate

1. Non omni homini reveles cor tuum, sed cum sapiente et timente Deum age causam tuam. Cum iuvenibus et extraneis rarus esto. Cum divitibus noli blandiri, et coram magnatibus non libenter appareas. Cum humilibus et simplicibus, cum devotis et morigeratis sociare, et quæ ædificationis sunt, pertracta. Non sis familiaris alicui mulieri; sed in communi omnes bonas mulieres Deo commenda. Soli Deo et angelis eius opta familiaris esse, et hominum notitiam devita.
2. Caritas habenda est ad omnes, sed familiaritas non expedit. Quandoque accidit, ut persona ignota ex bona fama lucescat; cuius tamen præsentia oculos intuentium offuscat. Putamus aliquando aliis placere ex coniunctione nostra, et incipimus magis displicere ex morum improbitate in nobis considerata.


CAPUT IX.
De obœdientia et subiectione

1. Valde magnum est, in obœdientia stare, sub prælato vivere, et sui iuris non esse. Multo tutius est, stare in subiectione, quam in prælatura. Multi sunt sub obœdientia magis ex necessitate, quam ex caritate; et illi pœnam habent et leviter murmurant. Nec libertatem mentis acquirent, nisi ex toto corde propter Deum se subiciant. Curre hic vel ibi: non invenies quietem, nisi in humili subiectione sub regimine prælati. Imaginatio locorum et mutatio multos fefellit.
2. Verum est, quod unusquisque libenter agit pro sensu suo, et inclinatur ad eos magis, qui secum sentiunt. Sed si Deus est inter nos, necesse est, ut relinquamus etiam quandoque nostrum sentire propter bonum pacis. Quis est ita sapiens, qui omnia plene scire potest? Ergo, noli nimis in sensu tuo confidere, sed velis etiam libenter aliorum sensum audire. Si bonum est tuum sentire, et hoc ipsum propter Deum dimittis et alium sequeris, magis exinde proficies.
3. Audivi enim sæpe, securius esse audire et accipere consilium, quam dare. Potest etiam contingere, ut bonum sit uniuscuiusque sentire; sed nolle aliis acquiescere, cum id ratio aut causa postulat, signum est superbiæ et pertinaciæ.


CAPUT X.
De cavenda superfluitate verborum

1. Caveas tumultum hominum, quantum potes; multum enim impedit tractatus sæcularium gestorum, etiam si simplici intentione proferantur. Cito enim inquinamur vanitate et captivamur. Vellem me pluries tacuisse et inter homines non fuisse. Sed quare tam libenter loquimur et invicem fabulamur, cum tamen raro sine læsione conscientiæ ad silentium redimus? Ideo tam libenter loquimur: quia per mutuas locutiones ab invicem consolari quærimus; et cor diversis cogitationibus fatigatum optamus relevare. Et multum libenter de his, quæ multum diligimus vel cupimus, vel quæ nobis contraria sentimus, libet loqui et cogitare.
2. Sed pro dolor, sæpe inaniter et frustra. Nam hæc exterior consolatio interioris et divinæ consolationis non modicum detrimentum est. Ideo vigilandum est et orandum, ne tempus otiose transeat. Si loqui licet et expedit, quæ ædificabilia sunt, loquere. Malus usus et neglegentia profectus nostri multum facit ad incustodiam oris nostri. Iuvat tamen non parum ad profectum spiritualem devota spiritualium rerum collatio; maxime ubi pares animo et spiritu in Deo sibi sociantur.


CAPUT XI.
De pace acquirenda et zelo proficiendi

1. Multam possemus pacem habere, si non vellemus nos cum aliorum dictis et factis, et quæ ad nostram curam non spectant, occupare. Quomodo potest ille diu in pace manere, qui alienis curis se intermiscet? qui occasiones forinsecus quærit? qui parum vel raro se intrinsecus colligit? Beati simplices, quoniam multam pacem habebunt.
2. Quare quidam sanctorum tam perfecti et contemplativi fuerunt? Quia omnino se ipsos mortificare ab omnibus terrenis desideriis studuerunt: et ideo totis medullis cordis Deo inhærere atque libere sibi vacare potuerunt. Nos nimium occupamur propriis passionibus, et de transitoriis nimis sollicitamur. Raro etiam unum vitium perfecte vincimus, et ad cotidianum profectum non accendimur: ideo frigidi et tepidi remanemus.
3. Si essemus nobis ipsis perfecte mortui, et interius minime implicati: tunc possemus etiam divina sapere, et de cælesti contemplatione aliquid experiri. Totum et maximum impedimentum est, quia non sumus a passionibus et concupiscentiis liberi, nec perfectam sanctorum viam conamur ingredi. Quando etiam modicum adversitatis occurrit, nimis cito deicimur, et ad humanas consolationes convertimur.
4. Si niteremur, sicut viri fortes, stare in prœlio: profecto auxilium Domini super nos videremus de cælo. Ipse enim certantes et de sua gratia sperantes paratus est adiuvare: qui nobis certandi occasiones procurat, ut vincamus. Si tantum in istis exterioribus observantiis profectum religionis ponimus, cito finem habebit devotio nostra. Sed ad radicem securim ponamus, ut purgati a passionibus pacificam mentem possideamus.
5. Si omni anno unum vitium exstirparemus, cito viri perfecti efficeremur. Sed modo e contrario sæpe sentimus, ut meliores et puriores in initio conversionis nos fuisse inveniamus, quam post multos annos professionis. Fervor et profectus cotidie deberet crescere; sed nunc pro magno videtur, si quis primi fervoris partem posset retinere. Si modicam violentiam faceremus in principio, tunc postea cuncta possemus facere cum levitate et gaudio.
6. Grave est, assueta dimittere: sed gravius est, contra propriam voluntatem ire. Sed si non vincis parva et levia, quando superabis difficiliora? Resiste in principio inclinationi tuæ, et malam dedisce consuetudinem, ne forte paulatim ad maiorem te ducat difficultatem. O si adverteres, quantam tibi pacem et aliis lætitiam faceres, te ipsum bene habendo, puto, quod sollicitior esses ad spiritualem profectum.


CAPUT XII.
De utilitate adversitatis

1. Bonum nobis est, quod aliquando habeamus aliquas gravitates et contrarietates; quia sæpe hominem ad cor revocant, quatinus se in exilio esse cognoscat, nec spem suam in aliqua re mundi ponat. Bonum est, quod patiamur quandoque contradictiones, et quod male et imperfecte de nobis sentiatur, etiam si bene agimus et intendimus. Ista sæpe iuvant ad humilitatem et a vana gloria nos defendunt. Tunc enim melius interiorem testem Deum quærimus, quando foris vilipendimur ab hominibus, et non bene nobis creditur.
2. Ideo deberet se homo in Deo taliter firmare, ut non esset ei necesse multas humanas consolationes quærere. Quando homo bonæ voluntatis tribulatur vel temptatur, aut malis cogitationibus affligitur: tunc Deum sibi magis necessarium intellegit, sine quo nihil boni se posse deprehendit. Tunc etiam tristatur, gemit, et orat pro miseriis, quas patitur. Tunc tædet eum diutius vivere, et mortem optat venire: ut possit dissolvi et cum Christo esse. Tunc etiam bene advertit, perfectam securitatem et plenam pacem in mundo non posse constare.


CAPUT XIII.
De temptationibus resistendis

1. Quamdiu in mundo vivimus, sine tribulatione et temptatione esse non possumus. Unde in Iob scriptum est: Temptatio est vita humana super terram. Ideo unusquisque sollicitus esse deberet circa temptationes suas et vigilare in orationibus, ne diabolus locum inveniret decipiendi, qui numquam dormitat, sed circuit quærens, quem devoret. Nemo tam perfectus est et sanctus, qui non habeat aliquando temptationes; et plene eis carere non possumus.
2. Sunt tamen temptationes homini sæpe valde utiles, licet molestæ sint et graves; quia in illis homo humiliatur, purgatur, et eruditur. Omnes sancti per multas tribulationes et temptationes transierunt et profecerunt. Et qui temptationes sustinere nequiverunt, reprobi facti sunt et defecerunt. Non est aliquis ordo tam sanctus, nec locus tam secretus, ubi non sint temptationes vel adversitates.
3. Non est homo securus a temptationibus totaliter, quamdiu vixerit: quia in nobis est, unde temptamur, ex quo in concupiscentia nati sumus. Una temptatione seu tribulatione recedente, alia supervenit, et semper aliquid ad patiendum habebimus, nam bonum felicitatis nostræ perdidimus. Multi quærunt temptationes fugere, et gravius incidunt in eas. Per solam fugam non possumus vincere; sed per patientiam et veram humilitatem omnibus hostibus efficimur fortiores.
4. Qui tantummodo exterius declinat nec radicem evellit, parum proficiet; immo citius ad eum temptationes redient et peius sentiet. Paulatim, et per patientiam cum longanimitate Deo iuvante melius superabis, quam cum duritia et importunitate propria. Sæpius accipe consilium in temptatione, et cum temptato noli duriter agere, sed consolationem ingere, sicut tibi optares fieri.
5. Initium omnium malarum temptationum inconstantia animi et parva ad Deum confidentia. Quia sicut navis sine gubernaculo hinc inde a fluctibus impellitur: ita homo remissus et suum propositum deserens varie temptatur. Ignis probat ferrum, et temptatio hominem iustum. Nescimus sæpe quid possumus; sed temptatio aperit quid sumus. Vigilandum est tamen præcipue circa initium temptationis; quia tunc facilius hostis vincitur, si ostium mentis nullatenus intrare sinitur; sed extra limen, statim ut pulsaverit, illi obviatur. Unde quidam dixit:
Principiis obsta, sero medicina paratur.
Nam primo occurrit menti simplex cogitatio, deinde fortis imaginatio, postea delectatio et motus pravus, et assensio. Sicque paulatim ingreditur hostis malignus ex toto, dum illi non resistitur in principio. Et quanto diutius ad resistendum quis torpuerit, tanto in se cotidie debilior fit, et hostis contra eum potentior.
6. Quidam in principio conversionis suæ graviores temptationes patiuntur, quidam autem in fine. Quidam vero quasi per totam vitam suam male habent. Nonnulli satis leniter temptantur, secundum divinæ ordinationis sapientiam et æquitatem, quæ statum et merita hominum pensat, et cuncta ad electorum suorum salutem præordinat.
7. Ideo non debemus desperare, cum temptamur, sed eo ferventius Deum exorare, quatinus nos in omni tribulatione dignetur adiuvare; qui utique, secundum dictum Pauli, talem faciet cum temptatione proventum, ut possimus sustinere. Humiliemus ergo animas nostras sub manu Dei in omni temptatione et tribulatione, quia humiles spiritu salvabit et exaltabit.
8. In temptationibus et tribulationibus probatur homo, quantum profecit; et ibi maius meritum consistit, et virtus melius patescit. Nec magnum est, si homo devotus sit et fervidus, cum gravitatem non sentit; sed si tempore adversitatis patenter se sustinet, spes magni profectus erit. Quidam a magnis temptationibus custodiuntur, et in parvis cotidianis sæpe vincuntur, ut humiliati numquam de se ipsis in magnis confidant, qui in tam modicis infirmantur.


CAPUT XIV.
De temerario iudicio vitando

1. Ad te ipsum oculos reflecte, et aliorum facta caveas iudicare. In iudicando alios homo frustra laborat, sæpius errat, et leviter peccat; se ipsum vero iudicando et discutiendo semper fructuose laborat. Sicut nobis res cordi est, sic de ea frequenter iudicamus; nam verum iudicium propter privatum amorem faciliter perdimus. Si Deus semper esset pura intentio nostri desiderii, non tam faciliter turbaremur pro resistentia sensus nostri.
2. Sed sæpe aliquid ab intra latet, vel etiam ab extra concurrit, quod nos etiam pariter trahit. Multi occulte se ipsos quærunt in rebus quas agunt, et nesciunt. Videntur etiam in bona pace stare, quando res pro eorum velle fiunt et sentire; si autem aliter fit quam cupiunt, cito moventur et tristes fiunt. Propter diversitatem sensuum et opinionum satis frequenter oriuntur dissensiones inter amicos et cives, inter religiosos et devotos.
3. Antiqua consuetudo difficulter relinquitur, et ultra proprium videre nemo libenter ducitur. Si rationi tuæ magis inniteris vel industriæ, quam virtuti subiectivæ Iesu Christi, raro et tarde eris homo illuminatus; quia Deus vult nos sibi perfecte subici, et omnem rationem per inflammatum amorem transcendere.


CAPUT XV.
De operibus ex caritate factis

1. Pro nulla re mundi et pro nullius hominis dilectione aliquod malum est faciendum; sed pro utilitate tamen indigentis opus bonum libere aliquando intermittendum est, aut etiam pro meliori mutandum. Hoc enim facto opus bonum non destruitur, sed in melius commutatur. Sine caritate opus externum nihil prodest; quidquid autem ex caritate agitur, quantumcumque etiam parvum sit et despectum, totum fructuosum efficitur. Magis siquidem Deus pensat ex quanto quis agit, quam opus quod facit.
2. Multum facit, qui multum diligit. Multum facit, qui rem bene facit. Bene facit, qui communitati magis quam suæ voluntati servit. Sæpe videtur esse caritas, et est magis carnalitas; quia naturalis inclinatio, propria voluntas, spes rertributionis, affectus commoditatis raro abesse volunt.
3. Qui veram et perfectam caritatem habet, in nulla re se ipsum quærit, sed Dei solummodo gloriam in omnibus fieri desiderat. Nulli etiam invidet, quia nullum privatum gaudium amat; nec in se ipso vult gaudere, sed in Deo, super omnia bona, optat beatificari. Nemini aliquid boni attribuit, sed totaliter ad Deum refert, a quo fontaliter omnia procedunt, in quo finaliter omnes sancti fruibiliter requiescunt. O qui scintillam haberet veræ caritatis, profecto omnia terrena sentiret plena fore vanitatis.


CAPUT XVI.
De sufferentia defectuum aliorum

1. Quæ homo in se vel in aliis emendare non valet, debet patienter sustinere, donec Deus aliter ordinet. Cogita, quia sic forte melius est pro tua probatione et patientia, sine qua non sunt multum ponderanda merita nostra. Debes tamen pro talibus impedimentis supplicare, ut Deus tibi dignetur subvenire, et possis benigne portare.
2. Si quis semel aut bis admonitus non acquiescit, noli cum eo contendere; sed totum Deo committe, ut fiat voluntas eius et honor in omnibus servis suis, qui scit bene mala in bonum convertere. Stude patiens esse in tolerando aliorum defectus, et qualescumque infirmitates; quia et tu multa habes, quæ ab aliis oportet tolerari. Si non potes te talem facere, qualem vis, quomodo poteris alium ad tuum habere beneplacitum? Libenter habemus alios perfectos, et tamen proprios non emendamus defectus.
3. Volumus, quod alii stricte corrigantur, et ipsi corrigi nolumus. Displicet larga aliorum licentia, et tamen nobis nolumus negari quod petimus. Alios restringi per statuta volumus, et ipsi nullatenus patimur amplius cohiberi. Sic ergo patet, quam raro proximum sicut nos ipsos pensamus. Si essent omnes perfecti, quid tunc haberemus ab aliis pro Deo pati?
4. Nunc autem Deus sic ordinavit, ut discamus alter alterius onera portare; quia nemo sine defectu, nemo sine onere, nemo sibi sufficiens, nemo sibi satis sapiens; sed oportet nos invicem portare, invicem consolari, pariter adiuvare, instruere et admonere. Quantæ autem virtutis quisque fuerit, melius patet occasione adversitatis. Occasiones namque hominem fragilem non faciunt, sed qualis sit, ostendunt.


CAPUT XVII.
De monastica vita

1. Oportet quod discas te ipsum in multis frangere, si vis pacem et concordiam cum aliis tenere. Non est parvum in monasteriis vel in congregatione habitare, et inibi sine querela conversari, et usque ad mortem fidelis perseverare. Beatus, qui ibidem bene vixerit et feliciter consummaverit. Si vis debite stare et proficere, teneas te tamquam exulem peregrinum super terram. Oportet te stultum fieri propter Christum, si vis religiosam ducere vitam.
2. Habitus et tonsura modicum confert; sed mutatio morum, et integra mortificatio passionum verum faciunt religiosum. Qui aliud quærit, quam pure Deum et animæ suæ salutem, non inveniet nisi tribulationem et dolorem. Non potest etiam diu stare pacificus, qui non nititur esse minimus et omnibus subiectus.
3. Ad serviendum venisti, non ad regendum; ad patiendum et laborandum scias te vocatum, non ad otiandum vel fabulandum. Hic ergo probantur homines, sicut aurum in fornace. Hic nemo potest stare, nisi ex toto corde se voluerit propter Deum humiliare.


CAPUT XVIII.
De exemplis sanctorum patrum

1. Intuere sanctorum patrum vivida exempla, in quibus vera perfectio refulsit et religio, et videbis, quam modicum sit, et pæne nihil, quod nos agimus. Heu, quid est vita nostra, si illis fuerit comparata? Sancti et amici Christi Domino servierunt in fame et siti, in frigore et nuditate, in labore et fatigatione, in vigiliis et ieiuniis, in orationibus et meditationibus sanctis, in persecutionibus et opprobriis multis.
2. O quam multas et graves tribulationes passi sunt apostoli, martyres, confessores, virgines, et reliqui omnes, qui Christi vestigia voluerunt sequi. Nam animas suas in hoc mundo oderunt, ut in æternam vitam eas possiderent. O quam strictam et abdicatam vitam sancti patres in eremo duxerunt! quam longas et graves temptationes pertulerunt! quam frequenter ab inimico vexati sunt! quam crebras et fervidas orationes Deo obtulerunt! quam rigidas abstinentias peregerunt! quam magnum zelum et fervorem ad spiritualem profectum habuerunt! quam forte bellum adversus edomationem vitiorum gesserunt! quam puram et rectam intentionem ad Deum tenuerunt! Per diem laborabant, et noctibus orationi diutinæ vacabant; quamquam laborando ab oratione mentali minime cessarent.
3. Omne tempus utiliter expendebant; omnis hora ad vacandum Deo brevis videbatur; et præ magna dulcedine contemplationis etiam oblivioni tradebatur necessitas corporalis refectionis. Omnibus divitiis, dignitatibus, honoribus, amicis et cognatis renuntiabant; nil de mundo habere cupiebant, vix necessaria vitæ sumebant; corpori servire, etiam in necessitate, dolebant. Pauperes igitur erant rebus terrenis, sed divites valde in gratia et virtutibus. Foris egebant, sed intus gratia et consolatione divina reficiebantur.
4. Mundo erant alieni, sed Deo proximi ac familiares amici. Sibi ipsis videbantur tamquam nihili, et huic mundo despecti; sed erant in oculis Dei pretiosi et dilecti. In vera humilitate stabant, in simplici obœdientia vivebant, in caritate et patientia ambulabant: et ideo cotidie in spiritu proficiebant, et magnam apud Deum gratiam obtinebant. Dati sunt in exemplum omnibus religiosis, et plus provocare nos debent ad bene proficiendum, quam tepidorum numerus ad relaxandum.
5. O quantus fervor omnium religiosorum in principio suæ sanctæ institutionis fuit! O quanta devotio orationis! quanta æmulatio virtutis! quam magna disciplina viguit! quanta reverentia et obœdientia sub regula magistri in omnibus effloruit! Testantur adhuc vestigia derelicta, quod vere viri sancti et perfecti fuerunt, qui tam strenue militantes mundum suppeditaverunt. Iam magnus putatur, si quis transgressor non fuerit, si quis, quod accepit, cum patientia tolerare potuerit.
6. Oh teporis et neglegentiæ status nostri, quod tam cito declinamus a pristino fervore, et iam tædet vivere præ lassitudine et tepore! Utinam in te penitus non dormitet profectus virtutum, qui multa sæpius exempla vidisti devotorum!


CAPUT XIX.
De exercitiis boni religiosi

1. Vita boni religiosi omnibus virtutibus pollere debet: ut sit talis interius, qualis videtur hominibus exterius. Et merito, multo plus debet esse intus, quam quod cernitur foris, quia inspector noster est Deus, quem summopere revereri debemus, ubicumque fuerimus, et tamquam angeli in conspectu eius mundi incedere. Omni die renovare debemus propositum nostrum et ad fervorem nos excitare, quasi hodie primum ad conversionem venissemus, atque dicere: Adiuva me, Domine Deus, in bono proposito et sancto servitio tuo, et da mihi nunc hodie perfecte incipere, quia nihil est quod hactenus feci.
2. Secundum propositum nostrum, cursus profectus nostri; et multa diligentia opus est bene proficere volenti. Quod si fortiter proponens sæpe deficit; quid ille, qui raro aut minus fixe aliquid proponit? Variis tamen modis contingit desertio propositi nostri; et levis omissio exercitiorum vix sine aliquo dispendio transit. Iustorum propositum in gratia Dei potius, quam in propria sapientia pendet: in quo et semper confidunt, quidquid arripiunt. Nam homo proponit, sed Deus disponit, nec est in homine via eius.
3. Si pietatis causa aut fraternæ utilitatis proposito quandoque consuetum omittitur exercitium, facile postea poterit recuperari. Si autem tædio animi aut neglegentia faciliter relinquitur, satis culpabile est et nocivum sentietur. Conemur quantum possumus, adhuc leviter deficiemus in multis. Semper tamen aliquid certi proponendum est, et contra illa præcipue, quæ amplius nos impediunt. Exteriora nostra et interiora pariter nobis scrutanda sunt et ordinanda, quia utraque expediunt ad profectum.
4. Si non continue te vales colligere, saltem interdum, et ad minus semel in die, mane videlicet, aut vespere. Mane propone, vespere discute mores tuos, qualis hodie fuisti in verbo, opere et cogitatione; quia in his sæpius forsitan Deum offendisti et proximum. Accinge te, sicut vir, contra diabolicas nequitias; frena gulam, et omnem carnis inclinationem facilius frenabis. Numquam sis ex toto otiosus, sed aut legens, aut scribens, aut orans, aut meditans, aut aliquid utilitatis pro communi laborans. Corporalia tamen exercitia discrete sunt agenda, nec omnibus æqualiter assumenda.
5. Quæ communia non sunt, non sunt foris ostendenda; nam in secreto tutius exercentur privata. Cavendum tamen, ne piger sis ad communia et ad singularia promptior; sed expletis integre et fideliter debitis et iniunctis, si iam ultra vacat, redde te tibi, prout devotio tua desiderat. Non possunt omnes habere unum exercitium, sed aliud isti, aliud illi magis deservit. Etiam pro temporis congruentia diversa placent exercitia, quia alia in festis, alia in feriatis magis sapiunt diebus. Aliis indigemus tempore temptationis, et aliis tempore pacis et quietis. Alia, cum tristamur, libet cogitare, et alia, cum læti in Domino fuerimus.
6. Circa principalia festa renovanda sunt bona exercitia et sanctorum suffragia ferventius imploranda. De festo in festum proponere debemus, quasi tunc de hoc sæculo migraturi et ad æternum festum perventuri. Ideoque sollicite nos præparare debemus in devotis temporibus et devotius conversari, atque omnem observantiam strictius custodire, tamquam in brevi præmium laboris nostri a Deo percepturi.
7. Et si dilatum fuerit, credamus nos minus bene præparatos atque indignos tantæ adhuc gloriæ, quæ revelabitur in nobis tempore præfinito: et studeamus nos melius ad exitum præparare. Beatus servus, ait evangelista Lucas, quem, cum venerit Dominus, invenerit vigilantem. Amen dico vobis, super omnia bona sua constituet eum.


CAPUT XX.
De amore solitudinis et silentii

1. Quære aptum tempus vacandi tibi, et de beneficiis Dei frequenter cogita. Relinque curiosa. Tales perlege materias, quæ compunctionem magis præstant, quam occupationem. Si te subtraxeris a superfluis locutionibus et otiosis circuitionibus, nec non a novitatibus et rumoribus audiendis: invenies tempus sufficiens et aptum pro bonis meditationibus insistendis. Maximi sanctorum humana consortia, ubi poterant, vitabant, et Deo in secreto servire eligebant.
2. Dixit quidam: Quotiens inter homines fui, minor homo redii. Hoc sæpius experimur, quando diu confabulamur. Facilius est omnino tacere, quam verbo non excedere. Facilius est domi latere, quam foris se posse sufficienter custodire. Oui igitur intendit ad interiora et spiritualia pervenire, oportet eum cum Iesu a turba declinare. Nemo secure apparet, nisi qui libenter latet. Nemo secure loquitur, nisi qui libenter tacet. Nemo secure præest, nisi qui libenter subest. Nemo secure præcipit, nisi qui bene obœdire didicit.
3. Nemo secure gaudet, nisi bonæ conscientiæ in se testimonium habeat. Semper tamen sanctorum securitas plena timoris Dei exstitit. Nec eo minus solliciti et humiles in se fuerunt, quia magnis virtutibus et gratia emicuerunt. Pravorum autem securitas ex superbia et præsumptione oritur, et in fine in deceptionem sui ipsius vertitur. Numquam promittas tibi securitatem in hac vita, quamvis bonus videaris cœnobita aut devotus eremita.
4. Sæpe meliores in æstimatione hominum gravius periclitati sunt propter suam nimiam confidentiam. Unde multis utilius est, ut non penitus temptationibus careant, sed sæpius impugnentur, ne nimium securi sint, ne forte in superbiam eleventur, ne etiam ad exteriores consolationes licentius declinent. O, qui numquam transitoriam lætitiam quæreret, qui numquam cum mundo se occuparet, quam bonam conscientiam servaret! O, qui omnem vanam sollicitudinem amputaret, et dumtaxat salutaria ac divina cogitaret, et totam spem suam in Deo constitueret, quam magnam pacem et quietem possideret!
5. Nemo dignus est cælesti consolatione, nisi diligenter se exercuerit in sancta compunctione. Si vis cordetenus compungi, intra cubile tuum, et exclude tumultus mundi, sicut scriptum est: In cubilibus vestris compungimini. In cella invenies, quod deforis sæpius amittes. Cella continuata dulcescit, et male custodita tædium generat. Si in principio conversionis tuæ bene eam incolueris et custodieris, erit tibi postea dilecta amica et gratissimum solacium.
6. In silentio et quiete proficit anima devota et discit abscondita scripturarum. Ibi invenit fluenta lacrimarum, quibus singulis noctibus se lavet et mundet, ut Conditori suo tanto familiarior fiat, quanto longius ab omni sæculari tumultu degit. Qui ergo se abstrahit a notis et amicis, approximabit illi Deus cum angelis sanctis. Melius est latere et sui curam agere, quam se neglecto signa facere. Laudabile est homini religioso, raro foras ire, fugere videri, nolle etiam homines videre.
7. Quid vis videre, quod non licet habere? Transit mundus et concupiscentia eius. Trahunt desideria sensualitatis ad spatiandum; sed cum hora transierit, quid nisi gravitatem conscientiæ et cordis dispersionem reportas? Lætus exitus tristem sæpe reditum parit, et læta vigilia serotina triste mane facit. Sic omne carnale gaudium blande intrat, sed in fine mordet et perimit. Quid potes alibi videre, quod hic non vides? Ecce cælum et terra et omnia elementa; nam ex istis omnia sunt facta.
8. Quid potes alicubi videre, quod diu potest sub sole permanere? Credis te forsitan satiari, sed non poteris pertingere. Si cuncta videres præsentia, quid esset nisi visio vana? Leva oculos tuos ad Deum in excelsis, et ora pro peccatis tuis et neglegentiis. Dimitte vana vanis; tu autem intende illis, quæ tibi præcepit Deus. Claude super te ostium tuum, et voca ad te Iesum, dilectum tuum. Mane cum eo in cella, quia non invenies alibi tantam pacem. Si non exisses nec quicquam de rumoribus audisses, melius in bona pace permansisses. Ex quo nova delectat aliquando audire, oportet te exinde turbationem cordis tolerare.


CAPUT XXI.
De compunctione cordis

1. Si vis aliquid proficere, conserva te in timore Dei, et noli esse nimis liber; sed sub disciplina cohibe omnes sensus tuos, nec ineptæ te tradas lætitiæ. Da te ad cordis compunctionem, et invenies devotionem. Compunctio multa bona aperit, quæ dissolutio cito perdere consuevit. Mirum est, quod homo potest umquam perfecte in hac vita lætari, qui suum exilium et tam multa pericula animæ suæ considerat et pensat.
2. Propter levitatem cordis et neglegentiam defectuum nostrorum non sentimus animæ nostræ dolores; sed sæpe vane ridemus, quando merito flere deberemus. Non est vera libertas nec bona lætitia, nisi in timore Dei cum bona conscientia. Felix, qui abicere potest omne impedimentum distractionis, et ad unionem se recolligere sanctæ compunctionis. Felix, qui a se abdicat, quidquid conscientiam suam maculare potest vel gravare. Certa viriliter, consuetudo consuetudine vincitur. Si tu scis homines dimittere, ipsi bene te dimittent tua facta facere.
3. Non attrahas tibi res aliorum, nec te implices causis maiorum. Habeas semper oculum super te primum, et admoneas te ipsum specialiter præ omnibus tibi dilectis. Si non habes favorem hominum, noli ex hoc tristari; sed hoc sit tibi grave, quia non habes te satis bene et circumspecte, sicut deceret Dei servum et devotum religiosum conversari. Utilius est sæpe et securius, quod homo non habeat multas consolationes in hac vita, secundum carnem præcipue. Tamen quod divinas non habemus, aut rarius sentimus, nos in culpa sumus, quia compunctionem cordis non quærimus, nec vanas et externas omnino abicimus.
4. Cognosce te indignum divina consolatione, sed magis dignum multa tribulatione. Quando homo est perfecte compunctus, tunc gravis et amarus est ei totus mundus. Bonus homo sufficientem invenit materiam dolendi et flendi. Sive enim se considerat, sive de proximo pensat, scit, quia nemo sine tribulatione hic vivit. Et quanto strictius sese considerat, tanto amplius dolet. Materiæ iusti doloris et internæ compunctionis sunt peccata et vitia nostra, quibus ita involuti iacemus, ut raro cælestia contemplari valeamus.
5. Si frequentius de morte tua, quam de longitudine vitæ cogitares, non dubium, quin ferventius te emendares. Si etiam futuras inferni sive purgatorii pœnas cordialiter perpenderes, credo, quod libenter laborem et dolorem sustineres, et nihil rigoris formidares. Sed quia ad cor ista non transeunt, et blandimenta adhuc amamus, ideo frigidi et valde pigri remanemus.
6. Sæpe est inopia spiritus, unde tam leviter conqueritur miserum corpus. Ora igitur humiliter ad Dominum, ut det tibi compunctionis spiritum, et dic cum propheta: Ciba me, Domine, pane lacrimarum, et potum da mihi in lacrimis in mensura.


CAPUT XXII.
De consideratione humanæ miseriæ

1. Miser es, ubicumque fueris et quocumque te verteris, nisi ad Deum te convertas. Quid turbaris, quia non succedit tibi, sicut vis et desideras? Quis est, qui habet omnia secundum suam voluntatem? Nec ego, nec tu, nec aliquis hominum super terram. Nemo est in mundo sine aliqua tribulatione vel angustia, quamvis rex sit vel papa. Quis est, qui melius habet? utique qui pro Deo aliquid pati valet.
2. Dicunt multi imbecilles et infirmi: Ecce, quam bonam vitam ille homo habet, quam dives, quam magnus, quam potens et excelsus! Sed attende ad cælestia bona, et videbis, quod omnia ista temporalia nulla sunt, sed valde incerta et magis gravantia, quia numquam sine sollicitudine et timore possidentur. Non est hominis felicitas, habere temporalia ad abundantiam, sed sufficit ei mediocritas. Vere miseria est vivere super terram. Quanto homo voluerit esse spiritualior, tanto præsens vita fit ei amarior, quia sentit melius et videt clarius humanæ corruptionis defectus. Nam comedere, bibere, vigilare, dormire, quiescere, laborare et ceteris necessitatibus naturæ subiacere, vere magna miseria est et afflictio homini devoto, qui libenter esset absolutus et liber ab omni peccato.
3. Valde enim gravatur interior homo necessitatibus corporalibus in hoc mundo. Unde propheta devote rogat, quatenus liber ab istis esse valeat, dicens: De necessitatibus meis erue me, Domine. Sed væ non cognoscentibus suam miseriam; et amplius væ illis, qui diligunt hanc miseram et corruptibilem vitam. Nam in tantum quidam hanc amplectuntur, licet etiam vix necessaria laborando aut mendicando habeant, ut, si possent hic semper vivere, de regno Dei nihil curarent.
4. O insani et infideles corde, qui tam profunde in terrenis iacent, ut nihil nisi carnalia sapiant. Sed miseri adhuc in fine graviter sentient, quam vile et nihilum erat, quod amaverunt. Sancti autem Dei et omnes devoti amici Christi non attenderunt, quæ carni placuerunt, nec quæ in hoc tempore floruerunt; sed tota spes eorum et intentio ad æterna bona anhelabat. Ferebatur totum desiderium eorum sursum ad mansura et invisibilia, ne amore visibilium traherentur ad infima. Noli, frater, amittere confidentiam proficiendi ad spiritualia; adhuc habes tempus et horam.
5. Quare vis procrastinare propositum tuum? Surge et in instanti incipe et dic: Nunc tempus est faciendi, nunc tempus est pugnandi, nunc aptum tempus est emendandi. Quando male habes et tribularis, tunc tempus est promerendi. Oportet te transire per ignem et aquam, antequam venias in refrigerium. Nisi tibi vim feceris, vitium non superabis. Quamdiu istud fragile corpus gerimus, sine peccato esse non possumus, nec sine tædio et dolore vivere. Libenter haberemus ab omni miseria quietem; sed quia per peccatum perdidimus innocentiam, amisimus etiam veram beatitudinem. Ideo oportet nos tenere patientiam et Dei expectare misericordiam; donec transeat iniquitas hæc et mortalitas absorbeatur a vita.
6. O quanta fragilitas humana, quæ semper prona est ad vitia! Hodie confiteris peccata tua, et cras iterum perpetras confessa. Nunc proponis cavere, et post horam agis quasi nihil proposuisses. Merito ergo nos ipsos humiliare possumus nec umquam aliquid magni de nobis sentire; quia tam fragiles et instabiles sumus. Cito etiam potest perdi per neglegentiam, quod multo labore vix tandem acquisitum est per gratiam.
7. Quid fiet de nobis adhuc in fine, qui tepescimus tam mane? Væ nobis, si sic volumus declinare ad quietem, quasi iam pax sit et securitas, cum necdum appareat vestigium veræ sanctitatis in conversatione nostra. Bene opus esset, quod adhuc iterum instrueremur, tamquam boni novicii, ad mores optimos; si forte spes esset de aliqua futura emendatione et maiori spirituali profectu.


CAPUT XXIII.
De meditatione mortis

1. Valde cito erit tecum hic factum, vide aliter quomodo te habeas: hodie homo est, et cras non comparet. Cum autem sublatus fuerit ab oculis, etiam cito transit a mente. O hebetudo et duritia cordis humani, quod solum præsentia meditatur, et futura non magis prævidet! Sic te in omni facto et cogitatu deberes tenere, quasi hodie esses moriturus. Si bonam conscientiam haberes, non multum mortem timeres. Melius esset peccata cavere, quam mortem fugere. Si hodie non es paratus, quomodo cras eris? Cras est dies incerta, et quid scis, si crastinum habebis?
2. Quid prodest diu vivere, quando tam parum emendamur? Ah, longa vita non semper emendat, sed sæpe culpam magis auget. Utinam per unam diem bene essemus conversati in hoc mundo! Multi annos computant conversionis, sed sæpe parvus est fructus emendationis. Si formidolosum est mori, forsitan periculosius erit diutius vivere. Beatus, qui horam mortis suæ semper ante oculos habet, et ad moriendum cotidie se disponit. Si vidisti aliquando hominem mori, cogita, quia et tu per eandem viam transibis.
3. Cum mane fuerit, puta te ad vesperum non perventurum. Vespere autem facto, mane non audeas tibi polliceri. Semper ergo paratus esto, et taliter vive, ut numquam te imparatum mors inveniat. Multi subito et improvise moriuntur. Nam hora, qua non putatur, filius hominis venturus est. Quando illa extrema hora venerit, multum aliter sentire incipies de tota vita tua præterita; et valde dolebis, quia tam neglegens et remissus fuisti.
4. Quam felix et prudens, qui talis nunc nititur esse in vita, qualis optat inveniri in morte! Dabit namque magnam fiduciam feliciter moriendi perfectus contemptus mundi, fervens desiderium in virtutibus proficiendi, amor disciplinæ, labor pænitentiæ, promptitudo obœdientiæ, abnegatio sui et supportatio cuiuslibet adversitatis pro amore Christi. Multa bona potes operari, dum sanus es; sed infirmatus, nescio quid poteris. Pauci ex infirmitate meliorantur; sic et qui multum peregrinantur, raro sanctificantur.
5. Noli confidere super amicos et proximos, nec in futurum tuam differas salutem: quia citius obliviscentur tui homines, quam æstimas. Melius est nunc tempestive providere et aliquid boni præmittere, quam super aliorum auxilio sperare. Si non es pro te ipso sollicitus modo, quis erit sollicitus pro te in futuro? Nunc tempus est valde pretiosum. Nunc sunt dies salutis: nunc tempus acceptabile. Sed pro dolor, quod hoc utilius non expendis, in quo promereri vales, unde æternaliter vivas! Veniet, quando unam diem seu horam pro emendatione desiderabis, et nescio, an impetrabis.
6. Eia carissime, de quanto periculo te poteris liberare, de quam magno timore eripere, si modo semper timoratus fueris et suspectus de morte? Stude nunc taliter vivere, ut in hora mortis valeas potius gaudere, quam timere. Disce nunc mori mundo, ut tunc incipias vivere cum Christo. Disce nunc omnia contemnere, ut tunc possis libere ad Christum pergere. Castiga nunc corpus tuum per pænitentiam, ut tunc certam valeas habere confidentiam.
7. Ah stulte, quid cogitas te diu victurum, cum nullum diem habeas securum? Quam multi decepti sunt, et insperate de corpore extracti? Quotiens audisti a dicentibus, quia ille gladio cecidit, ille submersus est, ille ab alto ruens cervicem fregit, ille manducando obriguit, ille ludendo finem fecit? Alius igne, alius ferro, alius peste, alius latrocinio interiit: et sic omnium finis mors est, et vita hominum tamquam umbra subito pertransit.
8. Quis memorabitur tui post mortem? et quis orabit pro te? Age, age nunc, carissime, quidquid agere potes: quia nescis, quando morieris, nescis etiam, quid tibi post mortem sequetur. Dum tempus habes, congrega divitias immortales. Præter salutem tuam, nihil cogites; solum quæ Dei sunt, cures. Fac nunc tibi amicos, venerando Dei sanctos, et eorum actus imitando, ut, cum defeceris in hac vita, illi te recipiant in æterna tabernacula.
9. Serva te tamquam peregrinum et hospitem super terram, ad quem nihil spectat de mundi negotiis. Serva cor liberum et ad Deum sursum erectum, quia non habes hic manentem civitatem. Illuc preces et gemitus cotidianos cum lacrimis dirige, ut spiritus tuus mereatur ad Dominum post mortem feliciter transire. Amen.


CAPUT XXIV.
De iudicio et pœnis peccatorum

1. In omnibus rebus respice finem, et qualiter ante districtum stabis iudicem, cui nihil est occultum; qui muneribus non placatur nec excusationes recipit; sed, quod iustum est, iudicabit. O miserrime et insipiens peccator, quid respondebis Deo, omnia mala tua scienti, qui interdum formidas vultum hominis irati? Utquid non prævides tibi in die iudicii, quando nemo poterit per alium excusari vel defendi, sed unusquisque sufficiens onus erit sibi ipsi? Nunc labor tuus est fructuosus, fletus acceptabilis, gemitus exaudibilis, dolor satisfactorius et purgativus.
2. Habet magnum et salubre purgatorium patiens homo, qui suscipiens iniurias, plus dolet de alterius malitia, quam de sua iniuria; qui pro contrariantibus sibi libenter orat et ex corde culpas indulget: qui veniam ab aliis petere non retardat; qui facilius miseretur, quam irascitur; qui sibi ipsi violentiam frequenter facit, et carnem omnino spiritui subiugare conatur. Melius est, modo purgare peccata et vitia resecare, quam in futuro purganda reservare. Vere nos ipsos decipimus per inordinatum amorem, quem ad carnem habemus.
3. Quid aliud ignis ille devorabit, nisi peccata tua? Quanto amplius tibi ipsi nunc parcis et carnem sequeris, tanto durius postea lues et maiorem materiam comburendi reservas. In quibus homo peccavit, in illis gravius punietur. Ibi acediosi ardentibus stimulis perurgentur, et gulosi ingenti siti ac fame cruciabuntur. Ibi luxuriosi et voluptatum amatores ardenti pice et fœtido sulphure perfundentur, et sicut furiosi canes præ dolore invidiosi ululabunt.
4. Nullum vitium erit, quod suum proprium cruciatum non habebit. Ibi superbi omni confusione replebuntur, et avari miserrima egestate arctabuntur. Ibi erit una hora gravior in pœna, quam hic centum anni in gravissima pænitentia. Ibi nulla requies est, nulla consolatio damnatis; hic tamen interdum cessatur a laboribus, atque amicorum fruitur solaciis. Esto modo sollicitus et dolens pro peccatis tuis, ut in die iudicii securus sis cum beatis. Tunc enim iusti stabunt in magna constantia adversus eos, qui se angustiaverunt et depresserunt. Tunc stabit ad iudicandum, qui modo se subicit humiliter iudiciis hominum. Tunc magnam fiduciam habebit pauper et humilis, et pavebit undique superbus.
5. Tunc videbitur sapiens in hoc mundo fuisse, qui pro Christo didicit stultus et despectus esse. Tunc placebit omnis tribulatio patienter perpessa, et omnis iniquitas oppilabit os suum. Tunc gaudebit omnis devotus, et mærebit omnis irreligiosus. Tunc plus exultabit caro afflicta, quam si in deliciis fuisset semper nutrita. Tunc splendebit habitus vilis, et obtenebrescet vestis subtilis. Tunc plus laudabitur pauperculum domicilium, quam deauratum palatium. Tunc iuvabit plus constans patientia, quam omnis mundi potentia. Tunc amplius exaltabitur simplex obœdientia, quam omnis sæcularis astutia.
6. Tunc plus lætificabit pura et bona conscientia, quam docta philosophia. Tunc plus ponderabit contemptus divitiarum, quam totus thesaurus terrigenarum. Tunc magis consolaberis super devota oratione, quam super delicata comestione. Tunc potius gaudebis de servato silentio, quam de longa fabulatione. Tunc plus valebunt sancta opera, quam multa pulchra verba. Tunc plus placebit stricta vita et ardua pænitentia, quam omnis delectatio terrena. Disce nunc in modico pati, ut tunc a gravioribus valeas liberari. Hic primo proba, quid possis postea. Si nunc tam parum vales sustinere, quomodo æterna tormenta poteris sufferre? Si modo modica passio tam impatientem efficit, quid gehenna tunc faciet? Ecce, vere non potes duo gaudia habere, delectari hic in mundo, et postea regnare cum Christo.
7. Si usque in hodiernam diem semper in honoribus et voluptatibus vixisses: quid totum tibi profuisset, si iam mori in instanti contingeret? Omnia ergo vanitas, præter amare Deum et illi soli servire. Qui enim Deum ex toto corde amat, nec mortem, nec supplicium, nec iudicium, nec infernum metuit; quia perfectus amor securum ad Deum accessum facit. Quem autem adhuc peccare delectat, non mirum, si mortem et iudicium timeat. Bonum tamen est, ut, si necdum amor a malo te revocat, saltem timor gehennalis coërceat. Qui vero timorem Dei postponit, diu stare in bono non valebit, sed diaboli laqueos citius incurret.


CAPUT XXV.
De ferventi emendatione totius vitæ nostræ

1. Esto vigilans et diligens in Dei servitio, et cogita frequenter: Ad quid venisti, et cur sæculum reliquisti? Nonne ut Deo viveres, et spiritualis homo fieres? Igitur ad profectum ferveas, quia mercedem laborum tuorum in brevi recipies; nec erit tunc amplius timor aut dolor in finibus tuis. Modicum nunc laborabis, et magnam requiem, immo perpetuam lætitiam invenies. Si tu permanseris fidelis et fervidus in agendo, Deus procul dubio erit fidelis et locuples in retribuendo. Spem bonam retinere debes, quod ad palmam pervenies; sed securitatem capere non oportet, ne torpeas aut elatus fias.
2. Cum quidam anxius inter metum et spem frequenter fluctuaret, et quadam vice mærore confectus, in ecclesia ante quoddam altare se in oratione prostravisset, hæc intra se revolvit, dicens: O si scirem, quod adhuc perseveraturus essem! statimque audivit divinum intus responsum: Quodsi hoc scires, quid facere velles? Fac nunc, quod tunc facere velles, et bene securus eris. Moxque consolatus et confortatus, divinæ se commisit voluntati, et cessavit anxia fluctuatio. Noluitque curiose investigare, ut sciret, quæ sibi essent futura: sed magis studuit inquirere, quæ esset voluntas Dei beneplacens et perfecta, ad omne opus bonum inchoandum et perficiendum.
3. Spera in Domino, et fac bonitatem ait propheta, et inhabita terram, et pasceris in divitiis eius. Unum est, quod multos a profectu et ferventi emendatione retrahit: horror difficultatis seu labor certaminis. Enimvero illi maxime præ ceteris in virtutibus proficiunt, qui ea, quæ sibi magis gravia et contraria sunt, virilius vincere nituntur. Nam ibi homo plus proficit et gratiam meretur ampliorem, ubi magis se ipsum vincit et in spiritu mortificat.
4. Sed non omnes habent æque multum ad vincendum et moriendum. Diligens tamen æmulator valentior erit ad proficiendum, etiamsi plures habeat passiones, quam alius bene morigeratus, minus tamen fervens ad virtutes. Duo specialiter ad magnam emendationem iuvant: videlicet, subtrahere se violenter, ad quod natura vitiose inclinatur, et ferventer instare pro bono, quo amplius quis indiget. Illa etiam studeas magis cavere et vincere, quæ tibi frequentius in aliis displicent.
5. Ubique profectum tuum capias, ut, si bona exempla videas vel audias, ad imitandum accendaris. Si quid autem reprehensibile consideraveris, cave, ne idem facias; aut si aliquando fecisti, citius emendare te studeas. Sicut oculus tuus alios considerat, sic iterum ab aliis notaris. Quam iucundum et dulce est, videre fervidos et devotos fratres, bene morigeratos et disciplinatos! Quam triste est et grave, videre inordinate ambulantes, qui ea, ad quæ vocati sunt, non exercent! Quam nocivum est, neglegere vocationis suæ propositum, et ad non commissa sensum inclinare!
6. Memor esto arrepti propositi, et imaginem tibi propone Crucifixi. Bene verecundari potes inspecta vita Iesu Chiristi, quia necdum magis illi te conformare studuisti, licet diu in via Dei fuisti. Religiosus, qui se intente et devote in sanctissima vita et passione Domini exercet, omnia utilia et necessaria sibi abundanter ibi inveniet; nec opus est, ut extra Iesum aliquid melius quærat. O si Iesus crucifixus in cor nostrum veniret, quam cito et sufficienter docti essemus!
7. Religiosus fervidus omnia bene portat et capit, quæ illi iubentur. Religiosus neglegens et tepidus habet tribulationem super tribulationem, et ex omni parte patitur angustiam; quia interiori consolatione caret, et exteriorem quærere prohibetur. Religiosus extra disciplinam vivens gravi patet ruinæ. Qui laxiora quærit et remissiora, semper in angustiis erit: quia unum aut reliquum sibi displicebit.
8. Quomodo faciunt tam multi alii religiosi, qui satis arctati sunt sub disciplina claustrali? Raro exeunt, abstracte vivunt, pauperrime comedunt, grosse vestiuntur, multum laborant, parum loquuntur, diu vigilant, mature surgunt, orationes prolongant, frequenter legunt et se in omni disciplina custodiunt. Attende Carthusienses, Cistercienses et diversæ religionis monachos ac moniales: qualiter omni nocte ad psallendum Domino assurgunt. Et ideo turpe esset, ut tu deberes in tam sancto opere pigritare, ubi tanta multitudo religiosorum incipit Deo iubilare.
9. O si nihil aliud faciendum incumberet, nisi Dominum Deum nostrum toto corde et ore laudare! O si numquam indigeres comedere, nec bibere, nec dormire, sed semper posses Deum laudare, et solummodo spiritualibus studiis vacare: tunc multo felicior esses quam modo, cum carni ex qualicumque necessitate servis. Utinam non essent istæ necessitates, sed solum spirituales animæ refectiones, quas heu! satis raro degustamus!
10. Quando homo ad hoc pervenit, quod de nulla creatura consolationem suam quærit, tunc ei Deus primo perfecte sapere incipit; tunc etiam bene contentus de omni eventu rerum erit. Tunc nec pro magno lætabitur, nec pro modico contristabitur; sed ponit se integre et fiducialiter in Deo, qui est ei omnia in omnibus: cui nihil utique perit nec moritur, sed omnia ei vivunt et ad nutum incunctanter deserviunt.
11. Memento semper finis, et quia perditum non redit tempus. Sine sollicitudine et diligentia numquam acquires virtutes. Si incipis tepescere, incipies male habere. Si autem dederis te ad fervorem, invenies magnam pacem et senties leviorem laborem, propter Dei gratiam et virtutis amorem. Homo fervidus et diligens ad omnia est paratus. Maior labor est resistere vitiis et passionibus, quam corporalibus insudare laboribus. Qui parvos non vitat defectus, paulatim labitur ad maiores. Gaudebis semper vespere, si diem expendas fructuose. Vigila super te ipsum, excita te ipsum, admone te ipsum; et quidquid de aliis sit, non neglegas te ipsum. Tantum proficies, quantum tibi ipsi vim intuleris. Amen.



\markboth{II}{}
LIBER II.
ADMONITIONES AD INTERNA TRAHENTES

\markright{I}
CAPUT I.
De interna conversatione

1. Regnum Dei intra vos est, dicit Dominus. Converte te ex toto corde ad Dominum et relinque hunc miserum mundum, et inveniet anima tua requiem. Disce exteriora contemnere et ad interiora te dare, et videbis regnum Dei in te venire. Est enim regnum Dei pax et gaudium in Spiritu Sancto, quod non datur impiis. Veniet ad te Christus, ostendens tibi consolationem suam, si dignam illi ab intus paraveris mansionem. Omnis gloria eius et decor ab intra est, et ibi complacet sibi. Frequens illi visitatio cum homine interno, dulcis sermocinatio, grata consolatio, multa pax, familiaritas stupenda nimis.
2. Eia, anima fidelis, præpara huic sponso cor tuum, quatenus ad te venire et in te habitare dignetur. Sic enim dicit: Si quis diligit me, sermonem meum servabit, et ad eum veniemus et mansionem apud eum faciemus. Da ergo Christo locum, et ceteris omnibus nega introitum. Cum Christum habueris, dives es, et sufficit tibi. Ipse erit provisor tuus et fidelis procurator in omnibus, ut non sit opus in hominibus sperare. Homines enim cito mutantur et deficiunt velociter; Christus autem manet in æternum et astat usque in finem firmiter.
3. Non est magna fiducia ponenda in homine fragili et mortali, etiam si utilis sit et dilectus; neque tristitia multa ex hoc capienda, si interdum adversetur et contradicat. Qui hodie tecum sunt, cras contrariari possunt; et e converso sæpe ut aura vertuntur. Pone totam fiduciam tuam in Deo, et sit ipse timor tuus et amor tuus. Ipse pro te respondebit et faciet bene, sicut melius fuerit. Non habes hic manentem civitatem; et ubicumque fueris, extraneus es et peregrinus; nec requiem aliquando habebis, nisi Christo intime fueris unitus.
4. Quid hic circumspicis, cum iste non sit locus tuæ requietionis? In cælestibus debet esse habitatio tua, et sicut in transitu cuncta terrena sunt aspicienda. Transeunt omnia et tu cum eis pariter. Vide, ut non inhæreas, ne capiaris et pereas. Apud Altissimum sit cogitatio tua, et deprecatio tua ad Christum sine intermissione dirigatur. Si nescis speculari alta et cælestia, requiesce in passione Christi, et in sacris vulneribus eius libenter habita. Si enim ad vulnera et pretiosa stigmata Iesu devote confugis, magnam in tribulatione confortationem senties; nec multum curabis hominum despectiones, faciliterque verba detrahentia perferes.
5. Christus fuit etiam in mundo ab hominibus despectus, et in maxima necessitate a notis et amicis inter opprobria derelictus. Christus pati voluit et despici, et tu audes de aliquo conqueri? Christus habuit adversarios et oblocutores, et tu vis omnes habere amicos et benefactores? Unde coronabitur patientia tua, si nihil adversitatis occurrerit? Si nihil contrarium vis pati, quomodo eris amicus Christi? Sustine te cum Christo et pro Christo, si vis regnare cum Christo.
6. Si semel perfecte introisses in interiora Iesu, et modicum de ardenti amore eius sapuisses, tunc de proprio commodo vel incommodo nihil curares, sed magis de opprobrio illato gauderes; quia amor Iesu facit hominem se ipsum contemnere. Amator Iesu et veritatis, et verus internus et liber ab affectionibus inordinatis potest se ad Deum libere convertere et elevare supra se ipsum in spiritu ac fruitive quiescere.
7. Cui sapiunt omnia, prout sunt, non ut dicuntur aut æstimantur: hic vere sapiens est et doctus magis a Deo, quam ab hominibus. Qui ab intra scit ambulare et modicum ab extra res ponderare, non requirit loca nec expectat tempora ad habenda devota exercitia. Homo internus cito se recolligit, quia numquam se totum ad exteriora effundit. Non illi obest labor exterior aut occupatio ad tempus necessaria, sed sicut res eveniunt, sic se illis accommodat. Qui intus bene dispositus est et ordinatus, non curat mirabiles et perversos hominum gestus. Tantum homo impeditur et distrahitur, quantum sibi res attrahit.
8. Si recte tibi esset, et bene purgatus esses, omnia tibi in bonum cederent et profectum. Ideo multa tibi displicent et sæpe conturbant, quia adhuc non es perfecte tibi ipsi mortuus nec segregatus ab omnibus terrenis. Nil sic maculat et implicat cor hominis, sicut impurus amor in creaturis. Si renuis consolari exterius, poteris speculari cælestia et frequenter iubilare interius.

\markright{II}
CAPUT II.
De humili submissione

1. Non magni pendas, quis pro te vel contra te sit; sed hoc age et cura, ut Deus tecum sit in omni re, quam facis. Habeas conscientiam bonam, et Deus bene te defensabit. Quem enim Deus adiuvare voluerit, nullius perversitas nocere poterit. Si tu scis tacere et pati, videbis procul dubio auxilium Domini. Ipse novit tempus et modum liberandi te, et ideo te debes illi resignare. Dei est adiuvare et ab omni confusione liberare. Sæpe valde prodest ad maiorem humilitatem servandam, quod defectus nostros alii sciunt et redarguunt.
2. Quando homo pro defectibus suis se humiliat, tunc faciliter alios placat et leviter satisfacit sibi irascentibus. Humilem Deus protegit et liberat, humilem diligit et consolatur; humili homini se inclinat, humili largitur gratiam magnam, et post suam depressionem levat ad gloriam. Humili sua secreta revelat et ad se dulciter trahit et invitat. Humilis accepta confusione satis bene est in pace: quia stat in Deo et non in mundo. Non reputes, te aliquid profecisse, nisi omnibus inferiorem te esse sentias.


CAPUT III.
De bono pacifico homine

1. Tene te primo in pace, et tunc poteris alios pacificare. Homo pacificus magis prodest, quam bene doctus. Homo passionatus etiam bonum in malum trahit et faciliter malum credit. Bonus pacificus homo omnia ad bonum convertit. Qui bene in pace est, de nullo suspicatur. Qui autem male contentus est et commotus, variis suspicionibus agitatur, nec ipse quiescit, nec alios quiescere permittit. Dicit sæpe, quod dicere non deberet, et omittit, quod sibi magis facere expediret. Considerat, quid alii facere tenentur, et neglegit, quid ipse tenetur. Habe ergo primo zelum super te ipsum, et tunc iuste zelare poteris etiam proximum tuum.
2. Tu bene scis facta tua excusare et colorare, et aliorum excusationes non vis recipere. Iustius esset, ut te accusares, ct fratrem tuum excusares. Si portari vis, porta et alium. Vide, quam longe es adhuc a vera caritate et humilitate, quæ nulli novit irasci vel indignari, nisi tantum sibi. Non est magnum, cum bonis et mansuetis conversari: hoc enim omnibus naturaliter placet; et unusquisque libenter pacem habet et secum sentientes magis diligit. Sed cum duris et perversis aut indisciplinatis aut nobis contrariantibus pacifice posse vivere, magna gratia est et laudabile nimis virileque factum.
3. Sunt, qui se ipsos in pace tenent et cum aliis etiam pacem habent. Et sunt, qui nec pacem habent, nec alios in pace dimittunt; aliis sunt graves, sed sibi semper graviores. Et sunt, qui se ipsos in pace retinent et ad pacem alios reducere student. Est tamen tota pax nostra in hac misera vita potius in humili sufferentia ponenda, quam in non sentiendo contraria. Qui melius scit pati, maiorem tenebit pacem. Iste est victor sui et dominus mundi, amicus Christi et heres cæli.


CAPUT IV.
De pura mente et simplici intentione

1. Duabus alis homo sublevatur a terrenis, simplicitate scilicet et puritate. Simplicitas debet esse in intentione, puritas in affectione. Simplicitas intendit Deum, puritas apprehendit et gustat. Nulla bona actio te impediet, si liber intus ab inordinato affectu fueris. Si nihil aliud, quam Dei beneplacitum et proximi utilitatem intendis et quæris, interna libertate perfrueris. Si rectum cor tuum esset, tunc omnis creatura speculum vitæ et liber sanctæ doctrinæ esset. Non est creatura tam parva et vilis, quæ Dei bonitatem non repræsentet.
2. Si tu esses intus bonus et purus, tunc omnia sine impedimento videres et bene caperes. Cor purum penetrat cælum et infernum. Qualis unusquisque intus est, taliter iudicat exterius. Si est gaudium in mundo, hoc utique possidet puri cordis homo. Et si est alicubi tribulatio et angustia, hoc melius novit mala conscientia. Sicut ferrum missum in ignem amittit rubiginem et totum candens efficitur: sic homo integre ad Deum se convertens a torpore exuitur, et in novum hominem transmutatur.
3. Quando homo incipit tepescere, tunc parvum metuit laborem et libenter externam accipit consolationem. Sed quando perfecte incipit se vincere et viriliter in via Dei ambulare, tunc minus ea reputat, quæ sibi prius gravia esse sentiebat.


CAPUT V.
De propria consideratione

1. Non possumus nobis ipsis nimis credere, quia sæpe gratia nobis deest et sensus. Modicum lumen est in nobis, et hoc cito per neglegentiam amittimus. Sæpe etiam non advertimus, quod tam cæci intus sumus. Sæpe male agimus, et peius excusamus. Passione interdum movemur, et zelum putamus. Parva in aliis reprehendimus, et nostra maiora pertransimus. Satis cito sentimus et ponderamus, quid ab aliis sustinemus; sed quantum alii de nobis sustinent, non advertimus. Qui bene et recte sua ponderaret, non esset, quod de alio graviter iudicaret.
2. Internus homo sui ipsius curam omnibus curis anteponit; et qui sibi ipsi diligenter intendit, faciliter de aliis tacet. Numquam eris internus et devotus, nisi de alienis silueris et ad te ipsum specialiter respexeris. Si tibi et Deo totaliter intendis, modicum te movebit, quod foris percipis. Ubi es, quando tibi ipsi præsens non es? Et quando omnia percurristi, quid te neglecto profecisti? Si debes habere pacem et unionem veram: oportet, quod totum adhuc postponas et te solum præ oculis habeas.
3. Multum proinde proficies, si te feriatum ab omni temporali cura conserves. Valde deficies, si aliquid temporale reputaveris. Nil magnum, nil altum, nil gratum, nil acceptum tibi sit, nisi pure Deus aut de Deo sit. Totum vanum existima, quidquid consolationis occurrerit de aliqua creatura. Amans Deum anima sub Deo despicit universa. Solus Deus æternus et immensus, implens omnia, solacium animæ et vera cordis lætitia.


CAPUT VI.
De lætitia bonæ conscientiæ

1. Gloria boni hominis testimonium bonæ conscientiæ. Habe bonam conscientiam et habebis semper lætitiam. Bona conscientia valde multa potest portare et valde læta est inter adversa. Mala conscientia semper timida est et inquieta. Suaviter requiesces, si cor tuum te non reprehenderit. Noli lætari, nisi cum benefeceris. Mali numquam habent veram lætitiam, nec internam sentiunt pacem: quia non est pax impiis, dicit Dominus. Et si dixerint: In pace sumus, non venient super nos mala: et quis nobis nocere audebit? ne credas eis; quoniam repente exurget ira Dei, et in nihilum redigentur actus eorum, et cogitationes eorum peribunt.
2. Gloriari in tribulatione non est grave amanti; sic enim gloriari est gloriari in cruce Domini. Brevis gloria, quæ ab hominibus datur et accipitur. Mundi gloriam semper comitatur tristitia. Bonorum gloria in conscientiis eorum et non in ore hominum. Iustorum lætitia de Deo et in Deo est, et gaudium eorum de veritate. Qui veram et æternam gloriam desiderat, temporalem non curat. Et qui temporalem requirit gloriam, aut non ex animo contemnit, minus amare convincitur cælestem. Magnam habet cordis tranquillitatem, qui nec laudes curat nec vituperia.
3. Facile erit contentus et pacatus, cuius conscientia munda est. Non es sanctior, si laudaris; nec vilior, si vituperaris. Quod es, hoc es; nec maior dici vales, quam Deo teste sis. Si attendis, quid apud te sis intus, non curabis, quid de te loquantur homines. Homo videt in facie, Deus autem in corde. Homo considerat actus, Deus vero pensat intentiones. Bene semper agere et modicum de se tenere, humilis animæ indicium est. Nolle consolari ab aliqua creatura, magnæ puritatis et internæ fiduciæ signum est.
4. Qui nullum extrinsecus pro se testimonium quærit, liquet, quod totaliter se Deo commisit. Non enim, qui se ipsum commendat, ille probatus est ait beatus Paulus: sed quem Deus commendat. Ambulare cum Deo intus, nec aliqua affectione teneri foris, status est interni hominis.


CAPUT VII.
De amore Iesu super omnia

1. Beatus, qui intellegit, quid sit amare Iesum, et contemnere se ipsum propter Iesum. Oportet dilectum pro dilecto relinquere, quia Iesus vult solus super omnia amari. Dilectio creaturæ fallax et instabilis; dilectio Iesu fidelis et perseverabilis. Qui adhæret creaturæ, cadet cum labili; qui amplectitur Iesum, firmabitur in ævum. Illum dilige et amicum tibi retine, qui omnibus recedentibus te non relinquet, nec patietur in fine perire. Ab omnibus oportet te aliquando separari, sive velis, sive nolis.
2. Teneas te apud Iesum vivens ac moriens, et illius fidelitati te committe, qui omnibus deficientibus solus te potest iuvare. Dilectus tuus talis est naturæ, ut alienum non velit admittere, sed solus vult cor tuum habere et tamquam rex in proprio throno sedere. Si scires te bene ab omni creatura evacuare, Iesus deberet libenter tecum habitare. Pæne totum perditum invenies, quidquid extra Iesum in hominibus posueris. Non confidas, nec innitaris super calamum ventosum; quia omnis caro fænum, et omnis gloria eius ut flos fæni cadet.
3. Cito decipieris, si ad externam hominum apparentiam tantum aspexeris. Si enim tuum in aliis quæris solacium et lucrum, senties sæpius detrimentum. Si quæris in omnibus Iesum, invenies utique Iesum. Si autem quæris te ipsum, invenies etiam te ipsum, sed ad tuam perniciem. Plus enim homo nocivior sibi, si Iesum non quærit, quam totus mundus et omnes sui adversarii.


CAPUT VIII.
De familiari amicitia Iesu

1. Quando Iesus adest, totum bonum est, nec quicquam difficile videtur; quando vero Iesus non adest, totum durum est. Quando Iesus intus non loquitur, consolatio vilis est; si autem Iesus unum tantum verbum loquitur, magna consolatio sentitur. Nonne Maria Magdalena statim surrexit de loco, in quo flevit, quando Martha illi dixit: Magister adest et vocat te? Felix hora, quando Iesus vocat de lacrimis ad gaudium spiritus! Quam aridus et durus es sine Iesu! Quam insipiens et vanus, si cupis aliquid extra Iesum! Nonne hoc est maius damnum, quam si totum perderes mundum?
2. Quid potest tibi mundus conferre sine Iesu? Esse sine Iesu, gravis est infernus, et esse cum Iesu, dulcis paradisus. Si fuerit tecum Iesus, nullus poterit nocere inimicus. Qui invenit Iesum, invenit thesaurum bonum, immo bonum super omne bonum. Et qui perdit Iesum, perdit nimis multum et plus quam totum mundum. Pauperrimus est, qui vivit sine Iesu, et ditissimus, qui bene est cum Iesu.
3. Magna ars est, scire cum Iesu conversari, et scire Iesum tenere, magna prudentia. Esto humilis et pacificus, et erit tecum Iesus. Sis devotus et quietus, et manebit tecum Iesus. Potes cito fugare Iesum et gratiam eius perdere, si volueris ad exteriora declinare. Et si illum effugaveris et perdideris, ad quem fugies, et quem tunc quæres amicum? Sine amico non potes bene vivere, et si Iesus non fuerit tibi præ omnibus amicus, eris nimis tristis et desolatus. Fatue igitur agis, si in aliquo altero confidis aut lætaris. Eligendum est magis, totum mundum habere contrarium quam Iesum offensum. Ex omnibus ergo caris sit Iesus solus dilectus specialis.
4. Diligantur omnes propter Iesum, Iesus autem propter se ipsum. Solus Iesus Christus singulariter est amandus, qui solus bonus et fidelis præ omnibus invenitur amicis. Propter ipsum et in ipso tam amici quam inimici sint tibi cari; et pro omnibus his exorandus est, ut omnes ipsum cognoscant et diligant. Numquam cupias singulariter laudari vel amari, quia hoc solius Dei est, qui similem sibi non habet. Nec velis, quod aliquis tecum in corde suo occupetur, neque tu cum alicuius occuperis amore; sed sit Iesus in te et in omni bono homine.
5. Esto purus et liber ab intus sine alicuius creaturæ implicamento. Oportet te esse nudum, et purum cor ad Deum gerere, si vis vacare et videre, quam suavis sit Dominus. Et revera ad hoc non pervenies, nisi gratia eius fueris præventus et intractus, ut omnibus evacuatis et licentiatis solus cum solo uniaris. Quando enim gratia Dei venit ad hominem, tunc potens fit ad omnia, et quando recedit, tunc pauper et infirmus erit, et quasi tantum ad flagella relictus. In his non debet deici nec desperare, sed ad voluntatem Dei æquanimiter stare, et cuncta supervenientia sibi ad laudem Iesu Christi perpeti; quia post hiemem sequitur æstas, post noctem redit dies, et post tempestatem magna serenitas.


CAPUT IX.
De carentia omnis solacii

1. Non est grave, humanum contemnere solacium, cum adest divinum. Magnum est et valde magnum, tam humano quam divino posse carere solacio, et pro honore Dei libenter exilium cordis velle sustinere, et in nullo se ipsum quærere, nec ad proprium meritum respicere. Quid magni est, si hilaris sis et devotus adveniente gratia? optabilis cunctis hæc hora. Satis suaviter equitat, quem gratia Dei portat. Et quid mirum, si onus non sentit, qui portatur ab omnipotente et ducitur a summo ductore?
2. Libenter habemus aliquid pro solacio, et difficulter homo exuitur a se ipso. Vicit sanctus martyr Laurentius sæculum cum suo sacerdote; quia omne, quod in mundo delectabile videbatur, despexit et summum Dei sacerdotem Sixtum, quem maxime diligebat, pro amore Christi etiam a se tolli clementer ferebat. Amore igitur creatoris amorem hominis superavit, et pro humano solacio divinum beneplacitum magis elegit. Ita et tu aliquem necessarium et dilectum amicum pro amore Dei disce relinquere. Nec graviter feras, cum ab amico derelictus fueris, sciens, quoniam oportet nos omnes tandem ab invicem separari.
3. Multum et diu oportet hominem in se ipso certare, antequam discat se ipsum plene superare et totum affectum suum in Deum trahere. Quando homo stat super se ipsum, facile labitur ad consolationes humanas. Sed verus amator Christi et studiosus sectator virtutum non cadit super consolationes, nec quærit tales sensibiles dulcedines, sed magis fortes exercitationes, et pro Christo duros sustinere labores.
4. Cum igitur spiritualis a Deo consolatio datur, cum gratiarum actione accipe eam; sed Dei munus intellege esse, non tuum meritum. Noli extolli, noli nimium gaudere, nec inaniter præsumere; sed esto magis humilior ex dono, cautior quoque et timoratior in cunctis actibus tuis; quoniam transibit hora illa, et sequetur temptatio. Cum ablata fuerit consolatio, non statim desperes: sed cum humilitate et patientia expecta cælestem visitationem, quoniam potens est Deus, ampliorem tibi redonare consolationem. Istud non est novum nec alienum viam Dei expertis, quia in magnis sanctis et in antiquis prophetis fuit sæpe talis alternationis modus.
5. Unde quidam, præsente iam gratia, dicebat: Ego dixi in abundantia mea: Non movebor in æternum. Absente vero gratia, quid in se fuerit expertus, adiungit dicens: Avertisti faciem tuam a me, et factus sum conturbatus. Inter hæc tamen nequaquam desperat, sed instantius Dominum rogat et dicit: Ad te, Domine, clamabo et ad Deum meum deprecabor. Denique orationis suæ fructum reportat et se exauditum testatur dicens: Audivit Dominus et misertus est mei; Dominus factus est adiutor meus. Sed in quo? Convertisti, inquit, planctum meum in gaudium mihi, et circumdedisti me lætitia. Si sic actum est cum magnis sanctis, non est desperandum nobis infirmis et pauperibus, si interdum in fervore et interdum in frigiditate sumus: quoniam spiritus venit et recedit secundum suæ beneplacitum voluntatis. Unde beatus Iob ait: Visitas eum diluculo, et subito probas illum.
6. Super quid igitur sperare possum, aut in quo confidere debeo, nisi in sola magna misericordia Dei et in sola spe gratiæ cælestis? Sive enim assint homines boni, sive devoti fratres, vel amici fideles, sive libri sancti, vel tractatus pulchri, sive dulcis cantus et hymni: omnia hæc modicum iuvant, modicum sapiunt, quando desertus sum a gratia et in propria paupertate relictus. Tunc non est melius remedium, quam patientia et abnegatio mei in voluntate Dei.
7. Numquam inveni aliquem tam religiosum et devotum, qui non habuerit interdum gratiæ subtractionem, aut non senserit fervoris diminutionem. Nullus sanctus fuit tam alte raptus et illuminatus, qui prius vel postea non fuerit temptatus. Non enim dignus est alta Dei contemplatione, qui pro Deo non est exercitatus aliqua tribulatione. Solet enim sequentis consolationis temptatio præcedens esse signum. Nam temptationibus probatis cælestis promittitur consolatio. Qui vicerit, inquit, dabo ei edere de ligno vitæ.
8. Datur autem consolatio divina, ut homo fortior sit ad sustinendum adversa. Sequitur etiam temptatio, ne se elevet de bono. Non dormit diabolus, nec caro adhuc mortua est: ideo non cesses te præparare ad certamen, quia a dextris et a sinistris hostes sunt, qui numquam quiescunt.


CAPUT X.
De gratitudine pro gratia Dei

1. Cur quæris quietem, cum natus sis ad laborem? Pone te ad patientiam magis, quam ad consolationes, et ad crucem portandam magis, quam ad lætitiam. Quis enim sæcularium non libenter consolationem et lætitiam spiritualem acciperet, si semper obtinere posset? Excedunt enim spirituales consolationes omnes mundi delicias et carnis voluptates. Nam omnes deliciæ mundanæ aut vanæ sunt aut turpes. Spirituales vero deliciæ solæ iucundæ et honestæ, ex virtutibus progenitæ et a Deo puris mentibus infusæ. Sed istis divinis consolationibus nemo semper pro suo affectu frui valet, quia tempus temptationis non diu cessat.
2. Multum autem contrariatur supernæ visitationi falsa libertas animi et magna confidentia sui. Deus bene facit consolationis gratiam dando, sed homo male agit, non totum Deo cum gratiarum actione retribuendo. Et ideo non possunt in nobis dona gratiæ fluere, quia ingrati sumus auctori, nec totum refundimus fontali origini. Semper enim debetur gratia digne gratias referenti, et auferetur ab elato, quod dari solet humili.
3. Nolo consolationem, quæ mihi aufert compunctionem; nec affecto contemplationem, quæ ducit in elationem. Non enim omne altum sanctum, nec omne dulce bonum, nec omne desiderium purum, nec omne carum Deo gratum. Libenter accepto gratiam, unde semper humilior et timoratior inveniar, atque ad relinquendum me paratior fiam. Doctus dono gratiæ et eruditus subtractionis verbere non sibi audebit quicquam boni attribuere, sed potius se pauperem et nudum confitebitur. Da Deo, quod Dei est, et tibi ascribe, quod tuum est; hoc est: Deo gratias pro gratia tribue, tibi autem soli culpam et dignam pœnam pro culpa deberi sentias.
4. Pone te semper ad infimum, et dabitur tibi summum; nam summum non stat sine infimo. Summi sancti apud Deum minimi sunt apud se, et quanto gloriosiores, tanto in se humiliores. Pleni veritate et gloria cælesti non sunt vanæ gloriæ cupidi. In Deo fundati et confirmati nullo modo possunt esse elati. Et qui totum Deo ascribunt, quidquid boni acceperunt, gloriam ab invicem non quærunt, sed gloriam, quæ a solo Deo est, volunt, et Deum in se et in omnibus sanctis laudari super omnia cupiunt et semper in id ipsum tendunt.
5. Esto igitur gratus pro minimo, et eris dignus maiora accipere. Sit tibi minimum etiam pro maximo, et magis contemptibile pro speciali dono. Si dignitas datoris inspicitur, nullum datum parvum aut nimis vile videbitur. Non enim parvum est, quod a summo Deo donatur. Etiam si pœnas et verbera dederit, gratum esse debet, quia semper pro salute nostra facit, quidquid nobis advenire permittit. Qui gratiam Dei retinere desiderat, sit gratus pro gratia data, patiens pro sublata. Oret, ut redeat; cautus sit et humilis, ne amittat.


CAPUT XI.
De paucitate amatorum crucis Iesu

1. Habet Iesus nunc multos amatores regni sui cælestis, sed paucos baiulatores suæ crucis. Multos habet desideratores consolationis, sed paucos tribulationis. Plures invenit socios mensæ, sed paucos abstinentiæ. Omnes cupiunt cum eo gaudere, pauci volunt pro eo aliquid sustinere. Multi Iesum sequuntur usque ad fractionem panis, sed pauci usque ad bibendum calicem passionis. Multi miracula eius venerantur, pauci ignominiam crucis sequuntur. Multi Iesum diligunt, quamdiu adversa non contingunt. Multi illum laudant et benedicunt, quamdiu consolationes aliquas ab ipso percipiunt. Si autem Iesus se absconderit et modicum eos reliquerit, aut in querimoniam vel in deiectionem nimiam cadunt.
2. Qui autem Iesum propter Iesum et non propter suam propriam aliquam consolationem diligunt, ipsum in omni tribulatione et angustia cordis, sicut in summa consolatione, benedicunt. Et si numquam eis consolationem dare vellet, ipsum tamen semper laudarent et semper gratias agere vellent.
3. O quantum potest amor Iesu purus, nullo proprio commodo vel amore permixtus! Nonne omnes mercenarii sunt dicendi, qui consolationes semper quærunt? Nonne amatores sui magis quam Christi probantur, qui sua commoda et lucra semper meditantur? Ubi invenietur talis, qui velit Deo servire gratis?
4. Raro invenitur tam spiritualis aliquis, qui omnibus sit nudatus. Nam verum pauperem spiritu et ab omni creatura nudum, quis inveniet? Procul et de ultimis finibus pretium eius. Si dederit homo omnem substantiam suam, adhuc nihil est. Et si fecerit pænitentiam magnam, adhuc exiguum est. Et si apprehenderit omnem scientiam, adhuc longe est. Et si habuerit virtutem magnam et devotionem nimis ardentem, adhuc multum sibi deest: scilicet unum, quod summe sibi necessarium est. Quid illud? Ut omnibus relictis se relinquat et a se totaliter exeat, nihilque de privato amore retineat. Cumque omnia fecerit, quæ facienda noverit, nil se fecisse sentiat.
5. Non grande ponderet, quod grande æstimari possit, sed in veritate servum inutilem se pronuntiet, sicut Veritas ait: Cum feceritis omnia, quæ præcepta sunt vobis, dicite: Servi inutiles sumus. Tunc vere pauper et nudus spiritu esse poterit, et cum propheta dicere: Quia unicus et pauper sum ego. Nemo tamen isto ditior, nemo potentior, nemo liberior, qui se et omnia relinquere scit et ad infimum se ponere.


CAPUT XII.
De regia via sanctæ crucis

1. Durus multis videtur hic sermo: Abnega temet ipsum, tolle crucem tuam et sequere Iesum. Sed multo durius erit, audire illud extremum verbum: Discedite a me, maledicti, in ignem æternum. Qui enim modo libenter audiunt et sequuntur verbum crucis, tunc non timebunt ab auditione æternæ damnationis. Hoc signum crucis erit in cælo, cum Dominus ad iudicandum venerit. Tunc omnes servi crucis, qui se Crucifixo conformaverunt in vita, accedent ad Christum iudicem cum magna fiducia.
2. Quid igitur times tollere crucem, per quam itur ad regnum? In cruce salus, in cruce vita, in cruce protectio ab hostibus; in cruce infusio supernæ suavitatis, in cruce robur mentis, in cruce gaudium spiritus; in cruce summa virtutis, in cruce perfectio sanctitatis. Non est salus animæ nec spes æternæ vitæ, nisi in cruce. Tolle ergo crucem tuam et sequere Iesum, et ibis in vitam æternam. Præcessit ille baiulans sibi crucem et mortuus est pro te in cruce, ut et tu tuam portes crucem, et mori affectes in cruce. Quia, si commortuus fueris, etiam cum illo pariter vives. Et si socius fueris pœnæ, eris et gloriæ.
3. Ecce, in cruce totum constat, et in moriendo totum iacet; et non est alia via ad vitam et ad veram internam pacem, nisi via sanctæ crucis et cotidianæ mortificationis. Ambula, ubi vis; quære, quodcumque volueris: et non invenies altiorem viam supra nec securiorem viam infra, nisi viam sanctæ crucis. Dispone et ordina omnia secundum tuum velle et videre: et non invenies, nisi semper aliquid pati debere aut sponte aut invite, et ita crucem semper invenies. Aut enim in corpore dolorem senties, aut in anima spiritus tribulationem sustinebis.
4. Interdum a Deo relinqueris, interdum a proximo exercitaberis, et quod amplius est, sæpe tibimet ipsi gravis eris; nec tamen aliquo remedio vel solacio liberari seu alleviari poteris, sed, donec Deus voluerit, oportet, ut sustineas. Vult enim Deus, ut tribulationem sine consolatione pati discas, et ut illi totaliter te subicias et humilior ex tribulatione fias. Nemo ita cordialiter sentit passionem Christi, sicut is, cui contigerit similia pati. Crux ergo semper parata est et ubique te exspectat. Non potes effugere, ubicumque cucurreris: quia, ubicumque veneris, te ipsum tecum portas, et semper te ipsum invenies. Converte te supra, converte te infra: converte te extra, converte te intra: et in his omnibus invenies crucem, et necesse est, te ubique tenere patientiam, si internam vis habere pacem et perpetuam promereri coronam.
5. Si libenter crucem portas, portabit te et ducet ad desideratum finem, ubi scilicet finis patiendi erit, quamvis hic non erit. Si invite portas, onus tibi facis, et te ipsum magis gravas; et tamen oportet, ut sustineas. Si abicis unam crucem, aliam procul dubio invenies et forsitan graviorem.
6. Credis tu evadere, quod nullus mortalium potuit præterire? Quis sanctorum in mundo sine cruce et tribulatione fuit? Nec enim Iesus Christus, Dominus noster, una hora sine dolore passionis fuit, quamdiu vixit. Oportebat, ait, Christum pati, et resurgere a mortuis, et ita intrare in gloriam suam. Et quomodo tu aliam viam quæris, quam hanc regiam viam, quæ est via sanctæ crucis?
7. Tota vita Christi crux fuit et martyrium; et tu tibi quæris requiem et gaudium? Erras, erras, si aliud quæris, quam pati tribulationes; quia tota ista vita mortalis plena est miseriis et circumsignata crucibus. Et quanto altius quis in spiritu profecerit, tanto graviores sæpe cruces invenerit: quia exilii sui pœna magis ex amore crescit.
8. Sed tamen iste sic multipliciter afflictus non est sine levamine consolationis, quia fructum maximum sibi sentit accrescere ex sufferentia suæ crucis. Nam dum sponte se illi subicit, omne onus tribulationis in fiduciam divinæ consolationis convertitur. Et quanto caro magis per afflictionem atteritur, tanto spiritus amplius per internam gratiam roboratur. Et nonnumquam in tantum confortatur ex affectu tribulationis et adversitatis, ob amorem conformitatis crucis Christi, ut se sine dolore et tribulatione esse non vellet; quoniam tanto se acceptiorem Deo credit, quanto plura et graviora pro eo perferre potuerit. Non est istud hominis virtus, sed gratia Christi, quæ tanta potest et agit in carne fragili, ut, quod naturaliter semper abhorret et fugit, hoc fervore spiritus aggrediatur et diligat.
9. Non est secundum hominem crucem portare, crucem amare, corpus castigare et servituti subicere, honores fugere, contumelias libenter sustinere, se ipsum despicere et despici optare, adversa quæque cum damnis perpeti et nihil prosperitatis in hoc mundo desiderare. Si ad te ipsum respicis, nihil huiusmodi ex te poteris. Sed si in Domino confidis, dabitur tibi fortitudo de cælo, et subicientur ditioni tuæ mundus et caro. Sed nec inimicum diabolum timebis, si fueris fide armatus et cruce Christi signatus.
10. Pone te ergo, sicut bonus et fidelis servus Christi, ad portandam viriliter crucem Domini tui, pro te ex amore crucifixi. Præpara te ad toleranda multa adversa et varia incommoda in hac misera vita; quia sic tecum erit, ubicumque fueris; et sic revera invenies, ubicumque latueris. Oportet ita esse; et non est remedium evadendi a tribulatione malorum et dolore, quam ut te patiaris. Calicem Domini affectanter bibe, si amicus eius esse et partem cum eo habere desideras. Consolationes Deo committe; faciat ipse cum talibus, sicut sibi magis placuerit. Tu vero, pone te ad sustinendum tribulationes, et reputa eas maximas consolationes, quia non sunt condignæ passiones huius temporis ad futuram gloriam promerendam, etiam si solus omnes posses sustinere.
11. Quando ad hoc veneris, quod tribulatio tibi dulcis est et sapit pro Christo: tunc bene tecum esse æstima, quia invenisti paradisum in terra. Quamdiu pati grave tibi est et fugere quæris: tam diu male habebis, et sequetur te ubique fuga tribulationis.
12. Si ponis te, ad quod esse debes, videlicet ad patiendum et moriendum, fiet cito melius, et pacem invenies. Etiam si raptus fueris usque ad tertium cælum cum Paulo, non es propterea securatus de nullo contrario patiendo. Ego, inquit Iesus, ostendam illi, quanta oporteat eum pro nomine meo pati. Pati ergo tibi remanet, si Iesum diligere et perpetuo illi servire placet.
13. Utinam dignus esses aliquid pro nomine Iesu pati; quam magna gloria remaneret tibi, quanta exultatio omnibus sanctis Dei, quanta quoque ædificatio esset proximi! Nam patientiam omnes recommendant, quamvis pauci tamen pati velint. Merito deberes libenter modicum pati pro Christo, cum multi graviora patiuntur pro mundo.
14. Scias pro certo, quia morientem te oportet ducere vitam. Et quanto quisque plus sibi moritur, tanto magis Deo vivere incipit. Nemo aptus est ad comprehendendum cælestia, nisi se submiserit ad portandum pro Christo adversa. Nihil Deo acceptius, nihil tibi salubrius in mundo isto, quam libenter pati pro Christo. Et si eligendum tibi esset, magis optare deberes pro Christo adversa pati, quam multis consolationibus recreari; quia Christo similior esses et omnibus sanctis magis conformior. Non enim stat meritum nostrum et profectus status nostri in multis suavitatibus et consolationibus, sed potius in magnis gravitatibus et tribulationibus perferendis.
15. Si quidem aliquid melius et utilius saluti hominum, quam pati, fuisset, Christus utique verbo et exemplo ostendisset. Nam et sequentes se discipulos omnesque eum sequi cupientes manifeste ad crucem portandam hortatur, et dicit: Si quis vult venire post me, abneget semet ipsum, et tollat crucem suam, et sequatur me. Omnibus ergo perlectis et scrutatis, sit hæc conclusio finalis: Quoniam per multas tribulationes oportet nos intrare in regnum Dei.




LIBER III.
DE INTERNA CONSOLATIONE


CAPUT I.
De interna Christi locutione ad animam fidelem

1. Audiam, quid loquatur in me Dominus Deus. Beata anima, quæ Dominum in se loquentem audit, et de ore eius consolationis verbum accipit. Beatæ aures, quæ venas divini susurri suscipiunt, et de mundi huius susurrationibus nihil advertunt. Beatæ plane aures, quæ non vocem foris sonantem, sed intus auscultant veritatem docentem. Beati oculi, qui exterioribus clausi, interioribus autem sunt intenti. Beati, qui interna penetrant, et ad capienda arcana cælestia magis ac magis per cotidiana exercitia se student præparare. Beati, qui Deo vacare gestiunt, et ab omni impedimento sæculi se excutiunt. Animadverte hæc, anima mea, et claude sensualitatis tuæ ostia, ut possis audire, quid in te loquatur Dominus Deus tuus.
2. Hæc dicit dilectus tuus: Salus tua ego sum, pax tua et vita tua. Serva te apud me, et pacem invenies. Dimitte omnia transitoria, quære æterna. Quid sunt omnia temporalia, nisi seductoria? Et quid iuvant omnes creaturæ, si fueris a creatore deserta? Omnibus ergo abdicatis creatori tuo te redde placitam ac fidelem, ut veram valeas apprehendere beatitudinem.


CAPUT II.
Quod veritas intus loquitur sine strepitu verborum

1. Loquere, Domine, quia audit servus tuus. Servus tuus sum ego; da mihi intellectum, ut sciam testimonia tua. Inclina cor meum in verba oris tui: fluat ut ros eloquium tuum. Dicebant olim filii Isræl ad Moysen: Loquere tu nobis, et audiemus; non loquatur nobis Dominus, ne forte moriamur. Non sic, Domine, non sic oro, sed magis cum Samuele propheta humiliter ac desideranter obsecro: Loquere, Domine, quia audit servus tuus. Non loquatur mihi Moyses aut aliquis ex prophetis: sed tu potius loquere, Domine Deus, inspirator et illuminator omnium prophetarum; quia tu solus sine eis potes me perfecte imbuere; illi autem sine te nihil proficient.
2. Possunt quidem verba sonare, sed spiritum non conferunt. Pulchriter dicunt, sed te tacente cor non accendunt. Litteras tradunt, sed tu sensum aperis. Mysteria proferunt, sed tu reseras intellectum signatorum. Mandata edicunt, sed tu iuvas ad perficiendum. Viam ostendunt, sed tu confortas ad ambulandum. Illi foris tantum agunt, sed tu corda instruis et illuminas. Illi exterius rigant, sed tu fecunditatem donas. Illi clamant verbis, sed tu auditui intellegentiam tribuis.
3. Non ergo loquatur mihi Moyses, sed tu, Domine Deus meus, æterna veritas, ne forte moriar et sine fructu efficiar, si fuero tantum foris admonitus et intus non accensus; ne sit mihi ad iudicium verbum auditum et non factum, cognitum nec amatum, creditum et non servatum. Loquere igitur, Domine, quia audit servus tuus; verba enim vitæ æternæ habes. Loquere mihi ad qualemcumque animæ meæ consolationem et ad totius vitæ meæ emendationem, tibi autem ad laudem et gloriam et perpetuum honorem.


CAPUT III.
Quod verba Dei cum humilitate sunt audienda, et quod multi ea non ponderant

1. Audi, fili, verba mea, verba suavissima, omnem philosophorum et sapientium huius mundi scientiam excedentia. Verba mea spiritus et vita sunt, nec humano sensu pensanda. Non sunt ad vanam complacentiam trahenda, sed in silentio audienda, et cum omni humilitate atque magno affectu suscipienda.
2. Et dixi: Beatus, quem tu erudieris, Domine, et de lege tua docueris eum, ut mitiges ei a diebus malis, et non desoletur in terra.
3. Ego, inquit Dominus, docui prophetas ab initio, et usque nunc non cesso omnibus loqui; sed multi ad vocem meam surdi sunt et duri. Plures mundum libentius audiunt, quam Deum; facilius sequuntur carnis suæ appetitum, quam Dei beneplacitum. Promittit mundus temporalia et parva, et servitur ei aviditate magna: ego promitto summa et æterna, et torpescunt mortalium corda. Quis tanta cura mihi in omnibus servit et obœdit, sicut mundo et dominis eius servitur? Erubesce, Sidon, ait mare, et si causam quæris, audi, quare. Pro modica præbenda longa via curritur; pro æterna vita a multis vix pes semel a terra levatur. Vile pretium quæritur; pro uno numismate interdum turpiter litigatur; pro vana re et parva promissione die noctuque fatigari non timetur.
4. Sed, pro pudor! pro bono incommutabili, pro præmio inæstimabili, pro summo honore et gloria interminabili vel ad modicum fatigari pigritatur. Erubesce ergo, serve piger et querulose, quod illi paratiores inveniuntur ad perditionem, quam tu ad vitam. Gaudent illi amplius ad vanitatem, quam tu ad veritatem. Equidem a spe sua nonnumquam frustrantur, sed promissio mea neminem fallit, nec confidentem mihi dimittit inanem. Quod promisi, dabo; quod dixi, implebo; si tamen usque in finem fidelis in dilectione mea quis permanserit. Ego remunerator sum omnium bonorum, et fortis probator omnium devotorum.
5. Scribe verba mea in corde tuo, et pertracta diligenter; erunt enim in tempore temptationis valde necessaria. Quod non intellegis, cum legis, cognosces in die visitationis. Dupliciter soleo electos meos visitare, temptatione scilicet et consolatione. Et duas lectiones eis cotidie lego: unam increpando eorum vitia, alteram exhortando ad virtutum incrementa. Qui habet verba mea et spernit ea, habet, qui iudicet eum in novissimo die.
Oratio ad implorandam devotionis gratiam
6. Domine Deus meus, omnia bona mea tu es. Et quis ego sum, ut audeam ad te loqui? Ego sum pauperrimus servulus tuus, et abiectus vermiculus, multo pauperior et contemptibilior, quam scio et dicere audeo. Memento, tamen, Domine, quia nihil sum, nihil habeo, nihilque valeo. Tu solus bonus, iustus et sanctus; tu omnia potes, omnia præstas, omnia imples, solum peccatorem inanem relinquens. Reminiscere miserationum tuarum, et imple cor meum gratia tua, qui non vis esse vacua opera tua.
7. Quomodo possum me tolerare in hac misera vita, nisi me confortaveris misericordia et gratia tua? Noli avertere faciem tuam a me; noli visitationem tuam prolongare; noli consolationem tuam abstrahere, ne fiat anima mea sicut terra sine aqua tibi. Domine, doce me facere voluntatem tuam; doce me coram te digne et humiliter conversari; quia sapientia mea tu es, qui in veritate me cognoscis, et cognovisti, antequam fieret mundus, et antequam natus essem in mundo.


CAPUT IV.
Quod in veritate et humilitate coram Deo conversandum est

1. Fili, ambula coram me in veritate, et in simplicitate cordis tui quære me semper. Qui ambulat coram me in veritate, tutabitur ab incursibus malis, et veritas liberabit eum a seductoribus et detractionibus iniquorum. Si veritas te liberaverit, vere liber eris, et non curabis de vanis hominum verbis.
Domine, verum est, sicut dicis; ita, quæso, mecum fiat. Veritas tua me doceat, ipsa me custodiat et usque ad salutarem finem conservet. Ipsa me liberet ab omni affectione mala et inordinata dilectione; et ambulabo tecum in magna cordis libertate.
2. Ego te docebo, ait veritas, quæ recta sunt et placita coram me. Cogita peccata tua cum displicentia magna et mærore; et numquam reputes, te aliquid esse propter opera bona. Revera peccator es et multis passionibus obnoxius et implicatus. Ex te semper ad nihil tendis; cito laberis, cito vinceris, cito turbaris, cito dissolveris. Non habes quicquam, unde possis gloriari, sed multa, unde te debeas vilificare; quia multo infirmior es, quam vales comprehendere.
3. Nil ergo magnum tibi videatur ex omnibus, quæ agis. Nil grande, nil pretiosum et admirabile, nil reputatione appareat dignum; nil altum, nil vere laudabile et desiderabile, nisi quod æternum est. Placeat tibi super omnia æterna veritas, displiceat tibi semper tua maxima vilitas. Nil sic timeas, sic vituperes et fugias, sicut vitia et peccata tua, quæ magis displicere debent, quam quælibet rerum damna. Quidam non sincere coram me ambulant, sed quadam curiositate et arrogantia ducti volunt secreta mea scire et alta Dei intellegere, se et suam salutem neglegentes. Hi sæpe in magnas temptationes et peccata propter suam superbiam et curiositatem, me eis adversante, labuntur.
4. Time iudicia Dei, expavesce iram omnipotentis. Noli autem discutere opera altissimi, sed tuas iniquitates perscrutare, in quantis deliquisti, et quam multa bona neglexisti. Quidam solum portant suam devotionem in libris, quidam in imaginibus, quidam autem in signis exterioribus et figuris. Quidam habent me in ore, sed modicum est in corde. Sunt alii, qui intellectu illuminati et affectu purgati ad æterna semper anhelant, de terrenis graviter audiunt, necessitatibus naturæ dolenter inserviunt; et hi sentiunt, quid spiritus veritatis loquitur in eis. Quia docet eos terrena despicere et amare cælestia, mundum neglegere et cælum tota die ac nocte desiderare.


CAPUT V.
De mirabili effectu divini amoris

1. Benedico te, Pater cælestis, Pater Domini mei Iesu Christi, quia mei pauperis dignatus es recordari. O Pater misericordiarum et Deus totius consolationis, gratias tibi, qui me indignum omni consolatione quandoque tua recreas consolatione. Benedico te semper et glorifico, cum unigenito Filio tuo et Spiritu Sancto paraclito in sæcula sæculorum. Eia, Domine Deus, amator sancte meus, cum tu veneris in cor meum, exultabunt omnia interiora mea. Tu es gloria mea et exultatio cordis mei. Tu spes mea et refugium meum in die tribulationis meæ.
2. Sed quia adhuc debilis sum in amore et imperfectus in virtute, ideo necesse habeo a te confortari et consolari; propterea visita me sæpius et instrue disciplinis sanctis. Libera me a passionibus malis, et sana cor meum ab omnibus affectionibus inordinatis, ut intus sanatus et bene purgatus, aptus efficiar ad amandum, fortis ad patiendum, stabilis ad perseverandum.
3. Magna res est amor, magnum omnino bonum, quod solum leve facit omne onerosum et fert æqualiter omne inæquale. Nam onus sine onere portat et omne amarum dulce ac sapidum efficit. Amor Iesu nobilis ad magna operanda impellit et ad desideranda semper perfectiora excitat. Amor vult esse sursum, nec ullis infimis rebus retineri. Amor vult esse liber et ab omni mundana affectione alienus, ne internus eius impediatur aspectus; ne per aliquod commodum temporale implicationes sustineat aut per incommodum succumbat. Nihil dulcius est amore, nihil fortius, nihil altius, nihil latius, nihil iucundius, nihil plenius nec melius in cælo et in terra: quia amor ex Deo natus est, nec potest, nisi in Deo, super omnia creata quiescere.
4. Amans volat, currit et lætatur; liber est et non tenetur. Dat omnia pro
omnibus, et habet omnia in omnibus; quia in uno summo super omnia quiescit, ex quo omne bonum fluit et procedit. Non respicit ad dona, sed ad donantem se convertit super omnia bona. Amor modum sæpe nescit, sed super omnem modum fervescit. Amor onus non sentit, labores non reputat; plus affectat, quam valet; de impossibilitate non causatur, quia cuncta sibi posse et licere arbitratur. Valet igitur ad omnia, et multa implet et effectui mancipat, ubi non amans deficit et iacet.
5. Amor vigilat et dormiens non dormitat. Fatigatus non lassatur, arctatus non arctatur, territus non conturbatur; sed sicut vivax flamma et ardens facula sursum erumpit secureque pertransit. Si quis amat, novit, quid hæc vox clamet. Magnus clamor in auribus Dei est ipse ardens affectus animæ, quæ dicit: Deus meus, amor meus, tu totus meus, et ego totus tuus.
6. Dilata me in amore, ut discam interiori cordis ore degustare, quam suave sit amare et in amore liquefieri et natare. Tenear amore, vadens supra me, præ nimio fervore et stupore. Cantem amoris canticum, sequar te dilectum meum in altum, deficiat in laude tua anima mea, iubilans ex amore. Amem te plus quam me, nec me nisi propter te, et omnes in te, qui vere amant te, sicut iubet lex amoris lucens ex te.
7. Est amor velox, sincerus, pius, iucundus et amœnus; fortis, patiens, fidelis, prudens, longanimis, virilis et se ipsum numquam quærens. Ubi enim se ipsum aliquis quærit, ibi ab amore cadit. Est amor circumspectus, humilis et rectus, non mollis, non levis, nec vanis intendens rebus; sobrius, castus, stabilis, quietus et in cunctis sensibus custoditus. Est amor subiectus et obœdiens prælatis, sibi vilis et despectus, Deo devotus et gratificus, fidens et sperans semper in eo, etiam cum sibi non sapit Deus: quia sine dolore non vivitur in amore.
8. Qui non est paratus omnia pati, et ad voluntatem stare dilecti, non est dignus amator appellari. Oportet amantem omnia dura et amara propter dilectum libenter amplecti, nec ob contraria accidentia ab eo deflecti.


CAPUT VI.
De probatione veri amatoris

1. Fili, non es adhuc fortis et prudens amator.
Ouare, Domine?
Quia propter modicam contrarietatem deficis a cœptis, et nimis avide consolationem quæris. Fortis amator stat in temptationibus, nec callidis credit persuasionibus inimici. Sicut ei in prosperis placeo, ita nec in adversis displiceo.
2. Prudens amator non tam donum amantis considerat, quam dantis amorem. Affectum potius attendit, quam censum, et infra dilectum omnia data ponit. Nobilis amator non quiescit in dono, sed in me super omne donum. Non est ideo totum perditum, si quandoque minus bene de me, vel de sanctis meis sentis, quam velles. Affectus ille bonus et dulcis, quem interdum percipis, effectus gratiæ præsentis est, et quidam prægustus patriæ cælestis, super quo non nimium innitendum, quia vadit et venit. Certare autem adversus incidentes malos motus animi, suggestionemque spernere diaboli, insigne est virtutis et magni meriti.
3. Non ergo te conturbent alienæ phantasiæ de quacumque materia ingestæ. Forte serva propositum et intentionem rectam ad Deum. Nec est illusio, quod aliquando in excessum subito raperis et statim ad solitas ineptias cordis reverteris. Illas enim invite magis pateris, quam agis; et quamdiu displicent et reniteris, meritum est et non perditio.
4. Scito, quod antiquus inimicus omnino nititur impedire desiderium tuum in bono et ab omni devoto exercitio evacuare: a sanctorum scilicet cultu, a pia passionis meæ memoria, a peccatorum utili recordatione, a proprii cordis custodia et a firmo proposito proficiendi in virtute. Multas malas cogitationes ingerit, ut tædium tibi faciat et horrorem: ut ab oratione revocet et sacra lectione. Displicet sibi humilis confessio, et, si posset, a communione cessare faceret. Non credas ei, neque cures illum, licet sæpius tibi deceptionis tetenderit laqueos. Sibi imputa, cum mala inserit et immunda. Dicito illi: Vade, immunde spiritus, erubesce, miser, valde immundus es tu, qui talia infers auribus meis. Discede a me, seductor pessime, non habebis in me partem ullam; sed Iesus mecum erit, tamquam bellator fortis, et tu stabis confusus. Malo mori et omnem pœnam subire, quam tibi consentire. Tace et obmutesce, non audiam te amplius, licet plures mihi moliaris molestias. Dominus illuminatio mea et salus mea, quem timebo? Si consistant adversum me castra, non timebit cor meum. Dominus adiutor meus et redemptor meus.
5. Certa tamquam miles bonus; et si interdum ex fragilitate corruis, resume vires fortiores prioribus, confidens de ampliori gratia mea, et multum præcave a vana complacentia et superbia. Propter hoc multi in errorem ducuntur, et in cæcitatem pæne incurabilem quandoque labuntur. Sit tibi in cautelam et perpetuam humilitatem ruina hæc superborum de se stulte præsumentium.


CAPUT VII.
De occultanda gratia sub humilitatis custodia

1. Fili, utilius est tibi et securius, devotionis gratiam abscondere, nec in altum te efferre, nec multum inde loqui, nec multum ponderare; sed magis temet ipsum despicere et tamquam indigno datam timere. Non est huic affectioni tenacius inhærendum, quæ citius potest mutari in contrarium. Cogita in gratia, quam miser et inops esse soles sine gratia. Nec est in eo tantum spiritualis vitæ profectus, cum consolationis habueris gratiam; sed cum humiliter et abnegate patienterque tuleris eius subtractionem: ita, quod tunc ab orationis studio non torpeas, nec reliqua opera tua ex usu facienda omnino dilabi permittas, sed sicut melius potueris et intellexeris, libenter quod in te est, facias, nec propter ariditatem seu anxietatem mentis, quam sentis, te totaliter neglegas.
2. Multi enim sunt, qui, cum non bene eis successerit, statim impatientes fiunt aut desides. Non enim semper est in potestate hominis via eius, sed Dei est dare et consolari, quando vult et quantum vult et cui vult, sicut sibi placuerit, et non amplius. Quidam incauti propter devotionis gratiam se ipsos destruxerunt, quia plus agere voluerunt, quam potuerunt, non pensantes suæ parvitatis mensuram, sed magis cordis affectum sequentes, quam rationis iudicium. Et quia maiora præsumpserunt, quam Deo placitum fuit, idcirco gratiam cito perdiderunt. Facti sunt inopes et viles relicti, qui in cælum posuerunt nidum sibi, ut humiliati et depauperati discant non in alis suis volare, sed sub pennis meis sperare. Qui adhuc novi sunt et imperiti in via Domini, nisi consilio discretorum se regant, faciliter decipi possunt et elidi.
3. Quodsi suum sentire magis sequi, quam aliis exercitatis credere volunt, erit eis periculosus exitus, si tamen retrahi a proprio conceptu noluerint. Raro sibi ipsis sapientes ab aliis regi humiliter patiuntur. Melius est, sapere modicum cum humilitate et parva intellegentia, quam magni scientiarum thesauri cum vana complacentia. Melius est tibi minus habere, quam multum, unde posses superbire. Non satis discrete agit, qui se totum lætitiæ tradit, obliviscens pristinæ inopiæ suæ et casti timoris Domini, qui timet gratiam oblatam amittere. Nec etiam satis virtuose sapit, qui tempore adversitatis et cuiusque gravitatis nimis desperate se gerit, et minus fidenter de me, quam oportet, recogitat ac sentit.
4. Qui tempore pacis nimis securus esse voluerit, sæpe tempore belli nimis deiectus et formidolosus reperietur. Si scires semper humilis et modicus in te permanere, nec non spiritum tuum bene moderare et regere, non incideres tam cito in periculum et offensam. Consilium bonum est, ut fervoris spiritu concepto mediteris, quid futurum sit abscedente lumine. Quod dum contigerit, recogita et denuo lucem posse reverti, quam ad cautelam tibi, mihi autem ad gloriam, ad tempus subtraxi.
5. Utilior est sæpe talis probatio, quam si semper prospera pro tua haberes voluntate. Nam merita non sunt ex hoc existimanda, si quis plures visiones aut consolationes habeat, vel si peritus sit in scripturis, aut in altiori ponatur gradu: sed si vera fuerit humilitate fundatus et divina caritate repletus; si Dei honorem pure et integre semper quærat; si se ipsum nihil reputet et in veritate despiciat, atque ab aliis etiam despici et humiliari magis gaudeat, quam honorari.


CAPUT VIII.
De vili æstimatione sui ipsius in oculis Dei

1. Loquar ad Dominum meum, cum sim pulvis et cinis. Si me amplius reputavero, ecce, tu stas contra me, et dicunt testimonium verum iniquitates meæ, nec possum contradicere. Si autem me vilificavero et ad nihilum redegero, et ab omni propria reputatione defecero, atque, sicut sum, pulverizavero, erit mihi propitia gratia tua et vicina cordi meo lux tua; et omnis æstimatio, quantulacumque minima, in valle nihileitatis meæ submergetur et peribit in æternum. Ibi ostendis me mihi, quid sum, quid fui, et quo deveni: quia nihil sum, et nescivi. Si mihi ipsi relinquor, ecce nihil et tota infirmitas; si autem subito me respexeris, statim fortis efficior, et novo repleor gaudio. Et mirum valde, quod sic repente sublevor et tam benigne a te complector, qui proprio pondere semper ad ima feror.
2. Facit hoc amor tuus, gratis præveniens me, et in tam multis subveniens necessitatibus, a gravibus quoque custodiens me periculis, et ab innumeris, ut vere dicam, eripiens malis. Me siquidem male amando me perdidi; et te solum quærendo et pure amando me et te pariter inveni, atque ex amore profundius ad nihilum me redegi. Quia tu, o dulcissime, facis mecum supra meritum omne, et supra id, quam audeo sperare vel rogare.
3. Benedictus sis, Deus meus, quia, licet ego omnibus bonis sim indignus, tua tamen nobilitas et infinita bonitas numquam cessat benefacere etiam ingratis et longe a te aversis. Converte nos ad te, ut simus grati, humiles et devoti, quia salus nostra tu es, virtus et fortitudo nostra.


CAPUT IX.
Quod omnia ad Deum, sicut ad finem ultimum, sunt referenda

1. Fili, ego debeo esse finis tuus supremus et ultimatus, si vere desideras esse beatus. Ex hac intentione purificabitur affectus tuus, sæpius ad se ipsum et ad creaturas male incurvatus. Nam si te ipsum in aliquo quæris, statim in te deficis et arescis. Omnia ergo ad me principaliter referas, quia ego sum, qui omnia dedi. Sic singula considera, sicut ex summo bono manantia; et ideo ad me, tamquam ad suam originem, cuncta sunt reducenda.
2. Ex me pusillus et magnus, pauper et dives, tamquam ex fonte vivo, aquam vivam hauriunt; et qui mihi sponte et libere deserviunt, gratiam pro gratia accipient. Qui autem extra me voluerit gloriari, vel in aliquo privato bono delectari, non stabilietur in vero gaudio neque in corde suo dilatabitur, sed multipliciter impedietur et angustiabitur. Nihil ergo tibi de bono ascribere debes, nec alicui homini virtutem attribuas; sed totum da Deo, sine quo nihil habet homo. Ego totum dedi, ego totum rehabere volo, et cum magna districtione gratiarum actiones requiro.
3. Hæc est veritas, qua fugatur gloriæ vanitas. Et si intraverit cælestis gratia et vera caritas, non erit aliqua invidia nec contractio cordis, neque privatus amor occupabit. Vincit enim omnia divina caritas et dilatat omnes animæ vires. Si recte sapis, in me solo gaudebis, in me solo sperabis; quia nemo bonus, nisi solus Deus, qui est super omnia laudandus et in omnibus benedicendus.


CAPUT X.
Quod spreto mundo dulce est servire Deo

1. Nunc iterum loquar, Domine, et non silebo; dicam in auribus Dei mei, Domini mei et Regis mei, qui est in excelso. O quam magna multitudo dulcedinis tuæ, Domine, quam abscondisti timentibus te! Sed quid es amantibus? quid toto corde tibi servientibus? Vere ineffabilis dulcedo contemplationis tuæ, quam largiris amantibus te. In hoc maxime ostendisti mihi dulcedinem caritatis tuæ, quia, cum non essem, fecisti me, et cum errarem longe a te, reduxisti me, ut servirem tibi, et præcepisti, ut diligam te.
2. O fons amoris perpetui, quid dicam de te? Quomodo potero tui oblivisci, qui mei dignatus es recordari, etiam postquam contabui et perii? Fecisti ultra omnem spem misericordiam cum servo tuo, et ultra omne meritum gratiam et amicitiam exhibuisti. Quid retribuam tibi pro gratia ista? Non enim omnibus datum est, ut omnibus abdicatis sæculo renuntient et monasticam vitam assumant. Numquid magnum est, ut tibi serviam, cui omnis creatura servire tenetur? Non magnum mihi videri debet servire tibi; sed hoc potius magnum mihi et mirandum apparet, quod tam pauperem et indignum dignaris in servum recipere et dilectis servis tuis adunare.
3. Ecce, omnia tua sunt, quæ habeo, et unde tibi servio. Verumtamen vice versa, tu magis mihi servis, quam ego tibi. Ecce, cælum et terra, quæ in ministerium hominis creasti, præsto sunt et faciunt cotidie, quæcumque mandasti. Et hoc parum est, quin etiam angelos in ministerium hominis ordinasti. Transcendit autem hæc omnia, quia tu ipse homini servire dignatus es, et te ipsum daturum ei promisisti.
4. Quid dabo tibi pro omnibus istis milibus bonis? Utinam possem tibi servire cunctis diebus vitæ meæ! Utinam vel uno die dignum servitium exhibere sufficerem! Vere tu es dignus omni servitio, omni honore et laude æterna. Vere Dominus meus es, et ego pauper servus tuus, qui totis viribus teneor tibi servire, nec umquam in laudibus tuis debeo fastidire. Sic volo, sic desidero; et quidquid mihi deest, tu digneris supplere.
5. Magnus honor, magna gloria, tibi servire et omnia propter te contemnere. Habebunt enim gratiam magnam, qui sponte se subiecerint tuæ sanctissimæ servituti. Invenient suavissimam Spiritus Sancti consolationem, qui pro amore tuo omnem carnalem abiecerint delectationem. Consequentur magnam mentis libertatem, qui arctam pro nomine tuo ingrediuntur viam et omnem mundanam neglexerint curam.
6. O grata et iucunda Dei servitus, qua homo veraciter efficitur liber et sanctus! O sacer status religiosi famulatus, qui hominem angelis reddit æqualem, Deo placabilem, dæmonibus terribilem et cunctis fidelibus commendabilem! O amplectendum et semper optandum servitium, quo summum promeretur bonum et gaudium acquiritur sine fine mansurum!


CAPUT XI.
Quod desideria cordis examinanda sunt et moderanda

1. Fili, oportet te adhuc multa addiscere, quæ necdum bene didicisti.
Quæ sunt hæc, Domine?
Ut desiderium tuum ponas totaliter secundum beneplacitum meum, et tui ipsius amator non sis, sed meæ voluntatis cupidus æmulator. Desideria te sæpe accendunt et vehementer impellunt; sed considera, an propter honorem meum, an propter tuum commodum magis movearis. Si ego sum in causa, bene contentus eris, quomodocumque ordinavero; si autem de proprio quæsitu aliquid latet, ecce, hoc est, quod te impedit et gravat.
2. Cave ergo, ne nimium innitaris super desiderio præconcepto, me non consulto: ne forte postea pæniteat aut displiceat, quod primo placuit, et quasi pro meliore zelasti. Non enim omnis affectio, quæ videtur bona, statim est sequenda; sed neque omnis contraria affectio ad primum fugienda. Expedit interdum refrenatione uti, etiam in bonis studiis et desideriis, ne per importunitatem mentis distractionem incurras, ne aliis per indisciplinationem scandalum generes, vel etiam per resistentiam aliorum subito turberis et corruas.
3. Interdum vero oportet violentia uti et viriliter appetitui sensitivo contraire, nec advertere, quid velit caro et quid non velit; sed hoc magis satagere, ut subiecta sit etiam nolens spiritui. Et tam diu castigari debet, et cogi servituti subesse, donec parata sit ad omnia, paucisque contentari discat et simplicibus delectari, nec contra aliquod inconveniens mussitare.


CAPUT XII.
De informatione patientiæ et luctamine adversus concupiscentias

1. Domine Deus, ut video, patientia est mihi valde necessaria; multa enim in hac vita accidunt contraria. Nam qualitercumque ordinavero de pace mea, non potest esse sine bello et dolore vita mea.
2. Ita est, fili. Sed volo te non talem quærere pacem, quæ temptationibus careat aut contraria non sentiat, sed tunc etiam æstimare te pacem invenisse, cum fueris variis tribulationibus exercitatus et in multis contrarietatibus probatus. Si dixeris te non posse multa pati, quomodo tunc sustinebis ignem purgatorii? De duobus malis minus est semper eligendum. Ut ergo æterna futura supplicia possis evadere, mala præsentia studeas pro Deo æquanimiter tolerare. An putas, quod homines sæculi huius nihil vel parum patiantur? Nec hoc invenies, etiam si delicatissimos quæsieris.
3. Sed habent, inquis, multas delectationes et proprias sequuntur voluntates; ideoque parum ponderant suas tribulationes.
4. Esto, ita sit, ut habeant, quidquid voluerint; sed quamdiu, putas, durabit? Ecce, quemadmodum fumus deficient abundantes in sæculo, et nulla erit recordatio præteritorum gaudiorum. Sed et cum adhuc vivunt, non sine amaritudine et tædio ac timore in eis quiescunt. Ex eadem namque re, unde sibi delectationem concipiunt, inde doloris pœnam frequenter recipiunt. Iuste illis fit, ut, quia inordinate delectationes quærunt et sequuntur, non sine confusione et amaritudine eas expleant. O quam breves, quam falsæ, quam inordinatæ et turpes omnes sunt! Verumtamen præ ebrietate et cæcitate non intellegunt, sed velut muta animalia propter modicum corruptibilis vitæ delectamentum mortem animæ incurrunt. Tu ergo, fili, post concupiscentias tuas non eas, et a voluntate tua avertere. Delectare in Domino, et dabit tibi petitiones cordis tui.
5. Etenim si veraciter vis delectari et abundantius a me consolari, ecce, in contemptu omnium mundanorum et in abscissione omnium infimarum delectationum erit benedictio tua, et copiosa tibi reddetur consolatio. Et quanto te plus ab omni creaturarum solacio subtraxeris, tanto in me suaviores et potentiores consolationes invenies. Sed primo non sine quadam tristitia et labore certaminis ad has pertinges. Obsistet inolita consuetudo, sed meliori consuetudine devincetur. Remurmurabit caro, sed fervore spiritus frenabitur. Instigabit et exacerbabit te serpens antiquus, sed oratione fugabitur; insuper et labore utili aditus ei magnus obstruetur.


CAPUT XIII.
De obœdientia humilis subditi ad exemplum Iesu Christi

1. Fili, qui se subtrahere nititur ab obœdientia, ipse se subtrahit a gratia; et qui quærit habere privata, amittit communia. Qui non libenter et sponte suo superiori se subdit, signum est, quod caro sua necdum perfecte sibi obœdit, sed sæpe recalcitrat et remurmurat. Disce ergo celeriter superiori tuo te submittere, si carnem propriam optas subiugare. Citius namque exterior vincitur inimicus, si interior homo non fuerit devastatus. Non est molestior et peior animæ hostis, quam tu ipse tibi, non bene concordans spiritui. Oportet omnino verum te assumere tui ipsius contemptum, si vis prævalere adversus carnem et sanguinem. Quia adhuc nimis inordinate te diligis, ideo plene te resignare aliorum voluntati trepidas.
2. Sed quid magnum, si tu, qui pulvis es et nihil, propter Deum te homini subdis, quando ego, omnipotens et altissimus, qui cuncta creavi ex nihilo, me homini propter te humiliter subieci? Factus sum omnium humillimus et infimus, ut tuam superbiam mea humilitate vinceres. Disce obtemperare, pulvis. Disce te humiliare, terra et limus, et sub omnium pedibus incurvare. Disce voluntates tuas frangere et ad omnem subiectionem te dare.
3. Exardesce contra te, nec patiaris tumorem in te vivere, sed ita subiectum et parvulum te exhibe, ut omnes super te ambulare possint et sicut lutum platearum conculcare. Quid habes, homo inanis, conqueri? Quid, sordide peccator, potes contradicere exprobrantibus tibi, qui totiens Deum offendisti et infernum multotiens meruisti? Sed pepercit tibi oculus meus, quia pretiosa fuit anima tua in conspectu meo: ut cognosceres dilectionem meam et gratus semper beneficiis meis existeres, et ut ad veram subiectionem et humilitatem te iugiter dares, patienterque proprium contemptum ferres.


CAPUT XIV.
De occultis Dei iudiciis considerandis, ne extollamur in bonis

1. Intonas super me iudicia tua, Domine, et timore ac tremore concutis omnia ossa mea, et expavescit anima mea valde. Sto attonitus et considero, quia cæli non sunt mundi in conspectu tuo. Si in angelis repperisti pravitatem, nec tamen pepercisti, quid fiet de me? Ceciderunt stellæ de cælo, et ego pulvis, quid præsumo? Quorum opera videbantur laudabilia, ceciderunt ad infima, et qui comedebant panem angelorum, vidi siliquis delectari porcorum.
2. Nulla est ergo sanctitas, si manum tuam, Domine, subtrahas. Nulla prodest sapientia, si gubernare desistas. Nulla iuvat fortitudo, si conservare desinas. Nulla secura castitas, si eam non protegas. Nulla propria prodest custodia, si non assit tua sacra vigilantia. Nam relicti mergimur et perimus, visitati vero erigimur et vivimus. Instabiles quippe sumus, sed per te confirmamur; tepescimus, sed a te accendimur.
3. O quam humiliter et abiecte mihi de me ipso sentiendum est! quam nihili pendendum, si quid boni videar habere! O quam profunde submittere me debeo sub abyssalibus iudiciis tuis, Domine, ubi nihil aliud me esse invenio, quam nihil et nihil! O pondus immensum! o pelagus intransnatabile, ubi nihil de me reperio, quam in toto nihil! Ubi est ergo latebra gloriæ? ubi confidentia de virtute concepta? Absorpta est omnis gloriatio vana in profunditate iudiciorum tuorum super me.
4. Quid est omnis caro in conspectu tuo? Numquid gloriabitur lutum contra formantem se? Quomodo potest erigi vaniloquio, cuius cor in veritate subiectum est Deo? Non eum totus mundus eriget, quem veritas sibi subiecit; nec omnium laudantium ore movebitur, qui totam spem suam in Deo firmavit. Nam et ipsi, qui loquuntur, ecce, omnes nihil; deficient enim cum sonitu verborum. Veritas autem Domini manet in æternum.


CAPUT XV.
Qualiter standum sit ac dicendum in omni re desiderabili

1. Fili, sic dicas in omni re: Domine, si tibi placitum fuerit, fiat hoc ita. Domine, si fuerit honor tuus, fiat hoc in nomine tuo. Domine, si mihi videris expedire, et utile esse probaveris, tunc dona mihi hoc uti ad honorem tuum. Sed si mihi nocivum fore cognoveris, nec animæ meæ saluti prodesse, aufer a me tale desiderium. Non enim omne desiderium est a Spiritu Sancto, etiam si homini videatur rectum et bonum. Difficile est pro vero iudicare, utrum spiritus bonus an alienus te impellat ad desiderandum hoc vel illud, an etiam ex proprio movearis spiritu. Multi in fine sunt decepti, qui primo bono spiritu videbantur inducti.
2. Igitur semper cum timore Dei et cordis humilitate desiderandum est et petendum, quidquid desiderabile menti occurrit, maximeque cum propria resignatione, mihi totum committendum est, atque dicendum: Domine, tu scis, qualiter melius est; fiat hoc vel illud, sicut volueris. Da, quod vis et quantum vis et quando vis. Fac mecum, sicut scis et sicut tibi magis placuerit et maior honor tuus fuerit. Pone me, ubi vis, et libere age mecum in omnibus. In manu tua sum, gyra et reversa me per circuitum. En, servus tuus ego, paratus ad omnia, quoniam non desidero mihi vivere, sed tibi: utinam digne et perfecte.
Oratio pro beneplacito Dei perficiendo
3. Concede mihi, benignissime Iesu, gratiam tuam, ut mecum sit et mecum laboret, mecumque usque in finem perseveret. Da mihi hoc semper desiderare et velle, quod tibi magis acceptum est et carius placet. Tua voluntas mea sit, et mea voluntas tuam semper sequatur et optime ei concordet. Sit mihi unum velle et nolle tecum, nec aliud posse velle aut nolle, nisi quod tu vis et nolis.
4. Da mihi omnibus mori, quæ in mundo sunt, et propter te amare contemni et nesciri in hoc sæculo. Da mihi super omnia desiderata in te requiescere et cor meum in te pacificare. Tu vera pax cordis, tu sola requies: extra te dura sunt omnia et inquieta. In hac pace in id ipsum, hoc est, in te uno summo æterno bono, dormiam et requiescam. Amen.


CAPUT XVI.
Quod verum solacium in solo Deo est quærendum

1. Quidquid desiderare possum vel cogitare ad solacium meum, non hic expecto, sed in posterum. Quodsi omnia solacia mundi solus haberem, et omnibus deliciis frui possem, certum est, quod diu durare non possent. Unde non poteris, anima mea, plene consolari nec perfecte recreari, nisi in Deo, consolatore pauperum ac susceptore humilium. Expecta modicum, anima mea, expecta divinum promissum, et habebis abundantiam omnium bonorum in cælo. Si nimis inordinate ista appetis præsentia, perdes æterna et cælestia. Sint temporalia in usu, æterna in desiderio. Non potes aliquo bono temporali satiari, quia ad hæc fruenda non es creata.
2. Etiam si omnia creata bona haberes, non posses esse felix et beata; sed in Deo, qui cuncta creavit, tota beatitudo tua et felicitas consistit, non qualis videtur et laudatur a stultis mundi amatoribus, sed qualem expectant boni Christi fideles, et prægustant interdum spirituales ac mundi cordes, quorum conversatio est in cælis. Vanum est et breve omne humanum solacium. Beatum et verum solacium, quod intus a veritate percipitur. Devotus homo ubique secum fert consolatorem suum Iesum, et dicit ad eum: Adesto mihi, Domine Iesu, in omni loco et tempore. Hæc mihi sit consolatio, libenter velle carere omni humano solacio. Et si tua defuerit consolatio, sit mihi tua voluntas et iusta probatio pro summo solacio. Non enim in perpetuum irasceris, neque in æternum comminaberis.


CAPUT XVII.
Quod omnis sollicitudo in Deo statuenda sit

1. Fili, sine me tecum agere, quod volo; ego scio, quid expedit tibi. Tu cogitas, sicut homo, tu sentis in multis, sicut humanus suadet affectus.
2. Domine, verum est, quod dicis. Maior est sollicitudo tua pro me, quam omnis cura, quam ego gerere possum pro me. Nimis enim casualiter stat, qui non proicit omnem sollicitudinem suam in te. Domine, dummodo voluntas mea recta et firma ad te permaneat, fac de me, quidquid tibi placuerit. Non enim potest esse nisi bonum, quidquid de me feceris. Si me vis esse in tenebris, sis benedictus, et si me vis esse in luce, sis iterum benedictus. Si me dignaris consolari, sis benedictus, et si me vis tribulari, sis æque semper benedictus.
3. Fili, sic oportet te stare, si mecum desideras ambulare. Ita promptus esse debes ad patiendum, sicut ad gaudendum. Ita libenter debes esse inops et pauper, sicut plenus et dives.
4. Domine, libenter patiar pro te, quidquid volueris venire super me. Indifferenter volo de manu tua bonum et malum, dulce et amarum, lætum et triste suscipere, et pro omnibus mihi contingentibus gratias agere. Custodi me ab omni peccato, et non timebo mortem nec infernum. Dummodo in æternum me non proicias, nec deleas de libro vitæ, non mihi nocebit, quidquid venerit tribulationis super me.


CAPUT XVIII.
Quod temporales miseriæ exemplo Christi æquanimiter sunt ferendæ

1. Fili, ego descendi de cælo pro tua salute; suscepi tuas miserias, non necessitate, sed caritate trahente, ut patientiam disceres et temporales miserias non indignanter ferres. Nam ab hora ortus mei usque ad exitum in cruce non defuit mihi tolerantia doloris. Defectum rerum temporalium magnum habui, multas querimonias de me frequenter audivi, confusiones et obprobria benigne sustinui, pro beneficiis ingratitudinem recepi, pro miraculis blasphemias, pro doctrina reprehensiones.
2. Domine, quia tu patiens fuisti in vita tua, in hoc maxime implendo præceptum Patris tui: dignum est, ut ego, misellus peccator, secundum voluntatem tuam patienter me sustineam; et donec ipse volueris, onus corruptibilis vitæ pro salute mea portem. Nam et si onerosa sentitur præsens vita, facta est tamen iam per gratiam tuam valde meritoria, atque exemplo tuo et sanctorum tuorum vestigiis infirmis tolerabilior et clarior. Sed et multo magis consolatoria, quam olim in lege veteri fuerat, cum porta cæli clausa persisteret, et obscurior etiam via ad cælum videbatur, quando tam pauci regnum cælorum quærere curabant. Sed neque qui tunc iusti erant et salvandi, ante passionem tuam et sacræ mortis debitum cæleste regnum poterant introire.
3. O quantas tibi gratias teneor referre, quod viam rectam et bonam dignatus es mihi et cunctis fidelibus ad æternum regnum tuum ostendere? Nam vita tua via nostra: et per sanctam patientiam ambulamus ad te, qui es corona nostra. Nisi tu nos præcessisses et docuisses, quis sequi curaret? Heu, quanti longe retroque manerent, nisi tua præclara exempla inspicerent! Ecce, adhuc tepescimus, auditis tot signis tuis et doctrinis: quid fieret, si tantum lumen ad sequendum te non haberemus?


CAPUT XIX.
De tolerantia iniuriarum, et quis verus patiens probetur

1. Quid est, quod loqueris, fili? Cessa conqueri, considerata mea et aliorum sanctorum passione. Nondum usque ad sanguinem restitisti. Parum est, quod tu pateris, in comparatione eorum, qui tam multa passi sunt, tam fortiter temptati, tam graviter tribulati, tam multipliciter probati et exercitati. Oportet te igitur aliorum graviora ad mentem reducere, ut levius feras tua minima. Et si tibi minima non videntur, vide, ne et hoc tua faciat impatientia. Sive tamen parva sive magna sint, stude cuncta patienter sufferre.
2. Quanto melius ad patiendum te disponis, tanto sapientius agis et amplius promereris; feres quoque levius animo et usu ad hoc non segniter paratis. Nec dicas: Non valeo hæc ab homine tali pati, nec huiuscemodi mihi patienda sunt; grave enim intulit damnum, et improperat mihi, quæ numquam cogitaveram; sed ab alio libenter patiar, et sicut patienda videro. Insipiens est talis cogitatio, quæ virtutem patientiæ non considerat, nec a quo coronanda erit, sed magis personas et offensas sibi illatas perpendit.
3. Non est verus patiens, qui pati non vult, nisi quantum sibi visum fuerit, et a quo sibi placuerit. Verus autem patiens non attendit, a quo homine, utrum a prælato suo an ab aliquo æquali aut inferiori, utrum a bono et sancto viro, vel a perverso et indigno exerceatur. Sed indifferenter ab omni creatura, quantumcumque et quotienscumque ei aliquid adversi acciderit, totum hoc de manu Dei gratanter accipit et ingens lucrum reputat, quia nil apud Deum, quamlibet parvum, pro Deo tamen passum, poterit sine merito transire.
4. Esto itaque expeditus ad pugnam, si vis habere victoriam. Sine certamine non potes venire ad patientiæ coronam. Si pati non vis, recusas coronari. Si autem coronari desideras, certa viriliter, sustine patienter. Sine labore non tenditur ad requiem, nec sine pugna pervenitur ad victoriam.
5. Fiat, Domine, mihi possibile per gratiam, quod mihi impossibile videtur per naturam. Tu scis, quod modicum possum pati, et quod cito deicior, levi exurgente adversitate. Efficiatur mihi quælibet exercitatio tribulationis pro nomine tuo amabilis et optabilis: nam pati et vexari pro te valde salubre est animæ meæ.


CAPUT XX.
De confessione propriæ infirmitatis et huius vitæ miseriis

1. Confitebor adversum me iniustitiam meam, confitebor tibi, Domine, infirmitatem meam. Sæpe parva res est, quæ me deicit et contristat. Propono me fortiter acturum; sed cum modica temptatio venerit, magna mihi angustia fit. Valde vilis quandoque res est, unde gravis temptatio provenit. Et dum puto me aliquantulum tutum, cum non sentio, invenio me nonnumquam pæne devictum ex levi flatu.
2. Vide ergo, Domine, humilitatem meam et fragilitatem tibi undique notam. Miserere et eripe me de luto, ut non infigar, ne permaneam deiectus usquequaque. Hoc est, quod me frequenter reverberat, et coram te confundit, quod tam labilis sum, et infirmus ad resistendum passionibus. Et si non omnino ad consensionem, tamen mihi etiam molesta et gravis est earum insectatio, et tædet valde sic cotidie vivere in lite. Ex hinc nota mihi fit infirmitas mea, quia multo facilius irruunt abominandæ semper phantasiæ, quam discedunt.
3. Utinam, fortissime Deus Isræl, zelator animarum fidelium, respicias servi tui laborem et dolorem, assistasque illi in omnibus, ad quæcumque perrexerit. Robora me cælesti fortitudine, ne vetus homo, misera caro spiritui necdum plene subacta, prævaleat dominari, adversus quam certare oportebit, quamdiu spiratur in hac vita miserrima. Heu, qualis est hæc vita, ubi non desunt tribulationes et miseriæ, ubi plena laqueis et hostibus sunt omnia! Nam una tribulatione seu temptatione recedente, alia accedit; sed et priore adhuc durante conflictu, alii plures superveniunt, et insperate.
4. Et quomodo potest amari vita, tantas habens amaritudines, tot subiecta calamitatibus et miseriis? Quomodo etiam dicitur vita, tot generans mortes et pestes? Et tamen amatur, et delectari in ea a multis quæritur. Reprehenditur frequenter mundus, quod fallax sit et vanus, nec tamen facile relinquitur, quia concupiscentiæ carnis nimis dominantur. Sed alia trahunt ad amandum, alia ad contemnendum. Trahunt ad amorem mundi desiderium carnis, desiderium oculorum et superbia vitæ; sed pœnæ ac miseriæ iuste sequentes ea odium mundi pariunt et tædium.
5. Sed vincit, pro dolor, delectatio prava mentem mundo deditam, et esse sub sentibus delicias reputat, quia Dei suavitatem et internam virtutis amœnitatem nec vidit, nec gustavit. Qui autem mundum perfecte contemnunt, et Deo vivere sub sancta disciplina student, isti divinam dulcedinem, veris abrenuntiatoribus promissam, non ignorant; et quam graviter mundus errat et varie fallitur, clarius vident.


CAPUT XXI.
Quod in Deo super omnia bona et dona requiescendum est

1. Super omnia et in omnibus requiesces, anima mea, in Domino semper, quia ipse sanctorum æterna requies. Da mihi, dulcissime et amantissime Iesu, in te super omnem creaturam requiescere: super omnem salutem et pulchritudinem, super omnem gloriam et honorem, super omnem potentiam et dignitatem, super omnem scientiam et subtilitatem, super omnes divitias et artes, super omnem lætitiam et exultationem, super omnem famam et laudem, super omnem suavitatem et consolationem, super omnem spem et promissionem, super omne meritum et desiderium, super omnia dona et munera, quæ potes dare et infundere, super omne gaudium et iubilationem, quam potest mens capere et sentire, denique super angelos et archangelos, et super omnem exercitum cæli, super omnia visibilia et invisibilia, et super omne, quod tu, Deus meus, non es.
2. Quia tu, Domine Deus meus, super omnia optimus es, tu solus altissimus, tu solus potentissimus, tu solus sufficientissimus et plenissimus, tu solus suavissimus et solaciosissimus, tu solus pulcherrimus et amantissimus, tu solus nobilissimus et gloriosissimus super omnia, in quo cuncta bona simul et perfecte sunt, et semper fuerunt, et erunt: atque ideo minus est et insufficiens, quidquid præter te ipsum mihi donas, aut de te ipso revelas, vel promittis, te non viso, nec plene adepto: quoniam quidem non potest cor meum veraciter requiescere, nec totaliter contentari, nisi in te requiescat, et omnia dona omnemque creaturam transcendat.
3. O mi dilectissime sponse, Iesu Christe, amator purissime, dominator universæ creaturæ, quis mihi det pennas veræ libertatis, ad volandum et pausandum in te? O quando ad plenum dabitur mihi vacare et videre, quam suavis es, Domine Deus meus? Quando ad plenum me recolligam in te, ut præ amore tuo non sentiam me, sed te solum, supra omnem sensum et modum, in modo non omnibus noto? Nunc autem frequenter gemo, et infelicitatem meam cum dolore porto. Quia multa mala in hac valle miseriarum occurrunt, quæ me sæpius conturbant, contristant et obnubilant; sæpius impediunt et distrahunt, alliciunt et implicant, ne liberum habeam accessum ad te, et ne iucundis fruar amplexibus, præsto semper beatis spiritibus. Moveat te suspirium meum et desolatio multiplex in terra.
4. O Iesu, splendor æternæ gloriæ, solamen peregrinantis animæ, apud te est os meum sine voce, et silentium meum loquitur tibi. Usquequo tardat venire Dominus meus? Veniat ad me pauperculum suum, et lætum faciat. Mittat manum suam, et miserum eripiat de omni angustia. Veni, veni: quia sine te nulla erit læta dies aut hora, quia tu lætitia mea, et sine te vacua est mensa mea. Miser sum et quodammodo incarceratus et compedibus gravatus, donec luce præsentiæ tuæ me reficias ac libertati dones, vultumque amicabilem demonstres.
5. Quærant alii pro te aliud, quodcumque libuerit: mihi aliud interim nil placet nec placebit, nisi tu Deus meus, spes mea, salus æterna. Non reticebo, nec deprecari cessabo, donec gratia tua revertatur, mihique tu intus loquaris.
6. Ecce, assum, ecce, ego ad te, quia invocasti me. Lacrimæ tuæ et desiderium animæ tuæ, humiliatio tua et contritio cordis inclinaverunt me et adduxerunt ad te.
7. Et dixi: Domine, vocavi te et desideravi frui te, paratus omnia respuere propter te. Tu enim prior excitasti me, ut quærerem te. Sis ergo benedictus, Domine, qui fecisti hanc bonitatem cum servo tuo, secundum multitudinem misericordiæ tuæ. Quid habet ultra dicere servus tuus coram te, nisi ut humiliet se valde ante te, memor semper propriæ iniquitatis et vilitatis? Non enim est similis tui in cunctis mirabilibus cæli et terræ. Sunt opera tua bona valde, iudicia vera, et providentia tua reguntur universa. Laus ergo tibi et gloria, o Patris sapientia: te laudet et benedicat os meum, anima mea et cuncta creata simul.


CAPUT XXII.
De recordatione multiplicium beneficiorum Dei

1. Aperi, Domine, cor meum in lege tua, et in præceptis tuis doce me ambulare. Da mihi intellegere voluntatem tuam, et cum magna reverentia ac diligenti consideratione beneficia tua tam in generali quam in speciali memorari, ut digne tibi ex hinc valeam gratias referre. Verum scio et confiteor, nec pro minimo puncto me posse debitas gratiarum laudes persolvere. Minor ego sum omnibus bonis mihi præstitis; et cum tuam nobilitatem attendo, deficit præ magnitudine spiritus meus.
2. Omnia, quæ in anima habemus et corpore, et quæcumque exterius vel interius, naturaliter et supernaturaliter possidemus, tua sunt beneficia et te beneficum, pium ac bonum commendant, a quo bona cuncta accepimus. Et si alius plura, alius pauciora accepit, omnia tamen tua sunt, et sine te nec minimum potest haberi. Ille, qui maiora accepit, non potest merito suo gloriari, neque super alios extolli, nec minori insultare; quia ille maior et melior, qui sibi minus ascribit et in regratiando humilior est atque devotior. Et qui omnibus viliorem se existimat et indigniorem se iudicat, aptior est ad percipienda maiora.
3. Qui autem pauciora accepit, contristari non debet, nec indignanter ferre, neque ditiori invidere, sed te potius attendere et tuam bonitatem maxime laudare, quod tam affluenter, tam gratis et libenter, sine personarum acceptione tua munera largiris. Omnia ex te, et ideo in omnibus es laudandus. Tu scis, quid unicuique donari expediat; et cur iste minus, et ille amplius habeat, non nostrum, sed tuum est hoc discernere, apud quem singulorum definita sunt merita.
4. Unde, Domine Deus, pro magno etiam reputo beneficio, non multa habere, unde exterius et secundum homines laus et gloria apparet: ita ut considerata quis paupertate et vilitate personæ suæ, non modo gravitatem aut tristitiam vel deiectionem inde concipiat, sed potius consolationem et hilaritatem magnam; quia tu, Deus, pauperes et humiles atque huic mundo despectos tibi elegisti in familiares et domesticos. Testes sunt ipsi apostoli tui, quos principes super omnem terram constituisti. Fuerunt tamen sine querela conversati in mundo, tam humiles et simplices, sine omni malitia et dolo, ut etiam pati contumelias gauderent pro nomine tuo; et quæ mundus abhorret, ipsi amplecterentur affectu magno.
5. Nihil ergo amatorem tuum et cognitorem beneficiorum tuorum ita lætificare debet, sicut voluntas tua in eo, et beneplacitum æternæ dispositionis tuæ: de qua tantum contentari debet et consolari, ut ita libenter velit esse minimus, sicut aliquis optaret esse maximus, et ita pacificus et contentus in novissimo, sicut in loco primo, atque ita libenter despicabilis et abiectus, nullius quoque nominis et famæ, sicut ceteris honorabilior et maior in mundo. Nam voluntas tua et amor honoris tui omnia excedere debet, et plus eum consolari magisque placere, quam omnia beneficia sibi data vel danda.


CAPUT XXIII.
De quatuor magnam importantibus pacem

1. Fili, nunc docebo te viam pacis et veræ libertatis.
2. Fac, Domine, quod dicis, quia hoc mihi gratum est audire.
3. Stude, fili, alterius potius facere voluntatem, quam tuam. Elige, semper minus, quam plus habere. Quære semper inferiorem locum et omnibus subesse. Opta semper et ora, ut voluntas Dei integre in te fiat. Ecce, talis homo ingreditur fines pacis et quietis.
4. Domine, sermo tuus iste brevis multum in se continet perfectionis. Parvus est dictu, sed plenus sensu et uber in fructu. Nam si posset a me fideliter custodiri, non deberet tam facilis in me turbatio oriri. Nam quotiens me impacatum sentio et gravatum, ab hac doctrina me recessisse invenio. Sed tu, qui omnia potes et animæ profectum semper diligis, adauge maiorem gratiam, ut possim tuum complere sermonem et meam perficere salutem.
Oratio contra cogitationes malas
5. Domine Deus meus, ne elongeris a me; Deus meus, in auxilium meum respice: quoniam insurrexerunt in me variæ cogitationes et timores magni, affligentes animam meam. Quomodo pertransibo illæsus? quomodo perfringam eas?
6. Ego, inquit, ante te ibo, et gloriosos terræ humiliabo. Aperiam ianuas carceris, et arcana secretorum revelabo tibi.
7. Fac, Domine, ut loqueris, et fugiant a facie tua omnes iniquæ cogitationes. Hæc spes et unica consolatio mea, ad te in omni tribulatione confugere, tibi confidere, ex intimo invocare et patienter consolationem tuam expectare.
Oratio pro illuminatione mentis
8. Clarifica me, Iesu bone, claritate interni luminis, et educ de habitaculo cordis mei tenebras universas. Cohibe evagationes multas, et vim facientes elide temptationes. Pugna fortiter pro me, et expugna malas bestias, concupiscentias dico illecebrosas: ut fiat pax in virtute tua, et abundantia laudis tuæ resonet in aula sancta, hoc est in conscientia pura. Impera ventis et tempestatibus; dic mari: Quiesce; et aquiloni: Ne flaveris; et erit tranquillitas magna.
9. Emitte lucem tuam et veritatem, ut luceant super terram; quia terra sum inanis et vacua, donec illumines me. Effunde gratiam desuper; perfunde cor meum rore cælesti; ministra devotionis aquas ad irrigandam faciem terræ, ad producendum fructum bonum et optimum. Eleva mentem pressam mole peccatorum, et ad cælestia totum desiderium meum suspende, ut gustata suavitate supernæ felicitatis pigeat de terrenis cogitare.
10. Rape me et eripe ab omni creaturarum indurabili consolatione, quia nulla res creata appetitum meum valet plenarie quietare et consolari. Iunge me tibi inseparabili dilectionis vinculo, quoniam tu solus sufficis amanti, et absque te frivola sunt universa.


CAPUT XXIV.
De evitatione curiosæ inquisitionis super alterius vita

1. Fili, noli esse curiosus nec vacuas gerere sollicitudines. Quid hoc vel illud ad te? Tu me sequere. Quid enim ad te, utrum ille sit talis vel talis, aut iste sic et sic agit vel loquitur? Tu non indiges respondere pro aliis, sed pro te ipso rationem reddes. Quid ergo te implicas? Ecce, ego omnes cognosco, et cuncta, quæ sub sole fiunt, video; et scio, qualiter cum unoquoque sit, quid cogitet, quid velit, et ad quem finem tendat eius intentio. Mihi igitur omnia committenda sunt: tu vero serva te in bona pace, et dimitte agitantem agitare, quantum voluerit. Veniet super eum, quidquid fecerit vel dixerit, quia me fallere non potest.
2. Non sit tibi curæ de magni nominis umbra, non de multorum familiaritate, nec de privata hominum dilectione. Ista enim generant distractiones et magnas in corde obscuritates. Libenter loquerer tibi verbum meum et abscondita revelarem, si adventum meum diligenter observares et ostium cordis mihi aperires. Esto providus, et vigila in orationibus, et humilia te in omnibus.


CAPUT XXV.
In quibus firma pax cordis et verus profectus consistit

1. Fili, ego locutus sum: Pacem relinquo vobis, pacem meam do vobis: non quomodo mundus dat, ego do vobis. Pacem omnes desiderant: sed quæ ad veram pacem pertinent, non omnes curant. Pax mea cum humilibus et mansuetis corde. Pax tua erit in multa patientia. Si me audieris et vocem meam secutus fueris, poteris multa pace frui.
2. Quid igitur faciam?
3. In omni re attende tibi, quid facias et quid dicas, et omnem intentionem tuam ad hoc dirige, ut mihi soli placeas et extra me nihil cupias vel quæras. Sed et de aliorum dictis vel factis nil temere iudices, nec cum rebus tibi non commissis te implices; et poterit fieri, ut parum vel raro turberis. Numquam autem sentire aliquam turbationem, nec aliquam pati cordis vel corporis molestiam, non est præsentis temporis, sed status æternæ quietis. Non ergo æstimes, te veram pacem invenisse, si nullam senseris gravitatem, nec tunc totum esse bonum, si neminem pateris adversarium; nec hoc esse perfectum, si cuncta fiant secundum tuum affectum. Neque tunc aliquid magni te reputes aut specialiter dilectum existimes, si in magna fueris devotione atque dulcedine: quia in istis non cognoscitur verus amator virtutis; nec in istis consistit profectus et perfectio hominis.
4. In quo ergo, Domine?
5. In offerendo te ex toto corde tuo voluntati divinæ, non quærendo, quæ tua sunt, nec in parvo nec in magno, nec in tempore nec in æternitate; ita ut una æquali facie in gratiarum actione permaneas inter prospera et contraria, omnia æqua lance pensando. Si fueris tam fortis et longanimus in spe, ut subtracta interiori consolatione etiam ad ampliora sustinenda cor tuum præparaveris, nec te iustificaveris, quasi hæc tantaque pati non deberes, sed me in omnibus dispositionibus iustificaveris et sanctum laudaveris; tunc in vera et recta via pacis ambulas, et spes indubitata erit, quod rursus in iubilo faciem meam sis visurus. Quodsi ad plenum tui ipsius contemptum perveneris, scito, quod tunc abundantia pacis perfrueris, secundum possibilitatem tui incolatus.


CAPUT XXVI.
De eminentia liberæ mentis, quam supplex oratio magis meretur, quam lectio

1. Domine, hoc opus est perfecti viri, numquam ab intentione cælestium animum relaxare, et inter multas curas quasi sine cura transire, non more torpentis, sed prærogativa quadam liberæ mentis, nulli creaturæ inordinata affectione adhærendo.
2. Obsecro te, piissime Deus meus, præserva me a curis huius vitæ, ne nimis implicer; a multis necessitatibus corporis, ne voluptate capiar; ab universis animæ obstaculis, ne molestiis fractus deiciar. Non dico: ab his rebus, quas toto affectu ambit vanitas mundana, sed ab his miseriis, quæ animam servi tui communi maledicto mortalitatis pœnaliter gravant et retardant, ne in libertatem spiritus, quotiens libuerit, valeat introire.
3. O Deus meus, dulcedo ineffabilis, verte mihi in amaritudinem omnem consolationem carnalem, ab æternorum amore me abstrahentem et ad se intuitu cuiusdam boni delectabilis præsentis male allicientem. Non me vincat, Deus meus, non vincat caro et sanguis, non me decipiat mundus et brevis gloria eius, non me supplantet diabolus et astutia illius. Da mihi fortitudinem resistendi, patientiam tolerandi, constantiam perseverandi. Da pro omnibus mundi consolationibus suavissimam spiritus tui unctionem, et pro carnali amore tui nominis infunde amorem.
4. Ecce, cibus, potus, vestis ac cetera utensilia ad corporis sustentaculum pertinentia ferventi spiritui sunt onerosa. Tribue, talibus fomentis temperate uti, non desiderio nimio implicari. Abicere omnia non licet, quia natura sustentanda est; requirere autem superflua et quæ magis delectant, lex sancta prohibet: nam alias caro adversus spiritum insolesceret. Inter hæc, quæso, manus tua me regat et doceat, ne quid nimium fiat.


CAPUT XXVII.
Quod privatus amor a summo bono maxime retardat

1. Fili, oportet te dare totum pro toto, et nihil tui ipsius esse. Scito, quod amor tui ipsius magis nocet tibi, quam aliqua res mundi. Secundum amorem et affectum, quem geris, quælibet res plus vel minus adhæret. Si fuerit amor tuus purus, simplex et bene ordinatus, eris sine captivitate rerum. Noli concupiscere, quod non licet habere; noli habere, quod te potest impedire et libertate interiori privare. Mirum, quod non ex toto fundo cordis te ipsum mihi committis cum omnibus, quæ desiderare potes vel habere.
2. Ouare vano mærore consumeris? Cur superfluis curis fatigaris? Sta ad beneplacitum meum, et nullum patieris detrimentum. Si quæris hoc vel illud, et volueris esse ibi vel ibi propter tuum commodum et proprium beneplacitum magis habendum: numquam eris in quietudine, nec liber a sollicitudine, quia in omni re reperietur aliquis defectus, et in omni loco erit, qui adversetur.
3. Iuvat igitur non quælibet res adepta vel multiplicata exterius, sed potius contempta et decisa ex corde radicitus. Quod non tantum de censu æris et divitiarum intellegas, sed de honoris etiam ambitu ac vanæ laudationis desiderio, quæ omnia transeunt cum mundo. Munit parum locus, si deest spiritus fervoris; nec diu stabit pax illa quæsita forinsecus, si vacat a vero fundamento status cordis, hoc est, nisi steteris in me; permutare te potes, sed non meliorare. Nam occasione orta et accepta, invenies, quod fugisti, et amplius.
Oratio pro purgatione cordis et cælesti sapientia
4. Confirma me, Deus, per gratiam Sancti Spiritus. Da virtutem corroborari in interiori homine, et cor meum ab omni inutili sollicitudine et angore evacuare, nec variis desideriis trahi cuiuscumque rei, vilis aut pretiosæ; sed omnia inspicere sicut transeuntia, et me pariter cum illis transiturum: quia nihil permanens sub sole, ubi omnia vanitas et afflictio spiritus. O quam sapiens, qui ita considerat!
5. Da mihi, Domine, cælestem sapientiam, ut discam te super omnia quærere et invenire, super omnia sapere et diligere, et cetera secundum ordinem sapientiæ tuæ, prout sunt, intellegere. Da prudenter declinare blandientem et patienter ferre adversantem, quia hæc magna sapientia, non moveri omni vento verborum, nec aurem male blandienti præbere Sirenæ; sic enim incepta pergitur via secure.


CAPUT XXVIII.
Contra linguas obtrectatorum

1. Fili, non ægre feras, si quidam de te male senserint et dixerint, quod non libenter audias. Tu deteriora de te ipso sentire debes, et neminem infirmiorem te credere. Si ambulas ab intra, non multum ponderabis volantia verba. Est non parva prudentia, silere in tempore malo et introrsus ad me converti, nec humano iudicio disturbari.
2. Non sit pax tua in ore hominum; sive enim bene sive male interpretati fuerint, non es ideo alter homo. Ubi est vera pax et vera gloria? Nonne in me? Et qui non appetit hominibus placere, nec timet displicere, multa perfruetur pace. Ex inordinato amore et vano timore oritur omnis inquietudo cordis et distractio sensuum.


CAPUT XXIX.
Qualiter instante tribulatione Deus invocandus est et benedicendus

1. Sit nomen tuum, Domine, benedictum in sæcula, qui voluisti hanc temptationem et tribulationem venire super me. Non possum eam effugere, sed necesse habeo ad te confugere, ut me adiuves et in bonum mihi convertas. Domine, modo sum in tribulatione, et non est cordi meo bene, sed multum vexor a præsenti passione. Et nunc, pater dilecte, quid dicam? Deprehensus sum inter angustias. Salvifica me ex hora hac. Sed propterea veni in hanc horam, ut tu clarificeris, cum fuero valide humiliatus et per te liberatus. Complaceat tibi, Domine, ut eruas me: nam ego pauper quid agere possum, et quo ibo sine te? Da patientiam, Domine, etiam hac vice. Adiuva me, Deus meus, et non timebo, quantumcumque gravatus fuero.
2. Et nunc inter hæc quid dicam? Domine, fiat voluntas tua. Ego bene merui tribulari et gravari. Oportet utique, ut sustineam, et utinam patienter, donec transeat tempestas et melius fiat! Potens est autem omnipotens manus tua, etiam hanc temptationem a me auferre et eius impetum mitigare, ne penitus succumbam, quemadmodum et prius sæpius egisti mecum, Deus meus, misericordia mea. Et quanto mihi difficilius, tanto tibi facilior est hæc mutatio dexteræ excelsi.


CAPUT XXX.
De divino petendo auxilio et confidentia recuperandæ gratiæ

1. Fili, ego Dominus, confortans in die tribulationis. Venias ad me, cum tibi non fuerit bene. Hoc est, quod maxime consolationem impedit cælestem, quia tardius te convertis ad orationem. Nam antequam me intente roges, multa interim solacia quæris, et recreas te in externis. Ideoque fit, ut parum omnia prosint, donec advertas, quia ego sum, qui eruo sperantes in me, nec est extra me valens auxilium, neque utile consilium, sed neque durabile remedium. Sed iam resumpto spiritu post tempestatem reconvalesce in luce miserationum mearum; quia prope sum, dicit Dominus, ut restaurem universa, non solum integre, sed et abundanter ac cumulate.
2. Numquid mihi quicquam est difficile? aut ero similis dicenti et non facienti? Ubi est fides tua? Sta firmiter et perseveranter. Esto longanimis et vir fortis; veniet tibi consolatio in tempore suo. Expecta me, expecta; veniam et curabo te. Temptatio est, quæ te vexat, et formido vana, quæ te exterret. Quid importat sollicitudo de futuris contingentibus, nisi ut tristitiam super tristitiam habeas? Sufficiat diei malitia sua. Vanum est et inutile, de futuris conturbari vel gratulari, quæ forte numquam evenient.
3. Sed humanum est, huiusmodi imaginationibus illudi, et parvi adhuc animi signum, tam leviter trahi ad suggestionem inimici. Ipse enim non curat, utrum veris an falsis illudat et decipiat, utrum præsentium amore, aut futurorum formidine prosternat. Non ergo turbetur cor tuum, neque formidet. Crede in me, et in misericordia mea habeto fiduciam. Quando tu putas te elongatum a me, sæpe sum propinquior. Quando tu æstimas pæne totum perditum, tunc sæpe maius merendi instat lucrum. Non est totum perditum, quando res accidit in contrarium. Non debes iudicare secundum præsens sentire, nec sic gravitati alicui, undecumque venienti, inhærere et accipere, tamquam omnis spes sit ablata emergendi.
4. Noli putare te relictum ex toto, quamvis ad tempus tibi miserim aliquam tribulationem vel etiam optatam subtraxerim consolationem: sic enim transitur ad regnum cælorum. Et hoc sine dubio magis expedit tibi et ceteris servis meis, ut exercitemini adversis, quam si cuncta ad libitum haberetis. Ego novi cogitationes absconditas, quia multum expedit pro salute tua, ut interdum sine sapore relinquaris, ne forte elevens in bono successu, et tibi ipsi placere velis in eo, quod non es. Quod dedi, auferre possum, et restituere, cum mihi placuerit.
5. Cum dedero, meum est, cum retraxero, tuum non tuli, quia meum est omne datum bonum et omne donum perfectum. Si tibi admisero gravitatem aut quamlibet contrarietatem, ne indigneris, neque concidat cor tuum: ego cito sublevare possum et omne onus in gaudium transmutare. Verumtamen iustus sum et recommendabilis multum, cum sic facio tecum.
6. Si recte sapis et in veritate aspicis, numquam debes propter adversa tam deiecte contristari, sed magis gaudere et gratias agere, immo hoc unicum reputare gaudium, quod affligens te doloribus non parco tibi. Sicut dilexit me Pater, et ego vos diligo, dixi dilectis discipulis meis, quos utique non misi ad gaudia temporalia, sed ad magna certamina, non ad honores, sed ad despectiones, non ad otium, sed ad labores, non ad requiem, sed ad afferendum fructum multum in patientia. Horum memento, fili mi, verborum.


CAPUT XXXI.
De neglectu omnis creaturæ, ut Creator possit inveniri

1. Domine, bene indigeo adhuc maiori gratia, si debeam illuc pervenire, ubi me nemo poterit nec ulla creatura impedire. Nam quamdiu res aliqua me retinet, non possum libere ad te volare. Cupiebat libere volare, qui dicebat: Quis dabit mihi pennas sicut columbæ, et volabo et requiescam? Quid simplici oculo quietius? Et quid liberius nil desiderante in terris? Oportet igitur omnem supertransire creaturam, et se ipsum perfecte deserere, ac in excessu mentis stare, et videre, te omnium conditorem cum creaturis nil simile habere. Et nisi quis ab omnibus creaturis fuerit expeditus, non poterit libere intendere divinis. Ideo enim pauci inveniuntur contemplativi, quia pauci sciunt se a perituris et creaturis ad plenum sequestrari.
2. Ad hoc magna requiritur gratia, quæ animam levet et supra semet ipsam rapiat. Et nisi homo sit in spiritu elevatus et ab omnibus creaturis liberatus ac Deo totus unitus, quidquid scit, quidquid etiam habet, non est magni ponderis. Diu parvus erit et infra iacebit, qui aliquid magnum æstimat, nisi solum unum immensum, æternum bonum. Et quidquid Deus non est, nihil est et pro nihilo computari debet. Est magna differentia, sapientia illuminati et devoti viri et scientia litterati atque studiosi clerici. Multo nobilior est illa doctrina, quæ de sursum ex divina influentia manat, quam quæ laboriose humano acquiritur ingenio.
3. Plures reperiuntur contemplationem desiderare; sed quæ ad eam requiruntur, non student exercere. Est magnum impedimentum, quia in signis et sensibilibus rebus statur, et parum de perfecta mortificatione habetur. Nescio quid est, quo spiritu ducimur, et quid prætendimus, qui spirituales dici videmur, quod tantum laborem et ampliorem sollicitudinem pro transitoriis et vilibus rebus agimus, et de interioribus nostris vix raro plene recollectis sensibus cogitamus.
4. Pro dolor! statim post modicam recollectionem foras erumpimus, nec opera nostra districta examinatione trutinamus. Ubi iacent affectus nostri, non attendimus, et quam impura sint omnia, non deploramus. Omnis quippe caro corruperat viam suam, et ideo sequebatur diluvium magnum. Cum ergo interior affectus noster multum corruptus sit, necesse est, ut actio sequens, index carentiæ interioris vigoris, corrumpatur. Ex puro corde procedit fructus bonæ vitæ.
5. Quantum quis fecerit, quæritur; sed ex quanta virtute agit, non tam studiose pensatur. Si fuerit fortis, dives, pulcher, habilis, vel bonus scriptor, bonus cantor, bonus laborator, investigatur; quam pauper sit spiritu, quam patiens et mitis, quam devotus et internus, a multis tacetur. Natura exteriora hominis respicit, gratia ad interiora se convertit. Illa frequenter fallitur; ista in Deo sperat, ut non decipiatur.


CAPUT XXXII.
De abnegatione sui et abdicatione omnis cupiditatis

1. Fili, non potes perfectam possidere libertatem, nisi totaliter abneges temet ipsum. Compediti sunt omnes proprietarii et sui ipsius amatores cupidi, curiosi, gyrovagi, quærentes semper mollia, non quæ Iesu Christi, sed hoc sæpe fingentes et componentes, quod non stabit. Peribit enim totum, quod non est ex Deo ortum. Tene breve et consummatum verbum: Dimitte omnia, et invenies omnia; relinque cupidinem, et reperies requiem. Hoc mente pertracta; et cum impleveris, intelleges omnia.
2. Domine, hoc non est opus unius diei, nec ludus parvulorum; immo in hoc brevi includitur omnis perfectio religiosorum.
3. Fili, non debes averti, nec statim deici, audita via perfectorum; sed magis ad sublimiora provocari, et ad minus ad hæc ex desiderio suspirare. Utinam sic tecum esset, et ad hoc pervenisses, ut tui ipsius amator non esses, sed ad nutum meum pure stares, et eius, quem tibi præposui, Patris: tunc mihi valde placeres, et tota vita tua in gaudio et pace transiret. Adhuc multa habes ad relinquendum: quæ nisi mihi ex integro resignaveris, non acquires, quod postulas. Suadeo tibi emere a me aurum ignitum, ut locuples fias, id est, cælestem sapientiam omnia infima conculcantem. Postpone terrenam sapientiam, omnem humanam et propriam complacentiam.
4. Dixi, viliora tibi emenda pro pretiosis et altis in rebus humanis. Nam valde vilis et parva ac pæne oblivioni tradita videtur vera cælestis sapientia; non sapiens alta de se, nec magnificari quærens in terra: quam multi ore tenus prædicant, sed vita longe dissentiunt: ipsa tamen est pretiosa margarita a multis abscondita.


CAPUT XXXIII.
De instabilitate cordis et de intentione finali ad Deum habenda

1. Fili, noli credere affectui tuo; qui nunc est, cito mutabitur in aliud. Quamdiu vixeris, mutabilitati subiectus es, etiam nolens: ut modo lætus, modo tristis, modo pacatus, modo turbatus, nunc devotus, nunc indevotus, nunc studiosus, nunc acediosus, nunc gravis, nunc levis inveniaris. Sed stat super hæc mutabilia sapiens et bene doctus in spiritu, non attendens, quid in se sentiat, vel qua parte flet ventus instabilitatis, sed ut tota intentio mentis eius ad debitum et optatum proficiat finem. Nam sic poterit unus et idem inconcussusque permanere, simplici intentionis oculo per tot varios eventus ad me imprætermisse directo.
2. Quanto autem purior fuerit intentionis oculus, tanto constantius inter diversas itur procellas. Sed in multis caligat oculus puræ intentionis; respicitur enim cito in aliquod delectabile, quod occurrit. Nam et raro totus liber quis invenitur a nævo propriæ exquisitionis. Sic Iudæi olim venerant in Bethaniam ad Martham et Mariam, non propter Iesum tantum, sed et ut Lazarum viderent. Mundandus est ergo intentionis oculus, ut sit simplex et rectus, atque ultra omnia varia media ad me dirigendus.


CAPUT XXXIV.
Quod amanti sapit Deus super omnia et in omnibus

1. Ecce, Deus meus et omnia. Quid volo amplius, et quid felicius desiderare possum? O sapidum et dulce verbum! sed amanti Verbum, non mundum, nec ea, quæ in mundo sunt. Deus meus et omnia. Intellegenti satis dictum est, et sæpe repetere iucundum est amanti. Te siquidem præsente iucunda sunt omnia; te autem absente fastidiunt cuncta. Tu facis cor tranquillum et pacem magnam lætitiamque festivam. Tu facis bene sentire de omnibus et in omnibus te laudare, nec potest aliquid sine te diu placere; sed si debet gratum esse et bene sapere, oportet gratiam tuam adesse et condimento tuæ sapientiæ condiri.
2. Cui tu sapis, quid ei recte non sapiet? Et cui tu non sapis, quid ei ad iucunditatem esse poterit? Sed deficiunt in sapientia tua mundi sapientes, et qui carnem sapiunt: quia ibi plurima vanitas, et hic mors invenitur. Qui autem te per contemptum mundanorum et carnis mortificationem sequuntur, vere sapientes esse cognoscuntur: quia de vanitate ad veritatem, de carne ad spiritum transferuntur. Istis sapit Deus: et quidquid boni invenitur in creaturis, totum ad laudem referunt sui conditoris. Dissimilis tamen, et multum dissimilis sapor Creatoris et creaturæ, æternitatis et temporis, lucis increatæ et lucis illuminatæ.
3. O lux perpetua, cuncta creata transcendens lumina, fulgura coruscationem de sublimi penetrantem omnia cordis mei intima. Purifica, lætifica, clarifica et vivifica spiritum meum, cum suis potentiis ad inhærendum tibi iubilosis excessibus. O quando veniet hæc beata et desiderabilis hora, ut tua me saties præsentia et sis mihi omnia in omnibus? Quamdiu hoc datum non fuerit, nec plenum gaudium erit. Adhuc, pro dolor, vivit in me vetus homo, non est totus crucifixus, non est perfecte mortuus. Adhuc concupiscit fortiter contra spiritum, bella movet intestina, nec regnum animæ patitur esse quietum.
4. Sed tu, qui dominaris potestati maris et motum fluctuum eius mitigas, exurge, adiuva me. Dissipa gentes, quæ bella volunt; contere eas in virtute tua. Ostende, quæso, magnalia tua, et glorificetur dextera tua: quia non est spes alia nec refugium mihi, nisi in te, Domine Deus meus.


CAPUT XXXV.
Quod non est securitas a temptatione in hac vita

1. Fili, numquam securus es in hac vita, sed quoad vixeris, semper arma spiritualia tibi sunt necessaria. Inter hostes versaris, et a dextris et a sinistris impugnaris. Si ergo non uteris undique scuto patientiæ, non eris diu sine vulnere. Insuper, si non ponis cor tuum fixe in me, cum mera voluntate cuncta patiendi propter me, non poteris ardorem istum sustinere, nec ad palmam pertingere beatorum. Oportet te ergo viriliter omnia pertransire et potenti manu uti adversus obiecta. Nam vincenti datur manna, et torpenti relinquitur multa miseria.
2. Si quæris in hac vita requiem, quomodo tunc pervenies ad æternam requiem? Non ponas te ad multam requiem, sed ad magnam patientiam. Quære veram pacem, non in terris, sed in cælis, non in hominibus nec in ceteris creaturis, sed in Deo solo. Pro amore Dei debes omnia libenter subire, labores scilicet et dolores, temptationes, vexationes, anxietates, necessitates, infirmitates, iniurias, oblocutiones, reprehensiones, humiliationes, confusiones, correctiones et despectiones. Hæc iuvant ad virtutem, hæc probant Christi tironem, hæc fabricant cælestem coronam. Ego reddam mercedem æternam pro brevi labore, et infinitam gloriam pro transitoria confusione.
3. Putas tu, quod semper habebis pro tua voluntate consolationes spirituales? Sancti mei non semper habuerunt tales, sed multas gravitates et temptationes varias magnasque desolationes. Sed patienter sustinuerunt se in omnibus, et magis confisi sunt Deo, quam sibi, scientes, quia non sunt condignæ passiones huius temporis ad futuram gloriam promerendam. Vis tu statim habere, quod multi post multas lacrimas et magnos labores vix obtinuerunt? Expecta Dominum, viriliter age, et confortare; noli diffidere, noli discedere, sed corpus et animam expone constanter pro gloria Dei. Ego reddam plenissime, ego tecum ero in omni tribulatione.


CAPUT XXXVI.
Contra vana hominum iudicia

1. Fili, iacta cor tuum firmiter in Domino, et humanum ne metuas iudicium, ubi te conscientia pium reddit et insontem. Bonum est et beatum taliter pati, nec hoc erit grave humili cordi et Deo magis quam sibi ipsi confidenti. Multi multa loquuntur, et ideo parva fides est adhibenda. Sed et omnibus satis esse, non est possibile. Etsi Paulus omnibus studuit in Domino placere et omnibus omnia factus est, tamen etiam pro minimo duxit, quod ab humano die iudicatus fuit.
2. Egit satis pro aliorum ædificatione et salute, quantum in se erat et poterat; sed ne ab aliis aliquando iudicaretur, vel non despiceretur, cohibere non potuit. Ideo totum Deo commisit, qui totum noverat; et patientia ac humilitate contra ora loquentium iniqua aut etiam vana ac mendosa cogitantium atque pro libitu suo quæque iactantium se defendit. Respondit tamen interdum, ne infirmis ex sua taciturnitate generaretur scandalum.
3. Quis tu, ut timeas a mortali homine? Hodie est, et cras non comparet. Deum time, et hominum terrores non expavesces. Quid potest aliquis in te verbis aut iniuriis? Sibi potius nocet, quam tibi; nec poterit iudicium Dei effugere, quicumque est ille. Tu habe Deum præ oculis, et noli contendere verbis querulosis. Quodsi ad præsens tu videris succumbi et confusionem pati, quam non meruisti: ne indigneris ex hoc, neque per impatientiam minuas coronam tuam, sed ad me potius respice in cælum, qui potens sum eripere ab omni confusione et iniuria, et unicuique reddere secundum opera sua.


CAPUT XXXVII.
De pura et integra resignatione sui ad obtinendam cordis libertatem

1. Fili, relinque te, et invenies me. Sta sine electione et omni proprietate, et lucraberis semper. Nam et adicietur tibi amplior gratia, statim ut te resignaveris nec resumpseris.
2. Domine, quotiens me resignabo, et in quibus me relinquam?
3. Semper et omni hora: sicut in parvo, sic et in magno. Nihil excipio, sed in omnibus te nudatum inveniri volo. Alioquin, quomodo poteris esse meus et ego tuus, nisi fueris ab omni propria voluntate intus et foris spoliatus? Quanto celerius hoc agis, tanto melius habebis, et quanto plenius et sincerius, tanto mihi plus placebis, et amplius lucraberis.
4. Quidam se resignant, sed cum aliqua exceptione: non enim plene Deo confidunt, ideo sibi providere satagunt. Quidam etiam primo totum offerunt, sed postea temptatione pulsati ad propria redeunt, ideo minime in virtute proficiunt. Hi ad veram puri cordis libertatem et iucundæ familiaritatis meæ gratiam non pertingent, nisi integra resignatione et cotidiana sui immolatione prius facta; sine qua non stat nec stabit unio fruitiva.
5. Dixi tibi sæpissime, et nunc iterum dico: Relinque te, resigna te, et frueris magna interna pace. Da totum pro toto; nil exquire, nil repete; sta pure et inhæsitanter in me, et habebis me. Eris liber in corde, et tenebræ non conculcabunt te. Ad hoc conare, hoc ora, hoc desidera, ut ab omni proprietate possis expoliari, et nudus nudum Iesum sequi, tibi mori et mihi æternaliter vivere. Tunc deficient omnes vanæ phantasiæ, conturbationes iniquæ et curæ superfluæ. Tunc etiam recedet immoderatus timor, et inordinatus amor morietur.


CAPUT XXXVIII.
De bono regimine in externis et recursu ad Deum in periculis

1. Fili, ad istud diligenter tendere debes, ut in omni loco et actione seu occupatione externa sis intimus liber et tui ipsius potens, et sint omnia sub te, et tu non sub eis: ut sis dominus actionum tuarum et rector, non servus, nec emptitius, sed magis exemptus verusque Hebræus, in sortem ac libertatem transiens filiorum Dei: qui stant super præsentia et speculantur æterna, qui transitoria sinistro intuentur oculo et dextro cælestia: quos temporalia non trahunt ad inhærendum, sed trahunt ipsi magis ea ad bene serviendum, prout ordinata sunt a Deo et instituta a summo opifice, qui nil inordinatum in sua reliquit creatura.
2. Si etiam in omni eventu stas non in apparentia externa, nec oculo carnali lustras visa vel audita, sed mox in qualibet causa intras cum Moyse in tabernaculum ad consulendum Dominum: audies nonnumquam divinum responsum, et redies instructus de multis præsentibus et futuris. Semper enim Moyses recursum habuit ad tabernaculum pro dubiis et quæstionibus solvendis, fugitque ad orationis adiutorium, pro periculis et improbitatibus hominum sublevandis. Sic et tu confugere debes in cordis tui secretarium, divinum intentius implorando suffragium. Propterea namque Iosue et filii Isræl a Gabaonitis leguntur decepti: quia os Domini prius non interrogaverunt, sed nimium creduli dulcibus sermonibus, falsa pietate delusi sunt.


CAPUT XXXIX.
Quod homo non sit importunus in negotiis

1. Fili, committe mihi semper causam tuam, ego bene disponam in tempore suo. Expecta ordinationem meam, et senties inde profectum.
2. Domine, satis libenter tibi omnes res committo, quia parum potest cogitatio mea proficere. Utinam non multum adhærerem futuris eventibus, sed ad beneplacitum tuum incunctanter me offerrem!
3. Fili, sæpe homo rem aliquam vehementer agitat, quam desiderat; sed cum ad eam pervenerit, aliter sentire incipit: quia affectiones circa idem non sunt durabiles, sed magis de uno ad aliud impellunt. Non est ergo minimum, etiam in minimis se ipsum relinquere.
4. Verus profectus hominis est abnegatio sui ipsius, et homo abnegatus valde liber est et securus. Sed antiquus hostis, omnibus bonis adversans, a temptatione non cessat; sed die noctuque graves molitur insidias, si forte in laqueum deceptionis possit præcipitare incautum. Vigilate et orate, dicit Dominus, ut non intretis in temptationem.


CAPUT XL.
Quod homo nihil boni ex se habet et de nullo gloriari potest

1. Domine, quid est homo, quod memor es eius, aut filius hominis, quia visitas eum? Quid promeruit homo, ut dares illi gratiam tuam? Domine, quid possum conqueri, si me deseris; aut quid iuste obtendere possum, si, quod peto, non feceris? Certe hoc in veritate cogitare possum et dicere: Domine, nihil sum, nihil possum, nihil boni ex me habeo; sed in omnibus deficio et ad nihil semper tendo. Et nisi a te fuero adiutus et interius informatus, totus efficior tepidus et dissolutus.
2. Tu autem, Domine, semper idem ipse es, et permanes in æternum semper bonus, iustus et sanctus, bene, iuste ac sancte agens omnia, et disponens in sapientia. Sed ego, qui ad defectum sum magis pronus quam ad profectum, non sum semper in uno statu perdurans, quia semper tempora mutantur super me. Verumtamen cito melius fit, cum tibi placuerit, et manum porrexeris adiutricem, quia tu solus sine humano suffragio poteris auxiliari et in tantum confirmare, ut vultus meus amplius in diversa non mutetur, sed in te uno cor meum convertatur et quiescat.
3. Unde, si bene scirem omnem humanam consolationem abicere, sive propter devotionem adipiscendam, sive propter necessitatem, qua compellor te quærere, quia non est homo, qui me consoletur, tunc possem merito de gratia tua sperare et de dono novæ consolationis exultare.
4. Gratias tibi, unde totum venit, quotienscumque mihi bene succedit. Ego autem vanitas et nihilum ante te, inconstans homo et infirmus. Unde ergo possum gloriari, aut cur appeto reputari? Numquid de nihilo? et hoc vanissimum est. Vere inanis gloria mala pestis, vanitas maxima: quia a vera trahit gloria et cælesti spoliat gratia. Dum enim homo complacet sibi, displicet tibi: dum inhiat laudibus humanis, privatur veris virtutibus.
5. Est autem vera gloria et exultatio sancta, gloriari in te et non in se, gaudere in nomine tuo, non in propria virtute, nec in aliqua creatura delectari, nisi propter te. Laudetur nomen tuum, non meum; magnificetur opus tuum, non meum; benedicatur nomen sanctum tuum, nihil mihi autem attribuatur de laudibus hominum. Tu gloria mea, tu exultatio cordis mei. In te gloriabor et exultabo tota die; pro me autem nihil, nisi in infirmitatibus meis.
6. Quærant Iudæi gloriam, quæ ab invicem est; ego hanc requiram, quæ a solo Deo est. Omnis quidem gloria humana, omnis honor temporalis, omnis altitudo mundana, æternæ gloriæ tuæ comparata, vanitas est et stultitia. O veritas mea et misericordia mea, Deus meus, Trinitas beata, tibi soli laus, honor, virtus, gloria, per infinita sæculorum sæcula.


CAPUT XLI.
De contemptu omnis temporalis honoris

1. Fili, noli tibi attrahere, si videas alios honorari et elevari, te autem despici et humiliari. Erige cor tuum ad me in cælum, et non contristabit te contemptus hominum in terris.
2. Domine, in cæcitate sumus, et vanitate cito seducimur. Si recte me inspicio, numquam mihi facta est iniuria ab aliqua creatura, unde nec iuste habeo conqueri adversus te. Quia autem frequenter et graviter peccavi tibi, merito armatur contra me omnis creatura. Mihi igitur iuste debetur confusio et contemptus; tibi autem laus, honor et gloria. Et nisi me ad hoc præparavero, quod velim libenter ab omni creatura despici et relinqui, atque penitus nihil videri, non possum interius pacificari et stabiliri, nec spiritualiter illuminari, neque plene tibi uniri.


CAPUT XLII.
Quod pax non est ponenda in hominibus

1. Fili, si ponis pacem tuam cum aliqua persona, propter tuum sentire et convivere, instabilis eris et implicatus. Sed si recursum habes ad semper viventem et manentem veritatem, non contristabit amicus recedens aut moriens. In me debet amici dilectio stare, et propter me diligendus est, quisquis tibi bonus visus est et multum carus in hac vita. Sine me non valet nec durabit amicitia, nec
est vera et munda dilectio, quam ego non copulo. Ita mortuus debes esse talibus affectionibus dilectorum hominum, ut, quantum ad te pertinet, sine omni humano optares esse consortio. Tanto homo Deo magis appropinquat, quanto ab omni solacio terreno longius recedit. Tanto etiam altius ad Deum ascendit, quanto profundius in se descendit et plus sibi ipsi vilescit.
2. Qui autem sibi aliquid boni attribuit, gratiam Dei in se venire impedit, quia gratia Spiritus Sancti cor humile semper quærit. Si scires te perfecte adnihilare atque ab omni creato amore evacuare, tunc deberem in te cum magna gratia emanare. Quando tu respicis ad creaturas, subtrahitur tibi aspectus Creatoris. Disce te in omnibus propter Creatorem vincere, tunc ad divinam valebis cognitionem pertingere. Quantumcumque modicum sit, si inordinate diligitur et respicitur, retardat a summo et vitiat.


CAPUT XLIII.
Contra vanam et sæcularem scientiam

1. Fili, non te moveant pulchra et subtilia hominum dicta. Non enim est regnum Dei in sermone, sed in virtute. Attende verba mea, quæ corda accendunt et mentes illuminant, inducunt compunctionem et variam ingerunt consolationem. Numquam ad hoc legas verbum, ut doctior aut sapientior possis videri. Stude mortificationi vitiorum, quia hoc amplius tibi proderit, quam notitia multarum difficilium quæstionum.
2. Cum multa legeris et cognoveris, ad unum semper oportet redire principium. Ego sum, qui doceo hominem scientiam, et clariorem intellegentiam parvulis tribuo, quam ab homine possit doceri. Cui ego loquor, cito sapiens erit et multum in spiritu proficiet. Væ eis, qui multa curiosa ab hominibus inquirunt, et de via mihi serviendi parum curant. Veniet tempus, quando apparebit magister magistrorum Christus, dominus angelorum, cunctorum auditurus lectiones, hoc est singulorum examinaturus conscientias. Et tunc scrutabitur Ierusalem in lucernis, et manifesta erunt abscondita tenebrarum, tacebuntque argumenta linguarum.
3. Ego sum, qui humilem in puncto elevo mentem, ut plures æternæ veritatis capiat rationes, quam si quis decem annis studuisset in scholis. Ego doceo sine strepitu verborum, sine confusione opinionum, sine fastu honoris, sine pugnatione argumentorum. Ego sum, qui doceo terrena despicere, præsentia fastidire, æterna quærere, æterna sapere, honores fugere, scandala sufferre, omnem spem in me ponere, extra me nil cupere, et super omnia me ardenter amare.
4. Nam quidam, amando me intime, didicit divina et loquebatur mirabilia. Plus profecit in relinquendo omnia, quam in studendo subtilia. Sed aliis loquor communia, aliis specialia; aliquibus in signis et figuris dulciter appareo, quibusdam vero in multo lumine revelo mysteria. Una vox librorum, sed non omnes æque informat: quia intus sum doctor veritatis, scrutator cordis, cogitationum intellector, actionum promotor, distribuens singulis, sicut dignum iudicavero.


CAPUT XLIV.
De non attrahendo sibi res exteriores

1. Fili, in multis oportet te esse inscium, et æstimare te tamquam mortuum super terram, et cui totus mundus crucifixus sit. Multa etiam oportet surda aure pertransire, et quæ tuæ pacis sunt, magis cogitare. Utilius est oculos a rebus displicentibus avertere et unicuique suum sentire relinquere, quam contentiosis sermonibus deservire. Si bene steteris cum Deo et eius iudicium aspexeris, facilius te victum portabis.
2. O Domine, quousque venimus? Ecce, damnum defletur temporale, pro modico quæstu laboratur et curritur, et spirituale detrimentum in oblivionem transit, et vix sero reditur. Quod parum vel nihil prodest, attenditur; et quod summe necessarium est, neglegenter præteritur: quia totus homo ad externa defluit; et nisi cito resipiscat, libens in exterioribus iacet.


CAPUT XLV.
Quod omnibus non est credendum, et de facili lapsu verborum

1. Da mihi auxilium, Domine, de tribulatione, quia vana salus hominis. Quam sæpe ibi non inveni fidem, ubi me habere putavi? Quotiens etiam ibi repperi, ubi minus præsumpsi? Vana ergo spes in hominibus, salus autem iustorum in te, Deus. Benedictus sis, Domine Deus meus, in omnibus, quæ accidunt nobis. Infirmi sumus et instabiles, cito fallimur et permutamur.
2. Quis est homo, qui ita caute et circumspecte in omnibus se custodire valet, ut aliquando in aliquam deceptionem vel perplexitatem non veniat? Sed qui in te, Domine, confidit, ac simplici ex corde quærit, non tam facile labitur. Et si inciderit aliquam tribulationem, quocumque modo fuerit etiam implicatus, citius per te eruetur, aut a te consolabitur: quia tu non deseres in te sperantem usque in finem. Rarus fidus amicus, in cunctis amici perseverans pressuris. Tu, Domine, tu solus es fidelissimus in omnibus, et præter te non est alter talis.
3. O quam bene sapuit sancta illa anima, quæ dixit: Mens mea solidata est et in Christo fundata. Si ita mecum foret, non tam facile timor humanus me sollicitaret, nec verborum iacula moverent. Quis omnia prævidere, quis præcavere futura mala sufficit? Si prævisa sæpe etiam lædunt, quid improvisa nisi graviter feriunt? Sed quare mihi misero non melius providi? Cur etiam tam facile aliis credidi? Sed homines sumus, nec aliud quam fragiles homines sumus, etiam si angeli a multis æstimamur et dicimur. Cui credam, Domine? cui, nisi tibi? Veritas es, quæ non fallis, nec falli potes. Et rursum: Omnis homo mendax, infirmus, instabilis et labilis maxime in verbis, ita ut statim vix credi debeat, quod rectum in facie sonare videtur.
4. Quam prudenter præmonuisti, cavendum ab hominibus, et quia inimici hominis domestici eius; nec credendum, si quis dixerit: Ecce hic, aut: Ecce illic. Doctus sum damno meo, et utinam ad cautelam maiorem, et non ad insipientiam mihi. Cautus esto, ait quidam, cautus esto, serva apud te, quod dico. Et dum ego sileo et absconditum credo, nec ille silere potest, quod silendum petiit, sed statim prodit me et se, et abiit. Ab huiusmodi fabulis et incautis hominibus protege me, Domine, ne in manus eorum incidam, nec umquam talia committam. Verbum verum et stabile da in os meum, et linguam callidam longe fac a me. Quod pati nolo, omnimode cavere debeo.
5. O quam bonum et pacificum de aliis silere, nec indifferenter omnia credere, neque de facili ulterius effari, paucis se ipsum revelare, te semper inspectorem cordis quærere, nec omni vento verborum circumferri, sed omnia intima et externa secundum placitum tuæ voluntatis optare perfici! Quam tutum pro conservatione cælestis gratiæ humanam fugere apparentiam, nec appetere, quæ foris admirationem videntur præbere, sed ea tota sedulitate sectari, quæ vitæ emendationem dant et fervorem! Quam multis nocuit virtus scita ac præpropere laudata! Quam sane profuit gratia silentio servata in hac fragili vita, quæ tota temptatio fertur et militia!


CAPUT XLVI.
De confidentia in Deo habenda, quando insurgunt verborum iacula

1. Fili, sta firmiter et spera in me. Quid enim sunt verba, nisi verba? Per aërem volant, sed lapidem non lædunt. Si reus es, cogita, quod libenter emendare te velis; si nihil tibi conscius es, pensa, quod velis libenter pro Deo hoc sustinere. Parum satis est, ut vel verba interdum sustineas, qui necdum fortia verbera tolerare vales. Et quare tam parva tibi ad cor transeunt, nisi quia adhuc carnalis es, et homines magis, quam oportet, attendis? Nam quia despici metuis, reprehendi pro excessibus non vis, et excusationum umbracula quæris.
2. Sed inspice te melius, et agnosces, quia vivit adhuc in te mundus et vanus amor placendi hominibus. Cum enim bassari refugis et confundi pro defectibus, constat utique, quod nec verus humilis sis, nec vere mundo mortuus, nec mundus tibi crucifixus. Sed audi verbum meum, et non curabis decem milia verba hominum. Ecce, si cuncta contra te dicerentur, quæ fingi malitiosissime possent; quid tibi nocerent, si omnino transire permitteres, nec plus quam festucam perpenderes? Numquid vel unum capillum tibi extrahere possent?
3. Sed qui cor intus non habet, nec Deum præ oculis, faciliter verbo movetur vituperationis. Qui autem in me confidit, nec proprio iudicio stare appetit, absque humano terrore erit. Ego enim sum iudex, et cognitor omnium secretorum; ego scio, qualiter res acta est; ego iniuriantem novi et sustinentem. A me exiit verbum istud, me permittente hoc accidit, ut revelentur ex multis cordibus cogitationes. Ego reum et innocentem iudicabo, sed occulto iudicio utrumque ante probare volui.
4. Testimonium hominum sæpe fallit; meum iudicium verum est, stabit, et non subvertetur. Latet plerumque et paucis ad singula patet; numquam tamen errat, nec errare potest, etiam si oculis insipientium non rectum videatur. Ad me ergo currendum est in omni iudicio, nec proprio innitendum arbitrio. Iustus enim non conturbabitur, quidquid a Deo ei acciderit. Etiam si iniuste aliquid contra eum prolatum fuerit, non multum curabit. Sed nec vane exultabit, si per alios rationabiliter excusetur. Pensat namque, quia ego sum scrutans corda et renes, qui non iudico secundum faciem et humanam apparentiam. Nam sæpe in oculis meis reperitur culpabile, quod hominum iudicio creditur laudabile.
5. Domine Deus, iudex iuste, fortis et patiens, qui hominum nosti fragilitatem et pravitatem, esto robur meum et tota fiducia mea: non enim mihi sufficit conscientia mea. Tu nosti, quod ego non novi; et ideo in omni reprehensione humiliare me debui et mansuete sustinere. Ignosce quoque mihi propitius, quotiens sic non egi, et dona iterum gratiam amplioris sufferentiæ. Melior est enim mihi tua copiosa misericordia, ad consecutionem indulgentiæ, quam mea opinata iustitia, pro defensione latentis conscientiæ. Et si nihil mihi conscius sum, tamen in hoc iustificare me non possum: quia remota misericordia tua, non iustificabitur in conspectu tuo omnis vivens.


CAPUT XLVII.
Quod omnia gravia pro æterna vita sunt toleranda

1. Fili, non te frangant labores, quos assumpsisti propter me, nec tribulationes te deiciant usquequaque; sed mea promissio in omni eventu te roboret et consoletur. Ego sufficiens sum ad reddendum supra omnem modum et mensuram. Non diu hic laborabis, nec semper gravaberis doloribus. Expecta paulisper, et videbis celerem finem malorum. Veniet una hora, quando cessabit omnis labor et tumultus. Modicum est et breve omne, quod transit cum tempore.
2. Age, quod agis; fideliter labora in vinea mea, ego ero merces tua. Scribe, lege, canta, geme, tace, ora, sustine viriliter contraria: digna est his omnibus et maioribus prœliis vita æterna. Veniet pax in die una, quæ nota est Domino; et erit non dies neque nox huius scilicet temporis, sed lux perpetua, claritas infinita, pax firma, et requies secura. Non dices tunc: Quis me liberabit de corpore mortis huius? Neque clamabis: Heu mihi, quia incolatus meus prolongatus est! quoniam præcipitabitur mors, et salus erit indefectiva, anxietas nulla, iucunditas beata, societas dulcis et decora.
3. O si vidisses sanctorum in cælo coronas perpetuas, quanta quoque nunc exultant gloria, qui huic mundo olim contemptibiles et quasi vita ipsa indigni putabantur; profecto te statim humiliares usque ad terram, et affectares potius omnibus subesse, quam uni præesse; nec huius vitæ lætos dies concupisceres, sed magis pro Deo tribulari gauderes, et pro nihilo inter homines computari, maximum lucrum duceres.
4. O si tibi hæc saperent, et profunde ad cor transirent, quomodo auderes vel semel conqueri? Nonne pro vita æterna cuncta laboriosa sunt toleranda? Non est parvum quid, perdere aut lucrari regnum Dei. Leva igitur faciem tuam in cælo. Ecce, ego et omnes sancti mei mecum, qui in hoc sæculo magnum habuere certamen, modo gaudent, modo consolantur, modo securi sunt, modo requiescunt, et sine fine mecum in regno patris mei permanebunt.


CAPUT XLVIII.
De die æternitatis et huius vitæ angustiis

1. O supernæ civitatis mansio beatissima! O dies æternitatis clarissima, quam nox non obscurat, sed summa veritas semper irradiat; dies semper læta, semper secura, et numquam statum mutans in contraria! O utinam dies illa illuxisset, et cuncta hæc temporalia finem accepissent! Lucet quidem sanctis perpetua claritate splendida, sed non nisi a longe et per speculum peregrinantibus in terra.
2. Norunt cæli cives, quam gaudiosa sit illa; gemunt exules filii Evæ, quod amara et tædiosa sit ista. Dies huius temporis parvi et mali, pleni doloribus et angustiis: ubi homo multis peccatis inquinatur, multis passionibus irretitur, multis timoribus stringitur, multis curis distenditur, multis curiositatibus distrahitur, multis vanitatibus implicatur, multis erroribus circumfunditur, multis laboribus atteritur, temptationibus gravatur, deliciis enervatur, egestate cruciatur.
3. O quando finis horum malorum? quando liberabor a misera servitute vitiorum? quando memorabor, Domine, tui solius? quando ad plenum lætabor in te? Quando ero sine omni impedimento in vera libertate, sine omni gravamine mentis et corporis? Quando erit pax solida, pax imperturbabilis et secura, pax intus et foris, pax ex omni parte firma? Iesu bone, quando stabo ad videndum te? quando contemplabor gloriam regni tui? quando eris mihi omnia in omnibus? O quando ero tecum in regno tuo, quod præparasti dilectis tuis ab æterno! Relictus sum pauper et exul in terra hostili, ubi bella cotidiana et infortunia maxima.
4. Consolare exilium meum, mitiga dolorem meum, quia ad te suspirat omne desiderium meum. Nam onus mihi totum est, quidquid hic mundus offert ad solacium. Desidero te intime frui, sed nequeo apprehendere. Opto inhærere cælestibus, sed deprimunt res temporales et immortificatæ passiones. Mente omnibus rebus superesse volo, carne autem invite subesse cogor. Sic ego homo infelix mecum pugno, et factus sum mihimet ipsi gravis, dum spiritus sursum et caro quærit esse deorsum.
5. O quid intus patior, dum mente cælestia tracto, et mox carnalium turba occurrit oranti! Deus meus, ne elongeris a me, neque declines in ira a servo tuo. Fulgura coruscationem tuam et dissipa eas; emitte sagittas tuas, et conturbentur omnes phantasiæ inimici. Recollige sensus meos ad te; fac me oblivisci omnium mundanorum; da cito abicere et contemnere phantasmata vitiorum. Succurre mihi, æterna veritas, ut nulla me moveat vanitas. Adveni, cælestis suavitas, et fugiat a facie tua omnis impuritas. Ignosce quoque mihi et misericorditer indulge, quotiens præter te aliud in oratione revolvo. Confiteor etenim vere, quia valde distracte me habere consuevi. Nam ibi multotiens non sum, ubi corporaliter sto aut sedeo; sed ibi magis sum, quo cogitationibus feror. Ibi sum, ubi cogitatio mea est. Ibi est frequenter cogitatio mea, ubi est, quod amo. Hoc mihi cito occurrit, quod naturaliter delectat aut ex usu placet.
6. Unde tu, veritas, aperte dixisti: Ubi enim est thesaurus tuus, ibi est et cor tuum. Si cælum diligo, libenter de cælestibus penso. Si mundum amo, mundi felicitatibus congaudeo et de adversitatibus eius tristor. Si carnem diligo, quæ carnis sunt, sæpe imaginor. Si spiritum amo, de spiritualibus cogitare delector. Quæcumque enim diligo, de his libenter loquor et audio, atque talium imagines mecum ad domum reporto. Sed beatus ille homo, qui propter te, Domine, omnibus creaturis licentiam abeundi tribuit, qui naturæ vim facit et concupiscentias carnis fervore spiritus crucifigit, ut serenata conscientia puram tibi orationem offerat, dignusque sit angelicis interesse choris, omnibus terrenis foris et intus exclusis.


CAPUT XLIX.
De desiderio æternæ vitæ, et quanta sint certantibus bona promissa

1. Fili, cum tibi desiderium æternæ beatitudinis desuper infundi sentis, et de tabernaculo corporis exire concupiscis, ut claritatem meam sine vicissitudinis umbra contemplari possis: dilata cor tuum et omni desiderio hanc sanctam inspirationem suscipe. Redde amplissimas supernæ bonitati gratias, quæ tecum sic dignanter agit, clementer visitat, ardenter excitat, potenter sublevat, ne proprio pondere ad terrena labaris. Neque enim hoc cogitatu tuo aut conatu accipis, sed sola dignatione supernæ gratiæ et divini respectus: quatenus in virtutibus et maiori humilitate proficias, et ad futura certamina te præpares, mihique toto cordis affectu adhærere ac ferventi voluntate studeas deservire.
2. Fili, sæpe ignis ardet, sed sine fumo flamma non ascendit. Sic et aliquorum desideria ad cælestia flagrant, et tamen a temptatione carnalis affectus liberi non sunt. Idcirco nec omnino pure pro honore Dei agunt, quod tam desideranter ab eo petunt. Tale est et tuum sæpe desiderium, quod insinuasti fore tam importunum. Non enim est hoc purum et perfectum, quod propria commoditate est infectum.
3. Pete, non quod tibi est delectabile et commodum, sed quod mihi est acceptabile atque honorificum: quia, si recte iudicas, meam ordinationem tuo desiderio et omni desiderata præferre debes ac sequi. Novi desiderium tuum, et frequentes gemitus audivi. Iam velles esse in libertate gloriæ filiorum Dei; iam te delectat domus æterna et cælestis patria gaudio plena, sed nondum venit hora ista: sed est adhuc aliud tempus, scilicet tempus belli, tempus laboris et probationis. Optas summo repleri bono, sed non potes hoc assequi modo. Ego sum: expecta me, dicit Dominus, donec veniat regnum Dei.
4. Probandus es adhuc in terris et in multis exercitandus. Consolatio tibi interdum dabitur, sed copiosa satietas non conceditur. Confortare igitur, et esto robustus, tam in agendo quam in patiendo naturæ contraria. Oportet te novum induere hominem et in alterum virum mutari. Oportet te sæpe agere, quod non vis; et quod vis, oportet relinquere. Quod aliis placet, processum habebit; quod tibi placet, ultra non proficiet. Quod alii dicunt, audietur; quod tu dicis, pro nihilo computabitur. Petent alii, et accipient; tu petes, nec impetrabis.
5. Erunt alii magni in ore hominum, de te autem tacebitur. Aliis hoc vel illud committetur, tu autem ad nihil utilis iudicaberis. Propter hoc natura quandoque contristabitur; et magnum, si silens portaveris. In his et similibus multis probari solet fidelis Domini servus, qualiter se abnegare, et in omnibus frangere quiverit. Vix est aliquid tale, in quo tantundem mori indiges, sicut videre et pati, quæ voluntati tuæ adversa sunt; maxime autem, cum disconvenientia et quæ minus utilia tibi apparent, fieri iubentur. Et quia non audes resistere altiori potestati, sub dominio constitutus: ideo durum tibi videtur ad nutum alterius ambulare et omne proprium sentire omittere.
6. Sed pensa, fili, horum fructum laborum, celerem finem atque præmium nimis magnum; et non habebis inde gravamen, sed fortissimum patientiæ tuæ solamen. Nam et pro modica hac voluntate, quam nunc sponte deseris, habebis semper voluntatem tuam in cælis. Ibi quippe invenies omne, quod volueris, omne, quod desiderare poteris. Ibi aderit tibi totius facultas boni sine timore amittendi. Ibi voluntas tua una semper mecum, nil cupiet extraneum vel privatum. Ibi nullus resistet tibi, nemo de te conqueretur, nemo impediet, nihil obviabit; sed cuncta desiderata simul erunt præsentia, totumque affectum tuum reficient et adimplebunt usque ad summum. Ibi reddam gloriam pro contumelia perpessa, pallium laudis pro mærore, pro loco novissimo sedem regni in sæcula. Ibi apparebit fructus obœdientiæ, gaudebit labor pænitentiæ, et humilis subiectio coronabitur gloriose.
7. Nunc ergo inclina te humiliter sub omnium manibus; nec sit tibi curæ, quis hoc dixerit vel iusserit. Sed hoc magnopere curato, ut sive prælatus seu iunior aut æqualis aliquid a te exposcerit vel innuerit, pro bono totum accipias, et sincera voluntate studeas adimplere. Quærat alius hoc, alius illud, glorietur ille in illo, et iste in isto, laudeturque millies mille: tu autem nec in isto, nec in illo, sed in tui ipsius gaude contemptu, et in mei solius beneplacito ac honore. Hoc optandum est tibi, ut sive per vitam, sive per mortem Deus semper in te glorificetur.


CAPUT L.
Qualiter homo desolatus se debet in manus Dei offerre

1. Domine Deus, sancte Pater, sis nunc et in æternum benedictus, quia, sicut vis, sic factum est, et quod facis, bonum est. Lætetur in te servus tuus, non in se, nec in aliquo alio; quia tu solus lætitia vera, tu spes mea et corona mea, tu gaudium meum et honor meus, Domine. Quid habet servus tuus, nisi quod a te accepit, etiam sine merito suo? Tua sunt omnia, quæ dedisti et quæ fecisti. Pauper sum et in laboribus meis a iuventute mea: et contristatur anima mea nonnumquam usque ad lacrimas, quandoque etiam conturbatur ad se propter imminentes passiones.
2. Desidero pacis gaudium, pacem filiorum tuorum flagito, qui in lumine consolationis a te pascuntur. Si das pacem, si gaudium sanctum infundis, erit anima servi tui plena modulatione, et devota in laude tua. Sed si te subtraxeris, sicut sæpissime soles, non poterit currere viam mandatorum tuorum, sed magis ad tundendum pectus genua eius incurvantur: quia non est illi sicut heri et nudius tertius, quando splendebat lucerna tua super caput eius, et sub umbra alarum tuarum protegebatur a temptationibus irruentibus.
3. Pater iuste et semper laudande, venit hora, ut probetur servus tuus. Pater amande, dignum est, ut hac hora patiatur pro te aliquid servus tuus. Pater perpetuo venerande, venit hora, quam ab æterno præsciebas affuturam, ut ad modicum tempus succumbat foris servus tuus, vivat vero semper apud te intus. Paululum vilipendatur, humilietur et deficiat coram hominibus, passionibus conteratur et languoribus: ut iterum tecum in aurora novæ lucis resurgat et in cælestibus clarificetur. Pater sancte, tu sic ordinasti et sic voluisti; et hoc factum est, quod ipse præcepisti.
4. Hæc est enim gratia ad amicum tuum, pati et tribulari in mundo pro amore tuo, quotienscumque et a quocumque id permiseris fieri. Sine consilio et providentia tua et sine causa nihil fit in terra. Bonum mihi, Domine, quod humiliasti me, ut discam iustificationes tuas, et omnes elationes cordis atque præsumptiones abiciam. Utile mihi, quod confusio cooperuit faciem meam; ut te potius, quam homines ad consolandum requiram. Didici etiam ex hoc inscrutabile iudicium tuum expavescere: qui affligis iustum cum impio, sed non sine æquitate et iustitia.
5. Gratias tibi ago, quia non pepercisti malis meis, sed attrivisti me verberibus amaris, infligens dolores et immittens angustias foris et intus. Non est, qui me consoletur ex omnibus, quæ sub cælo sunt, nisi tu, Domine Deus meus, cælestis medicus animarum: qui percutis et sanas, deducis ad inferos et reducis. Disciplina tua super me, et virga tua ipsa me docebit.
6. Ecce, Pater dilecte, in manibus tuis sum ego, sub virga correctionis tuæ me inclino. Percute dorsum meum et collum meum, ut incurvem ad voluntatem tuam tortuositatem meam. Fac me pium et humilem discipulum, sicut bene facere consuevisti, ut ambulem ad omnem nutum tuum. Tibi me et omnia mea ad corrigendum commendo; melius est hic corripi, quam in futuro. Tu scis omnia et singula, et nil te latet in humana conscientia. Antequam fiant, nosti ventura: et non opus est tibi, ut quis te doceat aut admoneat de his, quæ geruntur in terra. Tu scis, quid expedit ad profectum meum, et quantum deservit tribulatio ad rubiginem vitiorum purgandam. Fac mecum desideratum beneplacitum tuum, et ne despicias peccaminosam vitam meam, nulli melius nec clarius, quam tibi soli, notam.
7. Da mihi, Domine, scire, quod sciendum est, hoc amare, quod amandum est, hoc laudare, quod tibi summe placet; hoc reputare, quod tibi pretiosum apparet, hoc vituperare, quod oculis tuis sordescit. Non me sinas secundum visionem oculorum exteriorum iudicare, neque secundum auditum aurium hominum imperitorum sententiare; sed in iudicio vero de visibilibus et spiritualibus discernere, atque super omnia voluntatem beneplaciti tui semper inquirere.
8. Falluntur sæpe hominum sensus in iudicando; falluntur et amatores sæculi, visibilia tantummodo amando. Quid est homo inde melior, quia reputatur ab homine maior? Fallax fallacem, vanus vanum, cæcus cæcum, infirmus infirmum decipit, dum exaltat; et veraciter magis confundit, dum inaniter laudat. Nam quantum unusquisque est in oculis tuis, tantum est, et non amplius, ait humilis sanctus Franciscus.


CAPUT LI.
Quod humilibus insistendum est operibus, cum deficitur a summis

1. Fili, non vales semper in ferventiori desiderio virtutum stare, nec in altiori gradu contemplationis consistere: sed necesse habes interdum ob originalem corruptelam ad inferiora descendere, et onus corruptibilis vitæ etiam invite et cum tædio portare. Quamdiu mortale corpus geris, tædium senties et gravamen cordis. Oportet ergo sæpe in carne de carnis onere gemere, eo quod non vales spiritualibus studiis et divinæ contemplationi indesinenter inhærere.
2. Tunc expedit tibi ad humilia et exteriora opera confugere, et in bonis actibus te recreare, adventum meum et supernam visitationem firma confidentia expectare, exilium tuum et ariditatem mentis patienter sufferre, donec iterum a me visiteris et ab omnibus anxietatibus libereris. Nam faciam te laborum oblivisci et interna quiete perfrui. Expandam coram te prata scripturarum, ut, dilatato corde, currere incipias viam mandatorum meorum. Et dices: Non sunt condignæ passiones huius temporis ad futuram gloriam, quæ revelabitur in nobis.


CAPUT LII.
Quod homo non reputet se consolatione dignum, sed magis verberibus reum

1. Domine, non sum dignus consolatione tua nec aliqua spirituali visitatione; et ideo iuste mecum agis, quando me inopem et desolatum relinquis. Si enim ad instar maris lacrimas fundere possem, adhuc consolatione tua dignus non essem. Unde nihil dignus sum, quam flagellari et puniri, quia graviter et sæpe te offendi et in multis valde deliqui. Ergo vera pensata ratione, nec minima sum dignus consolatione. Sed tu, clemens et misericors Deus, qui non vis perire opera tua, ad ostendendum divitias bonitatis tuæ in vasa misericordiæ, etiam præter omne proprium meritum dignaris consolari servum tuum supra humanum modum. Tuæ enim consolationes non sunt sicut humanæ confabulationes.
2. Quid egi, Domine, ut mihi conferres aliquam cælestem consolationem? Ego nihil boni me egisse recolo, sed semper ad vitia pronum et ad emendationem pigrum fuisse. Verum est, et negare non possum. Si aliter dicerem, tu stares contra me, et non esset, qui defenderet. Quid merui pro peccatis meis, nisi infernum et ignem æternum? In veritate confiteor, quoniam dignus sum omni ludibrio et contemptu, nec decet me inter tuos devotos commemorari. Et licet hoc ægre audiam, tamen adversum me pro veritate peccata mea arguam, ut facilius misericordiam tuam merear impetrare.
3. Quid dicam reus et omni confusione plenus? Non habeo os loquendi, nisi hoc tantum verbum: Peccavi, Domine, peccavi; miserere mei, ignosce mihi. Sine me paululum, ut plangam dolorem meum, antequam vadam ad terram tenebrosam et opertam mortis caligine. Quid tam maxime a reo et misero peccatore requiris, nisi ut conteratur et humiliet se pro delictis suis? In vera contritione et cordis humiliatione nascitur spes veniæ, reconciliatur perturbata conscientia, reparatur gratia perdita, tuetur homo a futura ira, et occurrunt sibi mutuo, in osculo sancto, Deus et pænitens anima.
4. Humilis peccatorum contritio acceptabile tibi est, Domine, sacrificium, longe suavius odorans in conspectu tuo, quam turis incensum. Hæc est gratum etiam unguentum, quod sacris pedibus tuis infundi voluisti: quia cor contritum et humiliatum numquam despexisti. Ibi est locus refugii a facie iræ inimici. Ibi emendatur et abluitur, quidquid aliunde contractum est et inquinatum.


CAPUT LIII.
Quod gratia Dei non miscetur terrena sapientibus

1. Fili, pretiosa est gratia mea, non patitur se misceri extraneis rebus nec consolationibus terrenis. Abicere ergo oportet omnia impedimenta gratiæ, si optas eius infusionem suscipere. Pete secretum tibi, ama solus habitare tecum, nullius require confabulationem: sed magis ad Deum devotam effunde precem, ut compunctam teneas mentem et puram conscientiam. Totum mundum nihil æstima; Dei vacationem omnibus exterioribus antepone. Non enim poteris mihi vacare et in transitoriis pariter delectari. A notis et a caris oportet elongari, et ab omni temporali solacio mentem tenere privatam. Sic obsecrat beatus apostolus Petrus, ut tamquam advenas et peregrinos in hoc mundo se contineant Christi fideles.
2. O quanta fiducia erit morituro, quem nullius rei affectus detinet in mundo. Sed sic segregatum cor habere ab omnibus, æger necdum capit animus, nec animalis homo novit interni hominis libertatem. Attamen si vere velit esse spiritualis, oportet eum renuntiare tam remotis quam propinquis, et a nemine magis cavere, quam a se ipso. Si temet ipsum perfecte viceris, cetera facilius subiugabis. Perfecta victoria est de semet ipso triumphare. Qui enim semet ipsum subiectum tenet, ut sensualitas rationi, et ratio in cunctis obœdiat mihi, hic vere victor est sui et dominus mundi.
3. Si ad hunc apicem scandere gliscis, oportet viriliter incipere, et securim ad radicem ponere, ut evellas et destruas occultam inordinatam inclinationem ad te ipsum et ad omne privatum et materiale bonum. Ex hoc vitio, quod homo semet ipsum nimis inordinate diligit, pæne totum pendet, quidquid radicaliter vincendum est: quo devicto et subacto malo pax magna et tranquillitas erit continuo. Sed quia pauci sibi ipsis perfecte mori laborant, nec plene extra se tendunt, propterea in se implicati remanent, nec supra se in spiritu elevari possunt. Qui autem libere mecum ambulare desiderat, necesse est, ut omnes pravas et inordinatas affectiones suas mortificet, atque nulli creaturæ privato amore concupiscenter inhæreat.
De diversis motibus naturæ et gratiæ

1. Fili, diligenter adverte motus naturæ et gratiæ, quia valde contrarie et subtiliter moventur, et vix, nisi a spirituali et intimo illuminato homine, discernuntur. Omnes quidem bonum appetunt, et aliquid boni in suis dictis vel factis prætendunt; ideo sub specie boni multi falluntur.
2. Natura callida est, et multos trahit, illaqueat et decipit, et se semper pro fine habet: sed gratia simpliciter ambulat, ab omni specie mala declinat, fallacias non prætendit, et omnia pure propter Deum agit, in quo et finaliter requiescit.
3. Natura invite vult mori, nec premi, nec superari, nec subesse, nec sponte subiugari: gratia vero studet mortificationi propriæ, resistit sensualitati, quærit subici, appetit vinci, nec propria vult libertate fungi; sub disciplina amat teneri, nec alicui cupit dominari, sed sub Deo semper vivere, stare et esse; atque propter Deum omni humanæ creaturæ humiliter parata est inclinari.
4. Natura pro suo commodo laborat, et quid lucri ex alio sibi proveniat, attendit: gratia autem, non quid sibi utile et commodosum sit, sed quod multis proficiat, magis considerat.
5. Natura libenter honorem et reverentiam accipit: gratia vero omnem honorem et gloriam Deo fideliter attribuit.
6. Natura confusionem timet et contemptum: gratia autem gaudet pro nomine Iesu pati contumeliam.
7. Natura otium amat et quietem corporalem: gratia vero vacua esse non potest, sed libenter amplectitur laborem.
8. Natura quærit habere curiosa et pulchra, abhorret vilia et grossa: gratia vero simplicibus delectatur et humilibus, aspera non aspernatur, nec vetustis refugit indui pannis.
9. Natura respicit temporalia, gaudet ad lucra terrena, tristatur de damno, irritatur levi iniuriæ verbo: sed gratia attendit æterna, non inhæret temporalibus, nec in perditione rerum turbatur, neque verbis durioribus acerbatur; quia thesaurum suum et gaudium in cælo, ubi nil perit, constituit.
10. Natura cupida est, et libentius accipit quam donat, amat propria et privata: gratia autem pia est et communis, vitat singularia, contentatur paucis, beatius dare iudicat quam accipere.
11. Natura inclinat ad creaturas, ad carnem propriam, ad vanitates et discursus: sed gratia trahit ad Deum et ad virtutes, renuntiat creaturis, fugit mundum, odit carnis desideria, restringit evagationes, erubescit in publico apparere.
12. Natura libenter aliquod solacium habet externum, in quo delectetur ad sensum: sed gratia in solo Deo quærit consolari, et in summo bono super omnia visibilia delectari.
13. Natura totum agit propter lucrum et commodum proprium, nihil gratis facere potest, sed aut æquale, aut melius, aut laudem vel favorem pro benefactis consequi sperat, et multum ponderari sua gesta et dona concupiscit: gratia vero nil temporale quærit, nec aliud præmium, quam Deum solum pro mercede postulat; nec amplius de temporalibus necessariis desiderat, nisi quantum hæc sibi ad assecutionem æternorum valeant deservire.
14. Natura gaudet de amicis multis et propinquis, gloriatur de nobili loco et ortu generis; arridet potentibus, blanditur divitibus, applaudit sibi similibus: gratia autem et inimicos diligit, nec de amicorum turba extollitur, nec locum nec ortum natalium reputat, nisi virtus maior ibi fuerit; favet magis pauperi quam diviti, compatitur plus innocenti quam potenti; congaudet veraci, non fallaci; exhortatur semper bonos meliora charismata æmulari, et Filio Dei per virtutes assimilari.
15. Natura de defectu et molestia cito conqueritur: gratia constanter fert inopiam.
16. Natura omnia ad se reflectit, pro se certat et arguit: gratia autem ad Deum cuncta reducit, unde originaliter emanant; nihil boni sibi ascribit, nec arroganter præsumit; non contendit, nec suam sententiam aliis præfert; sed in omni sensu et intellectu æternæ sapientiæ ac divino examini se submittit. Natura appetit scire secreta et nova audire; vult exterius apparere et multa per sensus experiri; desiderat agnosci et agere, unde laus et admiratio procedit: sed gratia non curat nova nec curiosa percipere; quia totum hoc de vetustate corruptionis est ortum, cum nihil novum et durabile sit super terram. Docet itaque sensus restringere, vanam complacentiam et ostentationem devitare, laudanda et digne miranda humiliter abscondere, et de omni re et in omni scientia utilitatis fructum atque Dei laudem et honorem quærere. Non vult se nec sua prædicari, sed Deum in donis suis optat benedici, qui cuncta ex mera caritate largitur.
17. Hæc gratia supernaturale lumen et quoddam Dei speciale donum est, et proprie electorum signaculum et pignus salutis æternæ: quæ hominem de terrenis ad cælestia amanda sustollit, et de carnali spiritualem efficit. Quanto igitur natura amplius premitur et vincitur, tanto maior gratia infunditur; et cotidie novis visitationibus interior homo secundum imaginem Dei reformatur.


CAPUT LV.
De corruptione naturæ et efficacia gratiæ divinæ

1. Domine Deus meus, qui me creasti ad imaginem et similitudinem tuam, concede mihi hanc gratiam, quam ostendisti tam magnam et necessariam ad salutem, ut vincam pessimam naturam meam, trahentem ad peccata et in perditionem. Sentio enim in carne mea legem peccati, contradicentem legi mentis meæ, et captivum me ducentem ad obœdiendum sensualitati in multis; nec possum resistere passionibus eius, nisi assistat tua sanctissima gratia, cordi meo ardenter infusa.
2. Opus est gratia tua, et magna gratia, ut vincatur natura ad malum semper prona ab adulescentia sua. Nam per primum hominem Adam lapsa, et vitiata per peccatum, in omnes homines pœna huius maculæ descendit: ut ipsa natura, quæ bene et recta a te condita fuit, pro vitio iam et infirmitate corruptæ naturæ ponatur, eo quod motus eius, sibi relictus, ad malum et inferiora trahit. Nam modica vis, quæ remansit, est tamquam scintilla quædam latens in cinere. Hæc est ipsa ratio naturalis, circumfusa magna caligine, adhuc iudicium habens boni et mali, veri falsique distantiam; licet impotens sit adimplere omne, quod approbat, nec pleno iam lumine veritatis, nec sanitate affectionum suarum potiatur.
3. Hinc est, Deus meus, quod condelector legi tuæ secundum interiorem hominem, sciens, mandatum tuum fore bonum, iustum et sanctum, arguens etiam omne malum, et peccatum fugiendum. Carne autem servio legi peccati, dum magis sensualitati obœdio, quam rationi. Hinc est, quod velle bonum mihi adiacet, perficere autem non invenio. Hinc sæpe multa bona propono, sed quia gratia deest ad iuvandum infirmitatem meam, ex levi resistentia resilio et deficio. Hinc accidit, quod viam perfectionis agnosco, et qualiter agere debeam, clare satis video: sed propriæ corruptionis pondere pressus ad perfectiora non assurgo.
4. O quam maxime est mihi necessaria, Domine, tua gratia, ad inchoandum bonum, ad proficiendum et ad perficiendum! Nam sine ea nihil possum facere: omnia autem possum in te, confortante me gratia. O vere cælestis gratia, sine qua nulla sunt propria merita, nulla quoque dona naturæ ponderanda! Nihil artes, nihil divitiæ, nihil pulchritudo vel fortitudo, nihil ingenium vel eloquentia valent apud te, Domine, sine gratia. Nam dona naturæ bonis et malis sunt communia: electorum autem proprium donum est gratia sive dilectio; qua insigniti digni habentur vita æterna. Tantum eminet hæc gratia, ut nec donum prophetiæ, nec signorum operatio, nec quantalibet alta speculatio aliquid æstimentur sine ea. Sed neque fides, neque spes, neque aliæ virtutes tibi acceptæ sunt sine caritate et gratia.
5. O beatissima gratia, quæ pauperem spiritu virtutibus divitem facis, et divitem multis bonis humilem corde reddis! Veni, descende ad me, reple me mane consolatione tua, ne deficiat præ lassitudine et ariditate mentis anima mea. Obsecro, Domine, ut inveniam gratiam in oculis tuis: sufficit enim mihi gratia tua, ceteris non obtentis, quæ desiderat natura. Si fuero temptatus et vexatus tribulationibus multis, non timebo mala, dum mecum fuerit gratia tua. Ipsa fortitudo mea, ipsa consilium confert et auxilium. Cunctis hostibus potentior est, et sapientior universis sapientibus.
6. Magistra est veritatis, doctrix disciplinæ, lumen cordis, solamen pressuræ, fugatrix tristitiæ, ablatrix timoris, nutrix devotionis, productrix lacrimarum. Quid sum sine ea, nisi aridum lignum, et stipes inutilis ad eiciendum? Tua ergo me, Domine, gratia semper et præveniat et sequatur, ac bonis operibus iugiter præstet esse intentum, per Iesum Christum, filium tuum. Amen.


CAPUT LVI.
Quod nos ipsos abnegare et Christum imitari debemus per crucem

1. Fili, quantum a te vales exire, tantum in me poteris transire. Sicut nihil foris concupiscere internam pacem facit, sic se interius relinquere Deo coniungit. Volo te addiscere perfectam abnegationem tui in voluntate mea sine contradictione et querela. Sequere me: Ego sum via, veritas et vita. Sine via non itur, sine veritate non cognoscitur, sine vita non vivitur. Ego sum via, quam sequi debes; veritas, cui credere debes; vita, quam sperare debes. Ego sum via inviolabilis, veritas infallibilis, vita interminabilis. Ego sum via rectissima, veritas suprema, vita vera, vita beata, vita increata. Si manseris in via mea, cognosces veritatem, et veritas liberabit te, et apprehendes vitam æternam.
2. Si vis ad vitam ingredi, serva mandata. Si vis veritatem cognoscere, crede mihi. Si vis perfectus esse, vende omnia. Si vis esse discipulus meus, abnega temet ipsum. Si vis beatam vitam possidere, præsentem vitam contemne. Si vis exaltari in cælo, humilia te in mundo. Si vis regnare mecum, porta crucem mecum. Soli enim servi crucis inveniunt viam beatitudinis et veræ lucis.
3. Domine Iesu, quia arcta erat vita tua et mundo despecta, dona mihi te cum mundi despectu imitari. Non enim maior est servus domino suo, nec discipulus supra magistrum. Exerceatur servus tuus in vita tua, quia ibi est salus mea et sanctitas vera. Quidquid extra eam lego vel audio, non me recreat nec delectat plene.
4. Fili, quia hæc scis et legisti omnia, beatus eris, si feceris ea. Qui habet mandata mea et servat ea, ipse est, qui diligit me: et ego diligam eum et manifestabo ei me ipsum, et faciam eum consedere mecum in regno Patris mei.
5. Domine Iesu, sicut dixisti et promisisti, sic utique fiat, et mihi promereri contingat. Suscepi, suscepi de manu tua crucem; portabo, et portabo eam usque ad mortem, sicut imposuisti mihi. Vere vita boni monachi crux est, sed dux paradisi. Inceptum est, retro abire non licet, nec relinquere oportet.
6. Eia fratres, pergamus simul, Iesus erit nobiscum. Propter Iesum suscepimus hanc crucem: propter Iesum perseveremus in cruce. Erit adiutor noster, qui est dux noster et præcessor. En, rex noster ingreditur ante nos, qui pugnabit pro nobis. Sequamur viriliter, nemo metuat terrores; simus parati mori fortiter in bello, nec inferamus crimen gloriæ nostræ, ut fugiamus a cruce.


CAPUT LVII.
Quod homo non sit nimis dejectus, quando in aliquos labitur defectus

1. Fili, magis placent mihi patientia et humilitas in adversis, quam multa consolatio et devotio in prosperis. Ut quid te contristat parvum factum contra te dictum? Si amplius fuisset, commoveri non debuisses. Sed nunc permitte transire; non est primum, nec novum, nec ultimum erit, si diu vixeris. Satis virilis es, quamdiu nil obviat adversi. Bene etiam consulis, et alios nosti roborare verbis; sed cum ad ianuam tuam venit repentina tribulatio, deficis consilio et robore. Attende magnam fragilitatem tuam, quam sæpius experiris in modicis obiectis; tamen pro salute tua ista fiunt, cum hæc et similia contingunt.
2. Pone, ut melius nosti, ex corde; et si te tetigit, non tamen deiciat nec diu implicet. Ad minus sustine patienter, si non potes gaudenter. Etiam si minus libenter audis, et indignationem sentis: reprime te, nec patiaris aliquid inordinatum ex ore tuo exire, unde parvuli scandalizentur. Cito conquiescet commotio excitata, et dolor internus revertente dulcorabitur gratia. Adhuc vivo ego, dicit Dominus, iuvare te paratus, et solito amplius consolari, si confisus fueris mihi, et devote invocaveris.
3. Animæquior esto, et ad maiorem sustinentiam accingere. Non est totum frustratum, si te sæpius percipis tribulatum vel graviter temptatum. Homo es, et non Deus; caro es, non angelus. Quomodo tu posses semper in eodem statu virtutis permanere, quando hoc defuit angelo in cælo, et primo homini in paradiso? Ego sum, qui mærentes erigo sospitate, et suam cognoscentes infirmitatem ad meam proveho divinitatem.
4. Domine, benedictum sit verbum tuum, dulce super mel et favum ori meo. Quid facerem in tantis tribulationibus et angustiis meis, nisi me confortares tuis sanctis sermonibus? Dummodo tandem ad portum salutis perveniam, quid curæ est, quæ et quanta passus fuero? Da finem bonum, da felicem ex hoc mundo transitum. Memento mei, Deus meus, et dirige me recto itinere in regnum tuum. Amen.


CAPUT LVIII.
De altioribus rebus et occultis iudiciis Dei non scrutandis

1. Fili, caveas disputare de altis materiis et de occultis Dei iudiciis: cur iste sic relinquitur et ille ad tantam gratiam assumitur, cur etiam iste tantum affligitur et ille tam eximie exaltatur. Ista omnem humanam facultatem excedunt, nec ad investigandum iudicium divinum ulla ratio prævalet, vel disputatio. Quando ergo hæc tibi suggerit inimicus, vel etiam quidam curiosi inquirunt homines, responde illud prophetæ: Iustus es, Domine, et rectum iudicium tuum. Et illud: Iudicia Domini vera, iustificata in semet ipsa. Iudicia mea metuenda sunt, non discutienda; quia humano intellectui sunt incomprehensibilia.
2. Noli etiam inquirere nec disputare de meritis sanctorum, quis alio sit sanctior, aut quis maior fuerit in regno cælorum. Talia generant sæpe lites et contentiones inutiles, nutriunt quoque superbiam et vanam gloriam: unde oriuntur invidiæ et dissensiones, dum iste illum sanctum, et alius alium conatur superbe præferre. Talia autem velle scire et investigare nullum fructum afferunt, sed magis sanctis displicent: quia non sum Deus dissensionis, sed pacis; quæ pax magis in humilitate vera, quam in propria exaltatione consistit.
3. Quidam zelo dilectionis trahuntur ad hos vel ad illos ampliori affectu, sed humano potius quam divino. Ego sum, qui cunctos condidi sanctos; ego donavi gratiam, ego præstiti gloriam. Ego novi singulorum merita; ego præveni eos in benedictionibus dulcedinis meæ. Ego præscivi dilectos ante sæcula; ego eos elegi de mundo, non ipsi me præelegerunt. Ego vocavi per gratiam, attraxi per misericordiam; ego perduxi eos per temptationes varias. Ego infudi consolationes magnificas, ego dedi perseverantiam, ego coronavi eorum patientiam.
4. Ego primum et novissimum agnosco, ego omnes inæstimabili dilectione amplector. Ego laudandus sum in omnibus sanctis meis; ego super omnia benedicendus sum, et honorandus in singulis, quos sic gloriose magnificavi et prædestinavi, sine ullis præcedentibus propriis meritis. Qui ergo unum de minimis meis contempserit, nec magnum honorat; quia pusillum et magnum ego feci. Et qui derogat alicui sanctorum, derogat et mihi, et ceteris omnibus in regno cælorum. Omnes unum sunt per caritatis vinculum; idem sentiunt, idem volunt, et omnes in unum se diligunt.
5. Adhuc autem, quod multo altius est, plus me, quam se et sua merita diligunt. Nam supra se rapti, et extra propriam dilectionem tracti, toti in amorem mei pergunt, in quo fruitive quiescunt. Nihil est, quod eos avertere possit aut deprimere: quippe qui æterna veritate pleni, igne ardescunt inextinguibilis caritatis. Taceant igitur carnales et animales homines de sanctorum statu disserere, qui non norunt nisi privata gaudia diligere. Demunt et addunt pro sua inclinatione, non prout placet æternæ veritati.
6. In multis est ignorantia, eorum maxime, qui parum illuminati, raro aliquem perfecta dilectione spirituali diligere norunt. Multum adhuc naturali affectu et humana amicitia ad hos vel ad illos trahuntur, et sicut in inferioribus se habent, ita et de cælestibus imaginantur. Sed est distantia incomparabilis, quæ imperfecti cogitant, et quæ illuminati viri per revelationem supernam speculantur.
7. Cave ergo, fili, de istis curiose tractare, quæ tuam scientiam excedunt; sed hoc magis satage et intende, ut vel minimus in regno Dei queas inveniri. Et si quispiam sciret, quis alio sanctior esset, vel maior haberetur in regno cælorum: quid ei hæc notitia prodesset, nisi se ex hac cognitione coram me humiliaret, et in maiorem nominis mei laudem exurgeret? Multo acceptius Deo facit, qui de peccatorum suorum magnitudine et virtutum suarum parvitate cogitat, et quam longe a perfectione sanctorum distat, quam is, qui de eorum maioritate vel parvitate disputat. Melius est sanctos devotis precibus et lacrimis exorare, et eorum gloriosa suffragia humili mente implorare, quam eorum secreta vana inquisitione perscrutari.
8. Illi bene et optime contentantur, si homines scirent contentari et vaniloquia sua compescere. Non gloriantur de propriis meritis, quippe qui sibi nihil bonitatis ascribunt, sed totum mihi, quoniam ipsis cuncta ex infinita caritate mea donavi. Tanto amore divinitatis et gaudio supereffluenti replentur, ut nihil eis desit gloriæ, nihilque possit deesse felicitatis. Omnes sancti, quanto altiores in gloria, tanto humiliores in se ipsis, et mihi viciniores et dilectiores existunt. Ideoque habes scriptum, quia mittebant coronas suas ante Deum, et ceciderunt in facies suas coram Agno, et adoraverunt Viventem in sæcula sæculorum.
9. Multi quærunt, quis maior sit in regno Dei; qui ignorant, an cum minimis erunt digni computari. Magnum est vel minimum esse in cælo, ubi omnes magni sunt: quia omnes filii Dei vocabuntur et erunt. Minimus erit in mille, et peccator centum annorum morietur. Cum enim quærerent discipuli, quis maior esset in regno cælorum, tale audierunt responsum: Nisi conversi fueritis et efficiamini sicut parvuli, non intrabitis in regnum cælorum. Quicumque ergo humiliaverit se, sicut parvulus iste, hic maior est in regno cælorum.
10. Væ eis, qui cum parvulis humiliare se sponte dedignantur, quoniam humilis ianua regni cælestis eos non admittet intrare. Væ etiam divitibus, qui habent hic consolationes suas, quia pauperibus intrantibus in regnum Dei ipsi stabunt foris eiulantes. Gaudete humiles, et exultate pauperes, quia vestrum est regnum Dei, si tamen in veritate ambulatis.


CAPUT LIX.
Quod omnis spes et fiducia in solo Deo est figenda

1. Domine, quæ est fiducia mea, quam in hac vita habeo? aut quod maius solacium meum ex omnibus apparentibus sub cælo? Nonne tu, Domine Deus meus, cuius misericordiæ non est numerus? Ubi mihi bene fuit sine te? Aut quando male esse potuit præsente te? Malo pauper esse propter te, quam dives sine te. Eligo potius tecum in terra peregrinari, quam sine te cælum possidere. Ubi tu, ibi cælum; atque ibi mors et infernus, ubi tu non es. Tu mihi in desiderio es; et ideo post te gemere, clamare et exorare necesse est. In nullo denique possum plene confidere, qui in necessitatibus auxilietur opportunis, nisi in te solo Deo meo. Tu es spes mea, tu fiducia mea, tu consolator meus et fidelissimus in omnibus.
2. Omnes, quæ sua sunt, quærunt: tu salutem meam et profectum meum solummodo prætendis, et omnia in bonum mihi convertis. Etiam si variis temptationibus et adversitatibus exponas me, hoc totum ad utilitatem meam ordinas, qui mille modis dilectos tuos probare consuevisti. In qua probatione non minus diligi debes et laudari, quam si cælestibus consolationibus me repleres.
3. In te ergo, Domine Deus, pono totam spem meam et refugium, in te omnem tribulationem et angustiam meam constituo; quia totum infirmum et instabile invenio, quidquid extra te conspicio. Non enim proderunt multi amici, neque fortes auxiliarii adiuvare poterunt, nec prudentes consiliarii responsum utile dare, neque libri doctorum consolari, nec aliqua pretiosa substantia liberare, nec locus aliquis secretus et amœnus contutari: si tu ipse non assistas, iuves, confortes, consoleris, instruas et custodias.
4. Omnia namque, quæ ad pacem videntur esse et felicitatem habendam, te absente nihil sunt, nihilque felicitatis in veritate conferunt. Finis ergo omnium bonorum et altitudo vitæ et profunditas eloquiorum tu es; et in te super omnia sperare fortissimum solacium servorum tuorum. Ad te sunt oculi mei, in te confido, Deus meus, misericordiarum Pater. Benedic et sanctifica animam meam benedictione cælesti, ut fiat habitatio sancta tua, et sedes æternæ gloriæ tuæ, nihilque in templo tuæ dignitatis inveniatur, quod oculos tuæ maiestatis offendat. Secundum magnitudinem bonitatis tuæ et multitudinem miserationum tuarum respice in me, et exaudi orationem pauperis servi tui, longe exulantis in regione umbræ mortis. Protege et conserva animam servuli tui inter tot discrimina vitæ corruptibilis, ac comitante gratia tua, dirige per viam pacis ad patriam perpetuæ claritatis. Amen.




LIBER IV.
DE SACRAMENTO
Devota exhortatio ad sacram communionem
Vox Christi.
Venite ad me, omnes, qui laboratis et onerati estis, et ego reficiam vos, dicit Dominus. Panis, quem ego dabo, caro mea est, pro mundi vita. Accipite et comedite, hoc est Corpus meum, quod pro vobis tradetur: hoc facite in meam commemorationem. Qui manducat meam carnem, et bibit meum sanguinem, in me manet, et ego in illo. Verba, quæ ego locutus sum vobis, spiritus et vita sunt.


CAPUT I.
Cum quanta reverentia Christus sit suscipiendus
Vox discipuli.

1. Hæc sunt verba tua, Christe, veritas æterna, quamvis non uno tempore prolata, nec uno in loco conscripta. Quia ergo tua sunt et vera, gratanter mihi et fideliter cuncta sunt accipienda. Tua sunt, et tu ea protulisti; et mea quoque sunt, quia pro salute mea ea edidisti. Libenter suscipio ea ex ore tuo, ut arctius inserantur cordi meo. Excitant me verba tantæ pietatis, plena dulcedinis et dilectionis; sed terrent me delicta propria, et ad capienda tanta mysteria me reverberat impura conscientia. Provocat me dulcedo verborum tuorum, sed onerat multitudo vitiorum meorum.
2. Iubes, ut fiducialiter ad te accedam, si tecum velim habere partem, et ut immortalitatis accipiam alimoniam, si æternam cupiam obtinere vitam et gloriam. Venite, inquis, ad me omnes, qui laboratis et onerati estis, et ego reficiam vos. O dulce et amicabile verbum in aure peccatoris, quod tu, Domine Deus meus, egenum et pauperem invitas ad communionem tui sanctissimi Corporis. Sed quis ego sum, Domine, ut ad te præsumam accedere? Ecce, cæli cælorum te non capiunt; et tu dicis: Venite ad me omnes.
3. Quid sibi vult ista piissima dignatio et tam amicabilis invitatio? Quomodo ausus ero venire, qui nihil boni mihi conscius sum, unde possim præsumere? Quomodo te introducam in domum meam, qui sæpius offendi benignissimam faciem tuam? Reverentur angeli et archangeli, metuunt sancti et iusti; et tu dicis: Venite ad me omnes? Nisi tu, Domine, hoc diceres, quis verum esse crederet? Et nisi tu iuberes, quis accedere attemptaret?
4. Ecce, Noë, vir iustus, in arcæ fabrica centum annis loboravit, ut cum paucis salvaretur: et ego quomodo me potero una hora præparare, ut mundi fabricatorem cum reverentia sumam? Moyses, famulus tuus magnus et specialis amicus tuus, arcam ex lignis imputribilibus fecit, quam et mundissimo vestivit auro, ut tabulas legis in ea reponeret: et ego, putrida creatura, audebo te, conditorem legis ac vitæ datorem, tam facile suscipere? Salomon, sapientissimus regum Isræl, templum magnificum septem annis in laudem nominis tui ædificavit, et octo diebus festum dedicationis eius celebravit, mille hostias pacificas obtulit et arcam fœderis in clangore buccinæ et iubilo in locum sibi præparatum sollemniter collocavit. Et ego infelix et pauperrimus hominum, quomodo te in domum meam introducam, qui vix mediam expendere devote novi horam et utinam vel semel digne fere mediam?
5. O mi Deus, quantum illi ad placendum tibi agere studuerunt! Heu, quam pusillum est, quod ago: quam breve expleo tempus, cum me ad communicandum dispono! Raro totus collectus, rarissime ab omni distractione purgatus. Et certe in tua salutari deitatis præsentia nulla deberet occurrere indecens cogitatio, nulla etiam occupare creatura: quia non angelum, sed angelorum dominum suscepturus sum hospitio.
6. Est tamen valde magna distantia inter arcam fœderis cum suis reliquiis et mundissimum corpus tuum cum suis ineffabilibus virtutibus; inter legalia illa sacrificia futurorum præfigurativa et veram corporis tui hostiam, omnium antiquorum sacrificiorum completivam.
7. Quare igitur non magis ad tuam venerabilem inardesco præsentiam? Cur non maiori me præparo sollicitudine ad tua sancta sumenda, quando illi antiqui sancti patriarchæ et prophetæ, reges quoque et principes, cum universo populo, tantum devotionis demonstrarunt affectum erga cultum divinum?
8. Saltavit devotissimus rex David coram arca Dei totis viribus, recolens beneficia olim indulta patribus; fecit diversi generis organa, psalmos edidit, et cantari instituit cum lætitia, cecinit et ipse frequenter in cithara, Spiritus Sancti afflatus gratia; docuit populum Isræl toto corde Deum laudare, et ore consono diebus singulis benedicere et prædicare. Si tanta agebatur tunc devotio, ac divinæ laudis extitit recordatio coram arca testamenti: quanta nunc mihi et omni populo christiano habenda est reverentia et devotio in præsentia sacramenti, in sumptione excellentissimi corporis Christi?
9. Currunt multi ad diversa loca pro visitandis reliquiis sanctorum, et mirantur auditis gestis eorum, ampla ædificia templorum inspiciunt, et osculantur sericis et auro involuta sacra ossa ipsorum. Et ecce, tu præsens es hic apud me in altari, Deus meus, Sanctus sanctorum, Creator hominum, et Dominus angelorum. Sæpe in talibus videndis curiositas est hominum et novitas invisorum, et modicus reportatur emendationis fructus, maxime ubi est tam levis sine vera contritione discursus. Hic autem in sacramento altaris totus præsens es, Deus meus, homo Christus Iesus: ubi et copiosus percipitur æternæ salutis fructus, quotienscumque fueris digne ac devote susceptus. Ad istud vero non trahit levitas aliqua, nec curiositas aut sensualitas, sed firma fides, devota spes et sincera caritas.
10. O invisibilis conditor mundi Deus, quam mirabiliter agis nobiscum; quam suaviter et gratiose cum electis tuis disponis, quibus temet ipsum in sacramento sumendum proponis! Hoc namque omnem intellectum superat; hoc specialiter devotorum corda trahit et accendit affectum. Ipsi enim veri fideles tui, qui totam vitam suam ad emendationem disponunt, ex hoc dignissimo sacramento magnam devotionis gratiam et virtutis amorem frequenter recipiunt.
11. O admirabilis et abscondita gratia sacramenti, quam norunt tantum Christi fideles, infideles autem et peccatis servientes experiri non possunt! In hoc sacramento confertur spiritualis gratia, et reparatur in anima virtus amissa, et per peccatum deformata redit pulchritudo. Tanta est aliquando hæc gratia, ut ex plenitudine collatæ devotionis non tantum mens, sed et debile corpus vires sibi præstitas sentiat ampliores.
12. Dolendum tamen valde et miserandum super tepiditate et neglegentia nostra, quod non maiori affectione trahimur ad Christum sumendum, in quo tota spes salvandorum consistit et meritum. Ipse enim est sanctificatio nostra et redemptio; ipse consolatio viatorum, et sanctorum æterna fruitio. Dolendum itaque valde, quod multi tam parum hoc salutare mysterium advertunt, quod cælum lætificat, et mundum conservat universum. Heu cæcitas et duritia cordis humani, tam ineffabile donum non magis attendere, et ex cotidiano usu etiam ad inadvertentiam defluere!
13. Si enim hoc sanctissimum sacramentum in uno tantum celebraretur loco, et ab uno tantum consecraretur sacerdote in mundo: quanto putas desiderio ad illum locum et ad talem Dei sacerdotem homines afficerentur, ut divina mysteria celebrari viderent? Nunc autem multi facti sunt sacerdotes, et in multis locis offertur Christus, ut tanto maior appareat gratia et dilectio Dei ad hominem, quanto latius est sacra communio diffusa per orbem. Gratias tibi, Iesu bone, pastor æterne, qui nos pauperes et exules dignatus es pretioso corpore et sanguine tuo reficere, et ad hæc mysteria percipienda etiam proprii oris tui alloquio invitare, dicendo: Venite ad me omnes, qui laboratis et onerati estis, et ego reficiam vos.


CAPUT II.
Quod magna bonitas et caritas Dei in sacramento homini exhibetur
Vox discipuli.

1. Super bonitate tua et magna misericordia tua, Domine, confisus, accedo æger ad Salvatorem, esuriens et sitiens ad fontem vitæ, egenus ad Regem cæli, servus ad Dominum, creatura ad Creatorem, desolatus ad meum pium Consolatorem. Sed unde hoc mihi, ut venias ad me? Quis ego sum, ut præstes mihi te ipsum? Quomodo audet peccator coram te apparere? et tu quomodo dignaris ad peccatorem venire? Tu nosti servum tuum, et scis, quia nil boni in se habet, unde hoc illi præstes. Confiteor igitur vilitatem meam, agnosco tuam bonitatem, laudo pietatem, et gratias ago propter nimiam caritatem. Propter temet ipsum enim hoc facis, non propter mea merita; ut bonitas tua mihi magis innotescat, caritas amplior ingeratur, et humilitas perfectius commendetur. Quia ergo tibi hoc placet, et tu sic fieri iussisti, placet et mihi dignatio tua; et utinam iniquitas mea non obsistat!
2. O dulcissime et benignissime Iesu, quanta tibi reverentia et gratiarum actio cum perpetua laude pro susceptione sacri corporis tui debetur, cuius dignitatem nullus hominum explicare potens invenitur! Sed quid cogitabo in hac communione, in accessu ad Dominum meum, quem debite venerari nequeo, et tamen devote suscipere desidero? Quid cogitabo melius et salubrius, nisi me ipsum totaliter humiliando coram te, et tuam infinitam bonitatem exaltando supra me? Laudo te, Deus meus, et exalto in æternum. Despicio me, et subicio tibi in profundum vilitatis meæ.
3. Ecce, tu Sanctus sanctorum, et ego sordes peccatorum. Ecce, tu inclinas te ad me, qui non sum dignus ad te respicere. Ecce, tu venis ad me, tu vis esse mecum, tu invitas ad convivium tuum. Tu mihi dare vis cælestem cibum et panem angelorum ad manducandum: non alium sane quam te ipsum, panem vivum, qui de cælo descendisti, et das vitam mundo.
4. Ecce, unde dilectio procedit, qualis dignatio illucescit! quam magnæ gratiarum actiones et laudes tibi pro his debentur! O quam salutare et utile consilium tuum, cum istud instituisti! quam suave et iucundum convivium, cum te ipsum in cibum donasti! O quam admirabilis operatio tua, Domine! quam potens virtus tua! quam infallibilis veritas tua! Dixisti enim, et facta sunt omnia; et hoc factum est, quod ipse iussisti.
5. Mira res, et fide digna, ac humanum vincens intellectum: quod tu, Domine Deus meus, verus Deus et homo, sub modica specie panis et vini integer contineris, et sine consumptione a sumente manducaris. Tu, Domine universorum, qui nullius habes indigentiam, voluisti per sacramentum tuum habitare in nobis; conserva cor meum et corpus immaculatum; ut læta et pura conscientia sæpius tua valeam celebrare mysteria, et ad meam perpetuam accipere salutem, quæ ad tuum præcipue honorem et memoriale perenne sanxisti et instituisti.
6. Lætare, anima mea, et gratias age Deo pro tam nobili munere et solacio singulari in hac lacrimarum valle tibi relicto. Nam quotiens hoc mysterium recolis et Christi Corpus accipis, totiens tuæ redemptionis opus agis et particeps omnium meritorum Christi efficeris. Caritas etenim Christi numquam minuitur, et magnitudo propitiationis eius numquam exhauritur. Ideo nova semper mentis renovatione ad hoc disponere te debes, et magnum salutis mysterium attenta consideratione pensare. Ita magnum, novum et iucundum tibi videri debet, cum celebras aut missam audis, ac si eodem die Christus primum in uterum virginis descendens homo factus esset, aut in cruce pendens pro salute hominum pateretur et moreretur.


CAPUT III.
Quod utile sit sæpe communicare
Vox discipuli.

1. Ecce, ego ad te venio, Domine, ut bene mihi sit ex munere tuo, et lætificer in convivio sancto tuo, quod parasti in dulcedine tua pauperi, Deus. Ecce, in te est totum, quod desiderare possum et debeo; tu salus mea et redemptio, spes et fortitudo, decus et gloria. Lætifica ergo hodie animam servi tui, quoniam ad te, Domine Iesu, animam meam levavi. Desidero te nunc devote ac reverenter suscipere; cupio te in domum meam inducere, quatenus cum Zachæo merear a te benedici ac inter filios Abrahæ computari. Anima mea corpus tuum concupiscit, cor meum tecum uniri desiderat.
2. Trade te mihi, et sufficit. Nam præter te nulla consolatio valet. Sine te esse nequeo, et sine visitatione tua vivere non valeo. Ideoque oportet me frequenter ad te accedere, et in remedium salutis meæ recipere; ne forte deficiam in via, si fuero cælesti fraudatus alimonia. Sic enim tu, misericordissime Iesu, prædicans populis et varios curans languores, aliquando dixisti: Nolo eos ieiunos dimittere in domum suam, ne deficiant in via. Age igitur hoc mecum modo, qui te pro fidelium consolatione in Sacramento reliquisti. Tu es enim suavis refectio animæ: et qui te digne manducaverit, particeps et heres erit æternæ gloriæ. Necessarium quidem mihi est, qui tam sæpe labor et pecco, tam cito torpesco et deficio, ut per frequentes orationes et confessiones ad sacram corporis tui perceptionem me renovem, mundem et accendam; ne forte diutius abstinendo a sancto proposito defluam.
3. Proni enim sunt sensus hominis ad malum ab adulescentia sua; et nisi succurrat divina medicina, labitur homo mox ad peiora. Retrahit ergo sancta communio a malo et confortat in bono. Si enim modo tam sæpe neglegens sum et tepidus, quando communico aut celebro; quid fieret, si medelam non sumerem, et tam grande iuvamen non quærerem? Et licet omni die non sim aptus, nec ad celebrandum bene dispositus; dabo tamen operam, congruis temporibus divina percipere mysteria, ac tantæ gratiæ participem me præbere. Nam hæc est una principalis fidelis animæ consolatio, quamdiu peregrinatur a te in mortali corpore, ut sæpius memor Dei sui dilectum suum devota suscipiat mente.
4. O mira circa nos tuæ pietatis dignatio, quod tu, Domine Deus, Creator et Vivificator omnium spirituum, ad pauperculam dignaris venire animam, et cum tota divinitate tua ac humanitate eius impinguare esuriem! O felix mens, et beata anima, quæ te, Dominum Deum suum, meretur devote suscipere, et in tua susceptione spiritali gaudio repleri! O quam magnum suscipit Dominum, quam dilectum inducit hospitem, quam iucundum recipit socium, quam fidelem acceptat amicum, quam speciosum et nobilem amplectitur sponsum præ omnibus dilectis, et super omnia desiderabilia amandum! Sileant a facie tua, dulcissime dilecte meus, cælum et terra et omnis ornatus eorum; quoniam quidquid laudis habent ac decoris, ex dignatione tuæ est largitatis, nec ad decorem tui pervenient nominis, cuius sapientiæ non est numerus.


CAPUT IV.
Quod multa bona præstantur devote communicantibus
Vox discipuli.

1. Domine Deus meus, præveni servum tuum in benedictionibus dulcedinis tuæ, ut ad tuum magnificum sacramentum digne ac devote merear accedere. Excita cor meum in te, et a gravi torpore exue me. Visita me in salutari tuo ad gustandam in spiritu suavitatem tuam, quæ in hoc sacramento tamquam in fonte plenarie latet. Illumina quoque oculos meos ad intuendum tantum mysterium, et ad credendum illud indubitata fide me robora. Est enim operatio tua, non humana potentia; tua sacra institutio, non hominis adinventio. Non enim ad hæc capienda et intellegenda aliquis idoneus per se reperitur, quæ angelicam etiam subtilitatem transcendunt. Quid ergo ego peccator indignus, terra et cinis, de tam alto sacro secreto potero investigare et capere?
2. Domine, in simplicitate cordis mei, in bona firma fide et in tua iussione ad te cum spe ac reverentia accedo; et vere credo, quia tu præsens es hic in sacramento, Deus et homo. Vis ergo, ut te suscipiam, et me ipsum tibi in caritate uniam. Unde tuam precor clementiam, et specialem ad hoc imploro mihi donari gratiam, ut totus in te liquefiam et amore pereffluam, atque de nulla aliena consolatione amplius me intromittam. Est enim hoc altissimum et dignissimum sacramentum, salus animæ et corporis, medicina omnis spiritalis languoris, in quo vitia mea curantur, passiones frenantur, temptationes vincuntur aut minuuntur, gratia maior infunditur, virtus incepta augetur, firmatur fides, spes roboratur, et caritas ignescit ac dilatatur.
3. Multa namque bona largitus es, et adhuc sæpius largiris in sacramento dilectis tuis devote communicantibus, Deus meus, susceptor animæ meæ, reparator infirmitatis humanæ, et totius dator consolationis internæ. Nam multam ipsis consolationem adversus variam tribulationem infundis, et de imo deiectionis propriæ ad spem tuæ protectionis erigis, atque nova quadam gratia eos intus recreas et illustras, ut, qui anxii primum, et sine affectione se ante communionem senserant, postea refecti cibo potuque cælesti in melius se mutatos inveniant. Quod idcirco cum electis tuis dispensanter agis, ut veraciter agnoscant et patenter experiantur, quantum infirmitatis ex se ipsis habeant, et quid bonitatis ac gratiæ ex te consequantur. Quia ex semet ipsis frigidi, duri et indevoti, ex te autem ferventes, alacres et devoti esse merentur. Quis enim ad fontem suavitatis humiliter accedens, non modicum suavitatis inde reportat? Aut quis iuxta copiosum ignem stans, non parum caloris inde percipit? Et tu fons es semper plenus et superabundans, ignis iugiter ardens et numquam deficiens.
4. Unde si mihi non licet haurire de plenitudine fontis, nec usque ad satietatem potare, apponam tamen os meum ad foramen cælestis fistulæ, ut saltem modicam inde guttulam capiam ad refocillandam sitim meam, et non penitus exarescam. Et si necdum totus cælestis et tam ignitus, ut Cherubim et Seraphim, esse possum, conabor tamen devotioni insistere et cor meum præparare, ut vel modicam divini incendii flammam ex humili sumptione vivifici sacramenti conquiram. Quidquid autem mihi deest, Iesu bone, salvator sanctissime, tu pro me supple benigne ac gratiose, qui omnes ad te dignatus es vocare, dicens: Venite ad me omnes, qui laboratis et onerati estis, et ego reficiam vos.
5. Ego quidem laboro in sudore vultus mei, dolore cordis torqueor, peccatis oneror, temptationibus inquietor, multis malis passionibus implicor et premor; et non est qui adiuvet, non est qui liberet et salvum faciat, nisi tu, Domine Deus, salvator meus: cui committo me et omnia mea, ut me custodias et perducas in vitam æternam. Suscipe me in laudem et gloriam nominis tui, qui corpus tuum et sanguinem in cibum et potum mihi parasti. Præsta, Domine Deus salutaris meus, ut cum frequentatione mysterii tui crescat meæ devotionis affectus.


CAPUT V.
De dignitate sacramenti et statu sacerdotali
Vox discipuli.

1. Si haberes angelicam puritatem et sancti Iohannis baptistæ sanctitatem, non esses dignus hoc sacramentum accipere nec tractare. Non enim hoc mentis debetur hominum, quod homo consecret et tractet Christi sacramentum, et sumat in cibum panem angelorum. Grande ministerium et magna dignitas sacerdotum, quibus datum est, quod angelis non est concessum. Soli namque sacerdotes rite in ecclesia ordinati potestatem habent celebrandi et corpus Christi consecrandi. Sacerdos quidem minister est Dei, utens verbo Dei, per iussionem et institutionem Dei; Deus autem ibi principalis est auctor et invisibilis operator, cui subest omne, quod voluerit, et paret omne, quod iusserit.
2. Plus ergo credere debes Deo omnipotenti in hoc excellentissimo sacramento, quam proprio sensui aut alicui signo visibili. Ideoque cum timore et reverentia ad hoc opus est accedendum. Attende tibi et vide, cuius ministerium tibi traditum est per impositionem manus episcopi. Ecce, sacerdos factus es, et ad celebrandum consecratus; vide nunc, ut fideliter et devote in suo tempore Deo sacrificium offeras, et te ipsum irreprehensibilem exhibeas. Non alleviasti onus tuum, sed arctiori iam alligatus es vinculo disciplinæ, et ad maiorem teneris perfectionem sanctitatis. Sacerdos omnibus virtutibus debet esse ornatus, et aliis bonæ vitæ exemplum præbere. Eius conversatio non cum popularibus et communibus hominum viis, sed cum angelis in cælo aut cum perfectis viris in terra.
3. Sacerdos sacris vestibus indutus Christi vices gerit, ut Deum pro se et pro omni populo suppliciter et humiliter roget. Habet ante se et retro Dominicæ crucis signum, ad memorandam iugiter Christi passionem. Ante se crucem in casula portat, ut Christi vestigia diligenter inspiciat, et sequi ferventer studeat. Post se cruce signatus est, ut adversa quælibet ab aliis illata clementer pro Deo toleret. Ante se crucem gerit, ut propria peccata lugeat; post se, ut aliorum etiam commissa per compassionem defleat, et se medium inter Deum et peccatorem constitutum esse sciat, nec ab oratione et oblatione sancta torpescat, donec gratiam et misericordiam impetrare mereatur. Quando sacerdos celebrat, Deum honorat, angelos lætificat, ecclesiam ædificat, vivos adiuvat, defunctis requiem præstat, et sese omnium bonorum participem efficit.


CAPUT VI.
Interrogatio de exercitio ante communionem
Vox discipuli.

1. Cum tuam dignitatem, Domine, et meam vilitatem penso, valde contremisco et in me ipso confundor. Si enim non accedo, vitam fugio; et si indigne me ingessero, offensam incurro. Quid ergo faciam, Deus meus, auxiliator meus et consiliator in necessitatibus?
2. Tu doce me viam rectam; propone breve aliquod exercitium sacræ communioni congruum. Utile est enim scire, qualiter scilicet devote ac reverenter tibi præparare debeam cor meum, ad recipiendum salubriter tuum sacramentum, seu etiam celebrandum tam magnum et divinum sacrificium.


CAPUT VII.
De discussione propriæ conscientiæ et emendationis proposito
Vox dilecti.

1. Super omnia cum summa humilitate cordis et supplici reverentia, cum plena fide et pia intentione honoris Dei, ad hoc sacramentum celebrandum, tractandum et sumendum oportet Dei accedere sacerdotem. Diligenter examina conscientiam tuam et pro posse tuo vera contritione et humili confessione eam munda et clarifica: ita ut nil grave habeas aut scias, quod te remordeat et liberum accessum impediat. Habeas displicentiam omnium peccatorum tuorum in generali, et pro cotidianis excessibus magis in speciali doleas et gemas. Et si tempus patitur, Deo in secreto cordis cunctas confitere passionum tuarum miserias.
2. Ingemisce et dole, quod adhuc ita carnalis sis et mundanus; tam immortificatus a passionibus, tam plenus concupiscentiarum motibus; tam incustoditus in sensibus exterioribus, tam sæpe multis vanis phantasiis implicatus; tam multum inclinatus ad exteriora, tam neglegens ad interiora; tam levis ad risum et dissolutionem, tam durus ad fletum et compunctionem; tam promptus ad laxiora et carnis commoda, tam segnis ad rigorem et fervorem; tam curiosus ad nova audienda et pulchra intuenda, tam remissus ad humilia et abiecta amplectenda; tam cupidus ad multa habenda, tam parcus ad dandum, tam tenax ad retinendum; tam inconsideratus in loquendo, tam incontinens in tacendo; tam incompositus in moribus, tam importunus in actibus; tam effusus super cibum, tam surdus ad Dei verbum; tam velox ad quietem, tam tardus ad laborem; tam vigilans ad fabulas, tam somnolentus ad vigilias sacras; tam festinus ad finem, tam vagus ad attendendum; tam neglegens in horis persolvendis, tam tepidus in celebrando, tam aridus in communicando; tam cito distractus, tam raro plene tibi collectus; tam subito commotus ad iram, tam facilis ad alterius displicentiam; tam pronus ad iudicandum, tam rigidus ad arguendum; tam lætus ad prospera, tam debilis in adversis; tam sæpe multa bona proponens, et modicum ad effectum perducens.
3. His et aliis defectibus tuis, cum dolore et magna displicentia propriæ infirmitatis confessis ac deploratis, firmum statue propositum semper emendandi vitam tuam et in melius proficiendi. Deinde cum plena resignatione et integra voluntate offer te ipsum in honorem nominis mei in ara cordis tui holocaustum perpetuum, corpus tuum scilicet et animam mihi fideliter committendo: quatenus et sic digne merearis ad offerendum Deo sacrificium accedere, et sacramentum corporis mei salubriter suscipere.
4. Non est enim oblatio dignior et satisfactio maior pro peccatis diluendis, quam se ipsum pure et integre cum oblatione corporis Christi in missa et in communione Deo offerre. Si fecerit homo, quod in se est, et vere pænituerit, quotienscumque pro venia et gratia ad me accesserit: Vivo ego, dicit Dominus, qui nolo mortem peccatoris, sed magis, ut convertatur et vivat: quoniam peccatorum suorum non recordabor amplius, sed cuncta sibi indulta erunt.


CAPUT VIII.
De oblatione Christi in cruce et propria resignatione
Vox dilecti.

1. Sicut ego me ipsum, expansis in cruce manibus et nudo corpore, pro peccatis tuis Deo Patri sponte obtuli, ita ut nihil in me remaneret, quin totum in sacrificium divinæ placationis transiret: ita debes et tu temet ipsum mihi voluntarie in oblationem puram et sanctam, cotidie in missa, cum omnibus viribus et affectibus tuis, quanto intimius vales, offerre. Quid magis a te requiro, quam ut te studeas mihi ex integro resignare? Quidquid præter te ipsum das, nihil curo: quia non quæro datum tuum, sed te.
2. Sicut non sufficeret tibi omnibus habitis, præter me: ita nec mihi placere poterit, quidquid dederis, te non oblato. Offer te mihi, et da te totum pro Deo, et erit accepta oblatio. Ecce, ego me totum obtuli Patri pro te; dedi etiam totum corpus meum et sanguinem in cibum, ut totus tuus essem, et tu meus permaneres. Si autem in te ipso steteris, nec sponte te ad voluntatem meam obtuleris, non est plena oblatio, nec integra erit inter nos unio. Igitur omnia opera tua præcedere debet spontanea tui ipsius in manus Dei oblatio, si libertatem consequi vis et gratiam. Ideo enim tam pauci illuminati et liberi intus efficiuntur, quia se ipsos ex toto abnegare nesciunt. Est firma sententia mea: Nisi quis renuntiaverit omnibus, non potest meus esse discipulus. Tu ergo si optas meus esse discipulus, offer te ipsum mihi cum omnibus affectibus tuis.


CAPUT IX.
Quod nos et omnia nostra Deo debemus offerre et pro omnibus orare
Vox discipuli.

1. Domine, omnia tua sunt, quæ in cælo sunt, et quæ in terra. Desidero me ipsum tibi in spontaneam oblationem offerre, et tuus perpetue permanere. Domine, in simplicitate cordis mei offero me ipsum tibi hodie in servum sempiternum, in obsequium et in sacrificium laudis perpetuæ. Suscipe me cum hac sancta oblatione tui pretiosi corporis; quam tibi hodie in præsentia angelorum, invisibiliter assistentium, offero, ut sit pro me et pro cuncto populo tuo in salutem.
2. Domine, offero tibi omnia peccata et delicta mea, quæ commisi coram te et sanctis angelis tuis a die, quo primum peccare potui, usque ad horam hanc, super placabili altari tuo: ut tu omnia pariter incendas et comburas igne caritatis tuæ, et deleas universas maculas peccatorum meorum, et conscientiam meam ab omni delicto emundes, et restituas mihi gratiam tuam, quam peccando amisi, omnia mihi plene indulgendo et in osculum pacis me misericorditer assumendo.
3. Quid possum agere pro peccatis meis, nisi humiliter ea confitendo et lamentando, et tuam propitiationem incessanter deprecando? Deprecor te, exaudi me propitius, ubi asto coram te, Deus meus. Omnia peccata mea mihi maxime displicent, nolo ea umquam amplius perpetrare; sed pro eis doleo et dolebo, quamdiu vixero, paratus pænitentiam agere et pro posse satisfacere. Dimitte mihi, Deus, dimitte mihi peccata mea, propter nomen sanctum tuum, salva animam meam, quam pretioso sanguine tuo redemisti. Ecce committo me misericordiæ tuæ, resigno me manibus tuis. Age mecum secundum bonitatem tuam, non secundum meam malitiam et iniquitatem.
4. Offero etiam tibi omnia bona mea, quamvis valde pauca et imperfecta; ut tu ea emendes et sanctifices; ut ea grata habeas et accepta tibi facias, et semper ad meliora trahas; nec non ad beatum ac laudabilem finem me pigrum et inutilem homuncionem perducas.
5. Offero quoque tibi omnia pia desideria devotorum, necessitates parentum, amicorum, fratrum, sororum, omniumque carorum meorum, et eorum, qui mihi vel aliis propter amorem tuum benefecerunt, et qui orationes et missas pro se suisque omnibus dici a me desideraverunt et petierunt; sive in carne adhuc vivant, sive iam sæculo defuncti sint; ut omnes sibi auxilium gratiæ tuæ, opem consolationis, protectionem a periculis, liberationem a pœnis advenire sentiant, et ut ab omnibus malis erepti, gratias tibi magnificas læti persolvant.
6. Offero etiam tibi preces et hostias placationis, pro illis specialiter, qui me in aliquo læserunt, contristaverunt, aut vituperaverunt, vel aliquod damnum vel gravamen intulerunt; pro his quoque omnibus, quos aliquando contristavi, conturbavi, gravavi et scandalizavi, verbis, factis, scienter vel ignoranter, ut nobis omnibus pariter indulgeas peccata nostra et mutuas offensiones. Aufer, Domine, a cordibus nostris omnem suspicionem, indignationem, iram et disceptationem, et quidquid potest caritatem lædere et fraternam dilectionem minuere. Miserere, miserere, Domine, misericordiam tuam poscentibus, da gratiam indigentibus; et fac nos tales existere, ut simus digni gratia tua perfrui, et ad vitam proficiamus æternam. Amen.


CAPUT X.
Quod sacra communio de facili non est relinquenda
Vox dilecti.

1. Frequenter recurrendum est ad fontem gratiæ et divinæ misericordiæ, ad fontem bonitatis et totius puritatis: quatenus a passionibus tuis et vitiis curari valeas, et contra universas temptationes et fallacias diaboli fortior atque vigilantior effici merearis. Inimicus sciens fructum et remedium maximum in sacra communione positum, omni modo et occasione nititur fideles et devotos, quantum prævalet, retrahere et impedire.
2. Cum enim quidam sacræ communioni se aptare disponunt, peiores satanæ immissiones patiuntur. Ipse nequam spiritus, ut in Iob scribitur, venit inter filios Dei, ut solita illos nequitia sua perturbet, aut timidos nimium reddat et perplexos: quatenus affectum eorum minuat vel fidem impugnando auferat: si forte aut omnino communionem relinquant, aut cum tepore accedant. Sed non est quidquam curandum de versutiis et phantasiis illius, quantumlibet turpibus et horridis; sed cuncta phantasmata in caput eius sunt retorquenda. Contemnendus est miser et deridendus; nec propter insultus eius et commotiones, quas suscitat, sacra est omittenda communio.
3. Sæpe etiam impedit nimia sollicitudo pro devotione habenda, et anxietas quædam de confessione facienda. Age secundum consilium sapientum, et depone anxietatem et scrupulum; quia gratiam Dei impedit et devotionem mentis destruit. Propter aliquam parvam turbationem vel gravitatem sacram ne dimittas communionem; sed vade citius confiteri, et omnes offensiones aliis libenter indulge. Si vero tu aliquem offendisti, veniam humiliter precare, et Deus libenter indulgebit tibi.
4. Quid prodest diu tardare confessionem, aut sacram differre communionem? Expurga te cumprimis, expue velociter venenum, festina accipere remedium, et senties melius, quam si diu distuleris. Si hodie propter istud dimittis, cras forsitan aliud maius eveniet: et sic diu posses a communione impediri et magis ineptus fieri. Quanto citius vales, a præsenti gravitate et inertia te excutias: quia nihil importat diu anxiari, diu cum turbatione transire, et ob cotidiana obstacula se a divinis sequestrare. Immo plurimum nocet diu communionem protelare; nam et gravem torporem consuevit inducere. Pro dolor! quidam tepidi et dissoluti moras confitendi libenter accipiunt, et communionem sacram idcirco differri cupiunt, ne ad maiorem sui custodiam se dare teneantur.
5. Heu, quam modicam caritatem et debilem devotionem habent, qui sacram communionem tam faciliter postponunt! Quam felix ille et Deo acceptus habetur, qui sic vivit, et in tali puritate conscientiam suam custodit, ut etiam omni die communicare paratus, et bene affectatus esset, si ei liceret, et sine nota agere posset. Si quis interdum abstinet humilitatis gratia, aut legitima impediente causa, laudandus est de reverentia. Si autem torpor obrepserit, se ipsum excitare debet, et facere, quod in se est; et Dominus aderit desiderio suo pro bona voluntate, quam specialiter respicit.
6. Cum vero legitime præpeditus est, habebit semper bonam voluntatem, et piam intentionem communicandi, et sic non carebit fructu sacramenti. Potest enim quilibet devotus, omni die et omni hora, ad spiritualem Christi communionem salubriter et sine prohibitione accedere; et tamen certis diebus et statuto tempore corpus sui Redemptoris cum affectuosa reverentia sacramentaliter debet suscipere, et magis laudem Dei et honorem prætendere, quam suam consolationem quærere. Nam totiens mystice communicat, et invisibiliter reficitur, quotiens incarnationis Christi mysterium passionemque devote recolit, et in amore eius accenditur.
7. Qui aliter se non præparat, nisi instante festo, vel consuetudine compellente, sæpius imparatus erit. Beatus, qui se Domino in holocaustum offert, quotiens celebrat aut communicat. Non sis in celebrando nimis prolixus aut festinus, sed serva bonum communem modum, cum quibus vivis. Non debes aliis generare molestiam et tædium, sed communem servare viam secundum maiorum institutionem, et potius aliorum servire utilitati, quam propriæ devotioni vel affectui.


CAPUT XI.
Quod corpus Christi et sacra scriptura maxime sint animæ fideli necessaria
Vox discipuli.

1. O dulcissime Domine Iesu, quanta est dulcedo devotæ animæ, tecum epulantis in convivio tuo, ubi ei non alius cibus manducandus proponitur, nisi tu, unicus dilectus eius, super omnia desideria cordis eius desiderabilis. Et mihi quidem dulce foret in præsentia tua ex intimo affectu lacrimas fundere, et cum pia Magdalena pedes tuos lacrimis irrigare. Sed ubi est hæc devotio? ubi lacrimarum sanctarum copiosa effusio? Certe in conspectu tuo et sanctorum angelorum tuorum totum cor meum ardere deberet et ex gaudio flere. Habeo enim te in sacramento vere præsentem, quamvis aliena specie occultatum.
2. Nam in propria et divina claritate te conspicere, oculi mei ferre non possent, sed neque totus mundus in fulgore gloriæ maiestatis tuæ subsisteret. In hoc ergo imbecillitati meæ consulis, quod te sub sacramento abscondis. Habeo vere et adoro, quem angeli adorant in cælo; sed ego adhuc interim in fide, illi autem in specie et sine velamine. Me oportet contentum esse in lumine veræ fidei, et in ea ambulare, donec aspiret dies æternæ claritatis, et umbræ figurarum inclinentur. Cum autem venerit, quod perfectum est, cessabit usus sacramentorum; quia beati in gloria cælesti non egent medicamine sacramentali; gaudent enim sine fine in præsentia Dei, facie ad faciem gloriam eius speculantes; et de claritate in claritatem abyssalis Deitatis transformati, gustant verbum Dei caro factum, sicut fuit ab initio et manet in æternum.
3. Memor horum mirabilium, grave mihi fit tædium etiam quodlibet spirituale solacium: quia quamdiu Dominum meum aperte in sua gloria non video, pro nihilo duco omne, quod in mundo conspicio et audio. Testis es tu mihi, Deus, quod nulla res me potest consolari, nulla creatura quietare, nisi tu, Deus meus, quem desidero æternaliter contemplari. Sed non est hoc possibile, durante me in hac mortalitate. Ideo oportet, ut me ponam ad magnam patientiam, et me ipsum in omni desiderio tibi submittam. Nam et sancti tui, Domine, qui tecum iam in regno cælorum exultant, in fide et patientia magna, dum viverent, adventum gloriæ tuæ expectabant. Quod illi crediderunt, ego credo; quod illi speraverunt, ego spero; quo illi pervenerunt, per gratiam tuam me venturum confido. Ambulabo interim in fide, exemplis confortatus sanctorum. Habebo etiam libros sanctos pro solacio et vitæ speculo; atque super hæc omnia sanctissimum corpus tuum pro singulari remedio et refugio.
4. Duo namque mihi necessaria permaxime sentio in hac vita, sine quibus mihi importabilis foret ista miserabilis vita. In carcere corporis huius detentus, duobus me egere fateor, cibo scilicet et lumine. Dedisti itaque mihi infirmo sacrum corpus tuum ad refectionem mentis et corporis, et posuisti lucernam pedibus meis verbum tuum. Sine his duobus bene vivere non possem; nam verbum Dei lux animæ meæ, et sacramentum tuum panis vitæ. Hæc possunt etiam dici mensæ duæ, hinc et inde in gazophylacio sanctæ Ecclesiæ positæ. Una mensa est sacri altaris, habens panem sanctum, id est, corpus Christi pretiosum; altera est divinæ legis, continens doctrinam sanctam, erudiens fidem rectam, et firmiter usque ad interiora velaminis, ubi sunt sancta sanctorum, perducens.
5. Gratias tibi, Domine Iesu, lux lucis æternæ, pro doctrinæ sacræ mensa, quam nobis per servos tuos prophetas et apostolos aliosque doctores ministrasti. Gratias tibi, Creator ac Redemptor hominum, qui ad declarandam toti mundo caritatem tuam, cenam parasti magnam, in qua non agnum typicum, sed tuum sanctissimum corpus et sanguinem proposuisti manducandum: lætificans omnes fideles convivio sacro, et calice inebrians salutari, in quo sunt omnes deliciæ paradisi, et epulantur nobiscum angeli sancti, sed suavitate feliciori.
6. O quam magnum et honorabile est officium sacerdotum, quibus datum est Dominum maiestatis verbis sacris consecrare, labiis benedicere, manibus tenere, ore proprio sumere, et ceteris ministrare! O quam mundæ debent esse manus illæ, quam purum os, quam sanctum corpus, quam immaculatum cor erit sacerdotis, ad quem totiens ingreditur auctor puritatis! Ex ore sacerdotis nihil nisi sanctum, nihil nisi honestum et utile procedere debet verbum, qui tam sæpe Christi accipit sacramentum.
7. Oculi eius simplices et pudici, qui Christi corpus solent intueri. Manus puræ et in cælum elevatæ, quæ Creatorem cæli et terræ solent contrectare. Sacerdotibus specialiter in lege dicitur: Sancti estote, quoniam ego sanctus sum, Dominus Deus vester.
8. Adiuvet nos gratia tua, omnipotens Deus, ut, qui officium sacerdotale suscepimus, digne ac devote tibi in omni puritate et conscientia bona famulari valeamus. Et si non possumus in tanta innocentia vitæ conversari, ut debemus: concede nobis tamen digne flere mala, quæ gessimus, et in spiritu humilitatis ac bonæ voluntatis proposito tibi ferventius de cetero deservire.


CAPUT XII.
Quod magna diligentia se debeat communicaturus Christo præparare
Vox dilecti.

1. Ego sum puritatis amator et dator omnis sanctitatis. Ego cor purum quæro, et ibi est locus requietionis meæ. Para mihi cenaculum grande stratum, et faciam apud te pascha cum discipulis meis. Si vis, ut veniam ad te et apud te maneam: expurga vetus fermentum, et munda cordis tui habitaculum. Exclude totum sæculum et omnem vitiorum tumultum; sede tamquam passer solitarius in tecto, et cogita excessus tuos in amaritudine animæ tuæ. Omnis namque amans suo dilecto amatori optimum et pulcherrimum præparat locum, quia in hoc cognoscitur affectus suscipientis dilectum.
2. Scito tamen te non posse satisfacere huic præparationi ex merito tuæ actionis, etiam si per integrum annum te præparares et nihil aliud in mente haberes. Sed ex sola pietate et gratia mea permitteris ad mensam meam accedere: ac si mendicus ad prandium vocaretur divitis, et ille nihil aliud habeat ad retribuendum beneficiis eius, nisi se humiliando et ei regratiando. Fac, quod in te est, et diligenter facito, non ex consuetudine, non ex necessitate, sed cum timore et reverentia et affectu accipe corpus dilecti Domini Dei tui, dignantis ad te venire. Ego sum, qui vocavi, ego iussi fieri; ego supplebo, quod tibi deest: veni, et suscipe me.
3. Cum gratiam devotionis tribuo, gratias age Deo tuo, non quia dignus es, sed quia tui misertus sum. Si non habes, sed magis aridum te sentis, insiste orationi, ingemisce et pulsa; nec desistas, donec merearis micam aut guttam gratiæ salutaris accipere. Tu mei indiges, non ego tui indigeo. Nec tu me sanctificare venis, sed ego te sanctificare et meliorare venio. Tu venis, ut ex me sanctificeris et mihi uniaris; ut novam gratiam recipias et de novo ad emendationem accendaris. Noli neglegere hanc gratiam, sed præpara cum omni diligentia cor tuum, et introduc ad te dilectum tuum.
4. Oportet autem, ut non solum te præpares ad devotionem ante communionem, sed ut etiam te sollicite conserves in ea post sacramenti perceptionem. Nec minor custodia post exigitur, quam devota præparatio prius. Nam bona postmodum custodia optima iterum est præparatio ad maiorem gratiam consequendam. Ex eo quippe valde indispositus quis redditur, si statim fuerit nimis effusus ad exteriora solacia. Cave a multiloquio, mane in secreto, et fruere Deo tuo: ipsum enim habes, quem totus mundus tibi auferre non potest. Ego sum, cui te totum dare debes: ita ut iam ultra non in te, sed in me absque omni sollicitudine vivas.


CAPUT XIII.
Quod toto corde anima devota Christi unionem in sacramento affectare debet
Vox discipuli.

1. Quis mihi det, Domine, ut inveniam te solum, et aperiam tibi totum cor meum, et fruar te, sicut desiderat anima mea, et iam me nemo despiciat, nec ulla creatura me moveat vel respiciat, sed tu solus mihi loquaris, et ego tibi, sicut solet dilectus ad dilectum loqui, et amicus cum amico convivari? Hoc oro, hoc desidero, ut tibi totus uniar, et cor meum ab omnibus creatis rebus abstraham, magisque per sacram communionem ac frequentem celebrationem cælestia et æterna sapere discam. Ah, Domine Deus, quando ero tecum totus unitus et absorptus, meique totaliter oblitus? Tu in me, et ego in te: et sic nos pariter in unum manere concede.
2. Vere tu es dilectus meus, electus ex millibus, in quo complacuit animæ meæ habitare omnibus diebus vitæ suæ. Vere tu pacificus meus, in quo pax summa et requies vera, extra quem labor et dolor et infinita miseria. Vere tu es Deus absconditus; et consilium tuum non est cum impiis, sed cum humilibus et simplicibus sermo tuus. O quam suavis est, Domine, spiritus tuus, qui ut dulcedinem tuam in filios demonstrares, pane suavissimo, de cælo descendente, illos reficere dignaris! Vere non est alia natio tam grandis, quæ habeat deos appropinquantes sibi, sicut tu, Deus noster, ades universis fidelibus tuis; quibus ob cotidianum solacium et cor erigendum in cælum te tribuis ad edendum et fruendum.
3. Quæ est enim alia gens tam inclita, sicut plebs christiana? Aut quæ creatura sub cælo tam dilecta, ut anima devota, ad quam ingreditur Deus, ut pascat eam carne sua gloriosa? O ineffabilis gratia! o admirabilis dignatio! o amor immensus, homini singulariter impensus! Sed quid retribuam Domino pro gratia ista, pro caritate tam eximia? Non est aliud, quod gratius donare queam, quam ut cor meum Deo meo totaliter tribuam et intime coniungam. Tunc exultabunt omnia interiora mea, cum perfecte fuerit unita Deo anima mea. Tunc dicet mihi: Si tu vis esse mecum, ego volo esse tecum. Et ego respondebo illi: Dignare, Domine, manere mecum, ego volo libenter esse tecum. Hoc est totum desiderium meum, ut cor meum tibi sit unitum.


CAPUT XIV.
De quorundam devotorum ardenti desiderio ad Corpus Christi
Vox discipuli.

1. O quam magna multitudo dulcedinis tuæ, Domine, quam abscondisti timentibus te! Quando recordor devotorum aliquorum ad sacramentum tuum, Domine, cum maxima devotione et affectu accedentium, tunc sæpius in me ipso confundor et erubesco, quod ad altare tuum et sacræ communionis mensam tam tepide et frigide accedo, quod ita aridus et sine affectione cordis maneo, quod non sum totaliter accensus coram te, Deo meo, nec ita vehementer attractus et affectus, sicut multi devoti fuerunt, qui præ nimio desiderio communionis et sensibili cordis amore a fletu se non potuerunt continere: sed ore cordis et corporis pariter ad te, Deum, fontem vivum, medullitus inhiabant, suam esuriem non valentes aliter temperare nec satiare, nisi corpus tuum cum omni iucunditate et spirituali aviditate accepissent.
2. O vera ardens fides eorum, probabile existens argumentum sacræ præsentiæ tuæ! Isti enim veraciter cognoscunt Dominum suum in fractione panis, quorum cor tam valide ardet in eis de Iesu ambulante cum eis. Longe est a me sæpe talis affectus et devotio, tam vehemens amor et ardor. Esto mihi propitius, Iesu bone, dulcis et benigne, et concede pauperi mendico tuo, vel interdum modicum de cordiali affectu amoris tui in sacra communione sentire, ut fides mea magis convalescat, spes in bonitate tua proficiat, et caritas semel perfecte accensa et cæleste manna experta numquam deficiat.
3. Potens est autem misericordia tua, etiam gratiam desideratam mihi præstare, et in spiritu ardoris, cum dies beneplaciti tui venerit, me clementissime visitare. Etenim licet tanto desiderio tam specialium devotorum tuorum non ardeo, tamen de gratia tua, illius magni inflammati desiderii desiderium habeo, orans et desiderans, omnium talium fervidorum amatorum tuorum participem me fieri, ac eorum sancto consortio annumerari.


CAPUT XV.
Quod gratia devotionis humilitate et sui ipsius abnegatione acquiritur
Vox dilecti.

1. Oportet te devotionis gratiam instanter quærere, desideranter petere, patienter et fiducialiter expectare, gratanter recipere, humiliter conservare, studiose cum ea operari, ac Deo terminum et modum supernæ visitationis, donec veniat, committere. Humiliare præcipue te debes, cum parum aut nihil devotionis interius sentis; sed non nimium deici, nec inordinate contristari. Dat sæpe Deus in uno brevi momento, quod longo negavit tempore; dat quandoque in fine, quod in principio orationis distulit dare.
2. Si semper cito gratia daretur, et pro voto adesset, non esset infirmo homini bene portabile. Propterea in bona spe et humili patientia expectanda est devotionis gratia. Tibi tamen et peccatis tuis imputa, cum non datur vel etiam occulte tollitur. Modicum quandoque est, quod gratiam impedit et abscondit; si tamen modicum, et non potius grande dici debeat, quod tantum bonum prohibet. Et si hoc ipsum modicum vel grande amoveris, et perfecte viceris, erit, quod petisti.
3. Statim namque, ut te Deo ex toto corde tradideris, nec hoc vel illud pro tuo libitu seu velle quæsieris, sed integre te in ipso posueris, unitum te invenies et pacatum; quia nil ita bene sapiet et placebit, sicut beneplacitum divinæ voluntatis. Quisquis ergo intentionem suam simplici corde sursum ad Deum levaverit, seque ab omni inordinato amore seu displicentia cuiuslibet rei creatæ evacuaverit, aptissimus gratiæ percipiendæ ac dignus devotionis munere erit. Dat enim Dominus ibi benedictionem suam, ubi vasa vacua invenerit. Et quanto perfectius infimis quis renuntiat, et magis sibi ipsi per contemptum sui moritur, tanto gratia celerius venit, copiosius intrat et altius liberum cor elevat.
4. Tunc videbit, et affluet, et mirabitur, et dilatabitur cor eius in ipso, quia manus Domini cum eo, et ipse se posuit totaliter in manu eius usque in sæculum. Ecce, sic benedicetur homo, qui quærit Deum in toto corde suo, nec in vanum accipit animam suam. Hic in accipiendo sacram eucharistiam magnam promeretur divinæ unionis gratiam, quia non respicit ad propriam devotionem et consolationem, sed super omnem devotionem et consolationem ad Dei gloriam et honorem.


CAPUT XVI.
Quod necessitates nostras Christo aperire et eius gratiam postulare debemus
Vox discipuli.

1. O dulcissime atque amantissime Domine, quem nunc devote desidero suscipere, tu scis infirmitatem meam et necessitatem, quam patior; in quantis malis et vitiis iaceo; quam sæpe sum gravatus, temptatus, turbatus et inquinatus. Pro remedio ad te venio, pro consolatione et sublevamine te deprecor. Ad omnia scientem loquor, cui manifesta sunt omnia interiora mea, et qui solus potes me perfecte consolari et adiuvare. Tu scis, quibus bonis indigeo præ omnibus, et quam pauper sum in virtutibus.
2. Ecce, sto ante te pauper et nudus, gratiam postulans et misericordiam implorans. Refice esurientem mendicum tuum, accende frigiditatem meam igne amoris tui, illumina cæcitatem meam claritate præsentiæ tuæ. Verte mihi omnia terrena in amaritudinem, omnia gravia et contraria in patientiam, omnia infima et creata in contemptum et oblivionem. Erige cor meum ad te in cælum, et ne dimittas me vagari super terram. Tu solus mihi ex hoc iam dulcescas usque in sæculum; quia tu solus cibus et potus meus, amor meus et gaudium meum, dulcedo mea et totum bonum meum.
3. Utinam me totaliter ex tua præsentia accendas, comburas et in te transmutes, ut unus tecum efficiar spiritus, per gratiam internæ unionis et liquefactionem ardentis amoris! Ne patiaris, me ieiunum et aridum a te recedere, sed operare mecum misericorditer, sicut sæpius operatus es cum sanctis tuis mirabiliter. Quid mirum, si totus ex te ignescerem, et in me ipso deficerem; cum tu sis ignis semper ardens et numquam deficiens, amor corda purificans et intellectum illuminans?


CAPUT XVII.
De ardenti amore et vehementi affectu suscipiendi Christum
Vox discipuli.

1. Cum summa devotione et ardenti amore, cum toto cordis affectu et fervore, desidero te, Domine, suscipere, quemadmodum multi sancti et devotæ personæ in communicando te desideraverunt, qui tibi maxime in sanctitate vitæ placuerunt et in ardentissima devotione fuerunt. O Deus meus, amor æternus, totum bonum meum, felicitas interminabilis, cupio te suscipere cum vehementissimo desiderio et dignissima reverentia, quam aliquis sanctorum umquam habuit et sentire potuit.
2. Et licet indignus sum omnia illa sentimenta devotionis habere, tamen offero tibi totum cordis mei affectum, ac si omnia illa gratissima inflammata desideria solus haberem. Sed et quæcumque potest pia mens concipere et desiderare, hæc omnia tibi cum summa veneratione et intimo favore præbeo et offero. Nihil opto mihi reservare, sed me et omnia mea tibi sponte et libentissime immolare. Domine Deus meus, Creator meus et Redemptor meus, cum tali affectu, reverentia, laude et honore, cum tali gratitudine, dignitate et amore, cum tali fide, spe et puritate te affecto hodie suscipere, sicut te suscepit et desideravit sanctissima mater tua, gloriosa virgo Maria, quando angelo evangelizanti sibi incarnationis mysterium, humiliter ac devote respondit: Ecce, ancilla Domini, fiat mihi secundum verbum tuum.
3. Et sicut beatus præcursor tuus, excellentissimus sanctorum, Johannes baptista, in præsentia tua lætabundus exultavit in gaudio Spiritus Sancti, dum adhuc maternis clauderetur visceribus, et postmodum cernens inter homines Iesum ambulantem, valde se humilians, devoto cum affectu dicebat: Amicus autem sponsi, qui stat et audit eum, gaudio gaudet propter vocem sponsi: sic et ego magnis et sacris desideriis opto inflammari, et tibi ex toto corde me ipsum præsentare. Unde et omnium devotorum cordium iubilationes, ardentes affectus, mentales excessus, ac supernaturales illuminationes, et cælicas visiones tibi offero et exhibeo, cum omnibus virtutibus et laudibus, ab omni creatura in cælo et in terra celebratis et celebrandis, pro me et omnibus mihi in oratione commendatis, quatenus ab omnibus digne lauderis, et in perpetuum glorificeris.
4. Accipe vota mea, Domine Deus meus, et desideria infinitæ laudationis ac immensæ benedictionis, quæ tibi secundum multitudinem ineffabilis magnitudinis tuæ iure debentur. Hæc tibi reddo et reddere desidero per singulos dies et momenta temporum, atque ad reddendum mecum tibi gratias et laudes omnes cælestes spiritus et cunctos fideles tuos precibus et affectibus invito et exoro.
5. Laudent te universi populi, tribus et linguæ, et sanctum ac mellifluum nomen tuum cum summa iubilatione et ardenti devotione magnificent. Et quicumque reverenter ac devote altissimum sacramentum tuum celebrant, et plena fide recipiunt, gratiam et misericordiam apud te invenire mereantur, et pro me peccatore suppliciter exorent. Cumque optata devotione ac fruibili unione potiti fuerint, et bene consolati ac mirifice refecti, de sacra mensa cælesti abscesserint, mei pauperis recordari dignentur.


CAPUT XVIII.
Quod homo non sit curiosus scrutator sacramenti, sed humilis imitator Christi, subdendo sensum suum sacræ fidei
Vox dilecti.

1. Cavendum est tibi a curiosa et inutili perscrutatione huius profundissimi sacramenti, si non vis in dubitationis profundum submergi. Qui scrutator est maiestatis, opprimetur a gloria. Plus valet Deus operari, quam homo intellegere potest. Tolerabilis est pia et humilis inquisitio veritatis, parata semper doceri, et per sanas patrum sententias studens ambulare.
2. Beata simplicitas, quæ difficiles quæstionum relinquit vias, et plana ac firma pergit semita mandatorum Dei. Multi devotionem perdiderunt, dum altiora scrutari voluerunt. Fides a te exigitur et sincera vita, non altitudo intellectus, neque profunditas mysteriorum Dei. Si non intellegis, nec capis, quæ infra te sunt, quomodo comprehendes, quæ supra te sunt? Subdere Deo, et humilia sensum tuum fidei, et dabitur tibi scientiæ lumen, prout tibi fuerit utile ac necessarium.
3. Quidam graviter temptantur de fide et sacramento; sed non est hoc ipsis imputandum, sed potius inimico. Noli curare, noli disputare cum cogitationibus tuis, nec ad immissas a diabolo dubitationes responde; sed crede verbis Dei, crede sanctis eius et prophetis, et fugiet a te nequam inimicus. Sæpe multum prodest, quod talia sustinet Dei servus. Nam infideles et peccatores non temptat, quos secure iam possidet; fideles autem devotos variis modis temptat et vexat.
4. Perge ergo cum simplici et indubitata fide, et cum supplici reverentia ad sacramentum accede. Et quidquid intellegere non vales, Deo omnipotenti secure committe. Non fallit te Deus; fallitur, qui sibi ipsi nimium credit. Graditur Deus cum simplicibus, revelat se humilibus, dat intellectum parvulis, aperit sensum puris mentibus, et abscondit gratiam curiosis et superbis. Ratio humana debilis est et falli potest; fides autem vera falli non potest.
5. Omnis ratio et naturalis investigatio fidem sequi debet, non præcedere nec infringere. Nam fides et amor ibi maxime præcellunt, et occultis modis in hoc sanctissimo et superexcellentissimo sacramento operantur. Deus æternus et immensus, infinitæque potentiæ, facit magna et inscrutabilia in cælo et in terra, nec est investigatio mirabilium operum eius. Si talia essent opera Dei, ut facile ab humana ratione caperentur, non essent mirabilia nec ineffabilia dicenda.

\end{document}
