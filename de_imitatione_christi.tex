\documentclass[twoside]{article}

% Geometry of the page:
\usepackage{geometry}
\geometry{paperwidth=14.85cm, paperheight=21cm, inner=1.3cm, outer=1.3cm, tmargin=1cm, bmargin=1.5cm, includehead}

% Main font:
\usepackage{luatextra}
\setmainfont{Arno Pro}

% Hyphenations etc.:
\usepackage[latin]{babel}

% Formatting:
\usepackage[explicit]{titlesec}
\titleformat{\section}{}{}{0cm}
{
    \vspace{1cm}
    \fontsize{14}{16}\selectfont
    {\centering\textbf{#1}}
}
\titlespacing{\section}{0cm}{0cm}{-.8cm}

% Entêtes et pieds de pages :
\usepackage{fancyhdr}
\pagestyle{fancy}
\fancyhead{}
\fancyhead[CE]{\fontsize{11}{12}\selectfont\textsc{De Imitatione Christi}}
\fancyhead[CO]{\fontsize{11}{12}\selectfont\textsc{Liber \leftmark, caput \rightmark}}
\fancyfoot[CE,CO]{\fontsize{11}{12}\selectfont\thepage}
\renewcommand{\headrulewidth}{0.3pt}
\renewcommand{\footrulewidth}{0pt}
\renewcommand{\headrule}{\vbox to 6pt{\hbox to\headwidth{\hrulefill}\vss}}
\setlength{\parindent}{0cm}
\setlength{\headsep}{0.2cm} % Distance entre le header et le corps du texte.
\setlength{\footskip}{0.6cm} % Distance entre le footer et le corps du texte.

% Style de paragraphe TitreA :
\newenvironment{TitreA}[1]{
    \setlength{\parindent}{0cm}
    \setlength{\leftskip}{0cm}
    \fontsize{24}{36}\selectfont
    \setlength{\parskip}{-0.3cm}
    \begin{center}
        \MakeUppercase{#1}
    \end{center}
}

% Style de paragraphe TitreB :
\newenvironment{TitreB}[1]{
    \setlength{\parindent}{0cm}
    \setlength{\leftskip}{0cm}
    \setlength{\parskip}{-0.3cm}
    \fontsize{16}{18}\selectfont
    \begin{center}
        \textsc{#1}
    \end{center}
}

% Style de paragraphe TitreC :
\newenvironment{TitreC}[1]{
    \setlength{\parindent}{0cm}
    \setlength{\leftskip}{0cm}
    \setlength{\parskip}{0cm}
    \fontsize{14}{18}\selectfont
    \begin{center}
        \textsc{#1}
    \end{center}
}

% % Style de paragraphe Normal :
% \newenvironment{Normal}[1]{
%     \setlength{\parindent}{0cm}
%     \setlength{\leftskip}{0cm}
%     \setlength{\parskip}{0cm}
%     \fontsize{14}{16}\selectfont#1\par
%     \vspace{0.1cm}
% }

\newcommand{\Verse}[1]{\textbf{#1}}

\begin{document}
\selectlanguage{latin}
\thispagestyle{empty}
\TitreA{DE IMITATIONE CHRISTI}
\TitreA{LIBRI QUATUOR}
\vspace{1cm}
\markboth{I}{}
\TitreB{LIBER I}
\TitreB{ADMONITIONES AD SPIRITUALEM VITAM UTILES}
\vspace{1cm}
\markright{I}
%LIBER I
\TitreC{CAPUT I}
\TitreC{De Imitatióne Christi et contémptu ómnium vanitátum mundi}
\vspace{1cm}
\Normal{\Verse{1.} Qui séquitur me, non ámbulat in ténebris, dicit Dóminus. Hæc sunt verba Christi, quibus admonémur, quátenus vitam eius et mores imitémur, si velímus veráciter illuminári et ab omni cæcitáte cordis liberári. Summum ígitur stúdium nostrum sit, in vita Iesu Christi meditári.}
\Normal{\Verse{2.} Doctrína Christi omnes doctrínas Sanctórum præcéllit~; et qui spíritum habéret, abscónditum ibi manna inveníret. Sed contíngit, quod multi ex frequénti audítu Evangélii parvum desidérium séntiunt, quia spíritum Christi non habent. Qui autem vult plene et sápide Christi verba intellégere, opórtet, ut totam vitam suam illi stúdeat conformáre.}
\Normal{\Verse{3.} Quid prodest tibi, alta de Trinitáte disputáre, si cáreas humilitáte, unde displíceas Trinitáti~? Vere, alta verba non fáciunt sanctum et iustum, sed virtuósa vita éfficit Deo carum. Opto magis sentíre compunctiónem, quam scire eius definitiónem. Si scires totam Bíbliam extérius, et ómnium philosophórum dicta~: quid totum prodésset sine caritáte Dei et grátia~? Vánitas vanitátum, et ómnia vánitas, præter amáre Deum et illi soli servíre. Ista est summa sapiéntia, per contémptum mundi téndere ad regna cæléstia.}
\Normal{\Verse{4.} Vánitas ígitur est, divítias peritúras quǽrere et in illis speráre. Vánitas quoque est, honóres ambíre, et in altum statum se extóllere. Vánitas est, carnis desidéria sequi, et illud desideráre, unde póstmodum gráviter opórtet puníri. Vánitas est, longam vitam optáre, et de bona vita parum curáre. Vánitas est, præséntem vitam solum atténdere, et quæ futúra sunt, non prævidére. Vánitas est, dilígere, quod cum omni celeritáte transit, et illic non festináre, ubi sempitérnum gáudium manet.}
\Normal{\Verse{5.} Meménto illíus frequénter provérbii~: Quia non satiátur óculus visu, nec auris implétur audítu. Stude ergo cor tuum ab amóre visibílium abstráhere et ad invisibília te transférre. Nam sequéntes suam sensualitátem máculant consciéntiam et perdunt Dei grátiam.}
\markright{II}
%LIBER I
\TitreC{CAPUT II}
\TitreC{De húmili sentíre sui ipsíus}
\Normal{\Verse{1.} Omnis homo naturáliter scire desíderat~; sed sciéntia sine timóre Dei quid impórtat~? Mélior est profécto húmilis rústicus, qui Deo servit, quam supérbus philósophus, qui se neglécto cursum cæli consíderat. Qui bene seípsum cognóscit, sibi ipsi viléscit, nec láudibus delectátur humánis. Si scirem ómnia, quæ in mundo sunt, et non essem in caritáte~: quid me iuváret coram Deo, qui me iudicatúrus est ex facto?}
\Normal{\Verse{2.} Quiésce a nímio sciéndi desidério, quia magna ibi invenítur distráctio et decéptio. Sciéntes libénter volunt vidéri et dici sapiéntes. Multa sunt, quæ scire parum vel nihil ánimæ prosunt. Et valde insípiens est, qui alíquibus inténdit quam his, quæ salúti suæ desérviunt. Multa verba non sátiant ánimam~; sed bona vita refrígerat mentem, et pura consciéntia magnam ad Deum præstat confidéntiam.}
\Normal{\Verse{3.} Quanto plus et mélius scis, tanto grávius inde iudicáberis, nisi sánctius víxeris. Noli ergo extólli de ulla arte vel sciéntia, sed pótius time de data tibi notítia. Si tibi vidétur, quod multa scis et satis bene intéllegis~: scito tamen, quia sunt multo plura, quæ nescis. Noli altum sápere, sed ignorántiam tuam magis fatére. Quid te vis alícui præférre, cum plures doctióres te inveniántur, et magis in lege períti~? Si vis utíliter áliquid scire et díscere~: ama nescíri et pro níhilo reputári.}
\Normal{\Verse{4.} Hæc est altíssima et utilíssima léctio~: sui ipsíus vera cognítio et despéctio. De se ipso nihil tenére, et de áliis semper bene et alte sentíre, magna sapiéntia est et perféctio. Si vidéres álium apérte peccáre, vel áliqua grávia perpetráre, non debéres te tamen meliórem æstimáre~; quia nescis, quam diu possis in bono stare. Omnes frágiles sumus, sed tu néminem fragiliórem te ipso tenébis.}
\markright{III}
%LIBER I
\TitreC{CAPUT III}
\TitreC{De doctrína veritátis}
\Normal{\Verse{1.} Felix, quem véritas per se docet, non per figúras et voces transeúntes, sed sícuti se habet. Nostra opínio et noster sensus sæpe nos fallit et módicum videt. Quid prodest magna cavillátio de occúltis et obscúris rebus, de quibus nec arguémur in iudício, quia ignorávimus~? Grandis insipiéntia, quod negléctis utílibus et necessáriis ultro inténdimus curiósis et damnósis. Oculos habéntes non vidémus.}
\Normal{\Verse{2.} Et quid curæ nobis de genéribus et speciébus~? Cui ætérnum Verbum lóquitur, a multis opiniónibus expedítur. Ex uno verbo ómnia, et unum loquúntur ómnia~: et hoc est princípium, quod et lóquitur nobis. Nemo sine illo intéllegit aut recte iúdicat. Cui ómnia unum sunt, et ómnia ad unum trahit, et ómnia in uno videt, potest stábilis corde esse et in Deo pacíficus permanére. O véritas Deus, fac me unum tecum in caritáte perpétua. Tædet me sæpe, multa légere et audíre~: in te est totum, quod volo et desídero. Táceant omnes doctóres, síleant univérsæ creatúræ in conspéctu tuo~: tu mihi lóquere solus.}
\Normal{\Verse{3.} Quanto áliquis magis sibi unítus et intérius simplificátus fúerit, tanto plura et altióra sine labóre intéllegit, quia désuper lumen intellegéntiæ áccipit. Purus, simplex et stábilis spíritus in multis opéribus non dissipátur, quia ómnia ad Dei honórem operátur, et in se otiósus ab omni própria exquisitióne esse nítitur. Quis te magis ímpedit et moléstat quam tua immortificáta afféctio cordis~? Bonus et devótus homo ópera sua prius intus dispónit, quæ foris ágere debet. Nec illa trahunt eum ad desidéria vitiósæ inclinatiónis, sed ipse infléctit ea ad arbítrium rectæ ratiónis. Quis habet fórtius certámen, quam qui nítitur víncere seípsum~? Et hoc debéret esse negótium nostrum~: víncere vidélicet seípsum, et cotídie se ipso fortiórem fíeri atque in mélius áliquid profícere.}
\Normal{\Verse{4.} Omnis perféctio in hac vita quandam imperfectiónem sibi habet annexam~; et omnis speculátio nostra quadam calígine non caret. Húmilis tui cognítio cértior via est ad Deum, quam profúnda sciéntiæ inquisítio. Non est culpánda sciéntia aut quǽlibet simplex rei notítia, quæ bona est in se consideráta et a Deo ordináta~; sed præferénda est semper bona consciéntia et virtuósa vita. Quia vero plures magis student scire, quam bene vívere~: ídeo sæpe errant et pene nullum vel módicum fructum ferunt.}
\Normal{\Verse{5.} O si tantam adhibérent diligéntiam ad exstirpánda vítia et virtútes inseréndas, sícuti ad movéndas quæstiónes, non fíerent tanta mala et scándala in pópulo, nec tanta dissolútio in cœnóbiis. Certe adveniénte die iudícii non quærétur a nobis, quid légimus, sed quid fécimus~; nec quam bene díximus, sed quam religióse víximus. Dic mihi~: Ubi sunt modo omnes illi dómini et magístri, quos bene novísti, dum adhuc víverent, et stúdiis florérent~? Iam eórum præbéndas álii póssident~; et néscio, utrum de iis recógitant. In vita sua áliquid esse videbántur, et modo de illis tacétur.}
\Normal{\Verse{6.} O quam cito transit glória mundi~! Utinam vita eórum sciéntiæ ipsórum concordásset~! Tunc bene studuíssent et legíssent. Quam multi péreunt per vanam sciéntiam in sǽculo, qui parum curant de Dei servítio~! Et quia magis éligunt magni esse quam húmiles, ídeo evanéscunt in cogitatiónibus suis. Vere magnus est, qui magnam habet caritátem. Vere magnus est, qui in se parvus est et pro níhilo omne culmen honóris ducit. Vere prudens est, qui ómnia terréna arbitrátur ut stércora, ut Christum lucrifáciat. Et vere bene doctus est, qui Dei voluntátem facit, et suam voluntátem relínquit.}
\markright{IV}
%LIBER I
\TitreC{CAPUT IV}
\TitreC{De providéntia in agéndis}
\Normal{\Verse{1.} Non est credéndum omni verbo nec instínctui~; sed caute et longanímiter res est secúndum Deum ponderánda. Proh dolor, sæpe malum facílius quam bonum de álio créditur et dícitur, ita infírmi sumus. Sed perfécti viri non fácile credunt omni enarránti, quia sciunt infirmitátem humánam ad malum proclívam et in verbis satis lábilem.}
\Normal{\Verse{2.} Magna sapiéntia, non esse præcípitem in agéndis, nec pertináciter in própriis stare sénsibus. Ad hanc étiam pértinet, non quibúslibet hóminum verbis crédere, nec audíta vel crédita mox ad aliórum aures effúndere. Cum sapiénte et conscientióso viro consílium habe~; et quære pótius a melióre ínstrui, quam tuas adinventiónes sequi. Bona vita facit hóminem sapiéntem secúndum Deum et expértum in multis. Quanto quis in se humílior fúerit et Deo subiéctior, tanto in ómnibus erit sapiéntior et pacátior.}
\markright{V}
%LIBER I
\TitreC{CAPUT V}
\TitreC{De lectióne Sanctárum Scripturárum}
\Normal{\Verse{1.} Véritas est in Scriptúris Sanctis quærénda, non eloquéntia. Omnis Scriptúra Sacra eo spíritu debet legi, quo facta est. Quǽrere pótius debémus utilitátem in Scriptúris, quam subtilitátem sermónis. Ita libénter devótos et símplices libros légere debémus, sicut altos et profúndos. Non te offéndat auctóritas scribéntis, utrum parvæ vel magnæ litteratúræ fúerit~; sed amor puræ veritátis te trahat ad legéndum. Non quæras, quis hoc díxerit~; sed quid dicátur, atténde.}
\Normal{\Verse{2.} Hómines tránseunt, sed véritas Dómini manet in ætérnum. Sine personárum acceptióne, váriis modis lóquitur nobis Deus. Curiósitas nostra sæpe nos ímpedit in lectióne Scripturárum, cum vólumus intellégere et discútere, ubi simplíciter esset transeúndum. Si vis proféctum hauríre, lege humíliter, simplíciter et fidéliter~; nec umquam velis habére nomen sciéntiæ. Intérroga libénter et audi tacens verba sanctórum~; nec displíceant tibi parábolæ seniórum~: sine causa enim non proferúntur.}
\markright{VI}
%LIBER I
\TitreC{CAPUT VI}
\TitreC{De inordinátis affectiónibus}
\Normal{\Verse{1.} Quandocúmque homo áliquid inordináte áppetit, statim in se inquiétus fit. Supérbus et avárus numquam quiéscunt~; pauper et húmilis spíritu in multitúdine pacis conversántur. Homo qui necdum perfécte in se mórtuus est, cito temptátur et víncitur in parvis et vílibus rebus. Infírmus in spíritu et quodámmodo adhuc carnális et ad sensibília inclinátus difficúlter se potest a terrénis desidériis ex toto abstráhere. Et ídeo sæpe habet tristítiam, cum se súbtrahit~; léviter étiam indignátur, si quis ei resístit.}
\Normal{\Verse{2.} Si autem prosecútus fúerit, quod concupíscit, statim ex reátu consciéntiæ gravátur~: quia secútus est passiónem suam, quæ nihil iuvat ad pacem, quam quæsívit. Resisténdo ígitur passiónibus invenítur pax vera cordis, non autem eis serviéndo. Non est ergo pax in corde hóminis carnális, non in hómine exterióribus dédito, sed in férvido et spirituáli.}
\markright{VII}
%LIBER I
\TitreC{CAPUT VII}
\TitreC{De vana spe et elatióne fugiénda}
\Normal{\Verse{1.} Vanus est, qui spem suam ponit in homínibus aut in creatúris. Non te púdeat áliis servíre amóre Iesu Christi, et páuperem in hoc sǽculo vidéri. Non stes super teípsum, sed in Deo spem tuam constítue. Fac quod in te est, et Deus áderit bonæ voluntáti tuæ. Non confídas in tua sciéntia vel astútia cuiuscúmque vivéntis, sed magis in Dei grátia, qui ádiuvat húmiles, et de se præsuméntes humíliat.}
\Normal{\Verse{2.} Ne gloriéris in divítiis, si adsunt, nec in amícis, quia poténtes sunt, sed in Deo, qui ómnia præstat et seípsum super ómnia dare desíderat. Non te extóllas de magnitúdine vel pulchritúdine córporis, quæ módica infirmitáte corrúmpitur et defœdátur. Non pláceas tibi ipsi de habilitáte aut ingénio tuo, ne displíceas Deo, cuius est totum quidquid boni naturáliter habúeris.}
\Normal{\Verse{3.} Non te réputes áliis meliórem, ne forte coram Deo detérior habeáris, qui scit, quid est in hómine. Non supérbias opéribus bonis, quia áliter sunt iudícia Dei quam hóminum, cui sæpe dísplicet, quod homínibus placet. Si áliquid boni habúeris, crede de áliis melióra, ut humilitátem consérves. Non nocet, si ómnibus te suppónas~; nocet autem plúrimum, si vel uni te præpónas. Iugis pax cum húmili~; in corde autem supérbi zelus et indignátio frequens.}
\markright{VIII}
%LIBER I
\TitreC{CAPUT VIII}
\TitreC{De cavénda nímia familiaritáte}
\Normal{\Verse{1.} Non omni hómini revéles cor tuum, sed cum sapiénte et timénte Deum age causam tuam. Cum iuvénibus et extráneis rarus esto. Cum divítibus noli blandíri, et coram magnátibus non libénter appáreas. Cum humílibus et simplícibus, cum devótis et morigerátis sociáre, et quæ ædificatiónis sunt, pertrácta. Non sis familiáris alícui mulíeri~; sed in commúni omnes bonas mulíeres Deo comménda. Soli Deo et ángelis eius opta familiáris esse, et hóminum notítiam devíta.}
\Normal{\Verse{2.} Cáritas habénda est ad omnes, sed familiáritas non éxpedit. Quandóque áccidit, ut persóna ignóta ex bona fama lucéscat~; cuius tamen præséntia óculos intuéntium offúscat. Putámus aliquándo áliis placére ex coniunctióne nostra, et incípimus magis displicére ex morum improbitáte in nobis consideráta.}
\markright{IX}
%LIBER I
\TitreC{CAPUT IX}
\TitreC{De obœdiéntia et subiectióne}
\Normal{\Verse{1.} Valde magnum est, in obœdiéntia stare, sub præláto vívere, et sui iuris non esse. Multo tútius est, stare in subiectióne, quam in prælatúra. Multi sunt sub obœdiéntia magis ex necessitáte, quam ex caritáte~; et illi pœnam habent et léviter múrmurant. Nec libertátem mentis acquírent, nisi ex toto corde propter Deum se subíciant. Curre hic vel ibi~: non invénies quiétem, nisi in húmili subiectióne sub regímine præláti. Imaginátio locórum et mutátio multos feféllit.}
\Normal{\Verse{2.} Verum est, quod unusquísque libénter agit pro sensu suo, et inclinátur ad eos magis, qui secum séntiunt. Sed si Deus est inter nos, necésse est, ut relinquámus étiam quandóque nostrum sentíre propter bonum pacis. Quis est ita sápiens, qui ómnia plene scire potest~? Ergo, noli nimis in sensu tuo confídere, sed velis étiam libénter aliórum sensum audíre. Si bonum est tuum sentíre, et hoc ipsum propter Deum dimíttis et álium séqueris, magis exínde profícies.}
\Normal{\Verse{3.} Audívi enim sæpe, secúrius esse audíre et accípere consílium, quam dare. Potest étiam contíngere, ut bonum sit uniuscuiúsque sentíre~; sed nolle áliis acquiéscere, cum id rátio aut causa póstulat, signum est supérbiæ et pertináciæ.}
\markright{X}
%LIBER I
\TitreC{CAPUT X}
\TitreC{De cavénda superfluitáte verbórum}
\Normal{\Verse{1.} Cáveas tumúltum hóminum, quantum potes~; multum enim ímpedit tractátus sæculárium gestórum, étiam si símplici intentióne proferántur. Cito enim inquinámur vanitáte et captivámur. Vellem me plúries tacuísse et inter hómines non fuísse. Sed quare tam libénter lóquimur et ínvicem fabulámur, cum tamen raro sine læsióne consciéntiæ ad siléntium redímus~? Ideo tam libénter lóquimur~: quia per mútuas locutiónes ab ínvicem consolári quǽrimus~; et cor divérsis cogitatiónibus fatigátum optámus releváre. Et multum libénter de his, quæ multum dilígimus vel cúpimus, vel quæ nobis contrária sentímus, libet loqui et cogitáre.}
\Normal{\Verse{2.} Sed proh dolor, sæpe inániter et frustra. Nam hæc extérior consolátio interióris et divínæ consolatiónis non módicum detriméntum est. Ideo vigilándum est et orándum, ne tempus otióse tránseat. Si loqui licet et éxpedit, quæ ædificabília sunt, lóquere. Malus usus et neglegéntia proféctus nostri multum facit ad incustódiam oris nostri. Iuvat tamen non parum ad proféctum spirituálem devóta spirituálium rerum collátio~; máxime ubi pares ánimo et spíritu in Deo sibi sociántur.}
\markright{XI}
%LIBER I
\TitreC{CAPUT XI}
\TitreC{De pace acquirénda et zelo proficiéndi}
\Normal{\Verse{1.} Multam possémus pacem habére, si non vellémus nos cum aliórum dictis et factis, et quæ ad nostram curam non spectant, occupáre. Quómodo potest ille diu in pace manére, qui aliénis curis se intermíscet~? qui occasiónes forínsecus quærit~? qui parum vel raro se intrínsecus cólligit~? Beáti símplices, quóniam multam pacem habébunt.}
\Normal{\Verse{2.} Quare quidam sanctórum tam perfécti et contemplatívi fuérunt~? Quia omníno seípsos mortificáre ab ómnibus terrénis desidériis studuérunt~: et ídeo totis medúllis cordis Deo inhærére atque líbere sibi vacáre potuérunt. Nos nímium occupámur própriis passiónibus, et de transitóriis nimis sollicitámur. Raro étiam unum vítium perfécte víncimus, et ad cotidiánum proféctum non accéndimur~: ídeo frígidi et tépidi remanémus.}
\Normal{\Verse{3.} Si essémus nobis ipsis perfécte mórtui, et intérius mínime implicáti~: tunc possémus étiam divína sápere, et de cælésti contemplatióne áliquid experíri. Totum et máximum impediméntum est, quia non sumus a passiónibus et concupiscéntiis líberi, nec perféctam sanctórum viam conámur íngredi. Quando étiam módicum adversitátis occúrrit, nimis cito deícimur, et ad humánas consolatiónes convértimur.}
\Normal{\Verse{4.} Si niterémur, sicut viri fortes, stare in prœ́lio~: profécto auxílium Dómini super nos viderémus de cælo. Ipse enim certántes et de sua grátia sperántes parátus est adiuváre~: qui nobis certándi occasiónes procúrat, ut vincámus. Si tantum in istis exterióribus observántiis proféctum religiónis pónimus, cito finem habébit devótio nostra. Sed ad radícem secúrim ponámus, ut purgáti a passiónibus pacíficam mentem possideámus.}
\Normal{\Verse{5.} Si omni anno unum vítium exstirparémus, cito viri perfécti efficerémur. Sed modo e contrário sæpe sentímus, ut melióres et purióres in inítio conversiónis nos fuísse inveniámus, quam post multos annos professiónis. Fervor et proféctus cotídie debéret créscere~; sed nunc pro magno vidétur, si quis primi fervóris partem posset retinére. Si módicam violéntiam facerémus in princípio, tunc póstea cuncta possémus fácere cum levitáte et gáudio.}
\Normal{\Verse{6.} Grave est, assuéta dimíttere~: sed grávius est, contra própriam voluntátem ire. Sed si non vincis parva et lévia, quando superábis difficilióra~? Resíste in princípio inclinatióni tuæ, et malam dedísce consuetúdinem, ne forte paulátim ad maiórem te ducat difficultátem. O si advérteres, quantam tibi pacem et áliis lætítiam fáceres, teípsum bene habéndo, puto, quod sollicítior esses ad spirituálem proféctum.}
\markright{XII}
%LIBER I
\TitreC{CAPUT XII}
\TitreC{De utilitáte adversitátis}
\Normal{\Verse{1.} Bonum nobis est, quod aliquándo habeámus áliquas gravitátes et contrarietátes~; quia sæpe hóminem ad cor révocant, quátinus se in exílio esse cognóscat, nec spem suam in áliqua re mundi ponat. Bonum est, quod patiámur quandóque contradictiónes, et quod male et imperfécte de nobis sentiátur, étiam si bene ágimus et inténdimus. Ista sæpe iuvant ad humilitátem et a vana glória nos deféndunt. Tunc enim mélius interiórem testem Deum quǽrimus, quando foris vilipéndimur ab homínibus, et non bene nobis créditur.}
\Normal{\Verse{2.} Ideo debéret se homo in Deo táliter firmáre, ut non esset ei necésse multas humánas consolatiónes quǽrere. Quando homo bonæ voluntátis tribulátur vel temptátur, aut malis cogitatiónibus afflígitur~: tunc Deum sibi magis necessárium intéllegit, sine quo nihil boni se posse deprehéndit. Tunc étiam tristátur, gemit, et orat pro misériis, quas pátitur. Tunc tædet eum diútius vívere, et mortem optat veníre~: ut possit dissólvi et cum Christo esse. Tunc étiam bene advértit, perféctam securitátem et plenam pacem in mundo non posse constáre.}
\markright{XIII}
%LIBER I
\TitreC{CAPUT XIII}
\TitreC{De temptatiónibus resisténdis}
\Normal{\Verse{1.} Quámdiu in mundo vívimus, sine tribulatióne et temptatióne esse non póssumus. Unde in Iob scriptum est~: Temptátio est vita humána super terram. Ideo unusquísque sollícitus esse debéret circa temptatiónes suas et vigiláre in oratiónibus, ne diábolus locum inveníret decipiéndi, qui numquam dormítat, sed círcuit quærens, quem dévoret. Nemo tam perféctus est et sanctus, qui non hábeat aliquándo temptatiónes~; et plene eis carére non póssumus.}
\Normal{\Verse{2.} Sunt tamen temptatiónes hómini sæpe valde útiles, licet moléstæ sint et graves~; quia in illis homo humiliátur, purgátur, et erudítur. Omnes sancti per multas tribulatiónes et temptatiónes transiérunt et profecérunt. Et qui temptatiónes sustinére nequivérunt, réprobi facti sunt et defecérunt. Non est áliquis ordo tam sanctus, nec locus tam secrétus, ubi non sint temptatiónes vel adversitátes.}
\Normal{\Verse{3.} Non est homo secúrus a temptatiónibus totáliter, quámdiu víxerit~: quia in nobis est, unde temptámur, ex quo in concupiscéntia nati sumus. Una temptatióne seu tribulatióne recedénte, ália supérvenit, et semper áliquid ad patiéndum habébimus, nam bonum felicitátis nostræ perdídimus. Multi quærunt temptatiónes fúgere, et grávius íncidunt in eas. Per solam fugam non póssumus víncere~; sed per patiéntiam et veram humilitátem ómnibus hóstibus effícimur fortióres.}
\Normal{\Verse{4.} Qui tantúmmodo extérius declínat nec radícem evéllit, parum profíciet~; immo cítius ad eum temptatiónes redient et peius séntiet. Paulátim, et per patiéntiam cum longanimitáte Deo iuvánte mélius superábis, quam cum durítia et importunitáte própria. Sǽpius áccipe consílium in temptatióne, et cum temptáto noli dúriter ágere, sed consolatiónem íngere, sicut tibi optáres fíeri.}
\Normal{\Verse{5.} Inítium ómnium malárum temptatiónum inconstántia ánimi et parva ad Deum confidéntia. Quia sicut navis sine gubernáculo hinc inde a flúctibus impéllitur~: ita homo remíssus et suum propósitum déserens várie temptátur. Ignis probat ferrum, et temptátio hóminem iustum. Nescímus sæpe quid póssumus~; sed temptátio áperit quid sumus. Vigilándum est tamen præcípue circa inítium temptatiónis~; quia tunc facílius hostis víncitur, si óstium mentis nullátenus intráre sínitur~; sed extra limen, statim ut pulsáverit, illi obviátur. Unde quidam dixit~: \textit{Princípiis obsta, sero medicína parátur}. Nam primo occúrrit menti simplex cogitátio, deínde fortis imaginátio, póstea delectátio et motus pravus, et assénsio. Sicque paulátim ingréditur hostis malígnus ex toto, dum illi non resístitur in princípio. Et quanto diútius ad resisténdum quis torpúerit, tanto in se cotídie debílior fit, et hostis contra eum poténtior.}
\Normal{\Verse{6.} Quidam in princípio conversiónis suæ gravióres temptatiónes patiúntur, quidam autem in fine. Quidam vero quasi per totam vitam suam male habent. Nonnúlli satis léniter temptántur, secúndum divínæ ordinatiónis sapiéntiam et æquitátem, quæ statum et mérita hóminum pensat, et cuncta ad electórum suórum salútem præórdinat.}
\Normal{\Verse{7.} Ideo non debémus desperáre, cum temptámur, sed eo fervéntius Deum exoráre, quátinus nos in omni tribulatióne dignétur adiuváre~; qui útique, secúndum dictum Pauli, talem fáciet cum temptatióne provéntum, ut possímus sustinére. Humiliémus ergo ánimas nostras sub manu Dei in omni temptatióne et tribulatióne, quia húmiles spíritu salvábit et exaltábit.}
\Normal{\Verse{8.} In temptatiónibus et tribulatiónibus probátur homo, quantum profécit~; et ibi maius méritum consístit, et virtus mélius patéscit. Nec magnum est, si homo devótus sit et férvidus, cum gravitátem non sentit~; sed si témpore adversitátis patiénter se sústinet, spes magni proféctus erit. Quidam a magnis temptatiónibus custodiúntur, et in parvis cotidiánis sæpe vincúntur, ut humiliáti numquam de se ipsis in magnis confídant, qui in tam módicis infirmántur.}
\markright{XIV}
%LIBER I
\TitreC{CAPUT XIV}
\TitreC{De temerário iudício vitándo}
\Normal{\Verse{1.} Ad teípsum óculos reflécte, et aliórum facta cáveas iudicáre. In iudicándo álios homo frustra labórat, sǽpius errat, et léviter peccat~; seípsum vero iudicándo et discutiéndo semper fructuóse labórat. Sicut nobis res cordi est, sic de ea frequénter iudicámus~; nam verum iudícium propter privátum amórem facíliter pérdimus. Si Deus semper esset pura inténtio nostri desidérii, non tam facíliter turbarémur pro resisténtia sensus nostri.}
\Normal{\Verse{2.} Sed sæpe áliquid ab intra latet, vel étiam ab extra concúrrit, quod nos étiam páriter trahit. Multi occúlte seípsos quærunt in rebus quas agunt, et nésciunt. Vidéntur étiam in bona pace stare, quando res pro eórum velle fiunt et sentíre~; si autem áliter fit quam cúpiunt, cito movéntur et tristes fiunt. Propter diversitátem sénsuum et opiniónum satis frequénter oriúntur dissensiónes inter amícos et cives, inter religiósos et devótos.}
\Normal{\Verse{3.} Antíqua consuetúdo difficúlter relínquitur, et ultra próprium vidére nemo libénter dúcitur. Si ratióni tuæ magis inníteris vel indústriæ, quam virtúti subiectívæ Iesu Christi, raro et tarde eris homo illuminátus~; quia Deus vult nos sibi perfécte súbici, et omnem ratiónem per inflammátum amórem transcéndere.}
\markright{XV}
%LIBER I
\TitreC{CAPUT XV}
\TitreC{De opéribus ex caritáte factis}
\Normal{\Verse{1.} Pro nulla re mundi et pro nullíus hóminis dilectióne áliquod malum est faciéndum~; sed pro utilitáte tamen indigéntis opus bonum líbere aliquándo intermitténdum est, aut étiam pro melióri mutándum. Hoc enim facto opus bonum non destrúitur, sed in mélius commutátur. Sine caritáte opus extérnum nihil prodest~; quidquid autem ex caritáte ágitur, quantumcúmque étiam parvum sit et despéctum, totum fructuósum effícitur. Magis síquidem Deus pensat ex quanto quis agit, quam opus quod facit.}
\Normal{\Verse{2.} Multum facit, qui multum díligit. Multum facit, qui rem bene facit. Bene facit, qui communitáti magis quam suæ voluntáti servit. Sæpe vidétur esse cáritas, et est magis carnálitas~; quia naturális inclinátio, própria volúntas, spes retributiónis, afféctus commoditátis raro abésse volunt.}
\Normal{\Verse{3.} Qui veram et perféctam caritátem habet, in nulla re seípsum quærit, sed Dei solúmmodo glóriam in ómnibus fíeri desíderat. Nulli étiam ínvidet, quia nullum privátum gáudium amat~; nec in se ipso vult gaudére, sed in Deo, super ómnia bona, optat beatificári. Némini áliquid boni attríbuit, sed totáliter ad Deum refert, a quo fontáliter ómnia procédunt, in quo fináliter omnes sancti fruibíliter requiéscunt. O qui scintíllam habéret veræ caritátis, profécto ómnia terréna sentíret plena fore vanitátis.}
\markright{XVI}
%LIBER I
\TitreC{CAPUT XVI}
\TitreC{De sufferéntia deféctuum aliórum}
\Normal{\Verse{1.} Quæ homo in se vel in áliis emendáre non valet, debet patiénter sustinére, donec Deus áliter órdinet. Cógita, quia sic forte mélius est pro tua probatióne et patiéntia, sine qua non sunt multum ponderánda mérita nostra. Debes tamen pro tálibus impediméntis supplicáre, ut Deus tibi dignétur subveníre, et possis benígne portáre.}
\Normal{\Verse{2.} Si quis semel aut bis admónitus non acquiéscit, noli cum eo conténdere~; sed totum Deo commítte, ut fiat volúntas eius et honor in ómnibus servis suis, qui scit bene mala in bonum convértere. Stude pátiens esse in tolerándo aliórum deféctus, et qualescúmque infirmitátes~; quia et tu multa habes, quæ ab áliis opórtet tolerári. Si non potes te talem fácere, qualem vis, quómodo póteris álium ad tuum habére beneplácitum~? Libénter habémus álios perféctos, et tamen próprios non emendámus deféctus.}
\Normal{\Verse{3.} Vólumus, quod álii stricte corrigántur, et ipsi córrigi nólumus. Dísplicet larga aliórum licéntia, et tamen nobis nólumus negári quod pétimus. Alios restríngi per statúta vólumus, et ipsi nullátenus pátimur ámplius cohibéri. Sic ergo patet, quam raro próximum sicut nos ipsos pensámus. Si essent omnes perfécti, quid tunc haberémus ab áliis pro Deo pati?}
\Normal{\Verse{4.} Nunc autem Deus sic ordinávit, ut discámus alter altérius ónera portáre~; quia nemo sine deféctu, nemo sine ónere, nemo sibi suffíciens, nemo sibi satis sápiens~; sed opórtet nos ínvicem portáre, ínvicem consolári, páriter adiuváre, instrúere et admonére. Quantæ autem virtútis quisque fúerit, mélius patet occasióne adversitátis. Occasiónes namque hóminem frágilem non fáciunt, sed qualis sit, osténdunt.}
\markright{XVII}
%LIBER I
\TitreC{CAPUT XVII}
\TitreC{De monástica vita}
\Normal{\Verse{1.} Opórtet quod discas teípsum in multis frángere, si vis pacem et concórdiam cum áliis tenére. Non est parvum in monastériis vel in congregatióne habitáre, et ínibi sine queréla conversári, et usque ad mortem fidélis perseveráre. Beátus, qui ibídem bene víxerit et felíciter consummáverit. Si vis débite stare et profícere, téneas te tamquam éxsulem peregrínum super terram. Opórtet te stultum fíeri propter Christum, si vis religiósam dúcere vitam.}
\Normal{\Verse{2.} Hábitus et tonsúra módicum confert~; sed mutátio morum, et íntegra mortificátio passiónum verum fáciunt religiósum. Qui áliud quærit, quam pure Deum et ánimæ suæ salútem, non invéniet nisi tribulatiónem et dolórem. Non potest étiam diu stare pacíficus, qui non nítitur esse mínimus et ómnibus subiéctus.}
\Normal{\Verse{3.} Ad serviéndum venísti, non ad regéndum~; ad patiéndum et laborándum scias te vocátum, non ad otiándum vel fabulándum. Hic ergo probántur hómines, sicut aurum in fornáce. Hic nemo potest stare, nisi ex toto corde se volúerit propter Deum humiliáre.}
\markright{XVIII}
%LIBER I
\TitreC{CAPUT XVIII}
\TitreC{De exémplis sanctórum patrum}
\Normal{\Verse{1.} Intuére sanctórum patrum vívida exémpla, in quibus vera perféctio refúlsit et relígio, et vidébis, quam módicum sit, et pene nihil, quod nos ágimus. Heu, quid est vita nostra, si illis fúerit comparáta~? Sancti et amíci Christi Dómino serviérunt in fame et siti, in frígore et nuditáte, in labóre et fatigatióne, in vigíliis et ieiúniis, in oratiónibus et meditatiónibus sanctis, in persecutiónibus et oppróbriis multis.}
\Normal{\Verse{2.} O quam multas et graves tribulatiónes passi sunt apóstoli, mártyres, confessóres, vírgines, et réliqui omnes, qui Christi vestígia voluérunt sequi. Nam ánimas suas in hoc mundo odérunt, ut in ætérnam vitam eas possidérent. O quam strictam et abdicátam vitam sancti patres in erémo duxérunt~! quam longas et graves temptatiónes pertulérunt~! quam frequénter ab inimíco vexáti sunt~! quam crebras et férvidas oratiónes Deo obtulérunt~! quam rígidas abstinéntias peregérunt~! quam magnum zelum et fervórem ad spirituálem proféctum habuérunt~! quam forte bellum advérsus edomatiónem vitiórum gessérunt~! quam puram et rectam intentiónem ad Deum tenuérunt~! Per diem laborábant, et nóctibus oratióni diútinæ vacábant~; quamquam laborándo ab oratióne mentáli mínime cessárent.}
\Normal{\Verse{3.} Omne tempus utíliter expendébant~; omnis hora ad vacándum Deo brevis videbátur~; et præ magna dulcédine contemplatiónis étiam oblivióni tradebátur necéssitas corporális refectiónis. Omnibus divítiis, dignitátibus, honóribus, amícis et cognátis renuntiábant~; nil de mundo habére cupiébant, vix necessária vitæ sumébant~; córpori servíre, étiam in necessitáte, dolébant. Páuperes ígitur erant rebus terrénis, sed dívites valde in grátia et virtútibus. Foris egébant, sed intus grátia et consolatióne divína reficiebántur.}
\Normal{\Verse{4.} Mundo erant aliéni, sed Deo próximi ac familiáres amíci. Sibi ipsis videbántur tamquam níhili, et huic mundo despécti~; sed erant in óculis Dei pretiósi et dilécti. In vera humilitáte stabant, in símplici obœdiéntia vivébant, in caritáte et patiéntia ambulábant~: et ídeo cotídie in spíritu proficiébant, et magnam apud Deum grátiam obtinébant. Dati sunt in exémplum ómnibus religiósis, et plus provocáre nos debent ad bene proficiéndum, quam tepidórum númerus ad relaxándum.}
\Normal{\Verse{5.} O quantus fervor ómnium religiosórum in princípio suæ sanctæ institutiónis fuit~! O quanta devótio oratiónis~! quanta æmulátio virtútis~! quam magna disciplína víguit~! quanta reveréntia et obœdiéntia sub régula magístri in ómnibus efflóruit~! Testántur adhuc vestígia derelícta, quod vere viri sancti et perfécti fuérunt, qui tam strénue militántes mundum suppeditavérunt. Iam magnus putátur, si quis transgréssor non fúerit, si quis, quod accépit, cum patiéntia toleráre potúerit.}
\Normal{\Verse{6.} Oh tepóris et neglegéntiæ status nostri, quod tam cito declinámus a prístino fervóre, et iam tædet vívere præ lassitúdine et tepóre~! Utinam in te pénitus non dormítet proféctus virtútum, qui multa sǽpius exémpla vidísti devotórum!}
\markright{XIX}
%LIBER I
\TitreC{CAPUT XIX}
\TitreC{De exercítiis boni religiósi}
\Normal{\Verse{1.} Vita boni religiósi ómnibus virtútibus pollére debet~: ut sit talis intérius, qualis vidétur homínibus extérius. Et mérito, multo plus debet esse intus, quam quod cérnitur foris, quia inspéctor noster est Deus, quem summópere reveréri debémus, ubicúmque fuérimus, et tamquam ángeli in conspéctu eius mundi incédere. Omni die renováre debémus propósitum nostrum et ad fervórem nos excitáre, quasi hódie primum ad conversiónem venissémus, atque dícere~: Adiuva me, Dómine Deus, in bono propósito et sancto servítio tuo, et da mihi nunc hódie perfécte incípere, quia nihil est quod háctenus feci.}
\Normal{\Verse{2.} Secúndum propósitum nostrum, cursus proféctus nostri~; et multa diligéntia opus est bene profícere volénti. Quod si fórtiter propónens sæpe déficit~; quid ille, qui raro aut minus fixe áliquid propónit~? Váriis tamen modis contíngit desértio propósiti nostri~; et levis omíssio exercitiórum vix sine áliquo dispéndio transit. Iustórum propósitum in grátia Dei pótius, quam in própria sapiéntia pendet~: in quo et semper confídunt, quidquid arrípiunt. Nam homo propónit, sed Deus dispónit, nec est in hómine via eius.}
\Normal{\Verse{3.} Si pietátis causa aut fratérnæ utilitátis propósito quandóque consuétum omíttitur exercítium, fácile póstea póterit recuperári. Si autem tǽdio ánimi aut neglegéntia facíliter relínquitur, satis culpábile est et nocívum sentiétur. Conémur quantum póssumus, adhuc léviter deficiémus in multis. Semper tamen áliquid certi proponéndum est, et contra illa præcípue, quæ ámplius nos impédiunt. Exterióra nostra et interióra páriter nobis scrutánda sunt et ordinánda, quia útraque expédiunt ad proféctum.}
\Normal{\Verse{4.} Si non contínue te vales collígere, saltem intérdum, et ad minus semel in die, mane vidélicet, aut véspere. Mane propóne, véspere díscute mores tuos, qualis hódie fuísti in verbo, ópere et cogitatióne~; quia in his sǽpius fórsitan Deum offendísti et próximum. Accínge te, sicut vir, contra diabólicas nequítias~; frena gulam, et omnem carnis inclinatiónem facílius frenábis. Numquam sis ex toto otiósus, sed aut legens, aut scribens, aut orans, aut méditans, aut áliquid utilitátis pro commúni labórans. Corporália tamen exercítia discréte sunt agénda, nec ómnibus æquáliter assuménda.}
\Normal{\Verse{5.} Quæ commúnia non sunt, non sunt foris ostendénda~; nam in secréto tútius exercéntur priváta. Cavéndum tamen, ne piger sis ad commúnia et ad singulária prómptior~; sed explétis íntegre et fidéliter débitis et iniúnctis, si iam ultra vacat, redde te tibi, prout devótio tua desíderat. Non possunt omnes habére unum exercítium, sed áliud isti, áliud illi magis desérvit. Etiam pro témporis congruéntia divérsa placent exercítia, quia ália in festis, ália in feriátis magis sápiunt diébus. Aliis indigémus témpore temptatiónis, et áliis témpore pacis et quiétis. Alia, cum tristámur, libet cogitáre, et ália, cum læti in Dómino fuérimus.}
\Normal{\Verse{6.} Circa principália festa renovánda sunt bona exercítia et sanctórum suffrágia fervéntius imploránda. De festo in festum propónere debémus, quasi tunc de hoc sǽculo migratúri et ad ætérnum festum perventúri. Ideóque sollícite nos præparáre debémus in devótis tempóribus et devótius conversári, atque omnem observántiam stríctius custodíre, tamquam in brevi prǽmium labóris nostri a Deo perceptúri.}
\Normal{\Verse{7.} Et si dilátum fúerit, credámus nos minus bene præparátos atque indígnos tantæ adhuc glóriæ, quæ revelábitur in nobis témpore præfiníto~: et studeámus nos mélius ad éxitum præparáre. Beátus servus, ait evangelísta Lucas, quem, cum vénerit Dóminus, invénerit vigilántem. Amen dico vobis, super ómnia bona sua constítuet eum.}
\markright{XX}
%LIBER I
\TitreC{CAPUT XX}
\TitreC{De amóre solitúdinis et siléntii}
\Normal{\Verse{1.} Quære aptum tempus vacándi tibi, et de benefíciis Dei frequénter cógita. Relínque curiósa. Tales pérlege matérias, quæ compunctiónem magis præstant, quam occupatiónem. Si te subtráxeris a supérfluis locutiónibus et otiósis circuitiónibus, nec non a novitátibus et rumóribus audiéndis~: invénies tempus suffíciens et aptum pro bonis meditatiónibus insisténdis. Máximi sanctórum humána consórtia, ubi póterant, vitábant, et Deo in secréto servíre eligébant.}
\Normal{\Verse{2.} Dixit quidam~: Quótiens inter hómines fui, minor homo rédii. Hoc sǽpius experímur, quando diu confabulámur. Facílius est omníno tacére, quam verbo non excédere. Facílius est domi latére, quam foris se posse sufficiénter custodíre. Qui ígitur inténdit ad interióra et spirituália perveníre, opórtet eum cum Iesu a turba declináre. Nemo secúre appáret, nisi qui libénter latet. Nemo secúre lóquitur, nisi qui libénter tacet. Nemo secúre præest, nisi qui libénter subest. Nemo secúre prǽcipit, nisi qui bene obœdíre dídicit.}
\Normal{\Verse{3.} Nemo secúre gaudet, nisi bonæ consciéntiæ in se testimónium hábeat. Semper tamen sanctórum secúritas plena timóris Dei éxstitit. Nec eo minus sollíciti et húmiles in se fuérunt, quia magnis virtútibus et grátia emicuérunt. Pravórum autem secúritas ex supérbia et præsumptióne orítur, et in fine in deceptiónem sui ipsíus vértitur. Numquam promíttas tibi securitátem in hac vita, quamvis bonus videáris cœnobíta aut devótus eremíta.}
\Normal{\Verse{4.} Sæpe melióres in æstimatióne hóminum grávius periclitáti sunt propter suam nímiam confidéntiam. Unde multis utílius est, ut non pénitus temptatiónibus cáreant, sed sǽpius impugnéntur, ne nímium secúri sint, ne forte in supérbiam elevéntur, ne étiam ad exterióres consolatiónes licéntius declínent. O, qui numquam transitóriam lætítiam quǽreret, qui numquam cum mundo se occupáret, quam bonam consciéntiam serváret~! O, qui omnem vanam sollicitúdinem amputáret, et dumtáxat salutária ac divína cogitáret, et totam spem suam in Deo constitúeret, quam magnam pacem et quiétem possidéret!}
\Normal{\Verse{5.} Nemo dignus est cælésti consolatióne, nisi diligénter se exercúerit in sancta compunctióne. Si vis corde tenus compúngi, intra cubíle tuum, et exclúde tumúltus mundi, sicut scriptum est~: In cubílibus vestris compungímini. In cella invénies, quod déforis sǽpius amíttes. Cella continuáta dulcéscit, et male custodíta tǽdium génerat. Si in princípio conversiónis tuæ bene eam incolúeris et custodíeris, erit tibi póstea dilécta amíca et gratíssimum solácium.}
\Normal{\Verse{6.} In siléntio et quiéte próficit ánima devóta et discit abscóndita scripturárum. Ibi ínvenit fluénta lacrimárum, quibus síngulis nóctibus se lavet et mundet, ut Conditóri suo tanto familiárior fiat, quanto lóngius ab omni sæculári tumúltu degit. Qui ergo se ábstrahit a notis et amícis, approximábit illi Deus cum ángelis sanctis. Mélius est latére et sui curam ágere, quam se neglécto signa fácere. Laudábile est hómini religióso, raro foras ire, fúgere vidéri, nolle étiam hómines vidére.}
\Normal{\Verse{7.} Quid vis vidére, quod non licet habére~? Transit mundus et concupiscéntia eius. Trahunt desidéria sensualitátis ad spatiándum~; sed cum hora transíerit, quid nisi gravitátem consciéntiæ et cordis dispersiónem repórtas~? Lætus éxitus tristem sæpe réditum parit, et læta vigília serótina triste mane facit. Sic omne carnále gáudium blande intrat, sed in fine mordet et périmit. Quid potes álibi vidére, quod hic non vides~? Ecce cælum et terra et ómnia eleménta~; nam ex istis ómnia sunt facta.}
\Normal{\Verse{8.} Quid potes alícubi vidére, quod diu potest sub sole permanére~? Credis te fórsitan satiári, sed non póteris pertíngere. Si cuncta vidéres præséntia, quid esset nisi vísio vana~? Leva óculos tuos ad Deum in excélsis, et ora pro peccátis tuis et neglegéntiis. Dimítte vana vanis~; tu autem inténde illis, quæ tibi præcépit Deus. Claude super te óstium tuum, et voca ad te Iesum, diléctum tuum. Mane cum eo in cella, quia non invénies álibi tantam pacem. Si non exísses nec quicquam de rumóribus audísses, mélius in bona pace permansísses. Ex quo nova deléctat aliquándo audíre, opórtet te exínde turbatiónem cordis toleráre.}
\markright{XXI}
%LIBER I
\TitreC{CAPUT XXI}
\TitreC{De compunctióne cordis}
\Normal{\Verse{1.} Si vis áliquid profícere, consérva te in timóre Dei, et noli esse nimis liber~; sed sub disciplína cóhibe omnes sensus tuos, nec inéptæ te tradas lætítiæ. Da te ad cordis compunctiónem, et invénies devotiónem. Compúnctio multa bona áperit, quæ dissolútio cito pérdere consuévit. Mirum est, quod homo potest umquam perfécte in hac vita lætári, qui suum exílium et tam multa perícula ánimæ suæ consíderat et pensat.}
\Normal{\Verse{2.} Propter levitátem cordis et neglegéntiam deféctuum nostrórum non sentímus ánimæ nostræ dolóres~; sed sæpe vane ridémus, quando mérito flere deberémus. Non est vera libértas nec bona lætítia, nisi in timóre Dei cum bona consciéntia. Felix, qui abícere potest omne impediméntum distractiónis, et ad uniónem se recollígere sanctæ compunctiónis. Felix, qui a se ábdicat, quidquid consciéntiam suam maculáre potest vel graváre. Certa viríliter, consuetúdo consuetúdine víncitur. Si tu scis hómines dimíttere, ipsi bene te dimíttent tua facta fácere.}
\Normal{\Verse{3.} Non áttrahas tibi res aliórum, nec te ímplices causis maiórum. Hábeas semper óculum super te primum, et admóneas teípsum speciáliter præ ómnibus tibi diléctis. Si non habes favórem hóminum, noli ex hoc tristári~; sed hoc sit tibi grave, quia non habes te satis bene et circumspécte, sicut decéret Dei servum et devótum religiósum conversári. Utílius est sæpe et secúrius, quod homo non hábeat multas consolatiónes in hac vita, secúndum carnem præcípue. Tamen quod divínas non habémus, aut rárius sentímus, nos in culpa sumus, quia compunctiónem cordis non quǽrimus, nec vanas et extérnas omníno abícimus.}
\Normal{\Verse{4.} Cognósce te indígnum divína consolatióne, sed magis dignum multa tribulatióne. Quando homo est perfécte compúnctus, tunc gravis et amárus est ei totus mundus. Bonus homo sufficiéntem ínvenit matériam doléndi et flendi. Sive enim se consíderat, sive de próximo pensat, scit, quia nemo sine tribulatióne hic vivit. Et quanto stríctius sese consíderat, tanto ámplius dolet. Matériæ iusti dolóris et intérnæ compunctiónis sunt peccáta et vítia nostra, quibus ita involúti iacémus, ut raro cæléstia contemplári valeámus.}
\Normal{\Verse{5.} Si frequéntius de morte tua, quam de longitúdine vitæ cogitáres, non dúbium, quin fervéntius te emendáres. Si étiam futúras inférni sive purgatórii pœnas cordiáliter perpénderes, credo, quod libénter labórem et dolórem sustinéres, et nihil rigóris formidáres. Sed quia ad cor ista non tránseunt, et blandiménta adhuc amámus, ídeo frígidi et valde pigri remanémus.}
\Normal{\Verse{6.} Sæpe est inópia spíritus, unde tam léviter conquéritur míserum corpus. Ora ígitur humíliter ad Dóminum, ut det tibi compunctiónis spíritum, et dic cum prophéta~: Ciba me, Dómine, pane lacrimárum, et potum da mihi in lácrimis in mensúra.}
\markright{XXII}
%LIBER I
\TitreC{CAPUT XXII}
\TitreC{De consideratióne humánæ misériæ}
\Normal{\Verse{1.} Miser es, ubicúmque fúeris et quocúmque te vérteris, nisi ad Deum te convértas. Quid turbáris, quia non succédit tibi, sicut vis et desíderas~? Quis est, qui habet ómnia secúndum suam voluntátem~? Nec ego, nec tu, nec áliquis hóminum super terram. Nemo est in mundo sine áliqua tribulatióne vel angústia, quamvis rex sit vel papa. Quis est, qui mélius habet~? útique qui pro Deo áliquid pati valet.}
\Normal{\Verse{2.} Dicunt multi imbecílles et infírmi~: Ecce, quam bonam vitam ille homo habet, quam dives, quam magnus, quam potens et excélsus~! Sed atténde ad cæléstia bona, et vidébis, quod ómnia ista temporália nulla sunt, sed valde incérta et magis gravántia, quia numquam sine sollicitúdine et timóre possidéntur. Non est hóminis felícitas, habére temporália ad abundántiam, sed súfficit ei mediócritas. Vere miséria est vívere super terram. Quanto homo volúerit esse spirituálior, tanto præsens vita fit ei amárior, quia sentit mélius et videt clárius humánæ corruptiónis deféctus. Nam comédere, bíbere, vigiláre, dormíre, quiéscere, laboráre et céteris necessitátibus natúræ subiacére, vere magna miséria est et afflíctio hómini devóto, qui libénter esset absolútus et liber ab omni peccáto.}
\Normal{\Verse{3.} Valde enim gravátur intérior homo necessitátibus corporálibus in hoc mundo. Unde prophéta devóte rogat, quátenus liber ab istis esse váleat, dicens~: De necessitátibus meis érue me, Dómine. Sed væ non cognoscéntibus suam misériam~; et ámplius væ illis, qui díligunt hanc míseram et corruptíbilem vitam. Nam in tantum quidam hanc amplectúntur, licet étiam vix necessária laborándo aut mendicándo hábeant, ut, si possent hic semper vívere, de regno Dei nihil curárent.}
\Normal{\Verse{4.} O insáni et infidéles corde, qui tam profúnde in terrénis iacent, ut nihil nisi carnália sápiant. Sed míseri adhuc in fine gráviter séntient, quam vile et níhilum erat, quod amavérunt. Sancti autem Dei et omnes devóti amíci Christi non attendérunt, quæ carni placuérunt, nec quæ in hoc témpore floruérunt~; sed tota spes eórum et inténtio ad ætérna bona anhelábat. Ferebátur totum desidérium eórum sursum ad mansúra et invisibília, ne amóre visibílium traheréntur ad ínfima. Noli, frater, amíttere confidéntiam proficiéndi ad spirituália~; adhuc habes tempus et horam.}
\Normal{\Verse{5.} Quare vis procrastináre propósitum tuum~? Surge et in instánti íncipe et dic~: Nunc tempus est faciéndi, nunc tempus est pugnándi, nunc aptum tempus est emendándi. Quando male habes et tribuláris, tunc tempus est promeréndi. Opórtet te transíre per ignem et aquam, ántequam vénias in refrigérium. Nisi tibi vim féceris, vítium non superábis. Quámdiu istud frágile corpus gérimus, sine peccáto esse non póssumus, nec sine tǽdio et dolóre vívere. Libénter haberémus ab omni miséria quiétem~; sed quia per peccátum perdídimus innocéntiam, amísimus étiam veram beatitúdinem. Ideo opórtet nos tenére patiéntiam et Dei exspectáre misericórdiam~; donec tránseat iníquitas hæc et mortálitas absorbeátur a vita.}
\Normal{\Verse{6.} O quanta fragílitas humána, quæ semper prona est ad vítia~! Hódie confitéris peccáta tua, et cras íterum pérpetras conféssa. Nunc propónis cavére, et post horam agis quasi nihil proposuísses. Mérito ergo nos ipsos humiliáre póssumus nec umquam áliquid magni de nobis sentíre~; quia tam frágiles et instábiles sumus. Cito étiam potest perdi per neglegéntiam, quod multo labóre vix tandem acquisítum est per grátiam.}
\Normal{\Verse{7.} Quid fiet de nobis adhuc in fine, qui tepéscimus tam mane~? Væ nobis, si sic vólumus declináre ad quiétem, quasi iam pax sit et secúritas, cum necdum appáreat vestígium veræ sanctitátis in conversatióne nostra. Bene opus esset, quod adhuc íterum instruerémur, tamquam boni novícii, ad mores óptimos~; si forte spes esset de áliqua futúra emendatióne et maióri spirituáli proféctu.}
\markright{XXIII}
%LIBER I
\TitreC{CAPUT XXIII}
\TitreC{De meditatióne mortis}
\Normal{\Verse{1.} Valde cito erit tecum hic factum, vide áliter quómodo te hábeas~: hódie homo est, et cras non compáret. Cum autem sublátus fúerit ab óculis, étiam cito transit a mente. O hebetúdo et durítia cordis humáni, quod solum præséntia meditátur, et futúra non magis prǽvidet~! Sic te in omni facto et cogitátu debéres tenére, quasi hódie esses moritúrus. Si bonam consciéntiam habéres, non multum mortem timéres. Mélius esset peccáta cavére, quam mortem fúgere. Si hódie non es parátus, quómodo cras eris~? Cras est dies incérta, et quid scis, si crástinum habébis?}
\Normal{\Verse{2.} Quid prodest diu vívere, quando tam parum emendámur~? Ah, longa vita non semper eméndat, sed sæpe culpam magis auget. Utinam per unam diem bene essémus conversáti in hoc mundo~! Multi annos cómputant conversiónis, sed sæpe parvus est fructus emendatiónis. Si formidolósum est mori, fórsitan periculósius erit diútius vívere. Beátus, qui horam mortis suæ semper ante óculos habet, et ad moriéndum cotídie se dispónit. Si vidísti aliquándo hóminem mori, cógita, quia et tu per eándem viam transíbis.}
\Normal{\Verse{3.} Cum mane fúerit, puta te ad vésperum non perventúrum. Véspere autem facto, mane non áudeas tibi pollicéri. Semper ergo parátus esto, et táliter vive, ut numquam te imparátum mors invéniat. Multi súbito et improvíse moriúntur. Nam hora, qua non putátur, fílius hóminis ventúrus est. Quando illa extréma hora vénerit, multum áliter sentíre incípies de tota vita tua prætérita~; et valde dolébis, quia tam néglegens et remíssus fuísti.}
\Normal{\Verse{4.} Quam felix et prudens, qui talis nunc nítitur esse in vita, qualis optat inveníri in morte~! Dabit namque magnam fidúciam felíciter moriéndi perféctus contémptus mundi, fervens desidérium in virtútibus proficiéndi, amor disciplínæ, labor pæniténtiæ, promptitúdo obœdiéntiæ, abnegátio sui et supportátio cuiúslibet adversitátis pro amóre Christi. Multa bona potes operári, dum sanus es~; sed infirmátus, néscio quid póteris. Pauci ex infirmitáte meliorántur~; sic et qui multum peregrinántur, raro sanctificántur.}
\Normal{\Verse{5.} Noli confídere super amícos et próximos, nec in futúrum tuam dífferas salútem~: quia cítius obliviscéntur tui hómines, quam ǽstimas. Mélius est nunc tempestíve providére et áliquid boni præmíttere, quam super aliórum auxílio speráre. Si non es pro te ipso sollícitus modo, quis erit sollícitus pro te in futúro~? Nunc tempus est valde pretiósum. Nunc sunt dies salútis~: nunc tempus acceptábile. Sed proh dolor, quod hoc utílius non expéndis, in quo promeréri vales, unde æternáliter vivas~! Véniet, quando unam diem seu horam pro emendatióne desiderábis, et néscio, an impetrábis.}
\Normal{\Verse{6.} Eia caríssime, de quanto perículo te póteris liberáre, de quam magno timóre erípere, si modo semper timorátus fúeris et suspéctus de morte~? Stude nunc táliter vívere, ut in hora mortis váleas pótius gaudére, quam timére. Disce nunc mori mundo, ut tunc incípias vívere cum Christo. Disce nunc ómnia contémnere, ut tunc possis líbere ad Christum pérgere. Castíga nunc corpus tuum per pæniténtiam, ut tunc certam váleas habére confidéntiam.}
\Normal{\Verse{7.} Ah stulte, quid cógitas te diu victúrum, cum nullum diem hábeas secúrum~? Quam multi decépti sunt, et insperáte de córpore extrácti~? Quótiens audísti a dicéntibus, quia ille gládio cécidit, ille submérsus est, ille ab alto ruens cervícem fregit, ille manducándo obríguit, ille ludéndo finem fecit~? Alius igne, álius ferro, álius peste, álius latrocínio intériit~: et sic ómnium finis mors est, et vita hóminum tamquam umbra súbito pertránsit.}
\Normal{\Verse{8.} Quis memorábitur tui post mortem~? et quis orábit pro te~? Age, age nunc, caríssime, quidquid ágere potes~: quia nescis, quando moriéris, nescis étiam, quid tibi post mortem sequétur. Dum tempus habes, cóngrega divítias immortáles. Præter salútem tuam, nihil cógites~; solum quæ Dei sunt, cures. Fac nunc tibi amícos, venerándo Dei sanctos, et eórum actus imitándo, ut, cum deféceris in hac vita, illi te recípiant in ætérna tabernácula.}
\Normal{\Verse{9.} Serva te tamquam peregrínum et hóspitem super terram, ad quem nihil spectat de mundi negótiis. Serva cor líberum et ad Deum sursum eréctum, quia non habes hic manéntem civitátem. Illuc preces et gémitus cotidiános cum lácrimis dírige, ut spíritus tuus mereátur ad Dóminum post mortem felíciter transíre. Amen.}
\markright{XXIV}
%LIBER I
\TitreC{CAPUT XXIV}
\TitreC{De iudício et pœnis peccatórum}
\Normal{\Verse{1.} In ómnibus rebus réspice finem, et quáliter ante distríctum stabis iúdicem, cui nihil est occúltum~; qui munéribus non placátur nec excusatiónes récipit~; sed, quod iustum est, iudicábit. O misérrime et insípiens peccátor, quid respondébis Deo, ómnia mala tua sciénti, qui intérdum formídas vultum hóminis iráti~? Ut quid non prǽvides tibi in die iudícii, quando nemo póterit per álium excusári vel deféndi, sed unusquísque suffíciens onus erit sibi ipsi~? Nunc labor tuus est fructuósus, fletus acceptábilis, gémitus exaudíbilis, dolor satisfactórius et purgatívus.}
\Normal{\Verse{2.} Habet magnum et salúbre purgatórium pátiens homo, qui suscípiens iniúrias, plus dolet de altérius malítia, quam de sua iniúria~; qui pro contrariántibus sibi libénter orat et ex corde culpas indúlget~: qui véniam ab áliis pétere non retárdat~; qui facílius miserétur, quam iráscitur~; qui sibi ipsi violéntiam frequénter facit, et carnem omníno spirítui subiugáre conátur. Mélius est, modo purgáre peccáta et vítia resecáre, quam in futúro purgánda reserváre. Vere nos ipsos decípimus per inordinátum amórem, quem ad carnem habémus.}
\Normal{\Verse{3.} Quid áliud ignis ille devorábit, nisi peccáta tua~? Quanto ámplius tibi ipsi nunc parcis et carnem séqueris, tanto dúrius póstea lues et maiórem matériam comburéndi resérvas. In quibus homo peccávit, in illis grávius puniétur. Ibi acediósi ardéntibus stímulis perurgéntur, et gulósi ingénti siti ac fame cruciabúntur. Ibi luxuriósi et voluptátum amatóres ardénti pice et fœ́tido súlphure perfundéntur, et sicut furiósi canes præ dolóre invidiósi ululábunt.}
\Normal{\Verse{4.} Nullum vítium erit, quod suum próprium cruciátum non habébit. Ibi supérbi omni confusióne replebúntur, et avári misérrima egestáte arctabúntur. Ibi erit una hora grávior in pœna, quam hic centum anni in gravíssima pæniténtia. Ibi nulla réquies est, nulla consolátio damnátis~; hic tamen intérdum cessátur a labóribus, atque amicórum frúitur soláciis. Esto modo sollícitus et dolens pro peccátis tuis, ut in die iudícii secúrus sis cum beátis. Tunc enim iusti stabunt in magna constántia advérsus eos, qui se angustiavérunt et depressérunt. Tunc stabit ad iudicándum, qui modo se súbicit humíliter iudíciis hóminum. Tunc magnam fidúciam habébit pauper et húmilis, et pavébit úndique supérbus.}
\Normal{\Verse{5.} Tunc vidébitur sápiens in hoc mundo fuísse, qui pro Christo dídicit stultus et despéctus esse. Tunc placébit omnis tribulátio patiénter perpéssa, et omnis iníquitas oppilábit os suum. Tunc gaudébit omnis devótus, et mærébit omnis irreligiósus. Tunc plus exsultábit caro afflícta, quam si in delíciis fuísset semper nutríta. Tunc splendébit hábitus vilis, et obtenebréscet vestis subtílis. Tunc plus laudábitur paupérculum domicílium, quam deaurátum palátium. Tunc iuvábit plus constans patiéntia, quam omnis mundi poténtia. Tunc ámplius exaltábitur simplex obœdiéntia, quam omnis sæculáris astútia.}
\Normal{\Verse{6.} Tunc plus lætificábit pura et bona consciéntia, quam docta philosóphia. Tunc plus ponderábit contémptus divitiárum, quam totus thesáurus terrigenárum. Tunc magis consoláberis super devóta oratióne, quam super delicáta comestióne. Tunc pótius gaudébis de serváto siléntio, quam de longa fabulatióne. Tunc plus valébunt sancta ópera, quam multa pulchra verba. Tunc plus placébit stricta vita et árdua pæniténtia, quam omnis delectátio terréna. Disce nunc in módico pati, ut tunc a gravióribus váleas liberári. Hic primo proba, quid possis póstea. Si nunc tam parum vales sustinére, quómodo ætérna torménta póteris sufférre~? Si modo módica pássio tam impatiéntem éfficit, quid gehénna tunc fáciet~? Ecce, vere non potes duo gáudia habére, delectári hic in mundo, et póstea regnáre cum Christo.}
\Normal{\Verse{7.} Si usque in hodiérnam diem semper in honóribus et voluptátibus vixísses~: quid totum tibi profuísset, si iam mori in instánti contíngeret~? Omnia ergo vánitas, præter amáre Deum et illi soli servíre. Qui enim Deum ex toto corde amat, nec mortem, nec supplícium, nec iudícium, nec inférnum métuit~; quia perféctus amor secúrum ad Deum accéssum facit. Quem autem adhuc peccáre deléctat, non mirum, si mortem et iudícium tímeat. Bonum tamen est, ut, si necdum amor a malo te révocat, saltem timor gehennális coerceat. Qui vero timórem Dei postpónit, diu stare in bono non valébit, sed diáboli láqueos cítius incúrret.}
\markright{XXV}
%LIBER I
\TitreC{CAPUT XXV}
\TitreC{De fervénti emendatióne totíus vitæ nostræ}
\Normal{\Verse{1.} Esto vígilans et díligens in Dei servítio, et cógita frequénter~: Ad quid venísti, et cur sǽculum reliquísti~? Nonne ut Deo víveres, et spirituális homo fíeres~? Igitur ad proféctum férveas, quia mercédem labórum tuórum in brevi recípies~; nec erit tunc ámplius timor aut dolor in fínibus tuis. Módicum nunc laborábis, et magnam réquiem, immo perpétuam lætítiam invénies. Si tu permánseris fidélis et férvidus in agéndo, Deus procul dúbio erit fidélis et lócuples in retribuéndo. Spem bonam retinére debes, quod ad palmam pervénies~; sed securitátem cápere non opórtet, ne tórpeas aut elátus fias.}
\Normal{\Verse{2.} Cum quidam ánxius inter metum et spem frequénter fluctuáret, et quadam vice mæróre conféctus, in ecclésia ante quoddam altáre se in oratióne prostravísset, hæc intra se revólvit, dicens~: O si scirem, quod adhuc perseveratúrus essem~! statímque audívit divínum intus respónsum~: Quod si hoc scires, quid fácere velles~? Fac nunc, quod tunc fácere velles, et bene secúrus eris. Moxque consolátus et confortátus, divínæ se commísit voluntáti, et cessávit ánxia fluctuátio. Noluítque curióse investigáre, ut sciret, quæ sibi essent futúra~: sed magis stúduit inquírere, quæ esset volúntas Dei benéplacens et perfécta, ad omne opus bonum inchoándum et perficiéndum.}
\Normal{\Verse{3.} Spera in Dómino, et fac bonitátem ait prophéta, et inhábita terram, et pascéris in divítiis eius. Unum est, quod multos a proféctu et fervénti emendatióne rétrahit~: horror difficultátis seu labor certáminis. Enimvéro illi máxime præ céteris in virtútibus profíciunt, qui ea, quæ sibi magis grávia et contrária sunt, virílius víncere nitúntur. Nam ibi homo plus próficit et grátiam merétur ampliórem, ubi magis seípsum vincit et in spíritu mortíficat.}
\Normal{\Verse{4.} Sed non omnes habent æque multum ad vincéndum et moriéndum. Díligens tamen æmulátor valéntior erit ad proficiéndum, etiámsi plures hábeat passiónes, quam álius bene morigerátus, minus tamen fervens ad virtútes. Duo speciáliter ad magnam emendatiónem iuvant~: vidélicet, subtráhere se violénter, ad quod natúra vitióse inclinátur, et fervénter instáre pro bono, quo ámplius quis índiget. Illa étiam stúdeas magis cavére et víncere, quæ tibi frequéntius in áliis dísplicent.}
\Normal{\Verse{5.} Ubíque proféctum tuum cápias, ut, si bona exémpla vídeas vel áudias, ad imitándum accendáris. Si quid autem reprehensíbile consideráveris, cave, ne idem fácias~; aut si aliquándo fecísti, cítius emendáre te stúdeas. Sicut óculus tuus álios consíderat, sic íterum ab áliis notáris. Quam iucúndum et dulce est, vidére férvidos et devótos fratres, bene morigerátos et disciplinátos~! Quam triste est et grave, vidére inordináte ambulántes, qui ea, ad quæ vocáti sunt, non exércent~! Quam nocívum est, neglégere vocatiónis suæ propósitum, et ad non commíssa sensum inclináre!}
\Normal{\Verse{6.} Memor esto arrépti propósiti, et imáginem tibi propóne Crucifíxi. Bene verecundári potes inspécta vita Iesu Christi, quia necdum magis illi te conformáre studuísti, licet diu in via Dei fuísti. Religiósus, qui se inténte et devóte in sanctíssima vita et passióne Dómini exércet, ómnia utília et necessária sibi abundánter ibi invéniet~; nec opus est, ut extra Iesum áliquid mélius quærat. O si Iesus crucifíxus in cor nostrum veníret, quam cito et sufficiénter docti essémus!}
\Normal{\Verse{7.} Religiósus férvidus ómnia bene portat et capit, quæ illi iubéntur. Religiósus néglegens et tépidus habet tribulatiónem super tribulatiónem, et ex omni parte pátitur angústiam~; quia interióri consolatióne caret, et exteriórem quǽrere prohibétur. Religiósus extra disciplínam vivens gravi patet ruínæ. Qui laxióra quærit et remissióra, semper in angústiis erit~: quia unum aut réliquum sibi displicébit.}
\Normal{\Verse{8.} Quómodo fáciunt tam multi álii religiósi, qui satis arctáti sunt sub disciplína claustráli~? Raro éxeunt, abstrácte vivunt, paupérrime cómedunt, grosse vestiúntur, multum labórant, parum loquúntur, diu vígilant, matúre surgunt, oratiónes prolóngant, frequénter legunt et se in omni disciplína custódiunt. Atténde Carthusiénses, Cisterciénses et divérsæ religiónis mónachos ac moniáles~: quáliter omni nocte ad psalléndum Dómino assúrgunt. Et ídeo turpe esset, ut tu debéres in tam sancto ópere pigritáre, ubi tanta multitúdo religiosórum íncipit Deo iubiláre.}
\Normal{\Verse{9.} O si nihil áliud faciéndum incúmberet, nisi Dóminum Deum nostrum toto corde et ore laudáre~! O si numquam indigéres comédere, nec bíbere, nec dormíre, sed semper posses Deum laudáre, et solúmmodo spirituálibus stúdiis vacáre~: tunc multo felícior esses quam modo, cum carni ex qualicúmque necessitáte servis. Utinam non essent istæ necessitátes, sed solum spirituáles ánimæ refectiónes, quas heu~! satis raro degustámus!}
\Normal{\Verse{10.} Quando homo ad hoc pérvenit, quod de nulla creatúra consolatiónem suam quærit, tunc ei Deus primo perfécte sápere íncipit~; tunc étiam bene conténtus de omni evéntu rerum erit. Tunc nec pro magno lætábitur, nec pro módico contristábitur~; sed ponit se íntegre et fiduciáliter in Deo, qui est ei ómnia in ómnibus~: cui nihil útique perit nec móritur, sed ómnia ei vivunt et ad nutum incunctánter desérviunt.}
\Normal{\Verse{11.} Meménto semper finis, et quia pérditum non redit tempus. Sine sollicitúdine et diligéntia numquam acquíres virtútes. Si íncipis tepéscere, incípies male habére. Si autem déderis te ad fervórem, invénies magnam pacem et sénties leviórem labórem, propter Dei grátiam et virtútis amórem. Homo férvidus et díligens ad ómnia est parátus. Maior labor est resístere vítiis et passiónibus, quam corporálibus insudáre labóribus. Qui parvos non vitat deféctus, paulátim lábitur ad maióres. Gaudébis semper véspere, si diem expéndas fructuóse. Vígila super teípsum, excíta teípsum, ádmone teípsum~; et quidquid de áliis sit, non néglegas teípsum. Tantum profícies, quantum tibi ipsi vim intúleris. Amen.}
\markboth{II}{}
LIBER II
ADMONITIONES AD INTERNA TRAHENTES
\markright{I}
%LIBER II
\TitreC{CAPUT I}
\TitreC{De intérna conversatióne}
\Normal{\Verse{1.} Regnum Dei intra vos est, dicit Dóminus. Convérte te ex toto corde ad Dóminum et relínque hunc míserum mundum, et invéniet ánima tua réquiem. Disce exterióra contémnere et ad interióra te dare, et vidébis regnum Dei in te veníre. Est enim regnum Dei pax et gáudium in Spíritu Sancto, quod non datur ímpiis. Véniet ad te Christus, osténdens tibi consolatiónem suam, si dignam illi ab intus paráveris mansiónem. Omnis glória eius et decor ab intra est, et ibi cómplacet sibi. Frequens illi visitátio cum hómine intérno, dulcis sermocinátio, grata consolátio, multa pax, familiáritas stupénda nimis.}
\Normal{\Verse{2.} Eia, ánima fidélis, prǽpara huic sponso cor tuum, quátenus ad te veníre et in te habitáre dignétur. Sic enim dicit~: Si quis díligit me, sermónem meum servábit, et ad eum veniémus et mansiónem apud eum faciémus. Da ergo Christo locum, et céteris ómnibus nega intróitum. Cum Christum habúeris, dives es, et súfficit tibi. Ipse erit provísor tuus et fidélis procurátor in ómnibus, ut non sit opus in homínibus speráre. Hómines enim cito mutántur et defíciunt velóciter~; Christus autem manet in ætérnum et astat usque in finem fírmiter.}
\Normal{\Verse{3.} Non est magna fidúcia ponénda in hómine frágili et mortáli, étiam si útilis sit et diléctus~; neque tristítia multa ex hoc capiénda, si intérdum adversétur et contradícat. Qui hódie tecum sunt, cras contrariári possunt~; et e convérso sæpe ut aura vertúntur. Pone totam fidúciam tuam in Deo, et sit ipse timor tuus et amor tuus. Ipse pro te respondébit et fáciet bene, sicut mélius fúerit. Non habes hic manéntem civitátem~; et ubicúmque fúeris, extráneus es et peregrínus~; nec réquiem aliquándo habébis, nisi Christo íntime fúeris unítus.}
\Normal{\Verse{4.} Quid hic circúmspicis, cum iste non sit locus tuæ requietiónis~? In cæléstibus debet esse habitátio tua, et sicut in tránsitu cuncta terréna sunt aspiciénda. Tránseunt ómnia et tu cum eis páriter. Vide, ut non inhǽreas, ne capiáris et péreas. Apud Altíssimum sit cogitátio tua, et deprecátio tua ad Christum sine intermissióne dirigátur. Si nescis speculári alta et cæléstia, requiésce in passióne Christi, et in sacris vulnéribus eius libénter hábita. Si enim ad vúlnera et pretiósa stígmata Iesu devóte cónfugis, magnam in tribulatióne confortatiónem sénties~; nec multum curábis hóminum despectiónes, facilitérque verba detrahéntia pérferes.}
\Normal{\Verse{5.} Christus fuit étiam in mundo ab homínibus despéctus, et in máxima necessitáte a notis et amícis inter oppróbria derelíctus. Christus pati vóluit et déspici, et tu audes de áliquo cónqueri~? Christus hábuit adversários et oblocutóres, et tu vis omnes habére amícos et benefactóres~? Unde coronábitur patiéntia tua, si nihil adversitátis occúrrerit~? Si nihil contrárium vis pati, quómodo eris amícus Christi~? Sústine te cum Christo et pro Christo, si vis regnáre cum Christo.}
\Normal{\Verse{6.} Si semel perfécte introísses in interióra Iesu, et módicum de ardénti amóre eius sapuísses, tunc de próprio cómmodo vel incómmodo nihil curáres, sed magis de oppróbrio illáto gaudéres~; quia amor Iesu facit hóminem seípsum contémnere. Amátor Iesu et veritátis, et verus intérnus et liber ab affectiónibus inordinátis potest se ad Deum líbere convértere et eleváre supra seípsum in spíritu ac fruitíve quiéscere.}
\Normal{\Verse{7.} Cui sápiunt ómnia, prout sunt, non ut dicúntur aut æstimántur~: hic vere sápiens est et doctus magis a Deo, quam ab homínibus. Qui ab intra scit ambuláre et módicum ab extra res ponderáre, non requírit loca nec exspéctat témpora ad habénda devóta exercítia. Homo intérnus cito se recólligit, quia numquam se totum ad exterióra effúndit. Non illi obest labor extérior aut occupátio ad tempus necessária, sed sicut res evéniunt, sic se illis accómmodat. Qui intus bene dispósitus est et ordinátus, non curat mirábiles et pervérsos hóminum gestus. Tantum homo impedítur et distráhitur, quantum sibi res áttrahit.}
\Normal{\Verse{8.} Si recte tibi esset, et bene purgátus esses, ómnia tibi in bonum céderent et proféctum. Ideo multa tibi dísplicent et sæpe contúrbant, quia adhuc non es perfécte tibi ipsi mórtuus nec segregátus ab ómnibus terrénis. Nil sic máculat et ímplicat cor hóminis, sicut impurus amor in creatúris. Si rénuis consolári extérius, póteris speculári cæléstia et frequénter iubiláre intérius.}
\markright{II}
%LIBER II
\TitreC{CAPUT II}
\TitreC{De húmili submissióne}
\Normal{\Verse{1.} Non magni pendas, quis pro te vel contra te sit~; sed hoc age et cura, ut Deus tecum sit in omni re, quam facis. Hábeas consciéntiam bonam, et Deus bene te defensábit. Quem enim Deus adiuváre volúerit, nullíus pervérsitas nocére póterit. Si tu scis tacére et pati, vidébis procul dúbio auxílium Dómini. Ipse novit tempus et modum liberándi te, et ídeo te debes illi resignáre. Dei est adiuváre et ab omni confusióne liberáre. Sæpe valde prodest ad maiórem humilitátem servándam, quod deféctus nostros álii sciunt et redárguunt.}
\Normal{\Verse{2.} Quando homo pro deféctibus suis se humíliat, tunc facíliter álios placat et léviter satísfacit sibi irascéntibus. Húmilem Deus prótegit et líberat, húmilem díligit et consolátur~; húmili hómini se inclínat, húmili largítur grátiam magnam, et post suam depressiónem levat ad glóriam. Húmili sua secréta revélat et ad se dúlciter trahit et invítat. Húmilis accépta confusióne satis bene est in pace~: quia stat in Deo et non in mundo. Non réputes, te áliquid profecísse, nisi ómnibus inferiórem te esse séntias.}
\markright{III}
%LIBER II
\TitreC{CAPUT III}
\TitreC{De bono pacífico hómine}
\Normal{\Verse{1.} Tene te primo in pace, et tunc póteris álios pacificáre. Homo pacíficus magis prodest, quam bene doctus. Homo passionatus étiam bonum in malum trahit et facíliter malum credit. Bonus pacíficus homo ómnia ad bonum convértit. Qui bene in pace est, de nullo suspicátur. Qui autem male conténtus est et commótus, váriis suspiciónibus agitátur, nec ipse quiéscit, nec álios quiéscere permíttit. Dicit sæpe, quod dícere non debéret, et omíttit, quod sibi magis fácere expedíret. Consíderat, quid álii fácere tenéntur, et néglegit, quid ipse tenétur. Habe ergo primo zelum super teípsum, et tunc iuste zeláre póteris étiam próximum tuum.}
\Normal{\Verse{2.} Tu bene scis facta tua excusáre et coloráre, et aliórum excusatiónes non vis recípere. Iústius esset, ut te accusáres, et fratrem tuum excusáres. Si portári vis, porta et álium. Vide, quam longe es adhuc a vera caritáte et humilitáte, quæ nulli novit irásci vel indignári, nisi tantum sibi. Non est magnum, cum bonis et mansuétis conversári~: hoc enim ómnibus naturáliter placet~; et unusquísque libénter pacem habet et secum sentiéntes magis díligit. Sed cum duris et pervérsis aut indisciplinátis aut nobis contrariántibus pacífice posse vívere, magna grátia est et laudábile nimis viriléque factum.}
\Normal{\Verse{3.} Sunt, qui seípsos in pace tenent et cum áliis étiam pacem habent. Et sunt, qui nec pacem habent, nec álios in pace dimíttunt~; áliis sunt graves, sed sibi semper gravióres. Et sunt, qui seípsos in pace rétinent et ad pacem álios redúcere student. Est tamen tota pax nostra in hac mísera vita pótius in húmili sufferéntia ponénda, quam in non sentiéndo contrária. Qui mélius scit pati, maiórem tenébit pacem. Iste est victor sui et dóminus mundi, amícus Christi et heres cæli.}
\markright{IV}
%LIBER II
\TitreC{CAPUT IV}
\TitreC{De pura mente et símplici intentióne}
\Normal{\Verse{1.} Duábus alis homo sublevátur a terrénis, simplicitáte scílicet et puritáte. Simplícitas debet esse in intentióne, púritas in affectióne. Simplícitas inténdit Deum, púritas apprehéndit et gustat. Nulla bona áctio te impédiet, si liber intus ab inordináto afféctu fúeris. Si nihil áliud, quam Dei beneplácitum et próximi utilitátem inténdis et quæris, intérna libertáte perfruéris. Si rectum cor tuum esset, tunc omnis creatúra spéculum vitæ et liber sanctæ doctrínæ esset. Non est creatúra tam parva et vilis, quæ Dei bonitátem non repræséntet.}
\Normal{\Verse{2.} Si tu esses intus bonus et purus, tunc ómnia sine impediménto vidéres et bene cáperes. Cor purum pénetrat cælum et inférnum. Qualis unusquísque intus est, táliter iúdicat extérius. Si est gáudium in mundo, hoc útique póssidet puri cordis homo. Et si est alícubi tribulátio et angústia, hoc mélius novit mala consciéntia. Sicut ferrum missum in ignem amíttit rubíginem et totum candens effícitur~: sic homo íntegre ad Deum se convértens a torpóre exúitur, et in novum hóminem transmutátur.}
\Normal{\Verse{3.} Quando homo íncipit tepéscere, tunc parvum métuit labórem et libénter extérnam áccipit consolatiónem. Sed quando perfécte íncipit se víncere et viríliter in via Dei ambuláre, tunc minus ea réputat, quæ sibi prius grávia esse sentiébat.}
\markright{V}
%LIBER II
\TitreC{CAPUT V}
\TitreC{De própria consideratióne}
\Normal{\Verse{1.} Non póssumus nobis ipsis nimis crédere, quia sæpe grátia nobis deest et sensus. Módicum lumen est in nobis, et hoc cito per neglegéntiam amíttimus. Sæpe étiam non advértimus, quod tam cæci intus sumus. Sæpe male ágimus, et peius excusámus. Passióne intérdum movémur, et zelum putámus. Parva in áliis reprehéndimus, et nostra maióra pertransímus. Satis cito sentímus et ponderámus, quid ab áliis sustinémus~; sed quantum álii de nobis sústinent, non advértimus. Qui bene et recte sua ponderáret, non esset, quod de álio gráviter iudicáret.}
\Normal{\Verse{2.} Intérnus homo sui ipsíus curam ómnibus curis antepónit~; et qui sibi ipsi diligénter inténdit, facíliter de áliis tacet. Numquam eris intérnus et devótus, nisi de aliénis silúeris et ad teípsum speciáliter respéxeris. Si tibi et Deo totáliter inténdis, módicum te movébit, quod foris pércipis. Ubi es, quando tibi ipsi præsens non es~? Et quando ómnia percurrísti, quid te neglécto profecísti~? Si debes habére pacem et uniónem veram~: opórtet, quod totum adhuc postpónas et te solum præ óculis hábeas.}
\Normal{\Verse{3.} Multum proínde profícies, si te feriátum ab omni temporáli cura consérves. Valde defícies, si áliquid temporále reputáveris. Nil magnum, nil altum, nil gratum, nil accéptum tibi sit, nisi pure Deus aut de Deo sit. Totum vanum exístima, quidquid consolatiónis occúrrerit de áliqua creatúra. Amans Deum ánima sub Deo déspicit univérsa. Solus Deus ætérnus et imménsus, implens ómnia, solácium ánimæ et vera cordis lætítia.}
\markright{VI}
%LIBER II
\TitreC{CAPUT VI}
\TitreC{De lætítia bonæ consciéntiæ}
\Normal{\Verse{1.} Glória boni hóminis testimónium bonæ consciéntiæ. Habe bonam consciéntiam et habébis semper lætítiam. Bona consciéntia valde multa potest portáre et valde læta est inter advérsa. Mala consciéntia semper tímida est et inquiéta. Suáviter requiésces, si cor tuum te non reprehénderit. Noli lætári, nisi cum beneféceris. Mali numquam habent veram lætítiam, nec intérnam séntiunt pacem~: quia non est pax ímpiis, dicit Dóminus. Et si díxerint~: In pace sumus, non vénient super nos mala~: et quis nobis nocére audébit~? ne credas eis~; quóniam repénte exúrget ira Dei, et in níhilum redigéntur actus eórum, et cogitatiónes eórum períbunt.}
\Normal{\Verse{2.} Gloriári in tribulatióne non est grave amánti~; sic enim gloriári est gloriári in cruce Dómini. Brevis glória, quæ ab homínibus datur et accípitur. Mundi glóriam semper comitátur tristítia. Bonórum glória in consciéntiis eórum et non in ore hóminum. Iustórum lætítia de Deo et in Deo est, et gáudium eórum de veritáte. Qui veram et ætérnam glóriam desíderat, temporálem non curat. Et qui temporálem requírit glóriam, aut non ex ánimo contémnit, minus amáre convíncitur cæléstem. Magnam habet cordis tranquillitátem, qui nec laudes curat nec vitupéria.}
\Normal{\Verse{3.} Fácile erit conténtus et pacátus, cuius consciéntia munda est. Non es sánctior, si laudáris~; nec vílior, si vituperáris. Quod es, hoc es~; nec maior dici vales, quam Deo teste sis. Si atténdis, quid apud te sis intus, non curábis, quid de te loquántur hómines. Homo videt in fácie, Deus autem in corde. Homo consíderat actus, Deus vero pensat intentiónes. Bene semper ágere et módicum de se tenére, húmilis ánimæ indícium est. Nolle consolári ab áliqua creatúra, magnæ puritátis et intérnæ fidúciæ signum est.}
\Normal{\Verse{4.} Qui nullum extrínsecus pro se testimónium quærit, liquet, quod totáliter se Deo commísit. Non enim, qui seípsum comméndat, ille probátus est ait beátus Paulus~: sed quem Deus comméndat. Ambuláre cum Deo intus, nec áliqua affectióne tenéri foris, status est intérni hóminis.}
\markright{VII}
%LIBER II
\TitreC{CAPUT VII}
\TitreC{De amóre Iesu super ómnia}
\Normal{\Verse{1.} Beátus, qui intéllegit, quid sit amáre Iesum, et contémnere seípsum propter Iesum. Opórtet diléctum pro dilécto relínquere, quia Iesus vult solus super ómnia amári. Diléctio creatúræ fallax et instábilis~; diléctio Iesu fidélis et perseverábilis. Qui adhǽret creatúræ, cadet cum lábili~; qui ampléctitur Iesum, firmábitur in ævum. Illum dílige et amícum tibi rétine, qui ómnibus recedéntibus te non relínquet, nec patiétur in fine períre. Ab ómnibus opórtet te aliquándo separári, sive velis, sive nolis.}
\Normal{\Verse{2.} Téneas te apud Iesum vivens ac móriens, et illíus fidelitáti te commítte, qui ómnibus deficiéntibus solus te potest iuváre. Diléctus tuus talis est natúræ, ut aliénum non velit admíttere, sed solus vult cor tuum habére et tamquam rex in próprio throno sedére. Si scires te bene ab omni creatúra evacuáre, Iesus debéret libénter tecum habitáre. Pene totum pérditum invénies, quidquid extra Iesum in homínibus posúeris. Non confídas, nec innitáris super cálamum ventósum~; quia omnis caro fænum, et omnis glória eius ut flos fæni cadet.}
\Normal{\Verse{3.} Cito decipiéris, si ad extérnam hóminum apparéntiam tantum aspéxeris. Si enim tuum in áliis quæris solácium et lucrum, sénties sǽpius detriméntum. Si quæris in ómnibus Iesum, invénies útique Iesum. Si autem quæris teípsum, invénies étiam teípsum, sed ad tuam perníciem. Plus enim homo nocívior sibi, si Iesum non quærit, quam totus mundus et omnes sui adversárii.}
\markright{VIII}
%LIBER II
\TitreC{CAPUT VIII}
\TitreC{De familiári amicítia Iesu}
\Normal{\Verse{1.} Quando Iesus adest, totum bonum est, nec quicquam diffícile vidétur~; quando vero Iesus non adest, totum durum est. Quando Iesus intus non lóquitur, consolátio vilis est~; si autem Iesus unum tantum verbum lóquitur, magna consolátio sentítur. Nonne María Magdaléna statim surréxit de loco, in quo flevit, quando Martha illi dixit~: Magíster adest et vocat te~? Felix hora, quando Iesus vocat de lácrimis ad gáudium spíritus~! Quam áridus et durus es sine Iesu~! Quam insípiens et vanus, si cupis áliquid extra Iesum~! Nonne hoc est maius damnum, quam si totum pérderes mundum?}
\Normal{\Verse{2.} Quid potest tibi mundus conférre sine Iesu~? Esse sine Iesu, gravis est inférnus, et esse cum Iesu, dulcis paradísus. Si fúerit tecum Iesus, nullus póterit nocére inimícus. Qui ínvenit Iesum, ínvenit thesáurum bonum, immo bonum super omne bonum. Et qui perdit Iesum, perdit nimis multum et plus quam totum mundum. Paupérrimus est, qui vivit sine Iesu, et ditíssimus, qui bene est cum Iesu.}
\Normal{\Verse{3.} Magna ars est, scire cum Iesu conversári, et scire Iesum tenére, magna prudéntia. Esto húmilis et pacíficus, et erit tecum Iesus. Sis devótus et quiétus, et manébit tecum Iesus. Potes cito fugáre Iesum et grátiam eius pérdere, si volúeris ad exterióra declináre. Et si illum effugáveris et perdíderis, ad quem fúgies, et quem tunc quæres amícum~? Sine amíco non potes bene vívere, et si Iesus non fúerit tibi præ ómnibus amícus, eris nimis tristis et desolátus. Fátue ígitur agis, si in áliquo áltero confídis aut lætáris. Eligéndum est magis, totum mundum habére contrárium quam Iesum offénsum. Ex ómnibus ergo caris sit Iesus solus diléctus speciális.}
\Normal{\Verse{4.} Diligántur omnes propter Iesum, Iesus autem propter seípsum. Solus Iesus Christus singuláriter est amándus, qui solus bonus et fidélis præ ómnibus invenítur amícis. Propter ipsum et in ipso tam amíci quam inimíci sint tibi cari~; et pro ómnibus his exorándus est, ut omnes ipsum cognóscant et díligant. Numquam cúpias singuláriter laudári vel amári, quia hoc solíus Dei est, qui símilem sibi non habet. Nec velis, quod áliquis tecum in corde suo occupétur, neque tu cum alicúius occupéris amóre~; sed sit Iesus in te et in omni bono hómine.}
\Normal{\Verse{5.} Esto purus et liber ab intus sine alicúius creatúræ implicaménto. Opórtet te esse nudum, et purum cor ad Deum gérere, si vis vacáre et vidére, quam suávis sit Dóminus. Et revéra ad hoc non pervénies, nisi grátia eius fúeris prævéntus et intráctus, ut ómnibus evacuátis et licentiatis solus cum solo uniáris. Quando enim grátia Dei venit ad hóminem, tunc potens fit ad ómnia, et quando recédit, tunc pauper et infírmus erit, et quasi tantum ad flagélla relíctus. In his non debet déici nec desperáre, sed ad voluntátem Dei æquanímiter stare, et cuncta superveniéntia sibi ad laudem Iesu Christi pérpeti~; quia post híemem séquitur æstas, post noctem redit dies, et post tempestátem magna serénitas.}
\markright{IX}
%LIBER II
\TitreC{CAPUT IX}
\TitreC{De caréntia omnis solácii}
\Normal{\Verse{1.} Non est grave, humánum contémnere solácium, cum adest divínum. Magnum est et valde magnum, tam humáno quam divíno posse carére solácio, et pro honóre Dei libénter exílium cordis velle sustinére, et in nullo seípsum quǽrere, nec ad próprium méritum respícere. Quid magni est, si hílaris sis et devótus adveniénte grátia~? optábilis cunctis hæc hora. Satis suáviter équitat, quem grátia Dei portat. Et quid mirum, si onus non sentit, qui portátur ab omnipoténte et dúcitur a summo ductóre?}
\Normal{\Verse{2.} Libénter habémus áliquid pro solácio, et difficúlter homo exúitur a se ipso. Vicit sanctus martyr Lauréntius sǽculum cum suo sacerdóte~; quia omne, quod in mundo delectábile videbátur, despéxit et summum Dei sacerdótem Sixtum, quem máxime diligébat, pro amóre Christi étiam a se tolli cleménter ferébat. Amóre ígitur creatóris amórem hóminis superávit, et pro humáno solácio divínum beneplácitum magis elégit. Ita et tu áliquem necessárium et diléctum amícum pro amóre Dei disce relínquere. Nec gráviter feras, cum ab amíco derelíctus fúeris, sciens, quóniam opórtet nos omnes tandem ab ínvicem separári.}
\Normal{\Verse{3.} Multum et diu opórtet hóminem in se ipso certáre, ántequam discat seípsum plene superáre et totum afféctum suum in Deum tráhere. Quando homo stat super seípsum, fácile lábitur ad consolatiónes humánas. Sed verus amátor Christi et studiósus sectátor virtútum non cadit super consolatiónes, nec quærit tales sensíbiles dulcédines, sed magis fortes exercitatiónes, et pro Christo duros sustinére labóres.}
\Normal{\Verse{4.} Cum ígitur spirituális a Deo consolátio datur, cum gratiárum actióne áccipe eam~; sed Dei munus intéllege esse, non tuum méritum. Noli extólli, noli nímium gaudére, nec inániter præsúmere~; sed esto magis humílior ex dono, cáutior quoque et timorátior in cunctis áctibus tuis~; quóniam transíbit hora illa, et sequétur temptátio. Cum abláta fúerit consolátio, non statim despéres~: sed cum humilitáte et patiéntia exspécta cæléstem visitatiónem, quóniam potens est Deus, ampliórem tibi redonáre consolatiónem. Istud non est novum nec aliénum viam Dei expértis, quia in magnis sanctis et in antíquis prophétis fuit sæpe talis alternatiónis modus.}
\Normal{\Verse{5.} Unde quidam, præsénte iam grátia, dicébat~: Ego dixi in abundántia mea~: Non movébor in ætérnum. Absénte vero grátia, quid in se fúerit expértus, adiúngit dicens~: Avertísti fáciem tuam a me, et factus sum conturbátus. Inter hæc tamen nequáquam despérat, sed instántius Dóminum rogat et dicit~: Ad te, Dómine, clamábo et ad Deum meum deprecábor. Dénique oratiónis suæ fructum repórtat et se exaudítum testátur dicens~: Audívit Dóminus et misértus est mei~; Dóminus factus est adiútor meus. Sed in quo~? Convertísti, inquit, planctum meum in gáudium mihi, et circumdedísti me lætítia. Si sic actum est cum magnis sanctis, non est desperándum nobis infírmis et paupéribus, si intérdum in fervóre et intérdum in frigiditáte sumus~: quóniam spíritus venit et recédit secúndum suæ beneplácitum voluntátis. Unde beátus Iob ait~: Vísitas eum dilúculo, et súbito probas illum.}
\Normal{\Verse{6.} Super quid ígitur speráre possum, aut in quo confídere débeo, nisi in sola magna misericórdia Dei et in sola spe grátiæ cæléstis~? Sive enim assint hómines boni, sive devóti fratres, vel amíci fidéles, sive libri sancti, vel tractátus pulchri, sive dulcis cantus et hymni~: ómnia hæc módicum iuvant, módicum sápiunt, quando desértus sum a grátia et in própria paupertáte relíctus. Tunc non est mélius remédium, quam patiéntia et abnegátio mei in voluntáte Dei.}
\Normal{\Verse{7.} Numquam invéni áliquem tam religiósum et devótum, qui non habúerit intérdum grátiæ subtractiónem, aut non sénserit fervóris diminutiónem. Nullus sanctus fuit tam alte raptus et illuminátus, qui prius vel póstea non fúerit temptátus. Non enim dignus est alta Dei contemplatióne, qui pro Deo non est exercitátus áliqua tribulatióne. Solet enim sequéntis consolatiónis temptátio præcédens esse signum. Nam temptatiónibus probátis cæléstis promíttitur consolátio. Qui vícerit, inquit, dabo ei édere de ligno vitæ.}
\Normal{\Verse{8.} Datur autem consolátio divína, ut homo fórtior sit ad sustinéndum advérsa. Séquitur étiam temptátio, ne se élevet de bono. Non dormit diábolus, nec caro adhuc mórtua est~: ídeo non cesses te præparáre ad certámen, quia a dextris et a sinístris hostes sunt, qui numquam quiéscunt.}
\markright{X}
%LIBER II
\TitreC{CAPUT X}
\TitreC{De gratitúdine pro grátia Dei}
\Normal{\Verse{1.} Cur quæris quiétem, cum natus sis ad labórem~? Pone te ad patiéntiam magis, quam ad consolatiónes, et ad crucem portándam magis, quam ad lætítiam. Quis enim sæculárium non libénter consolatiónem et lætítiam spirituálem accíperet, si semper obtinére posset~? Excédunt enim spirituáles consolatiónes omnes mundi delícias et carnis voluptátes. Nam omnes delíciæ mundánæ aut vanæ sunt aut turpes. Spirituáles vero delíciæ solæ iucúndæ et honéstæ, ex virtútibus progénitæ et a Deo puris méntibus infúsæ. Sed istis divínis consolatiónibus nemo semper pro suo afféctu frui valet, quia tempus temptatiónis non diu cessat.}
\Normal{\Verse{2.} Multum autem contrariátur supérnæ visitatióni falsa libértas ánimi et magna confidéntia sui. Deus bene facit consolatiónis grátiam dando, sed homo male agit, non totum Deo cum gratiárum actióne retribuéndo. Et ídeo non possunt in nobis dona grátiæ flúere, quia ingráti sumus auctóri, nec totum refúndimus fontáli orígini. Semper enim debétur grátia digne grátias referénti, et auferétur ab eláto, quod dari solet húmili.}
\Normal{\Verse{3.} Nolo consolatiónem, quæ mihi aufert compunctiónem~; nec affécto contemplatiónem, quæ ducit in elatiónem. Non enim omne altum sanctum, nec omne dulce bonum, nec omne desidérium purum, nec omne carum Deo gratum. Libénter accépto grátiam, unde semper humílior et timorátior invéniar, atque ad relinquéndum me parátior fiam. Doctus dono grátiæ et erudítus subtractiónis vérbere non sibi audébit quicquam boni attribúere, sed pótius se páuperem et nudum confitébitur. Da Deo, quod Dei est, et tibi ascribe, quod tuum est~; hoc est~: Deo grátias pro grátia tríbue, tibi autem soli culpam et dignam pœnam pro culpa debéri séntias.}
\Normal{\Verse{4.} Pone te semper ad ínfimum, et dábitur tibi summum~; nam summum non stat sine ínfimo. Summi sancti apud Deum mínimi sunt apud se, et quanto gloriosióres, tanto in se humilióres. Pleni veritáte et glória cælésti non sunt vanæ glóriæ cúpidi. In Deo fundáti et confirmáti nullo modo possunt esse eláti. Et qui totum Deo ascríbunt, quidquid boni accepérunt, glóriam ab ínvicem non quærunt, sed glóriam, quæ a solo Deo est, volunt, et Deum in se et in ómnibus sanctis laudári super ómnia cúpiunt et semper in id ipsum tendunt.}
\Normal{\Verse{5.} Esto ígitur gratus pro mínimo, et eris dignus maióra accípere. Sit tibi mínimum étiam pro máximo, et magis contemptíbile pro speciáli dono. Si dígnitas datóris inspícitur, nullum datum parvum aut nimis vile vidébitur. Non enim parvum est, quod a summo Deo donátur. Etiam si pœnas et vérbera déderit, gratum esse debet, quia semper pro salúte nostra facit, quidquid nobis adveníre permíttit. Qui grátiam Dei retinére desíderat, sit gratus pro grátia data, pátiens pro subláta. Oret, ut rédeat~; cautus sit et húmilis, ne amíttat.}
\markright{XI}
%LIBER II
\TitreC{CAPUT XI}
\TitreC{De paucitáte amatórum crucis Iesu}
\Normal{\Verse{1.} Habet Iesus nunc multos amatóres regni sui cæléstis, sed paucos baiulatóres suæ crucis. Multos habet desideratóres consolatiónis, sed paucos tribulatiónis. Plures ínvenit sócios mensæ, sed paucos abstinéntiæ. Omnes cúpiunt cum eo gaudére, pauci volunt pro eo áliquid sustinére. Multi Iesum sequúntur usque ad fractiónem panis, sed pauci usque ad bibéndum cálicem passiónis. Multi mirácula eius venerántur, pauci ignomíniam crucis sequúntur. Multi Iesum díligunt, quámdiu advérsa non contíngunt. Multi illum laudant et benedícunt, quámdiu consolatiónes áliquas ab ipso percípiunt. Si autem Iesus se abscónderit et módicum eos relíquerit, aut in querimóniam vel in deiectiónem nímiam cadunt.}
\Normal{\Verse{2.} Qui autem Iesum propter Iesum et non propter suam própriam áliquam consolatiónem díligunt, ipsum in omni tribulatióne et angústia cordis, sicut in summa consolatióne, benedícunt. Et si numquam eis consolatiónem dare vellet, ipsum tamen semper laudárent et semper grátias ágere vellent.}
\Normal{\Verse{3.} O quantum potest amor Iesu purus, nullo próprio cómmodo vel amóre permíxtus~! Nonne omnes mercenárii sunt dicéndi, qui consolatiónes semper quærunt~? Nonne amatóres sui magis quam Christi probántur, qui sua cómmoda et lucra semper meditántur~? Ubi inveniétur talis, qui velit Deo servíre gratis?}
\Normal{\Verse{4.} Raro invenítur tam spirituális áliquis, qui ómnibus sit nudátus. Nam verum páuperem spíritu et ab omni creatúra nudum, quis invéniet~? Procul et de últimis fínibus prétium eius. Si déderit homo omnem substántiam suam, adhuc nihil est. Et si fécerit pæniténtiam magnam, adhuc exíguum est. Et si apprehénderit omnem sciéntiam, adhuc longe est. Et si habúerit virtútem magnam et devotiónem nimis ardéntem, adhuc multum sibi deest~: scílicet unum, quod summe sibi necessárium est. Quid illud~? Ut ómnibus relíctis se relínquat et a se totáliter éxeat, nihílque de priváto amóre retíneat. Cumque ómnia fécerit, quæ faciénda nóverit, nil se fecísse séntiat.}
\Normal{\Verse{5.} Non grande pónderet, quod grande æstimári possit, sed in veritáte servum inútilem se pronúntiet, sicut Véritas ait~: Cum fecéritis ómnia, quæ præcépta sunt vobis, dícite~: Servi inútiles sumus. Tunc vere pauper et nudus spíritu esse póterit, et cum prophéta dícere~: Quia únicus et pauper sum ego. Nemo tamen isto dítior, nemo poténtior, nemo libérior, qui se et ómnia relínquere scit et ad ínfimum se pónere.}
\markright{XII}
%LIBER II
\TitreC{CAPUT XII}
\TitreC{De régia via sanctæ crucis}
\Normal{\Verse{1.} Durus multis vidétur hic sermo~: Abnega temetípsum, tolle crucem tuam et séquere Iesum. Sed multo dúrius erit, audíre illud extrémum verbum~: Discédite a me, maledícti, in ignem ætérnum. Qui enim modo libénter áudiunt et sequúntur verbum crucis, tunc non timébunt ab auditióne ætérnæ damnatiónis. Hoc signum crucis erit in cælo, cum Dóminus ad iudicándum vénerit. Tunc omnes servi crucis, qui se Crucifíxo conformavérunt in vita, accédent ad Christum iúdicem cum magna fidúcia.}
\Normal{\Verse{2.} Quid ígitur times tóllere crucem, per quam itur ad regnum~? In cruce salus, in cruce vita, in cruce protéctio ab hóstibus~; in cruce infúsio supérnæ suavitátis, in cruce robur mentis, in cruce gáudium spíritus~; in cruce summa virtútis, in cruce perféctio sanctitátis. Non est salus ánimæ nec spes ætérnæ vitæ, nisi in cruce. Tolle ergo crucem tuam et séquere Iesum, et ibis in vitam ætérnam. Præcéssit ille báiulans sibi crucem et mórtuus est pro te in cruce, ut et tu tuam portes crucem, et mori afféctes in cruce. Quia, si commórtuus fúeris, étiam cum illo páriter vives. Et si sócius fúeris pœnæ, eris et glóriæ.}
\Normal{\Verse{3.} Ecce, in cruce totum constat, et in moriéndo totum iacet~; et non est ália via ad vitam et ad veram intérnam pacem, nisi via sanctæ crucis et cotidiánæ mortificatiónis. Ambula, ubi vis~; quære, quodcúmque volúeris~: et non invénies altiórem viam supra nec securiórem viam infra, nisi viam sanctæ crucis. Dispóne et órdina ómnia secúndum tuum velle et vidére~: et non invénies, nisi semper áliquid pati debére aut sponte aut invíte, et ita crucem semper invénies. Aut enim in córpore dolórem sénties, aut in ánima spíritus tribulatiónem sustinébis.}
\Normal{\Verse{4.} Intérdum a Deo relinquéris, intérdum a próximo exercitáberis, et quod ámplius est, sæpe tibimet ipsi gravis eris~; nec tamen áliquo remédio vel solácio liberári seu alleviári póteris, sed, donec Deus volúerit, opórtet, ut sustíneas. Vult enim Deus, ut tribulatiónem sine consolatióne pati discas, et ut illi totáliter te subícias et humílior ex tribulatióne fias. Nemo ita cordiáliter sentit passiónem Christi, sicut is, cui contígerit simília pati. Crux ergo semper paráta est et ubíque te exspéctat. Non potes effúgere, ubicúmque cucúrreris~: quia, ubicúmque véneris, teípsum tecum portas, et semper teípsum invénies. Convérte te supra, convérte te infra~: convérte te extra, convérte te intra~: et in his ómnibus invénies crucem, et necésse est, te ubíque tenére patiéntiam, si intérnam vis habére pacem et perpétuam promeréri corónam.}
\Normal{\Verse{5.} Si libénter crucem portas, portábit te et ducet ad desiderátum finem, ubi scílicet finis patiéndi erit, quamvis hic non erit. Si invíte portas, onus tibi facis, et teípsum magis gravas~; et tamen opórtet, ut sustíneas. Si ábicis unam crucem, áliam procul dúbio invénies et fórsitan graviórem.}
\Normal{\Verse{6.} Credis tu evádere, quod nullus mortálium pótuit præteríre~? Quis sanctórum in mundo sine cruce et tribulatióne fuit~? Nec enim Iesus Christus, Dóminus noster, una hora sine dolóre passiónis fuit, quámdiu vixit. Oportébat, ait, Christum pati, et resúrgere a mórtuis, et ita intráre in glóriam suam. Et quómodo tu áliam viam quæris, quam hanc régiam viam, quæ est via sanctæ crucis?}
\Normal{\Verse{7.} Tota vita Christi crux fuit et martýrium~; et tu tibi quæris réquiem et gáudium~? Erras, erras, si áliud quæris, quam pati tribulatiónes~; quia tota ista vita mortális plena est misériis et circumsignáta crúcibus. Et quanto áltius quis in spíritu profécerit, tanto gravióres sæpe cruces invénerit~: quia exílii sui pœna magis ex amóre crescit.}
\Normal{\Verse{8.} Sed tamen iste sic multiplíciter afflíctus non est sine levámine consolatiónis, quia fructum máximum sibi sentit accréscere ex sufferéntia suæ crucis. Nam dum sponte se illi súbicit, omne onus tribulatiónis in fidúciam divínæ consolatiónis convértitur. Et quanto caro magis per afflictiónem attéritur, tanto spíritus ámplius per intérnam grátiam roborátur. Et nonnúmquam in tantum confortátur ex afféctu tribulatiónis et adversitátis, ob amórem conformitátis crucis Christi, ut se sine dolóre et tribulatióne esse non vellet~; quóniam tanto se acceptiórem Deo credit, quanto plura et gravióra pro eo perférre potúerit. Non est istud hóminis virtus, sed grátia Christi, quæ tanta potest et agit in carne frágili, ut, quod naturáliter semper abhórret et fugit, hoc fervóre spíritus aggrediátur et díligat.}
\Normal{\Verse{9.} Non est secúndum hóminem crucem portáre, crucem amáre, corpus castigáre et servitúti subícere, honóres fúgere, contumélias libénter sustinére, seípsum despícere et déspici optáre, advérsa quæque cum damnis pérpeti et nihil prosperitátis in hoc mundo desideráre. Si ad teípsum réspicis, nihil huiúsmodi ex te póteris. Sed si in Dómino confídis, dábitur tibi fortitúdo de cælo, et subiciéntur ditióni tuæ mundus et caro. Sed nec inimícum diábolum timébis, si fúeris fide armátus et cruce Christi signátus.}
\Normal{\Verse{10.} Pone te ergo, sicut bonus et fidélis servus Christi, ad portándam viríliter crucem Dómini tui, pro te ex amóre crucifíxi. Prǽpara te ad toleránda multa advérsa et vária incómmoda in hac mísera vita~; quia sic tecum erit, ubicúmque fúeris~; et sic revéra invénies, ubicúmque latúeris. Opórtet ita esse~; et non est remédium evadéndi a tribulatióne malórum et dolóre, quam ut te patiáris. Cálicem Dómini affectanter bibe, si amícus eius esse et partem cum eo habére desíderas. Consolatiónes Deo commítte~; fáciat ipse cum tálibus, sicut sibi magis placúerit. Tu vero, pone te ad sustinéndum tribulatiónes, et réputa eas máximas consolatiónes, quia non sunt condígnæ passiónes huius témporis ad futúram glóriam promeréndam, étiam si solus omnes posses sustinére.}
\Normal{\Verse{11.} Quando ad hoc véneris, quod tribulátio tibi dulcis est et sapit pro Christo~: tunc bene tecum esse ǽstima, quia invenísti paradísum in terra. Quámdiu pati grave tibi est et fúgere quæris~: tam diu male habébis, et sequétur te ubíque fuga tribulatiónis.}
\Normal{\Verse{12.} Si ponis te, ad quod esse debes, vidélicet ad patiéndum et moriéndum, fiet cito mélius, et pacem invénies. Etiam si raptus fúeris usque ad tértium cælum cum Paulo, non es proptérea securatus de nullo contrário patiéndo. Ego, inquit Iesus, osténdam illi, quanta opórteat eum pro nómine meo pati. Pati ergo tibi rémanet, si Iesum dilígere et perpétuo illi servíre placet.}
\Normal{\Verse{13.} Utinam dignus esses áliquid pro nómine Iesu pati~; quam magna glória remanéret tibi, quanta exsultátio ómnibus sanctis Dei, quanta quoque ædificátio esset próximi~! Nam patiéntiam omnes recommendant, quamvis pauci tamen pati velint. Mérito debéres libénter módicum pati pro Christo, cum multi gravióra patiúntur pro mundo.}
\Normal{\Verse{14.} Scias pro certo, quia moriéntem te opórtet dúcere vitam. Et quanto quisque plus sibi móritur, tanto magis Deo vívere íncipit. Nemo aptus est ad comprehendéndum cæléstia, nisi se submíserit ad portándum pro Christo advérsa. Nihil Deo accéptius, nihil tibi salúbrius in mundo isto, quam libénter pati pro Christo. Et si eligéndum tibi esset, magis optáre debéres pro Christo advérsa pati, quam multis consolatiónibus recreári~; quia Christo simílior esses et ómnibus sanctis magis confórmior. Non enim stat méritum nostrum et proféctus status nostri in multis suavitátibus et consolatiónibus, sed pótius in magnis gravitátibus et tribulatiónibus perferéndis.}
\Normal{\Verse{15.} Si quidem áliquid mélius et utílius salúti hóminum, quam pati, fuísset, Christus útique verbo et exémplo ostendísset. Nam et sequéntes se discípulos omnésque eum sequi cupiéntes maniféste ad crucem portándam hortátur, et dicit~: Si quis vult veníre post me, ábneget semetípsum, et tollat crucem suam, et sequátur me. Omnibus ergo perléctis et scrutátis, sit hæc conclúsio finális~: Quóniam per multas tribulatiónes opórtet nos intráre in regnum Dei.}
\markboth{III}{}
LIBER III
DE INTERNA CONSOLATIONE
\markright{I}
%LIBER III
\TitreC{CAPUT I}
\TitreC{De intérna Christi locutióne ad ánimam fidélem}
\Normal{\Verse{1.} Audiam, quid loquátur in me Dóminus Deus. Beáta ánima, quæ Dóminum in se loquéntem audit, et de ore eius consolatiónis verbum áccipit. Beátæ aures, quæ venas divíni susúrri suscípiunt, et de mundi huius susurratiónibus nihil advértunt. Beátæ plane aures, quæ non vocem foris sonántem, sed intus auscúltant veritátem docéntem. Beáti óculi, qui exterióribus clausi, interióribus autem sunt inténti. Beáti, qui intérna pénetrant, et ad capiénda arcána cæléstia magis ac magis per cotidiána exercítia se student præparáre. Beáti, qui Deo vacáre géstiunt, et ab omni impediménto sǽculi se excútiunt. Animadvérte hæc, ánima mea, et claude sensualitátis tuæ óstia, ut possis audíre, quid in te loquátur Dóminus Deus tuus.}
\Normal{\Verse{2.} Hæc dicit diléctus tuus~: Salus tua ego sum, pax tua et vita tua. Serva te apud me, et pacem invénies. Dimítte ómnia transitória, quære ætérna. Quid sunt ómnia temporália, nisi seductória~? Et quid iuvant omnes creatúræ, si fúeris a creatóre desérta~? Omnibus ergo abdicátis creatóri tuo te redde plácitam ac fidélem, ut veram váleas apprehéndere beatitúdinem.}
\markright{II}
%LIBER III
\TitreC{CAPUT II}
\TitreC{Quod véritas intus lóquitur sine strépitu verbórum}
\Normal{\Verse{1.} Lóquere, Dómine, quia audit servus tuus. Servus tuus sum ego~; da mihi intelléctum, ut sciam testimónia tua. Inclína cor meum in verba oris tui~: fluat ut ros elóquium tuum. Dicébant olim fílii Israel ad Móysen~: Lóquere tu nobis, et audiémus~; non loquátur nobis Dóminus, ne forte moriámur. Non sic, Dómine, non sic oro, sed magis cum Samuéle prophéta humíliter ac desideránter óbsecro~: Lóquere, Dómine, quia audit servus tuus. Non loquátur mihi Móyses aut áliquis ex prophétis~: sed tu pótius lóquere, Dómine Deus, inspirátor et illuminátor ómnium prophetárum~; quia tu solus sine eis potes me perfécte imbúere. illi autem sine te nihil profícient.}
\Normal{\Verse{2.} Possunt quidem verba sonáre, sed spíritum non cónferunt. Pulchriter dicunt, sed te tacénte cor non accéndunt. Lítteras tradunt, sed tu sensum áperis. Mystéria próferunt, sed tu réseras intelléctum signatórum. Mandáta edícunt, sed tu iuvas ad perficiéndum. Viam osténdunt, sed tu confórtas ad ambulándum. Illi foris tantum agunt, sed tu corda ínstruis et illúminas. Illi extérius rigant, sed tu fecunditátem donas. Illi clamant verbis, sed tu audítui intellegéntiam tríbuis.}
\Normal{\Verse{3.} Non ergo loquátur mihi Móyses, sed tu, Dómine Deus meus, ætérna véritas, ne forte móriar et sine fructu effíciar, si fúero tantum foris admónitus et intus non accénsus~; ne sit mihi ad iudícium verbum audítum et non factum, cógnitum nec amátum, créditum et non servátum. Lóquere ígitur, Dómine, quia audit servus tuus~; verba enim vitæ ætérnæ habes. Lóquere mihi ad qualemcúmque ánimæ meæ consolatiónem et ad totíus vitæ meæ emendatiónem, tibi autem ad laudem et glóriam et perpétuum honórem.}
\markright{III}
%LIBER III
\TitreC{CAPUT III}
\TitreC{Quod verba Dei cum humilitáte sunt audiénda, et quod multi ea non pónderant}
\Normal{\Verse{1.} Audi, fili, verba mea, verba suavíssima, omnem philosophórum et sapiéntium huius mundi sciéntiam excedéntia. Verba mea spíritus et vita sunt, nec humáno sensu pensánda. Non sunt ad vanam complacéntiam trahénda, sed in siléntio audiénda, et cum omni humilitáte atque magno afféctu suscipiénda.}
\Normal{\Verse{2.} Et dixi~: Beátus, quem tu erudíeris, Dómine, et de lege tua docúeris eum, ut mítiges ei a diébus malis, et non desolétur in terra.}
\Normal{\Verse{3.} Ego, inquit Dóminus, dócui prophétas ab inítio, et usque nunc non cesso ómnibus loqui~; sed multi ad vocem meam surdi sunt et duri. Plures mundum libéntius áudiunt, quam Deum~; facílius sequúntur carnis suæ appetítum, quam Dei beneplácitum. Promíttit mundus temporália et parva, et servítur ei aviditáte magna~: ego promítto summa et ætérna, et torpéscunt mortálium corda. Quis tanta cura mihi in ómnibus servit et obœ́dit, sicut mundo et dóminis eius servítur~? Erubésce, Sidon, ait mare, et si causam quæris, audi, quare. Pro módica præbénda longa via cúrritur~; pro ætérna vita a multis vix pes semel a terra levátur. Vile prétium quǽritur~; pro uno numísmate intérdum túrpiter litigátur~; pro vana re et parva promissióne die noctúque fatigári non timétur.}
\Normal{\Verse{4.} Sed, pro pudor~! pro bono incommutábili, pro prǽmio inæstimábili, pro summo honóre et glória interminábili vel ad módicum fatigári pigritátur. Erubésce ergo, serve piger et querulóse, quod illi paratióres inveniúntur ad perditiónem, quam tu ad vitam. Gaudent illi ámplius ad vanitátem, quam tu ad veritátem. Equidem a spe sua nonnúmquam frustrántur, sed promíssio mea néminem fallit, nec confidéntem mihi dimíttit inánem. Quod promísi, dabo~; quod dixi, implébo~; si tamen usque in finem fidélis in dilectióne mea quis permánserit. Ego remunerátor sum ómnium bonórum, et fortis probátor ómnium devotórum.}
\Normal{\Verse{5.} Scribe verba mea in corde tuo, et pertrácta diligénter~; erunt enim in témpore temptatiónis valde necessária. Quod non intéllegis, cum legis, cognósces in die visitatiónis. Duplíciter sóleo eléctos meos visitáre, temptatióne scílicet et consolatióne. Et duas lectiónes eis cotídie lego~: unam increpándo eórum vítia, álteram exhortándo ad virtútum increménta. Qui habet verba mea et spernit ea, habet, qui iúdicet eum in novíssimo die.}
\TitreD{Orátio ad implorándam devotiónis grátiam}
\Normal{\Verse{6.} Dómine Deus meus, ómnia bona mea tu es. Et quis ego sum, ut áudeam ad te loqui~? Ego sum paupérrimus sérvulus tuus, et abiéctus vermículus, multo paupérior et contemptibílior, quam scio et dícere áudeo. Meménto, tamen, Dómine, quia nihil sum, nihil hábeo, nihílque váleo. Tu solus bonus, iustus et sanctus~; tu ómnia potes, ómnia præstas, ómnia imples, solum peccatórem inánem relínquens. Reminíscere miseratiónum tuárum, et imple cor meum grátia tua, qui non vis esse vácua ópera tua.}
\Normal{\Verse{7.} Quómodo possum me toleráre in hac mísera vita, nisi me confortáveris misericórdia et grátia tua~? Noli avértere fáciem tuam a me~; noli visitatiónem tuam prolongáre~; noli consolatiónem tuam abstráhere, ne fiat ánima mea sicut terra sine aqua tibi. Dómine, doce me fácere voluntátem tuam~; doce me coram te digne et humíliter conversári~; quia sapiéntia mea tu es, qui in veritáte me cognóscis, et cognovísti, ántequam fíeret mundus, et ántequam natus essem in mundo.}
\markright{IV}
%LIBER III
\TitreC{CAPUT IV}
\TitreC{Quod in veritáte et humilitáte coram Deo conversándum est}
\Normal{\Verse{1.} Fili, ámbula coram me in veritáte, et in simplicitáte cordis tui quære me semper. Qui ámbulat coram me in veritáte, tutábitur ab incúrsibus malis, et véritas liberábit eum a seductóribus et detractiónibus iniquórum. Si véritas te liberáverit, vere liber eris, et non curábis de vanis hóminum verbis. Dómine, verum est, sicut dicis~; ita, quæso, mecum fiat. Véritas tua me dóceat, ipsa me custódiat et usque ad salutárem finem consérvet. Ipsa me líberet ab omni affectióne mala et inordináta dilectióne~; et ambulábo tecum in magna cordis libertáte.}
\Normal{\Verse{2.} Ego te docébo, ait véritas, quæ recta sunt et plácita coram me. Cógita peccáta tua cum displicéntia magna et mæróre~; et numquam réputes, te áliquid esse propter ópera bona. Revéra peccátor es et multis passiónibus obnóxius et implicátus. Ex te semper ad nihil tendis~; cito láberis, cito vínceris, cito turbáris, cito dissólveris. Non habes quicquam, unde possis gloriári, sed multa, unde te débeas vilificáre~; quia multo infírmior es, quam vales comprehéndere.}
\Normal{\Verse{3.} Nil ergo magnum tibi videátur ex ómnibus, quæ agis. Nil grande, nil pretiósum et admirábile, nil reputatióne appáreat dignum~; nil altum, nil vere laudábile et desiderábile, nisi quod ætérnum est. Pláceat tibi super ómnia ætérna véritas, displíceat tibi semper tua máxima vílitas. Nil sic tímeas, sic vitúperes et fúgias, sicut vítia et peccáta tua, quæ magis displicére debent, quam quǽlibet rerum damna. Quidam non sincére coram me ámbulant, sed quadam curiositáte et arrogántia ducti volunt secréta mea scire et alta Dei intellégere, se et suam salútem neglegéntes. Hi sæpe in magnas temptatiónes et peccáta propter suam supérbiam et curiositátem, me eis adversánte, labúntur.}
\Normal{\Verse{4.} Time iudícia Dei, expavésce iram omnipoténtis. Noli autem discútere ópera altíssimi, sed tuas iniquitátes perscrutáre, in quantis deliquísti, et quam multa bona neglexísti. Quidam solum portant suam devotiónem in libris, quidam in imagínibus, quidam autem in signis exterióribus et figúris. Quidam habent me in ore, sed módicum est in corde. Sunt álii, qui intelléctu illumináti et afféctu purgáti ad ætérna semper anhélant, de terrénis gráviter áudiunt, necessitátibus natúræ dolénter insérviunt~; et hi séntiunt, quid spíritus veritátis lóquitur in eis. Quia docet eos terréna despícere et amáre cæléstia, mundum neglégere et cælum tota die ac nocte desideráre.}
\markright{V}
%LIBER III
\TitreC{CAPUT V}
\TitreC{De mirábili efféctu divíni amóris}
\Normal{\Verse{1.} Benedíco te, Pater cæléstis, Pater Dómini mei Iesu Christi, quia mei páuperis dignátus es recordári. O Pater misericordiárum et Deus totíus consolatiónis, grátias tibi, qui me indígnum omni consolatióne quandóque tua récreas consolatióne. Benedíco te semper et glorífico, cum unigénito Fílio tuo et Spíritu Sancto paráclito in sǽcula sæculórum. Eia, Dómine Deus, amátor sancte meus, cum tu véneris in cor meum, exsultábunt ómnia interióra mea. Tu es glória mea et exsultátio cordis mei. Tu spes mea et refúgium meum in die tribulatiónis meæ.}
\Normal{\Verse{2.} Sed quia adhuc débilis sum in amóre et imperféctus in virtúte, ídeo necésse hábeo a te confortári et consolári~; proptérea vísita me sǽpius et ínstrue disciplínis sanctis. Líbera me a passiónibus malis, et sana cor meum ab ómnibus affectiónibus inordinátis, ut intus sanátus et bene purgátus, aptus effíciar ad amándum, fortis ad patiéndum, stábilis ad perseverándum.}
\Normal{\Verse{3.} Magna res est amor, magnum omníno bonum, quod solum leve facit omne onerósum et fert æquáliter omne inæquále. Nam onus sine ónere portat et omne amárum dulce ac sápidum éfficit. Amor Iesu nóbilis ad magna operánda impéllit et ad desideránda semper perfectióra éxcitat. Amor vult esse sursum, nec ullis ínfimis rebus retinéri. Amor vult esse liber et ab omni mundána affectióne aliénus, ne intérnus eius impediátur aspéctus~; ne per áliquod cómmodum temporále implicatiónes sustíneat aut per incómmodum succúmbat. Nihil dúlcius est amóre, nihil fórtius, nihil áltius, nihil látius, nihil iucúndius, nihil plénius nec mélius in cælo et in terra~: quia amor ex Deo natus est, nec potest, nisi in Deo, super ómnia creáta quiéscere.}
\Normal{\Verse{4.} Amans volat, currit et lætátur~; liber est et non tenétur. Dat ómnia pro ómnibus, et habet ómnia in ómnibus~; quia in uno summo super ómnia quiéscit, ex quo omne bonum fluit et procédit. Non réspicit ad dona, sed ad donántem se convértit super ómnia bona. Amor modum sæpe nescit, sed super omnem modum fervéscit. Amor onus non sentit, labóres non réputat~; plus afféctat, quam valet~; de impossibilitáte non causátur, quia cuncta sibi posse et licére arbitrátur. Valet ígitur ad ómnia, et multa implet et efféctui máncipat, ubi non amans déficit et iacet.}
\Normal{\Verse{5.} Amor vígilat et dórmiens non dormítat. Fatigátus non lassátur, arctátus non arctátur, térritus non conturbátur~; sed sicut vivax flamma et ardens fácula sursum erúmpit securéque pertránsit. Si quis amat, novit, quid hæc vox clamet. Magnus clamor in áuribus Dei est ipse ardens afféctus ánimæ, quæ dicit~: Deus meus, amor meus, tu totus meus, et ego totus tuus.}
\Normal{\Verse{6.} Diláta me in amóre, ut discam interióri cordis ore degustáre, quam suáve sit amáre et in amóre liquefíeri et natáre. Ténear amóre, vadens supra me, præ nímio fervóre et stupóre. Cantem amóris cánticum, sequar te diléctum meum in altum, defíciat in laude tua ánima mea, iúbilans ex amóre. Amem te plus quam me, nec me nisi propter te, et omnes in te, qui vere amant te, sicut iubet lex amóris lucens ex te.}
\Normal{\Verse{7.} Est amor velox, sincérus, pius, iucúndus et amœ́nus~; fortis, pátiens, fidélis, prudens, longánimis, virílis et seípsum numquam quærens. Ubi enim seípsum áliquis quærit, ibi ab amóre cadit. Est amor circumspéctus, húmilis et rectus, non mollis, non levis, nec vanis inténdens rebus~; sóbrius, castus, stábilis, quiétus et in cunctis sénsibus custodítus. Est amor subiéctus et obœ́diens prælátis, sibi vilis et despéctus, Deo devótus et gratíficus, fidens et sperans semper in eo, étiam cum sibi non sapit Deus~: quia sine dolóre non vívitur in amóre.}
\Normal{\Verse{8.} Qui non est parátus ómnia pati, et ad voluntátem stare dilécti, non est dignus amátor appellári. Opórtet amántem ómnia dura et amára propter diléctum libénter amplécti, nec ob contrária accidéntia ab eo deflécti.}
\markright{VI}
%LIBER III
\TitreC{CAPUT VI}
\TitreC{De probatióne veri amatóris}
\Normal{\Verse{1.} Fili, non es adhuc fortis et prudens amátor. Quare, Dómine? Quia propter módicam contrarietátem déficis a cœptis, et nimis ávide consolatiónem quæris. Fortis amátor stat in temptatiónibus, nec cállidis credit persuasiónibus inimíci. Sicut ei in prósperis pláceo, ita nec in advérsis displíceo.}
\Normal{\Verse{2.} Prudens amátor non tam donum amántis consíderat, quam dantis amórem. Afféctum pótius atténdit, quam censum, et infra diléctum ómnia data ponit. Nóbilis amátor non quiéscit in dono, sed in me super omne donum. Non est ídeo totum pérditum, si quandóque minus bene de me, vel de sanctis meis sentis, quam velles. Afféctus ille bonus et dulcis, quem intérdum pércipis, efféctus grátiæ præséntis est, et quidam prægustus pátriæ cæléstis, super quo non nímium inniténdum, quia vadit et venit. Certáre autem advérsus incidéntes malos motus ánimi, suggestionémque spérnere diáboli, insígne est virtútis et magni mériti.}
\Normal{\Verse{3.} Non ergo te contúrbent aliénæ phantásiæ de quacúmque matéria ingéstæ. Forte serva propósitum et intentiónem rectam ad Deum. Nec est illusio, quod aliquándo in excéssum súbito ráperis et statim ad sólitas inéptias cordis revérteris. Illas enim invíte magis páteris, quam agis~; et quámdiu dísplicent et reníteris, méritum est et non perdítio.}
\Normal{\Verse{4.} Scito, quod antíquus inimícus omníno nítitur impedíre desidérium tuum in bono et ab omni devóto exercítio evacuáre~: a sanctórum scílicet cultu, a pia passiónis meæ memória, a peccatórum útili recordatióne, a próprii cordis custódia et a firmo propósito proficiéndi in virtúte. Multas malas cogitatiónes íngerit, ut tǽdium tibi fáciat et horrórem~: ut ab oratióne révocet et sacra lectióne. Dísplicet sibi húmilis conféssio, et, si posset, a communióne cessáre fáceret. Non credas ei, neque cures illum, licet sǽpius tibi deceptiónis teténderit láqueos. Sibi ímputa, cum mala ínserit et immúnda. Dícito illi~: Vade, immúnde spíritus, erubésce, miser, valde immúndus es tu, qui tália infers áuribus meis. Discéde a me, sedúctor péssime, non habébis in me partem ullam~; sed Iesus mecum erit, tamquam bellátor fortis, et tu stabis confúsus. Malo mori et omnem pœnam subíre, quam tibi consentíre. Tace et obmutésce, non áudiam te ámplius, licet plures mihi moliáris moléstias. Dóminus illuminátio mea et salus mea, quem timébo~? Si consístant advérsum me castra, non timébit cor meum. Dóminus adiútor meus et redémptor meus.}
\Normal{\Verse{5.} Certa tamquam miles bonus~; et si intérdum ex fragilitáte córruis, resúme vires fortióres prióribus, confídens de amplióri grátia mea, et multum prǽcave a vana complacéntia et supérbia. Propter hoc multi in errórem ducúntur, et in cæcitátem pene incurábilem quandóque labúntur. Sit tibi in cautélam et perpétuam humilitátem ruína hæc superbórum de se stulte præsuméntium.}
\markright{VII}
%LIBER III
\TitreC{CAPUT VII}
\TitreC{De occultánda grátia sub humilitátis custódia}
\Normal{\Verse{1.} Fili, utílius est tibi et secúrius, devotiónis grátiam abscóndere, nec in altum te efférre, nec multum inde loqui, nec multum ponderáre~; sed magis temetípsum despícere et tamquam indígno datam timére. Non est huic affectióni tenácius inhæréndum, quæ cítius potest mutári in contrárium. Cógita in grátia, quam miser et inops esse soles sine grátia. Nec est in eo tantum spirituális vitæ proféctus, cum consolatiónis habúeris grátiam~; sed cum humíliter et abnegáte patientérque túleris eius subtractiónem~: ita, quod tunc ab oratiónis stúdio non tórpeas, nec réliqua ópera tua ex usu faciénda omníno dilábi permíttas, sed sicut mélius potúeris et intelléxeris, libénter quod in te est, fácias, nec propter ariditátem seu anxietátem mentis, quam sentis, te totáliter néglegas.}
\Normal{\Verse{2.} Multi enim sunt, qui, cum non bene eis succésserit, statim impatiéntes fiunt aut désides. Non enim semper est in potestáte hóminis via eius, sed Dei est dare et consolári, quando vult et quantum vult et cui vult, sicut sibi placúerit, et non ámplius. Quidam incáuti propter devotiónis grátiam seípsos destruxérunt, quia plus ágere voluérunt, quam potuérunt, non pensántes suæ parvitátis mensúram, sed magis cordis afféctum sequéntes, quam ratiónis iudícium. Et quia maióra præsumpsérunt, quam Deo plácitum fuit, idcírco grátiam cito perdidérunt. Facti sunt ínopes et viles relícti, qui in cælum posuérunt nidum sibi, ut humiliáti et depauperáti discant non in alis suis voláre, sed sub pennis meis speráre. Qui adhuc novi sunt et imperíti in via Dómini, nisi consílio discretórum se regant, facíliter décipi possunt et elídi.}
\Normal{\Verse{3.} Quod si suum sentíre magis sequi, quam áliis exercitátis crédere volunt, erit eis periculósus éxitus, si tamen rétrahi a próprio concéptu nolúerint. Raro sibi ipsis sapiéntes ab áliis regi humíliter patiúntur. Mélius est, sápere módicum cum humilitáte et parva intellegéntia, quam magni scientiárum thesáuri cum vana complacéntia. Mélius est tibi minus habére, quam multum, unde posses superbíre. Non satis discréte agit, qui se totum lætítiæ tradit, oblivíscens prístinæ inópiæ suæ et casti timóris Dómini, qui timet grátiam oblátam amíttere. Nec étiam satis virtuóse sapit, qui témpore adversitátis et cuiúsque gravitátis nimis desperáte se gerit, et minus fidénter de me, quam opórtet, recógitat ac sentit.}
\Normal{\Verse{4.} Qui témpore pacis nimis secúrus esse volúerit, sæpe témpore belli nimis deiéctus et formidolósus reperiétur. Si scires semper húmilis et módicus in te permanére, nec non spíritum tuum bene moderáre et régere, non incíderes tam cito in perículum et offénsam. Consílium bonum est, ut fervóris spíritu concépto meditéris, quid futúrum sit abscedénte lúmine. Quod dum contígerit, recógita et dénuo lucem posse revérti, quam ad cautélam tibi, mihi autem ad glóriam, ad tempus subtráxi.}
\Normal{\Verse{5.} Utílior est sæpe talis probátio, quam si semper próspera pro tua habéres voluntáte. Nam mérita non sunt ex hoc existimánda, si quis plures visiónes aut consolatiónes hábeat, vel si perítus sit in scriptúris, aut in altióri ponátur gradu~: sed si vera fúerit humilitáte fundátus et divína caritáte replétus~; si Dei honórem pure et íntegre semper quærat~; si seípsum nihil réputet et in veritáte despíciat, atque ab áliis étiam déspici et humiliári magis gáudeat, quam honorári.}
\markright{VIII}
%LIBER III
\TitreC{CAPUT VIII}
\TitreC{De vili æstimatióne sui ipsíus in óculis Dei}
\Normal{\Verse{1.} Loquar ad Dóminum meum, cum sim pulvis et cinis. Si me ámplius reputávero, ecce, tu stas contra me, et dicunt testimónium verum iniquitátes meæ, nec possum contradícere. Si autem me vilificávero et ad níhilum redégero, et ab omni própria reputatióne defécero, atque, sicut sum, pulverizávero, erit mihi propítia grátia tua et vicína cordi meo lux tua~; et omnis æstimátio, quantulacúmque mínima, in valle nihileitatis meæ submergétur et períbit in ætérnum. Ibi osténdis me mihi, quid sum, quid fui, et quo devéni~: quia nihil sum, et nescívi. Si mihi ipsi relínquor, ecce nihil et tota infírmitas~; si autem súbito me respéxeris, statim fortis effícior, et novo répleor gáudio. Et mirum valde, quod sic repénte súblevor et tam benígne a te compléctor, qui próprio póndere semper ad ima feror.}
\Normal{\Verse{2.} Facit hoc amor tuus, gratis prævéniens me, et in tam multis subvéniens necessitátibus, a grávibus quoque custódiens me perículis, et ab innúmeris, ut vere dicam, erípiens malis. Me síquidem male amándo me pérdidi~; et te solum quæréndo et pure amándo me et te páriter invéni, atque ex amóre profúndius ad níhilum me redégi. Quia tu, o dulcíssime, facis mecum supra méritum omne, et supra id, quam áudeo speráre vel rogáre.}
\Normal{\Verse{3.} Benedíctus sis, Deus meus, quia, licet ego ómnibus bonis sim indígnus, tua tamen nobílitas et infiníta bónitas numquam cessat benefácere étiam ingrátis et longe a te avérsis. Convérte nos ad te, ut simus grati, húmiles et devóti, quia salus nostra tu es, virtus et fortitúdo nostra.}
\markright{IX}
%LIBER III
\TitreC{CAPUT IX}
\TitreC{Quod ómnia ad Deum, sicut ad finem últimum, sunt referénda}
\Normal{\Verse{1.} Fili, ego débeo esse finis tuus suprémus et ultimátus, si vere desíderas esse beátus. Ex hac intentióne purificábitur afféctus tuus, sǽpius ad seípsum et ad creatúras male incurvátus. Nam si teípsum in áliquo quæris, statim in te déficis et aréscis. Omnia ergo ad me principáliter réferas, quia ego sum, qui ómnia dedi. Sic síngula consídera, sicut ex summo bono manántia~; et ídeo ad me, tamquam ad suam oríginem, cuncta sunt reducénda.}
\Normal{\Verse{2.} Ex me pusíllus et magnus, pauper et dives, tamquam ex fonte vivo, aquam vivam háuriunt~; et qui mihi sponte et líbere desérviunt, grátiam pro grátia accípient. Qui autem extra me volúerit gloriári, vel in áliquo priváto bono delectári, non stabiliétur in vero gáudio neque in corde suo dilatábitur, sed multiplíciter impediétur et angustiábitur. Nihil ergo tibi de bono ascribere debes, nec alícui hómini virtútem attríbuas~; sed totum da Deo, sine quo nihil habet homo. Ego totum dedi, ego totum rehabere volo, et cum magna districtióne gratiárum actiónes requíro.}
\Normal{\Verse{3.} Hæc est véritas, qua fugátur glóriæ vánitas. Et si intráverit cæléstis grátia et vera cáritas, non erit áliqua invídia nec contráctio cordis, neque privátus amor occupábit. Vincit enim ómnia divína cáritas et dilátat omnes ánimæ vires. Si recte sapis, in me solo gaudébis, in me solo sperábis~; quia nemo bonus, nisi solus Deus, qui est super ómnia laudándus et in ómnibus benedicéndus.}
\markright{X}
%LIBER III
\TitreC{CAPUT X}
\TitreC{Quod spreto mundo dulce est servíre Deo}
\Normal{\Verse{1.} Nunc íterum loquar, Dómine, et non silébo~; dicam in áuribus Dei mei, Dómini mei et Regis mei, qui est in excélso. O quam magna multitúdo dulcédinis tuæ, Dómine, quam abscondísti timéntibus te~! Sed quid es amántibus~? quid toto corde tibi serviéntibus~? Vere ineffábilis dulcédo contemplatiónis tuæ, quam largíris amántibus te. In hoc máxime ostendísti mihi dulcédinem caritátis tuæ, quia, cum non essem, fecísti me, et cum errárem longe a te, reduxísti me, ut servírem tibi, et præcepísti, ut díligam te.}
\Normal{\Verse{2.} O fons amóris perpétui, quid dicam de te~? Quómodo pótero tui oblivísci, qui mei dignátus es recordári, étiam postquam contábui et périi~? Fecísti ultra omnem spem misericórdiam cum servo tuo, et ultra omne méritum grátiam et amicítiam exhibuísti. Quid retríbuam tibi pro grátia ista~? Non enim ómnibus datum est, ut ómnibus abdicátis sǽculo renúntient et monásticam vitam assúmant. Numquid magnum est, ut tibi sérviam, cui omnis creatúra servíre tenétur~? Non magnum mihi vidéri debet servíre tibi~; sed hoc pótius magnum mihi et mirándum appáret, quod tam páuperem et indígnum dignáris in servum recípere et diléctis servis tuis adunáre.}
\Normal{\Verse{3.} Ecce, ómnia tua sunt, quæ hábeo, et unde tibi sérvio. Verúmtamen vice versa, tu magis mihi servis, quam ego tibi. Ecce, cælum et terra, quæ in ministérium hóminis creásti, præsto sunt et fáciunt cotídie, quæcúmque mandásti. Et hoc parum est, quin étiam ángelos in ministérium hóminis ordinásti. Transcéndit autem hæc ómnia, quia tu ipse hómini servíre dignátus es, et teípsum datúrum ei promisísti.}
\Normal{\Verse{4.} Quid dabo tibi pro ómnibus istis mílibus bonis~? Utinam possem tibi servíre cunctis diébus vitæ meæ~! Utinam vel uno die dignum servítium exhibére suffícerem~! Vere tu es dignus omni servítio, omni honóre et laude ætérna. Vere Dóminus meus es, et ego pauper servus tuus, qui totis víribus téneor tibi servíre, nec umquam in láudibus tuis débeo fastidíre. Sic volo, sic desídero~; et quidquid mihi deest, tu dignéris supplére.}
\Normal{\Verse{5.} Magnus honor, magna glória, tibi servíre et ómnia propter te contémnere. Habébunt enim grátiam magnam, qui sponte se subiécerint tuæ sanctíssimæ servitúti. Invénient suavíssimam Spíritus Sancti consolatiónem, qui pro amóre tuo omnem carnálem abiécerint delectatiónem. Consequéntur magnam mentis libertátem, qui arctam pro nómine tuo ingrediúntur viam et omnem mundánam negléxerint curam.}
\Normal{\Verse{6.} O grata et iucúnda Dei sérvitus, qua homo veráciter effícitur liber et sanctus~! O sacer status religiósi famulátus, qui hóminem ángelis reddit æquálem, Deo placábilem, dæmónibus terríbilem et cunctis fidélibus commendábilem~! O amplecténdum et semper optándum servítium, quo summum promerétur bonum et gáudium acquíritur sine fine mansúrum!}
\markright{XI}
%LIBER III
\TitreC{CAPUT XI}
\TitreC{Quod desidéria cordis examinánda sunt et moderánda}
\Normal{\Verse{1.} Fili, opórtet te adhuc multa addíscere, quæ necdum bene didicísti. Quæ sunt hæc, Dómine? Ut desidérium tuum ponas totáliter secúndum beneplácitum meum, et tui ipsíus amátor non sis, sed meæ voluntátis cúpidus æmulátor. Desidéria te sæpe accéndunt et veheménter impéllunt~; sed consídera, an propter honórem meum, an propter tuum cómmodum magis moveáris. Si ego sum in causa, bene conténtus eris, quomodocúmque ordinávero~; si autem de próprio quæsítu áliquid latet, ecce, hoc est, quod te ímpedit et gravat.}
\Normal{\Verse{2.} Cave ergo, ne nímium innitáris super desidério præconcepto, me non consúlto~: ne forte póstea pæníteat aut displíceat, quod primo plácuit, et quasi pro melióre zelásti. Non enim omnis afféctio, quæ vidétur bona, statim est sequénda~; sed neque omnis contrária afféctio ad primum fugiénda. Expedit intérdum refrenatióne uti, étiam in bonis stúdiis et desidériis, ne per importunitátem mentis distractiónem incúrras, ne áliis per indisciplinatiónem scándalum géneres, vel étiam per resisténtiam aliórum súbito turbéris et córruas.}
\Normal{\Verse{3.} Intérdum vero opórtet violéntia uti et viríliter appetítui sensitivo contraíre, nec advértere, quid velit caro et quid non velit~; sed hoc magis satágere, ut subiécta sit étiam nolens spirítui. Et tam diu castigári debet, et cogi servitúti subésse, donec paráta sit ad ómnia, paucísque contentári discat et simplícibus delectári, nec contra áliquod inconvéniens mussitáre.}
\markright{XII}
%LIBER III
\TitreC{CAPUT XII}
\TitreC{De informatióne patiéntiæ et luctámine advérsus concupiscéntias}
\Normal{\Verse{1.} Dómine Deus, ut vídeo, patiéntia est mihi valde necessária~; multa enim in hac vita áccidunt contrária. Nam qualitercúmque ordinávero de pace mea, non potest esse sine bello et dolóre vita mea.}
\Normal{\Verse{2.} Ita est, fili. Sed volo te non talem quǽrere pacem, quæ temptatiónibus cáreat aut contrária non séntiat, sed tunc étiam æstimáre te pacem invenísse, cum fúeris váriis tribulatiónibus exercitátus et in multis contrarietátibus probátus. Si díxeris te non posse multa pati, quómodo tunc sustinébis ignem purgatórii~? De duóbus malis minus est semper eligéndum. Ut ergo ætérna futúra supplícia possis evádere, mala præséntia stúdeas pro Deo æquanímiter toleráre. An putas, quod hómines sǽculi huius nihil vel parum patiántur~? Nec hoc invénies, étiam si delicatíssimos quæsíeris.}
\Normal{\Verse{3.} Sed habent, inquis, multas delectatiónes et próprias sequúntur voluntátes~; ideóque parum pónderant suas tribulatiónes.}
\Normal{\Verse{4.} Esto, ita sit, ut hábeant, quidquid volúerint~; sed quámdiu, putas, durábit~? Ecce, quemádmodum fumus defícient abundántes in sǽculo, et nulla erit recordátio præteritórum gaudiórum. Sed et cum adhuc vivunt, non sine amaritúdine et tǽdio ac timóre in eis quiéscunt. Ex eádem namque re, unde sibi delectatiónem concípiunt, inde dolóris pœnam frequénter recípiunt. Iuste illis fit, ut, quia inordináte delectatiónes quærunt et sequúntur, non sine confusióne et amaritúdine eas éxpleant. O quam breves, quam falsæ, quam inordinátæ et turpes omnes sunt~! Verúmtamen præ ebrietáte et cæcitáte non intéllegunt, sed velut muta animália propter módicum corruptíbilis vitæ delectaméntum mortem ánimæ incúrrunt. Tu ergo, fili, post concupiscéntias tuas non eas, et a voluntáte tua avértere. Delectáre in Dómino, et dabit tibi petitiónes cordis tui.}
\Normal{\Verse{5.} Etenim si veráciter vis delectári et abundántius a me consolári, ecce, in contémptu ómnium mundanórum et in abscissióne ómnium infimárum delectatiónum erit benedíctio tua, et copiósa tibi reddétur consolátio. Et quanto te plus ab omni creaturárum solácio subtráxeris, tanto in me suavióres et potentióres consolatiónes invénies. Sed primo non sine quadam tristítia et labóre certáminis ad has pertínges. Obsístet inólita consuetúdo, sed melióri consuetúdine devincétur. Remurmurábit caro, sed fervóre spíritus frenábitur. Instigábit et exacerbábit te serpens antíquus, sed oratióne fugábitur~; ínsuper et labóre útili áditus ei magnus obstruétur.}
\markright{XIII}
%LIBER III
\TitreC{CAPUT XIII}
\TitreC{De obœdiéntia húmilis súbditi ad exémplum Iesu Christi}
\Normal{\Verse{1.} Fili, qui se subtráhere nítitur ab obœdiéntia, ipse se súbtrahit a grátia~; et qui quærit habére priváta, amíttit commúnia. Qui non libénter et sponte suo superióri se subdit, signum est, quod caro sua necdum perfécte sibi obœ́dit, sed sæpe recálcitrat et remúrmurat. Disce ergo celériter superióri tuo te submíttere, si carnem própriam optas subiugáre. Cítius namque extérior víncitur inimícus, si intérior homo non fúerit devastátus. Non est moléstior et peior ánimæ hostis, quam tu ipse tibi, non bene concórdans spirítui. Opórtet omníno verum te assúmere tui ipsíus contémptum, si vis prævalére advérsus carnem et sánguinem. Quia adhuc nimis inordináte te díligis, ídeo plene te resignáre aliórum voluntáti trépidas.}
\Normal{\Verse{2.} Sed quid magnum, si tu, qui pulvis es et nihil, propter Deum te hómini subdis, quando ego, omnípotens et altíssimus, qui cuncta creávi ex níhilo, me hómini propter te humíliter subiéci~? Factus sum ómnium humillimus et ínfimus, ut tuam supérbiam mea humilitáte vínceres. Disce obtemperáre, pulvis. Disce te humiliáre, terra et limus, et sub ómnium pédibus incurváre. Disce voluntátes tuas frángere et ad omnem subiectiónem te dare.}
\Normal{\Verse{3.} Exardésce contra te, nec patiáris tumórem in te vívere, sed ita subiéctum et párvulum te éxhibe, ut omnes super te ambuláre possint et sicut lutum plateárum conculcáre. Quid habes, homo inánis, cónqueri~? Quid, sórdide peccátor, potes contradícere exprobrántibus tibi, qui tótiens Deum offendísti et inférnum multótiens meruísti~? Sed pepércit tibi óculus meus, quia pretiósa fuit ánima tua in conspéctu meo~: ut cognósceres dilectiónem meam et gratus semper benefíciis meis exsísteres, et ut ad veram subiectiónem et humilitátem te iúgiter dares, patientérque próprium contémptum ferres.}
\markright{XIV}
%LIBER III
\TitreC{CAPUT XIV}
\TitreC{De occúltis Dei iudíciis considerándis, ne extollámur in bonis}
\Normal{\Verse{1.} Intonas super me iudícia tua, Dómine, et timóre ac tremóre cóncutis ómnia ossa mea, et expavéscit ánima mea valde. Sto attónitus et consídero, quia cæli non sunt mundi in conspéctu tuo. Si in ángelis repperísti pravitátem, nec tamen pepercísti, quid fiet de me~? Cecidérunt stellæ de cælo, et ego pulvis, quid præsúmo~? Quorum ópera videbántur laudabília, cecidérunt ad ínfima, et qui comedébant panem angelórum, vidi síliquis delectári porcórum.}
\Normal{\Verse{2.} Nulla est ergo sánctitas, si manum tuam, Dómine, súbtrahas. Nulla prodest sapiéntia, si gubernáre desístas. Nulla iuvat fortitúdo, si conserváre désinas. Nulla secúra cástitas, si eam non prótegas. Nulla própria prodest custódia, si non adsit tua sacra vigilántia. Nam relícti mérgimur et perímus, visitáti vero erígimur et vívimus. Instábiles quippe sumus, sed per te confirmámur~; tepéscimus, sed a te accéndimur.}
\Normal{\Verse{3.} O quam humíliter et abiécte mihi de me ipso sentiéndum est~! quam níhili pendéndum, si quid boni vídear habére~! O quam profúnde submíttere me débeo sub abyssálibus iudíciis tuis, Dómine, ubi nihil áliud me esse invénio, quam nihil et nihil~! O pondus imménsum~! o pélagus intransnatábile, ubi nihil de me repério, quam in toto nihil~! Ubi est ergo látebra glóriæ~? ubi confidéntia de virtúte concépta~? Absórpta est omnis gloriátio vana in profunditáte iudiciórum tuórum super me.}
\Normal{\Verse{4.} Quid est omnis caro in conspéctu tuo~? Numquid gloriábitur lutum contra formántem se~? Quómodo potest érigi vanilóquio, cuius cor in veritáte subiéctum est Deo~? Non eum totus mundus ériget, quem véritas sibi subiécit~; nec ómnium laudántium ore movébitur, qui totam spem suam in Deo firmávit. Nam et ipsi, qui loquúntur, ecce, omnes nihil~; defícient enim cum sónitu verbórum. Véritas autem Dómini manet in ætérnum.}
\markright{XV}
%LIBER III
\TitreC{CAPUT XV}
\TitreC{Quáliter standum sit ac dicéndum in omni re desiderábili}
\Normal{\Verse{1.} Fili, sic dicas in omni re~: Dómine, si tibi plácitum fúerit, fiat hoc ita. Dómine, si fúerit honor tuus, fiat hoc in nómine tuo. Dómine, si mihi víderis expedíre, et útile esse probáveris, tunc dona mihi hoc uti ad honórem tuum. Sed si mihi nocívum fore cognóveris, nec ánimæ meæ salúti prodésse, aufer a me tale desidérium. Non enim omne desidérium est a Spíritu Sancto, étiam si hómini videátur rectum et bonum. Diffícile est pro vero iudicáre, utrum spíritus bonus an aliénus te impéllat ad desiderándum hoc vel illud, an étiam ex próprio moveáris spíritu. Multi in fine sunt decépti, qui primo bono spíritu videbántur indúcti.}
\Normal{\Verse{2.} Igitur semper cum timóre Dei et cordis humilitáte desiderándum est et peténdum, quidquid desiderábile menti occúrrit, maximéque cum própria resignatióne, mihi totum committéndum est, atque dicéndum~: Dómine, tu scis, quáliter mélius est~; fiat hoc vel illud, sicut volúeris. Da, quod vis et quantum vis et quando vis. Fac mecum, sicut scis et sicut tibi magis placúerit et maior honor tuus fúerit. Pone me, ubi vis, et líbere age mecum in ómnibus. In manu tua sum, gyra et revérsa me per circúitum. En, servus tuus ego, parátus ad ómnia, quóniam non desídero mihi vívere, sed tibi~: útinam digne et perfécte.}
\TitreD{Orátio pro beneplácito Dei perficiéndo}
\Normal{\Verse{3.} Concéde mihi, benigníssime Iesu, grátiam tuam, ut mecum sit et mecum labóret, mecúmque usque in finem persevéret. Da mihi hoc semper desideráre et velle, quod tibi magis accéptum est et cárius placet. Tua volúntas mea sit, et mea volúntas tuam semper sequátur et óptime ei concórdet. Sit mihi unum velle et nolle tecum, nec áliud posse velle aut nolle, nisi quod tu vis et nolis.}
\Normal{\Verse{4.} Da mihi ómnibus mori, quæ in mundo sunt, et propter te amáre contémni et nescíri in hoc sǽculo. Da mihi super ómnia desideráta in te requiéscere et cor meum in te pacificáre. Tu vera pax cordis, tu sola réquies~: extra te dura sunt ómnia et inquiéta. In hac pace in id ipsum, hoc est, in te uno summo ætérno bono, dórmiam et requiéscam. Amen.}
\markright{XVI}
%LIBER III
\TitreC{CAPUT XVI}
\TitreC{Quod verum solácium in solo Deo est quæréndum}
\Normal{\Verse{1.} Quidquid desideráre possum vel cogitáre ad solácium meum, non hic exspécto, sed in pósterum. Quod si ómnia solácia mundi solus habérem, et ómnibus delíciis frui possem, certum est, quod diu duráre non possent. Unde non póteris, ánima mea, plene consolári nec perfécte recreári, nisi in Deo, consolatóre páuperum ac susceptóre humílium. Exspécta módicum, ánima mea, exspécta divínum promíssum, et habébis abundántiam ómnium bonórum in cælo. Si nimis inordináte ista áppetis præséntia, perdes ætérna et cæléstia. Sint temporália in usu, ætérna in desidério. Non potes áliquo bono temporáli satiári, quia ad hæc fruénda non es creáta.}
\Normal{\Verse{2.} Etiam si ómnia creáta bona habéres, non posses esse felix et beáta~; sed in Deo, qui cuncta creávit, tota beatitúdo tua et felícitas consístit, non qualis vidétur et laudátur a stultis mundi amatóribus, sed qualem exspéctant boni Christi fidéles, et prægústant intérdum spirituáles ac mundi cordes, quorum conversátio est in cælis. Vanum est et breve omne humánum solácium. Beátum et verum solácium, quod intus a veritáte percípitur. Devótus homo ubíque secum fert consolatórem suum Iesum, et dicit ad eum~: Adésto mihi, Dómine Iesu, in omni loco et témpore. Hæc mihi sit consolátio, libénter velle carére omni humáno solácio. Et si tua defúerit consolátio, sit mihi tua volúntas et iusta probátio pro summo solácio. Non enim in perpétuum irascéris, neque in ætérnum commináberis.}
\markright{XVII}
%LIBER III
\TitreC{CAPUT XVII}
\TitreC{Quod omnis sollicitúdo in Deo statuénda sit}
\Normal{\Verse{1.} Fili, sine me tecum ágere, quod volo~; ego scio, quid éxpedit tibi. Tu cógitas, sicut homo, tu sentis in multis, sicut humánus suádet afféctus.}
\Normal{\Verse{2.} Dómine, verum est, quod dicis. Maior est sollicitúdo tua pro me, quam omnis cura, quam ego gérere possum pro me. Nimis enim casuáliter stat, qui non próicit omnem sollicitúdinem suam in te. Dómine, dúmmodo volúntas mea recta et firma ad te permáneat, fac de me, quidquid tibi placúerit. Non enim potest esse nisi bonum, quidquid de me féceris. Si me vis esse in ténebris, sis benedíctus, et si me vis esse in luce, sis íterum benedíctus. Si me dignáris consolári, sis benedíctus, et si me vis tribulári, sis æque semper benedíctus.}
\Normal{\Verse{3.} Fili, sic opórtet te stare, si mecum desíderas ambuláre. Ita promptus esse debes ad patiéndum, sicut ad gaudéndum. Ita libénter debes esse inops et pauper, sicut plenus et dives.}
\Normal{\Verse{4.} Dómine, libénter pátiar pro te, quidquid volúeris veníre super me. Indifferénter volo de manu tua bonum et malum, dulce et amárum, lætum et triste suscípere, et pro ómnibus mihi contingéntibus grátias ágere. Custódi me ab omni peccáto, et non timébo mortem nec inférnum. Dúmmodo in ætérnum me non proícias, nec déleas de libro vitæ, non mihi nocébit, quidquid vénerit tribulatiónis super me.}
\markright{XVIII}
%LIBER III
\TitreC{CAPUT XVIII}
\TitreC{Quod temporáles misériæ exémplo Christi æquanímiter sunt feréndæ}
\Normal{\Verse{1.} Fili, ego descéndi de cælo pro tua salúte~; suscépi tuas misérias, non necessitáte, sed caritáte trahénte, ut patiéntiam dísceres et temporáles misérias non indignánter ferres. Nam ab hora ortus mei usque ad éxitum in cruce non défuit mihi tolerántia dolóris. Deféctum rerum temporálium magnum hábui, multas querimónias de me frequénter audívi, confusiónes et obprobria benígne sustínui, pro benefíciis ingratitúdinem recépi, pro miráculis blasphémias, pro doctrína reprehensiónes.}
\Normal{\Verse{2.} Dómine, quia tu pátiens fuísti in vita tua, in hoc máxime impléndo præcéptum Patris tui~: dignum est, ut ego, miséllus peccátor, secúndum voluntátem tuam patiénter me sustíneam~; et donec ipse volúeris, onus corruptíbilis vitæ pro salúte mea portem. Nam et si onerósa sentítur præsens vita, facta est tamen iam per grátiam tuam valde meritória, atque exémplo tuo et sanctórum tuórum vestígiis infírmis tolerabílior et clárior. Sed et multo magis consolatória, quam olim in lege véteri fúerat, cum porta cæli clausa persísteret, et obscúrior étiam via ad cælum videbátur, quando tam pauci regnum cælórum quǽrere curábant. Sed neque qui tunc iusti erant et salvándi, ante passiónem tuam et sacræ mortis débitum cæléste regnum póterant introíre.}
\Normal{\Verse{3.} O quantas tibi grátias téneor reférre, quod viam rectam et bonam dignátus es mihi et cunctis fidélibus ad ætérnum regnum tuum osténdere~! Nam vita tua via nostra~: et per sanctam patiéntiam ambulámus ad te, qui es coróna nostra. Nisi tu nos præcessísses et docuísses, quis sequi curáret~? Heu, quanti longe retróque manérent, nisi tua præclára exémpla inspícerent~! Ecce, adhuc tepéscimus, audítis tot signis tuis et doctrínis~: quid fíeret, si tantum lumen ad sequéndum te non haberémus?}
\markright{XIX}
%LIBER III
\TitreC{CAPUT XIX}
\TitreC{De tolerántia iniuriárum, et quis verus pátiens probétur}
\Normal{\Verse{1.} Quid est, quod lóqueris, fili~? Cessa cónqueri, consideráta mea et aliórum sanctórum passióne. Nondum usque ad sánguinem restitísti. Parum est, quod tu páteris, in comparatióne eórum, qui tam multa passi sunt, tam fórtiter temptáti, tam gráviter tribuláti, tam multiplíciter probáti et exercitáti. Opórtet te ígitur aliórum gravióra ad mentem redúcere, ut lévius feras tua mínima. Et si tibi mínima non vidéntur, vide, ne et hoc tua fáciat impatiéntia. Sive tamen parva sive magna sint, stude cuncta patiénter sufférre.}
\Normal{\Verse{2.} Quanto mélius ad patiéndum te dispónis, tanto sapiéntius agis et ámplius promeréris~; feres quoque lévius ánimo et usu ad hoc non ségniter parátis. Nec dicas~: Non váleo hæc ab hómine tali pati, nec huiuscémodi mihi patiénda sunt~; grave enim íntulit damnum, et impróperat mihi, quæ numquam cogitáveram~; sed ab álio libénter pátiar, et sicut patiénda vídero. Insípiens est talis cogitátio, quæ virtútem patiéntiæ non consíderat, nec a quo coronánda erit, sed magis persónas et offénsas sibi illátas perpéndit.}
\Normal{\Verse{3.} Non est verus pátiens, qui pati non vult, nisi quantum sibi visum fúerit, et a quo sibi placúerit. Verus autem pátiens non atténdit, a quo hómine, utrum a præláto suo an ab áliquo æquáli aut inferióri, utrum a bono et sancto viro, vel a pervérso et indígno exerceátur. Sed indifferénter ab omni creatúra, quantumcúmque et quotiescúmque ei áliquid advérsi accíderit, totum hoc de manu Dei gratánter áccipit et ingens lucrum réputat, quia nil apud Deum, quámlibet parvum, pro Deo tamen passum, póterit sine mérito transíre.}
\Normal{\Verse{4.} Esto ítaque expedítus ad pugnam, si vis habére victóriam. Sine certámine non potes veníre ad patiéntiæ corónam. Si pati non vis, recúsas coronári. Si autem coronári desíderas, certa viríliter, sústine patiénter. Sine labóre non ténditur ad réquiem, nec sine pugna pervenítur ad victóriam.}
\Normal{\Verse{5.} Fiat, Dómine, mihi possíbile per grátiam, quod mihi impossíbile vidétur per natúram. Tu scis, quod módicum possum pati, et quod cito deícior, levi exurgénte adversitáte. Efficiátur mihi quǽlibet exercitátio tribulatiónis pro nómine tuo amábilis et optábilis~: nam pati et vexári pro te valde salúbre est ánimæ meæ.}
\markright{XX}
%LIBER III
\TitreC{CAPUT XX}
\TitreC{De confessióne própriæ infirmitátis et huius vitæ misériis}
\Normal{\Verse{1.} Confitébor advérsum me iniustítiam meam, confitébor tibi, Dómine, infirmitátem meam. Sæpe parva res est, quæ me déicit et contrístat. Propóno me fórtiter actúrum~; sed cum módica temptátio vénerit, magna mihi angústia fit. Valde vilis quandóque res est, unde gravis temptátio próvenit. Et dum puto me aliquántulum tutum, cum non séntio, invénio me nonnúmquam pene devíctum ex levi flatu.}
\Normal{\Verse{2.} Vide ergo, Dómine, humilitátem meam et fragilitátem tibi úndique notam. Miserére et éripe me de luto, ut non infígar, ne permáneam deiéctus usquequáque. Hoc est, quod me frequénter revérberat, et coram te confúndit, quod tam lábilis sum, et infírmus ad resisténdum passiónibus. Et si non omníno ad consensiónem, tamen mihi étiam molésta et gravis est eárum insectátio, et tædet valde sic cotídie vívere in lite. Ex hinc nota mihi fit infírmitas mea, quia multo facílius írruunt abominándæ semper phantásiæ, quam discédunt.}
\Normal{\Verse{3.} Utinam, fortíssime Deus Israel, zelátor animárum fidélium, respícias servi tui labórem et dolórem, assistásque illi in ómnibus, ad quæcúmque perréxerit. Róbora me cælésti fortitúdine, ne vetus homo, mísera caro spirítui necdum plene subácta, præváleat dominári, advérsus quam certáre oportébit, quámdiu spirátur in hac vita misérrima. Heu, qualis est hæc vita, ubi non desunt tribulatiónes et misériæ, ubi plena láqueis et hóstibus sunt ómnia~! Nam una tribulatióne seu temptatióne recedénte, ália accédit~; sed et prióre adhuc duránte conflíctu, álii plures supervéniunt, et insperáte.}
\Normal{\Verse{4.} Et quómodo potest amári vita, tantas habens amaritúdines, tot subiécta calamitátibus et misériis~? Quómodo étiam dícitur vita, tot génerans mortes et pestes~? Et tamen amátur, et delectári in ea a multis quǽritur. Reprehénditur frequénter mundus, quod fallax sit et vanus, nec tamen fácile relínquitur, quia concupiscéntiæ carnis nimis dominántur. Sed ália trahunt ad amándum, ália ad contemnéndum. Trahunt ad amórem mundi desidérium carnis, desidérium oculórum et supérbia vitæ~; sed pœnæ ac misériæ iuste sequéntes ea odium mundi páriunt et tǽdium.}
\Normal{\Verse{5.} Sed vincit, proh dolor, delectátio prava mentem mundo déditam, et esse sub séntibus delícias réputat, quia Dei suavitátem et intérnam virtútis amœnitátem nec vidit, nec gustávit. Qui autem mundum perfécte contémnunt, et Deo vívere sub sancta disciplína student, isti divínam dulcédinem, veris abrenuntiatóribus promíssam, non ignórant~; et quam gráviter mundus errat et várie fállitur, clárius vident.}
\markright{XXI}
%LIBER III
\TitreC{CAPUT XXI}
\TitreC{Quod in Deo super ómnia bona et dona requiescéndum est}
\Normal{\Verse{1.} Super ómnia et in ómnibus requiésces, ánima mea, in Dómino semper, quia ipse sanctórum ætérna réquies. Da mihi, dulcíssime et amantíssime Iesu, in te super omnem creatúram requiéscere~: super omnem salútem et pulchritúdinem, super omnem glóriam et honórem, super omnem poténtiam et dignitátem, super omnem sciéntiam et subtilitátem, super omnes divítias et artes, super omnem lætítiam et exsultatiónem, super omnem famam et laudem, super omnem suavitátem et consolatiónem, super omnem spem et promissiónem, super omne méritum et desidérium, super ómnia dona et múnera, quæ potes dare et infúndere, super omne gáudium et iubilatiónem, quam potest mens cápere et sentíre, dénique super ángelos et archángelos, et super omnem exércitum cæli, super ómnia visibília et invisibília, et super omne, quod tu, Deus meus, non es.}
\Normal{\Verse{2.} Quia tu, Dómine Deus meus, super ómnia óptimus es, tu solus altíssimus, tu solus potentíssimus, tu solus sufficientíssimus et pleníssimus, tu solus suavíssimus et solaciosíssimus, tu solus pulchérrimus et amantíssimus, tu solus nobilíssimus et gloriosíssimus super ómnia, in quo cuncta bona simul et perfécte sunt, et semper fuérunt, et erunt~: atque ídeo minus est et insuffíciens, quidquid præter teípsum mihi donas, aut de te ipso revélas, vel promíttis, te non viso, nec plene adépto~: quóniam quidem non potest cor meum veráciter requiéscere, nec totáliter contentári, nisi in te requiéscat, et ómnia dona omnémque creatúram transcéndat.}
\Normal{\Verse{3.} O mi dilectíssime sponse, Iesu Christe, amátor puríssime, dominátor univérsæ creatúræ, quis mihi det pennas veræ libertátis, ad volándum et pausándum in te~? O quando ad plenum dábitur mihi vacáre et vidére, quam suávis es, Dómine Deus meus~? Quando ad plenum me recólligam in te, ut præ amóre tuo non séntiam me, sed te solum, supra omnem sensum et modum, in modo non ómnibus noto~? Nunc autem frequénter gemo, et infelicitátem meam cum dolóre porto. Quia multa mala in hac valle miseriárum occúrrunt, quæ me sǽpius contúrbant, contrístant et obnúbilant~; sǽpius impédiunt et dístrahunt, allíciunt et ímplicant, ne líberum hábeam accéssum ad te, et ne iucúndis fruar ampléxibus, præsto semper beátis spirítibus. Móveat te suspírium meum et desolátio múltiplex in terra.}
\Normal{\Verse{4.} O Iesu, splendor ætérnæ glóriæ, solámen peregrinántis ánimæ, apud te est os meum sine voce, et siléntium meum lóquitur tibi. Usquequo tardat veníre Dóminus meus~? Véniat ad me paupérculum suum, et lætum fáciat. Mittat manum suam, et míserum erípiat de omni angústia. Veni, veni~: quia sine te nulla erit læta dies aut hora, quia tu lætítia mea, et sine te vácua est mensa mea. Miser sum et quodámmodo incarcerátus et compédibus gravátus, donec luce præséntiæ tuæ me refícias ac libertáti dones, vultúmque amicábilem demónstres.}
\Normal{\Verse{5.} Quærant álii pro te áliud, quodcúmque libúerit~: mihi áliud ínterim nil placet nec placébit, nisi tu Deus meus, spes mea, salus ætérna. Non reticébo, nec deprecári cessábo, donec grátia tua revertátur, mihíque tu intus loquáris.}
\Normal{\Verse{6.} Ecce, assum, ecce, ego ad te, quia invocásti me. Lácrimæ tuæ et desidérium ánimæ tuæ, humiliátio tua et contrítio cordis inclinavérunt me et adduxérunt ad te.}
\Normal{\Verse{7.} Et dixi~: Dómine, vocávi te et desiderávi frui te, parátus ómnia respúere propter te. Tu enim prior excitásti me, ut quǽrerem te. Sis ergo benedíctus, Dómine, qui fecísti hanc bonitátem cum servo tuo, secúndum multitúdinem misericórdiæ tuæ. Quid habet ultra dícere servus tuus coram te, nisi ut humíliet se valde ante te, memor semper própriæ iniquitátis et vilitátis~? Non enim est símilis tui in cunctis mirabílibus cæli et terræ. Sunt ópera tua bona valde, iudícia vera, et providéntia tua regúntur univérsa. Laus ergo tibi et glória, o Patris sapiéntia~: te laudet et benedícat os meum, ánima mea et cuncta creáta simul.}
\markright{XXII}
%LIBER III
\TitreC{CAPUT XXII}
\TitreC{De recordatióne multiplícium beneficiórum Dei}
\Normal{\Verse{1.} Aperi, Dómine, cor meum in lege tua, et in præcéptis tuis doce me ambuláre. Da mihi intellégere voluntátem tuam, et cum magna reveréntia ac diligénti consideratióne benefícia tua tam in generáli quam in speciáli memorári, ut digne tibi ex hinc váleam grátias reférre. Verum scio et confíteor, nec pro mínimo puncto me posse débitas gratiárum laudes persólvere. Minor ego sum ómnibus bonis mihi prǽstitis~; et cum tuam nobilitátem atténdo, déficit præ magnitúdine spíritus meus.}
\Normal{\Verse{2.} Omnia, quæ in ánima habémus et córpore, et quæcúmque extérius vel intérius, naturáliter et supernaturáliter possidémus, tua sunt benefícia et te benéficum, pium ac bonum comméndant, a quo bona cuncta accépimus. Et si álius plura, álius paucióra accépit, ómnia tamen tua sunt, et sine te nec mínimum potest habéri. Ille, qui maióra accépit, non potest mérito suo gloriári, neque super álios extólli, nec minóri insultáre~; quia ille maior et mélior, qui sibi minus ascríbit et in regratiándo humílior est atque devótior. Et qui ómnibus viliórem se exístimat et indigniórem se iúdicat, áptior est ad percipiénda maióra.}
\Normal{\Verse{3.} Qui autem paucióra accépit, contristári non debet, nec indignánter ferre, neque ditióri invidére, sed te pótius atténdere et tuam bonitátem máxime laudáre, quod tam affluénter, tam gratis et libénter, sine personárum acceptióne tua múnera largíris. Omnia ex te, et ídeo in ómnibus es laudándus. Tu scis, quid unicuíque donári expédiat~; et cur iste minus, et ille ámplius hábeat, non nostrum, sed tuum est hoc discérnere, apud quem singulórum definíta sunt mérita.}
\Normal{\Verse{4.} Unde, Dómine Deus, pro magno étiam réputo benefício, non multa habére, unde extérius et secúndum hómines laus et glória appáret~: ita ut consideráta quis paupertáte et vilitáte persónæ suæ, non modo gravitátem aut tristítiam vel deiectiónem inde concípiat, sed pótius consolatiónem et hilaritátem magnam~; quia tu, Deus, páuperes et húmiles atque huic mundo despéctos tibi elegísti in familiáres et domésticos. Testes sunt ipsi apóstoli tui, quos príncipes super omnem terram constituísti. Fuérunt tamen sine queréla conversáti in mundo, tam húmiles et símplices, sine omni malítia et dolo, ut étiam pati contumélias gaudérent pro nómine tuo~; et quæ mundus abhórret, ipsi amplecteréntur afféctu magno.}
\Normal{\Verse{5.} Nihil ergo amatórem tuum et cognitórem beneficiórum tuórum ita lætificáre debet, sicut volúntas tua in eo, et beneplácitum ætérnæ dispositiónis tuæ~: de qua tantum contentári debet et consolári, ut ita libénter velit esse mínimus, sicut áliquis optáret esse máximus, et ita pacíficus et conténtus in novíssimo, sicut in loco primo, atque ita libénter despicábilis et abiéctus, nullíus quoque nóminis et famæ, sicut céteris honorabílior et maior in mundo. Nam volúntas tua et amor honóris tui ómnia excédere debet, et plus eum consolári magísque placére, quam ómnia benefícia sibi data vel danda.}
\markright{XXIII}
%LIBER III
\TitreC{CAPUT XXIII}
\TitreC{De quátuor magnam importántibus pacem}
\Normal{\Verse{1.} Fili, nunc docébo te viam pacis et veræ libertátis.}
\Normal{\Verse{2.} Fac, Dómine, quod dicis, quia hoc mihi gratum est audíre.}
\Normal{\Verse{3.} Stude, fili, altérius pótius fácere voluntátem, quam tuam. Elige, semper minus, quam plus habére. Quære semper inferiórem locum et ómnibus subésse. Opta semper et ora, ut volúntas Dei íntegre in te fiat. Ecce, talis homo ingréditur fines pacis et quiétis.}
\Normal{\Verse{4.} Dómine, sermo tuus iste brevis multum in se cóntinet perfectiónis. Parvus est dictu, sed plenus sensu et uber in fructu. Nam si posset a me fidéliter custodíri, non debéret tam fácilis in me turbátio oríri. Nam quótiens me impacátum séntio et gravátum, ab hac doctrína me recessísse invénio. Sed tu, qui ómnia potes et ánimæ proféctum semper díligis, adáuge maiórem grátiam, ut possim tuum complére sermónem et meam perfícere salútem.}
\TitreD{Orátio contra cogitatiónes malas}
\Normal{\Verse{5.} Dómine Deus meus, ne elongéris a me~; Deus meus, in auxílium meum réspice~: quóniam insurrexérunt in me váriæ cogitatiónes et timóres magni, affligéntes ánimam meam. Quómodo pertransíbo illǽsus~? quómodo perfríngam eas?}
\Normal{\Verse{6.} Ego, inquit, ante te ibo, et gloriósos terræ humiliábo. Apériam iánuas cárceris, et arcána secretórum revelábo tibi.}
\Normal{\Verse{7.} Fac, Dómine, ut lóqueris, et fúgiant a fácie tua omnes iníquæ cogitatiónes. Hæc spes et única consolátio mea, ad te in omni tribulatióne confúgere, tibi confídere, ex íntimo invocáre et patiénter consolatiónem tuam exspectáre.}
\TitreD{Orátio pro illuminatióne mentis}
\Normal{\Verse{8.} Clarífica me, Iesu bone, claritáte intérni lúminis, et educ de habitáculo cordis mei ténebras univérsas. Cóhibe evagatiónes multas, et vim faciéntes elíde temptatiónes. Pugna fórtiter pro me, et expúgna malas béstias, concupiscéntias dico illecebrósas~: ut fiat pax in virtúte tua, et abundántia laudis tuæ résonet in aula sancta, hoc est in consciéntia pura. Impera ventis et tempestátibus~; dic mari~: Quiésce~; et aquilóni~: Ne fláveris~; et erit tranquíllitas magna.}
\Normal{\Verse{9.} Emítte lucem tuam et veritátem, ut lúceant super terram~; quia terra sum inánis et vácua, donec illúmines me. Effúnde grátiam désuper~; perfúnde cor meum rore cælésti~; minístra devotiónis aquas ad irrigándam fáciem terræ, ad producéndum fructum bonum et óptimum. Eleva mentem pressam mole peccatórum, et ad cæléstia totum desidérium meum suspénde, ut gustáta suavitáte supérnæ felicitátis pígeat de terrénis cogitáre.}
\Normal{\Verse{10.} Rape me et éripe ab omni creaturárum indurábili consolatióne, quia nulla res creáta appetítum meum valet plenárie quietáre et consolári. Iunge me tibi inseparábili dilectiónis vínculo, quóniam tu solus súfficis amánti, et absque te frívola sunt univérsa.}
\markright{XXIV}
%LIBER III
\TitreC{CAPUT XXIV}
\TitreC{De evitatióne curiósæ inquisitiónis super altérius vita}
\Normal{\Verse{1.} Fili, noli esse curiósus nec vácuas gérere sollicitúdines. Quid hoc vel illud ad te~? Tu me séquere. Quid enim ad te, utrum ille sit talis vel talis, aut iste sic et sic agit vel lóquitur~? Tu non índiges respondére pro áliis, sed pro te ipso ratiónem reddes. Quid ergo te ímplicas~? Ecce, ego omnes cognósco, et cuncta, quæ sub sole fiunt, vídeo~; et scio, quáliter cum unoquóque sit, quid cógitet, quid velit, et ad quem finem tendat eius inténtio. Mihi ígitur ómnia committénda sunt~: tu vero serva te in bona pace, et dimítte agitántem agitáre quantum volúerit. Véniet super eum, quidquid fécerit vel díxerit, quia me fállere non potest.}
\Normal{\Verse{2.} Non sit tibi curæ de magni nóminis umbra, non de multórum familiaritáte, nec de priváta hóminum dilectióne. Ista enim génerant distractiónes et magnas in corde obscuritátes. Libénter lóquerer tibi verbum meum et abscóndita revelárem, si advéntum meum diligénter observáres et óstium cordis mihi aperíres. Esto próvidus, et vígila in oratiónibus, et humília te in ómnibus.}
\markright{XXV}
%LIBER III
\TitreC{CAPUT XXV}
\TitreC{In quibus firma pax cordis et verus proféctus consístit}
\Normal{\Verse{1.} Fili, ego locútus sum~: Pacem relínquo vobis, pacem meam do vobis~: non quómodo mundus dat, ego do vobis. Pacem omnes desíderant~: sed quæ ad veram pacem pértinent, non omnes curant. Pax mea cum humílibus et mansuétis corde. Pax tua erit in multa patiéntia. Si me audíeris et vocem meam secútus fúeris, póteris multa pace frui.}
\Normal{\Verse{2.} Quid ígitur fáciam?}
\Normal{\Verse{3.} In omni re atténde tibi, quid fácias et quid dicas, et omnem intentiónem tuam ad hoc dírige, ut mihi soli pláceas et extra me nihil cúpias vel quæras. Sed et de aliórum dictis vel factis nil témere iúdices, nec cum rebus tibi non commíssis te ímplices~; et póterit fíeri, ut parum vel raro turbéris. Numquam autem sentíre áliquam turbatiónem, nec áliquam pati cordis vel córporis moléstiam, non est præséntis témporis, sed status ætérnæ quiétis. Non ergo ǽstimes, te veram pacem invenísse, si nullam sénseris gravitátem, nec tunc totum esse bonum, si néminem páteris adversárium~; nec hoc esse perféctum, si cuncta fiant secúndum tuum afféctum. Neque tunc áliquid magni te réputes aut speciáliter diléctum exístimes, si in magna fúeris devotióne atque dulcédine~: quia in istis non cognóscitur verus amátor virtútis~; nec in istis consístit proféctus et perféctio hóminis.}
\Normal{\Verse{4.} In quo ergo, Dómine?}
\Normal{\Verse{5.} In offeréndo te ex toto corde tuo voluntáti divínæ, non quæréndo, quæ tua sunt, nec in parvo nec in magno, nec in témpore nec in æternitáte~; ita ut una æquáli fácie in gratiárum actióne permáneas inter próspera et contrária, ómnia æqua lance pensándo. Si fúeris tam fortis et longánimus in spe, ut subtrácta interióri consolatióne étiam ad amplióra sustinénda cor tuum præparáveris, nec te iustificáveris, quasi hæc tantáque pati non debéres, sed me in ómnibus dispositiónibus iustificáveris et sanctum laudáveris~; tunc in vera et recta via pacis ámbulas, et spes indubitáta erit, quod rursus in iúbilo fáciem meam sis visúrus. Quod si ad plenum tui ipsíus contémptum pervéneris, scito, quod tunc abundántia pacis perfruéris, secúndum possibilitátem tui incolátus.}
\markright{XXVI}
%LIBER III
\TitreC{CAPUT XXVI}
\TitreC{De eminéntia líberæ mentis, quam supplex orátio magis merétur, quam léctio}
\Normal{\Verse{1.} Dómine, hoc opus est perfécti viri, numquam ab intentióne cæléstium ánimum relaxáre, et inter multas curas quasi sine cura transíre, non more torpéntis, sed prærogatíva quadam líberæ mentis, nulli creatúræ inordináta affectióne adhæréndo.}
\Normal{\Verse{2.} Obsecro te, piíssime Deus meus, præsérva me a curis huius vitæ, ne nimis ímplicer~; a multis necessitátibus córporis, ne voluptáte cápiar~; ab univérsis ánimæ obstáculis, ne moléstiis fractus deíciar. Non dico~: ab his rebus, quas toto afféctu ambit vánitas mundána, sed ab his misériis, quæ ánimam servi tui commúni maledícto mortalitátis pœnáliter gravant et retárdant, ne in libertátem spíritus, quótiens libúerit, váleat introíre.}
\Normal{\Verse{3.} O Deus meus, dulcédo ineffábilis, verte mihi in amaritúdinem omnem consolatiónem carnálem, ab æternórum amóre me abstrahéntem et ad se intúitu cuiúsdam boni delectábilis præséntis male alliciéntem. Non me vincat, Deus meus, non vincat caro et sanguis, non me decípiat mundus et brevis glória eius, non me supplántet diábolus et astútia illíus. Da mihi fortitúdinem resisténdi, patiéntiam tolerándi, constántiam perseverándi. Da pro ómnibus mundi consolatiónibus suavíssimam spíritus tui unctiónem, et pro carnáli amóre tui nóminis infúnde amórem.}
\Normal{\Verse{4.} Ecce, cibus, potus, vestis ac cétera utensília ad córporis sustentáculum pertinéntia fervénti spirítui sunt onerósa. Tríbue, tálibus foméntis temperáte uti, non desidério nímio implicári. Abícere ómnia non licet, quia natúra sustentánda est~; requírere autem supérflua et quæ magis deléctant, lex sancta próhibet~: nam álias caro advérsus spíritum insolésceret. Inter hæc, quæso, manus tua me regat et dóceat, ne quid nímium fiat.}
\markright{XXVII}
%LIBER III
\TitreC{CAPUT XXVII}
\TitreC{Quod privátus amor a summo bono máxime retárdat}
\Normal{\Verse{1.} Fili, opórtet te dare totum pro toto, et nihil tui ipsíus esse. Scito, quod amor tui ipsíus magis nocet tibi, quam áliqua res mundi. Secúndum amórem et afféctum, quem geris, quǽlibet res plus vel minus adhǽret. Si fúerit amor tuus purus, simplex et bene ordinátus, eris sine captivitáte rerum. Noli concupíscere, quod non licet habére~; noli habére, quod te potest impedíre et libertáte interióri priváre. Mirum, quod non ex toto fundo cordis teípsum mihi commíttis cum ómnibus, quæ desideráre potes vel habére.}
\Normal{\Verse{2.} Quare vano mæróre consúmeris~? Cur supérfluis curis fatigáris~? Sta ad beneplácitum meum, et nullum patiéris detriméntum. Si quæris hoc vel illud, et volúeris esse ibi vel ibi propter tuum cómmodum et próprium beneplácitum magis habéndum~: numquam eris in quietúdine, nec liber a sollicitúdine, quia in omni re reperiétur áliquis deféctus, et in omni loco erit, qui adversétur.}
\Normal{\Verse{3.} Iuvat ígitur non quǽlibet res adépta vel multiplicáta extérius, sed pótius contémpta et decísa ex corde radícitus. Quod non tantum de censu æris et divitiárum intéllegas, sed de honóris étiam ámbitu ac vanæ laudatiónis desidério, quæ ómnia tránseunt cum mundo. Munit parum locus, si deest spíritus fervóris~; nec diu stabit pax illa quæsíta forínsecus, si vacat a vero fundaménto status cordis, hoc est, nisi stéteris in me~; permutáre te potes, sed non melioráre. Nam occasióne orta et accépta, invénies, quod fugísti, et ámplius.}
\TitreD{Orátio pro purgatióne cordis et cælésti sapiéntia}
\Normal{\Verse{4.} Confírma me, Deus, per grátiam Sancti Spíritus. Da virtútem corroborári in interióri hómine, et cor meum ab omni inútili sollicitúdine et angóre evacuáre, nec váriis desidériis trahi cuiuscúmque rei, vilis aut pretiósæ~; sed ómnia inspícere sicut transeúntia, et me páriter cum illis transitúrum~: quia nihil pérmanens sub sole, ubi ómnia vánitas et afflíctio spíritus. O quam sápiens, qui ita consíderat!}
\Normal{\Verse{5.} Da mihi, Dómine, cæléstem sapiéntiam, ut discam te super ómnia quǽrere et inveníre, super ómnia sápere et dilígere, et cétera secúndum órdinem sapiéntiæ tuæ, prout sunt, intellégere. Da prudénter declináre blandiéntem et patiénter ferre adversántem, quia hæc magna sapiéntia, non movéri omni vento verbórum, nec aurem male blandiénti præbére Sirénæ~; sic enim incépta pérgitur via secúre.}
\markright{XXVIII}
%LIBER III
\TitreC{CAPUT XXVIII}
\TitreC{Contra linguas obtrectatórum}
\Normal{\Verse{1.} Fili, non ægre feras, si quidam de te male sénserint et díxerint, quod non libénter áudias. Tu deterióra de te ipso sentíre debes, et néminem infirmiórem te crédere. Si ámbulas ab intra, non multum ponderábis volántia verba. Est non parva prudéntia, silére in témpore malo et intrórsus ad me convérti, nec humáno iudício disturbári.}
\Normal{\Verse{2.} Non sit pax tua in ore hóminum~; sive enim bene sive male interpretáti fúerint, non es ídeo alter homo. Ubi est vera pax et vera glória~? Nonne in me~? Et qui non áppetit homínibus placére, nec timet displicére, multa perfruétur pace. Ex inordináto amóre et vano timóre orítur omnis inquietúdo cordis et distráctio sénsuum.}
\markright{XXIX}
%LIBER III
\TitreC{CAPUT XXIX}
\TitreC{Quáliter instánte tribulatióne Deus invocándus est et benedicéndus}
\Normal{\Verse{1.} Sit nomen tuum, Dómine, benedíctum in sǽcula, qui voluísti hanc temptatiónem et tribulatiónem veníre super me. Non possum eam effúgere, sed necésse hábeo ad te confúgere, ut me ádiuves et in bonum mihi convértas. Dómine, modo sum in tribulatióne, et non est cordi meo bene, sed multum vexor a præsénti passióne. Et nunc, pater dilécte, quid dicam~? Deprehénsus sum inter angústias. Salvífica me ex hora hac. Sed proptérea veni in hanc horam, ut tu clarificéris, cum fúero válide humiliátus et per te liberátus. Compláceat tibi, Dómine, ut éruas me~: nam ego pauper quid ágere possum, et quo ibo sine te~? Da patiéntiam, Dómine, étiam hac vice. Adiuva me, Deus meus, et non timébo, quantumcúmque gravátus fúero.}
\Normal{\Verse{2.} Et nunc inter hæc quid dicam~? Dómine, fiat volúntas tua. Ego bene mérui tribulári et gravári. Opórtet útique, ut sustíneam, et útinam patiénter, donec tránseat tempéstas et mélius fiat~! Potens est autem omnípotens manus tua, étiam hanc temptatiónem a me auférre et eius ímpetum mitigáre, ne pénitus succúmbam, quemádmodum et prius sǽpius egísti mecum, Deus meus, misericórdia mea. Et quanto mihi difficílius, tanto tibi facílior est hæc mutátio déxteræ excélsi.}
\markright{XXX}
%LIBER III
\TitreC{CAPUT XXX}
\TitreC{De divíno peténdo auxílio et confidéntia recuperándæ grátiæ}
\Normal{\Verse{1.} Fili, ego Dóminus, confórtans in die tribulatiónis. Vénias ad me, cum tibi non fúerit bene. Hoc est, quod máxime consolatiónem ímpedit cæléstem, quia tárdius te convértis ad oratiónem. Nam ántequam me inténte roges, multa ínterim solácia quæris, et récreas te in extérnis. Ideóque fit, ut parum ómnia prosint, donec advértas, quia ego sum, qui éruo sperántes in me, nec est extra me valens auxílium, neque útile consílium, sed neque durábile remédium. Sed iam resúmpto spíritu post tempestátem reconvalésce in luce miseratiónum meárum: quia prope sum, dicit Dóminus, ut restáurem univérsa, non solum íntegre, sed et abundánter ac cumuláte.}
\Normal{\Verse{2.} Numquid mihi quicquam est diffícile~? aut ero símilis dicénti et non faciénti~? Ubi est fides tua~? Sta fírmiter et perseveránter. Esto longánimis et vir fortis~; véniet tibi consolátio in témpore suo. Exspécta me, exspécta~; véniam et curábo te. Temptátio est, quæ te vexat, et formído vana, quæ te extérret. Quid impórtat sollicitúdo de futúris contingéntibus, nisi ut tristítiam super tristítiam hábeas~? Suffíciat diéi malítia sua. Vanum est et inútile, de futúris conturbári vel gratulári, quæ forte numquam evénient.}
\Normal{\Verse{3.} Sed humánum est, huiúsmodi imaginatiónibus illúdi, et parvi adhuc ánimi signum, tam léviter trahi ad suggestiónem inimíci. Ipse enim non curat, utrum veris an falsis illúdat et decípiat, utrum præséntium amóre, aut futurórum formídine prostérnat. Non ergo turbétur cor tuum, neque formídet. Crede in me, et in misericórdia mea habéto fidúciam. Quando tu putas te elongátum a me, sæpe sum propínquior. Quando tu ǽstimas pene totum pérditum, tunc sæpe maius meréndi instat lucrum. Non est totum pérditum, quando res áccidit in contrárium. Non debes iudicáre secúndum præsens sentíre, nec sic gravitáti alícui, undecúmque veniénti, inhærére et accípere, tamquam omnis spes sit abláta emergéndi.}
\Normal{\Verse{4.} Noli putáre te relíctum ex toto, quamvis ad tempus tibi míserim áliquam tribulatiónem vel étiam optátam subtráxerim consolatiónem~: sic enim transítur ad regnum cælórum. Et hoc sine dúbio magis éxpedit tibi et céteris servis meis, ut exercitémini advérsis, quam si cuncta ad líbitum haberétis. Ego novi cogitatiónes abscónditas, quia multum éxpedit pro salúte tua, ut intérdum sine sapóre relinquáris, ne forte elevéris in bono succéssu, et tibi ipsi placére velis in eo, quod non es. Quod dedi, auférre possum, et restitúere, cum mihi placúerit.}
\Normal{\Verse{5.} Cum dédero, meum est, cum retráxero, tuum non tuli, quia meum est omne datum bonum et omne donum perféctum. Si tibi admísero gravitátem aut quámlibet contrarietátem, ne indignéris, neque cóncidat cor tuum~: ego cito subleváre possum et omne onus in gáudium transmutáre. Verúmtamen iustus sum et recommendábilis multum, cum sic fácio tecum.}
\Normal{\Verse{6.} Si recte sapis et in veritáte áspicis, numquam debes propter advérsa tam deiécte contristári, sed magis gaudére et grátias ágere, immo hoc únicum reputáre gáudium, quod afflígens te dolóribus non parco tibi. Sicut diléxit me Pater, et ego vos díligo, dixi diléctis discípulis meis, quos útique non misi ad gáudia temporália, sed ad magna certámina, non ad honóres, sed ad despectiónes, non ad ótium, sed ad labóres, non ad réquiem, sed ad afferéndum fructum multum in patiéntia. Horum meménto, fili mi, verbórum.}
\markright{XXXI}
%LIBER III
\TitreC{CAPUT XXXI}
\TitreC{De negléctu omnis creatúræ, ut Creátor possit inveníri}
\Normal{\Verse{1.} Dómine, bene indígeo adhuc maióri grátia, si débeam illuc perveníre, ubi me nemo póterit nec ulla creatúra impedíre. Nam quámdiu res áliqua me rétinet, non possum líbere ad te voláre. Cupiébat líbere voláre, qui dicébat~: Quis dabit mihi pennas sicut colúmbæ, et volábo et requiéscam~? Quid símplici óculo quiétius~? Et quid libérius nil desideránte in terris~? Opórtet ígitur omnem supertransíre creatúram, et seípsum perfécte desérere, ac in excéssu mentis stare, et vidére, te ómnium conditórem cum creatúris nil símile habére. Et nisi quis ab ómnibus creatúris fúerit expedítus, non póterit líbere inténdere divínis. Ideo enim pauci inveniúntur contemplatívi, quia pauci sciunt se a peritúris et creatúris ad plenum sequestrári.}
\Normal{\Verse{2.} Ad hoc magna requíritur grátia, quæ ánimam levet et supra semetípsam rápiat. Et nisi homo sit in spíritu elevátus et ab ómnibus creatúris liberátus ac Deo totus unítus, quidquid scit, quidquid étiam habet, non est magni pónderis. Diu parvus erit et infra iacébit, qui áliquid magnum ǽstimat, nisi solum unum imménsum, ætérnum bonum. Et quidquid Deus non est, nihil est et pro níhilo computári debet. Est magna differéntia, sapiéntia illumináti et devóti viri et sciéntia litteráti atque studiósi clérici. Multo nobílior est illa doctrína, quæ de sursum ex divína influéntia manat, quam quæ laborióse humáno acquíritur ingénio.}
\Normal{\Verse{3.} Plures reperiúntur contemplatiónem desideráre~; sed quæ ad eam requirúntur, non student exercére. Est magnum impediméntum, quia in signis et sensibílibus rebus statur, et parum de perfécta mortificatióne habétur. Néscio quid est, quo spíritu dúcimur, et quid præténdimus, qui spirituáles dici vidémur, quod tantum labórem et ampliórem sollicitúdinem pro transitóriis et vílibus rebus ágimus, et de interióribus nostris vix raro plene recolléctis sénsibus cogitámus.}
\Normal{\Verse{4.} Proh dolor~! statim post módicam recollectiónem foras erúmpimus, nec ópera nostra distrícta examinatióne trutinámus. Ubi iacent afféctus nostri, non atténdimus, et quam impúra sint ómnia, non deplorámus. Omnis quippe caro corrúperat viam suam, et ídeo sequebátur dilúvium magnum. Cum ergo intérior afféctus noster multum corrúptus sit, necésse est, ut áctio sequens, index caréntiæ interióris vigóris, corrumpátur. Ex puro corde procédit fructus bonæ vitæ.}
\Normal{\Verse{5.} Quantum quis fécerit, quǽritur~; sed ex quanta virtúte agit, non tam studióse pensátur. Si fúerit fortis, dives, pulcher, hábilis, vel bonus scriptor, bonus cantor, bonus laborátor, investigátur~; quam pauper sit spíritu, quam pátiens et mitis, quam devótus et intérnus, a multis tacétur. Natúra exterióra hóminis réspicit, grátia ad interióra se convértit. Illa frequénter fállitur~; ista in Deo sperat, ut non decipiátur.}
\markright{XXXII}
%LIBER III
\TitreC{CAPUT XXXII}
\TitreC{De abnegatióne sui et abdicatióne omnis cupiditátis}
\Normal{\Verse{1.} Fili, non potes perféctam possidére libertátem, nisi totáliter ábneges temetípsum. Compedíti sunt omnes proprietárii et sui ipsíus amatóres cúpidi, curiósi, gyróvagi, quæréntes semper móllia, non quæ Iesu Christi, sed hoc sæpe fingéntes et componéntes, quod non stabit. Períbit enim totum, quod non est ex Deo ortum. Tene breve et consummátum verbum~: Dimítte ómnia, et invénies ómnia~; relínque cupídinem, et repéries réquiem. Hoc mente pertrácta~; et cum impléveris, intélleges ómnia.}
\Normal{\Verse{2.} Dómine, hoc non est opus uníus diéi, nec ludus parvulórum~; immo in hoc brevi inclúditur omnis perféctio religiosórum.}
\Normal{\Verse{3.} Fili, non debes avérti, nec statim déici, audíta via perfectórum~; sed magis ad sublimióra provocári, et ad minus ad hæc ex desidério suspiráre. Utinam sic tecum esset, et ad hoc pervenísses, ut tui ipsíus amátor non esses, sed ad nutum meum pure stares, et eius, quem tibi præpósui, Patris~: tunc mihi valde placéres, et tota vita tua in gáudio et pace transíret. Adhuc multa habes ad relinquéndum~: quæ nisi mihi ex íntegro resignáveris, non acquíres, quod póstulas. Suádeo tibi émere a me aurum ignítum, ut lócuples fias, id est, cæléstem sapiéntiam ómnia ínfima conculcántem. Postpóne terrénam sapiéntiam, omnem humánam et própriam complacéntiam.}
\Normal{\Verse{4.} Dixi, vilióra tibi eménda pro pretiósis et altis in rebus humánis. Nam valde vilis et parva ac pene oblivióni trádita vidétur vera cæléstis sapiéntia~; non sápiens alta de se, nec magnificári quærens in terra~: quam multi ore tenus prǽdicant, sed vita longe disséntiunt~: ipsa tamen est pretiósa margaríta a multis abscóndita.}
\markright{XXXIII}
%LIBER III
\TitreC{CAPUT XXXIII}
\TitreC{De instabilitáte cordis et de intentióne fináli ad Deum habénda}
\Normal{\Verse{1.} Fili, noli crédere afféctui tuo~; qui nunc est, cito mutábitur in áliud. Quámdiu víxeris, mutabilitáti subiéctus es, étiam nolens~: ut modo lætus, modo tristis, modo pacátus, modo turbátus, nunc devótus, nunc indevótus, nunc studiósus, nunc acediósus, nunc gravis, nunc levis inveniáris. Sed stat super hæc mutabília sápiens et bene doctus in spíritu, non atténdens, quid in se séntiat, vel qua parte flet ventus instabilitátis, sed ut tota inténtio mentis eius ad débitum et optátum profíciat finem. Nam sic póterit unus et idem inconcussúsque permanére, símplici intentiónis óculo per tot vários evéntus ad me imprætermísse dirécto.}
\Normal{\Verse{2.} Quanto autem púrior fúerit intentiónis óculus, tanto constántius inter divérsas itur procéllas. Sed in multis calígat óculus puræ intentiónis~; respícitur enim cito in áliquod delectábile, quod occúrrit. Nam et raro totus liber quis invenítur a nævo própriæ exquisitiónis. Sic Iudǽi olim vénerant in Bethániam ad Martham et Maríam, non propter Iesum tantum, sed et ut Lázarum vidérent. Mundándus est ergo intentiónis óculus, ut sit simplex et rectus, atque ultra ómnia vária média ad me dirigéndus.}
\markright{XXXIV}
%LIBER III
\TitreC{CAPUT XXXIV}
\TitreC{Quod amánti sapit Deus super ómnia et in ómnibus}
\Normal{\Verse{1.} Ecce, Deus meus et ómnia. Quid volo ámplius, et quid felícius desideráre possum~? O sápidum et dulce verbum~! sed amánti Verbum, non mundum, nec ea, quæ in mundo sunt. Deus meus et ómnia. Intellegénti satis dictum est, et sæpe repétere iucúndum est amánti. Te síquidem præsénte iucúnda sunt ómnia~; te autem absénte fastídiunt cuncta. Tu facis cor tranquíllum et pacem magnam lætitiámque festívam. Tu facis bene sentíre de ómnibus et in ómnibus te laudáre, nec potest áliquid sine te diu placére~; sed si debet gratum esse et bene sápere, opórtet grátiam tuam adésse et condiménto tuæ sapiéntiæ condíri.}
\Normal{\Verse{2.} Cui tu sapis, quid ei recte non sápiet~? Et cui tu non sapis, quid ei ad iucunditátem esse póterit~? Sed defíciunt in sapiéntia tua mundi sapiéntes, et qui carnem sápiunt~: quia ibi plúrima vánitas, et hic mors invenítur. Qui autem te per contémptum mundanórum et carnis mortificatiónem sequúntur, vere sapiéntes esse cognoscúntur~: quia de vanitáte ad veritátem, de carne ad spíritum transferúntur. Istis sapit Deus~: et quidquid boni invenítur in creatúris, totum ad laudem réferunt sui conditóris. Dissímilis tamen, et multum dissímilis sapor Creatóris et creatúræ, æternitátis et témporis, lucis increátæ et lucis illuminátæ.}
\Normal{\Verse{3.} O lux perpétua, cuncta creáta transcéndens lúmina, fúlgura coruscatiónem de sublími penetrántem ómnia cordis mei íntima. Purífica, lætífica, clarífica et vivífica spíritum meum, cum suis poténtiis ad inhæréndum tibi iubilósis excéssibus. O quando véniet hæc beáta et desiderábilis hora, ut tua me sáties præséntia et sis mihi ómnia in ómnibus~? Quámdiu hoc datum non fúerit, nec plenum gáudium erit. Adhuc, proh dolor, vivit in me vetus homo, non est totus crucifíxus, non est perfécte mórtuus. Adhuc concupíscit fórtiter contra spíritum, bella movet intestína, nec regnum ánimæ pátitur esse quiétum.}
\Normal{\Verse{4.} Sed tu, qui domináris potestáti maris et motum flúctuum eius mítigas, exúrge, ádiuva me. Díssipa gentes, quæ bella volunt~; cóntere eas in virtúte tua. Osténde, quæso, magnália tua, et glorificétur déxtera tua~: quia non est spes ália nec refúgium mihi, nisi in te, Dómine Deus meus.}
\markright{XXXV}
%LIBER III
\TitreC{CAPUT XXXV}
\TitreC{Quod non est secúritas a temptatióne in hac vita}
\Normal{\Verse{1.} Fili, numquam secúrus es in hac vita, sed quoad víxeris, semper arma spirituália tibi sunt necessária. Inter hostes versáris, et a dextris et a sinístris impugnáris. Si ergo non úteris úndique scuto patiéntiæ, non eris diu sine vúlnere. Insuper, si non ponis cor tuum fixe in me, cum mera voluntáte cuncta patiéndi propter me, non póteris ardórem istum sustinére, nec ad palmam pertíngere beatórum. Opórtet te ergo viríliter ómnia pertransíre et poténti manu uti advérsus obiécta. Nam vincénti datur manna, et torpénti relínquitur multa miséria.}
\Normal{\Verse{2.} Si quæris in hac vita réquiem, quómodo tunc pervénies ad ætérnam réquiem~? Non ponas te ad multam réquiem, sed ad magnam patiéntiam. Quære veram pacem, non in terris, sed in cælis, non in homínibus nec in céteris creatúris, sed in Deo solo. Pro amóre Dei debes ómnia libénter subíre, labóres scílicet et dolóres, temptatiónes, vexatiónes, anxietátes, necessitátes, infirmitátes, iniúrias, oblocutiónes, reprehensiónes, humiliatiónes, confusiónes, correctiónes et despectiónes. Hæc iuvant ad virtútem, hæc probant Christi tirónem, hæc fábricant cæléstem corónam. Ego reddam mercédem ætérnam pro brevi labóre, et infinítam glóriam pro transitória confusióne.}
\Normal{\Verse{3.} Putas tu, quod semper habébis pro tua voluntáte consolatiónes spirituáles~? Sancti mei non semper habuérunt tales, sed multas gravitátes et temptatiónes várias magnásque desolatiónes. Sed patiénter sustinuérunt se in ómnibus, et magis confísi sunt Deo, quam sibi, sciéntes, quia non sunt condígnæ passiónes huius témporis ad futúram glóriam promeréndam. Vis tu statim habére, quod multi post multas lácrimas et magnos labóres vix obtinuérunt~? Exspécta Dóminum, viríliter age, et confortáre~; noli diffídere, noli discédere, sed corpus et ánimam expóne constánter pro glória Dei. Ego reddam pleníssime, ego tecum ero in omni tribulatióne.}
\markright{XXXVI}
%LIBER III
\TitreC{CAPUT XXXVI}
\TitreC{Contra vana hóminum iudícia}
\Normal{\Verse{1.} Fili, iacta cor tuum fírmiter in Dómino, et humánum ne métuas iudícium, ubi te consciéntia pium reddit et insóntem. Bonum est et beátum táliter pati, nec hoc erit grave húmili cordi et Deo magis quam sibi ipsi confidénti. Multi multa loquúntur, et ídeo parva fides est adhibénda. Sed et ómnibus satis esse, non est possíbile. Etsi Paulus ómnibus stúduit in Dómino placére et ómnibus ómnia factus est, tamen étiam pro mínimo duxit, quod ab humáno die iudicátus fuit.}
\Normal{\Verse{2.} Egit satis pro aliórum ædificatióne et salúte, quantum in se erat et póterat~; sed ne ab áliis aliquándo iudicarétur, vel non despicerétur, cohibére non pótuit. Ideo totum Deo commísit, qui totum nóverat~; et patiéntia ac humilitáte contra ora loquéntium iníqua aut étiam vana ac mendósa cogitántium atque pro líbitu suo quæque iactántium se deféndit. Respóndit tamen intérdum, ne infírmis ex sua taciturnitáte generarétur scándalum.}
\Normal{\Verse{3.} Quis tu, ut tímeas a mortáli hómine~? Hódie est, et cras non compáret. Deum time, et hóminum terróres non expavésces. Quid potest áliquis in te verbis aut iniúriis~? Sibi pótius nocet, quam tibi~; nec póterit iudícium Dei effúgere, quicúmque est ille. Tu habe Deum præ óculis, et noli conténdere verbis querulósis. Quod si ad præsens tu vidéris succúmbi et confusiónem pati, quam non meruísti~: ne indignéris ex hoc, neque per impatiéntiam mínuas corónam tuam, sed ad me pótius réspice in cælum, qui potens sum erípere ab omni confusióne et iniúria, et unicuíque réddere secúndum ópera sua.}
\markright{XXXVII}
%LIBER III
\TitreC{CAPUT XXXVII}
\TitreC{De pura et íntegra resignatióne sui ad obtinéndam cordis libertátem}
\Normal{\Verse{1.} Fili, relínque te, et invénies me. Sta sine electióne et omni proprietáte, et lucráberis semper. Nam et adiciétur tibi ámplior grátia, statim ut te resignáveris nec resúmpseris.}
\Normal{\Verse{2.} Dómine, quótiens me resignábo, et in quibus me relínquam?}
\Normal{\Verse{3.} Semper et omni hora~: sicut in parvo, sic et in magno. Nihil excípio, sed in ómnibus te nudátum inveníri volo. Alióquin, quómodo póteris esse meus et ego tuus, nisi fúeris ab omni própria voluntáte intus et foris spoliátus~? Quanto celérius hoc agis, tanto mélius habébis, et quanto plénius et sincérius, tanto mihi plus placébis, et ámplius lucráberis.}
\Normal{\Verse{4.} Quidam se resígnant, sed cum áliqua exceptióne~: non enim plene Deo confídunt, ídeo sibi providére sátagunt. Quidam étiam primo totum ófferunt, sed póstea temptatióne pulsáti ad própria rédeunt, ídeo mínime in virtúte profíciunt. Hi ad veram puri cordis libertátem et iucúndæ familiaritátis meæ grátiam non pertíngent, nisi íntegra resignatióne et cotidiána sui immolatióne prius facta~; sine qua non stat nec stabit únio fruitíva.}
\Normal{\Verse{5.} Dixi tibi sæpíssime, et nunc íterum dico~: Relínque te, resígna te, et fruéris magna intérna pace. Da totum pro toto~; nil exquíre, nil répete~; sta pure et inhæsitánter in me, et habébis me. Eris liber in corde, et ténebræ non conculcábunt te. Ad hoc conáre, hoc ora, hoc desídera, ut ab omni proprietáte possis exspoliári, et nudus nudum Iesum sequi, tibi mori et mihi æternáliter vívere. Tunc defícient omnes vanæ phantásiæ, conturbatiónes iníquæ et curæ supérfluæ. Tunc étiam recédet immoderátus timor, et inordinátus amor moriétur.}
\markright{XXXVIII}
%LIBER III
\TitreC{CAPUT XXXVIII}
\TitreC{De bono regímine in extérnis et recúrsu ad Deum in perículis}
\Normal{\Verse{1.} Fili, ad istud diligénter téndere debes, ut in omni loco et actióne seu occupatióne extérna sis íntimus liber et tui ipsíus potens, et sint ómnia sub te, et tu non sub eis~: ut sis dóminus actiónum tuárum et rector, non servus, nec emptítius, sed magis exémptus verúsque Hebrǽus, in sortem ac libertátem tránsiens filiórum Dei~: qui stant super præséntia et speculántur ætérna, qui transitória sinístro intuéntur óculo et dextro cæléstia~: quos temporália non trahunt ad inhæréndum, sed trahunt ipsi magis ea ad bene serviéndum, prout ordináta sunt a Deo et institúta a summo opífice, qui nil inordinátum in sua relíquit creatúra.}
\Normal{\Verse{2.} Si étiam in omni evéntu stas non in apparéntia extérna, nec óculo carnáli lustras visa vel audíta, sed mox in quálibet causa intras cum Móyse in tabernáculum ad consuléndum Dóminum~: áudies nonnúmquam divínum respónsum, et rédies instrúctus de multis præséntibus et futúris. Semper enim Móyses recúrsum hábuit ad tabernáculum pro dúbiis et quæstiónibus solvéndis, fugítque ad oratiónis adiutórium, pro perículis et improbitátibus hóminum sublevándis. Sic et tu confúgere debes in cordis tui secretárium, divínum inténtius implorándo suffrágium. Proptérea namque Iósue et fílii Israel a Gabaonítis legúntur decépti~: quia os Dómini prius non interrogavérunt, sed nímium créduli dúlcibus sermónibus, falsa pietáte delúsi sunt.}
\markright{XXXIX}
%LIBER III
\TitreC{CAPUT XXXIX}
\TitreC{Quod homo non sit importúnus in negótiis}
\Normal{\Verse{1.} Fili, commítte mihi semper causam tuam, ego bene dispónam in témpore suo. Exspécta ordinatiónem meam, et sénties inde proféctum.}
\Normal{\Verse{2.} Dómine, satis libénter tibi omnes res commítto, quia parum potest cogitátio mea profícere. Utinam non multum adhærérem futúris evéntibus, sed ad beneplácitum tuum incunctánter me offérrem!}
\Normal{\Verse{3.} Fili, sæpe homo rem áliquam veheménter ágitat, quam desíderat~; sed cum ad eam pervénerit, áliter sentíre íncipit~: quia affectiónes circa idem non sunt durábiles, sed magis de uno ad áliud impéllunt. Non est ergo mínimum, étiam in mínimis seípsum relínquere.}
\Normal{\Verse{4.} Verus proféctus hóminis est abnegátio sui ipsíus, et homo abnegátus valde liber est et secúrus. Sed antíquus hostis, ómnibus bonis advérsans, a temptatióne non cessat~; sed die noctúque graves molítur insídias, si forte in láqueum deceptiónis possit præcipitáre incáutum. Vigiláte et oráte, dicit Dóminus, ut non intrétis in temptatiónem.}
\markright{XL}
%LIBER III
\TitreC{CAPUT XL}
\TitreC{Quod homo nihil boni ex se habet et de nullo gloriári potest}
\Normal{\Verse{1.} Dómine, quid est homo, quod memor es eius, aut fílius hóminis, quia vísitas eum~? Quid proméruit homo, ut dares illi grátiam tuam~? Dómine, quid possum cónqueri, si me déseris~; aut quid iuste obténdere possum, si, quod peto, non féceris~? Certe hoc in veritáte cogitáre possum et dícere~: Dómine, nihil sum, nihil possum, nihil boni ex me hábeo~; sed in ómnibus defício et ad nihil semper tendo. Et nisi a te fúero adiútus et intérius informátus, totus effícior tépidus et dissolútus.}
\Normal{\Verse{2.} Tu autem, Dómine, semper idem ipse es, et pérmanes in ætérnum semper bonus, iustus et sanctus, bene, iuste ac sancte agens ómnia, et dispónens in sapiéntia. Sed ego, qui ad deféctum sum magis pronus quam ad proféctum, non sum semper in uno statu perdúrans, quia semper témpora mutántur super me. Verúmtamen cito mélius fit, cum tibi placúerit, et manum porréxeris adiutrícem, quia tu solus sine humáno suffrágio póteris auxiliári et in tantum confirmáre, ut vultus meus ámplius in divérsa non mutétur, sed in te uno cor meum convertátur et quiéscat.}
\Normal{\Verse{3.} Unde, si bene scirem omnem humánam consolatiónem abícere, sive propter devotiónem adipiscéndam, sive propter necessitátem, qua compéllor te quǽrere, quia non est homo, qui me consolétur, tunc possem mérito de grátia tua speráre et de dono novæ consolatiónis exsultáre.}
\Normal{\Verse{4.} Grátias tibi, unde totum venit, quotiescúmque mihi bene succédit. Ego autem vánitas et níhilum ante te, incónstans homo et infírmus. Unde ergo possum gloriári, aut cur áppeto reputári~? Numquid de níhilo~? et hoc vaníssimum est. Vere inánis glória mala pestis, vánitas máxima~: quia a vera trahit glória et cælésti spóliat grátia. Dum enim homo cómplacet sibi, dísplicet tibi~: dum ínhiat láudibus humánis, privátur veris virtútibus.}
\Normal{\Verse{5.} Est autem vera glória et exsultátio sancta, gloriári in te et non in se, gaudére in nómine tuo, non in própria virtúte, nec in áliqua creatúra delectári, nisi propter te. Laudétur nomen tuum, non meum~; magnificétur opus tuum, non meum~; benedicátur nomen sanctum tuum, nihil mihi autem attribuátur de láudibus hóminum. Tu glória mea, tu exsultátio cordis mei. In te gloriábor et exsultábo tota die~; pro me autem nihil, nisi in infirmitátibus meis.}
\Normal{\Verse{6.} Quærant Iudǽi glóriam, quæ ab ínvicem est~; ego hanc requíram, quæ a solo Deo est. Omnis quidem glória humána, omnis honor temporális, omnis altitúdo mundána, ætérnæ glóriæ tuæ comparáta, vánitas est et stultítia. O véritas mea et misericórdia mea, Deus meus, Trínitas beáta, tibi soli laus, honor, virtus, glória, per infiníta sæculórum sǽcula.}
\markright{XLI}
%LIBER III
\TitreC{CAPUT XLI}
\TitreC{De contémptu omnis temporális honóris}
\Normal{\Verse{1.} Fili, noli tibi attráhere, si vídeas álios honorári et elevári, te autem déspici et humiliári. Erige cor tuum ad me in cælum, et non contristábit te contémptus hóminum in terris.}
\Normal{\Verse{2.} Dómine, in cæcitáte sumus, et vanitáte cito sedúcimur. Si recte me inspício, numquam mihi facta est iniúria ab áliqua creatúra, unde nec iuste hábeo cónqueri advérsus te. Quia autem frequénter et gráviter peccávi tibi, mérito armátur contra me omnis creatúra. Mihi ígitur iuste debétur confúsio et contémptus~; tibi autem laus, honor et glória. Et nisi me ad hoc præparávero, quod velim libénter ab omni creatúra déspici et relínqui, atque pénitus nihil vidéri, non possum intérius pacificári et stabilíri, nec spirituáliter illuminári, neque plene tibi uníri.}
\markright{XLII}
%LIBER III
\TitreC{CAPUT XLII}
\TitreC{Quod pax non est ponénda in homínibus}
\Normal{\Verse{1.} Fili, si ponis pacem tuam cum áliqua persóna, propter tuum sentíre et convívere, instábilis eris et implicátus. Sed si recúrsum habes ad semper vivéntem et manéntem veritátem, non contristábit amícus recédens aut móriens. In me debet amíci diléctio stare, et propter me diligéndus est, quisquis tibi bonus visus est et multum carus in hac vita. Sine me non valet nec durábit amicítia, nec est vera et munda diléctio, quam ego non cópulo. Ita mórtuus debes esse tálibus affectiónibus dilectórum hóminum, ut, quantum ad te pértinet, sine omni humáno optáres esse consórtio. Tanto homo Deo magis appropínquat, quanto ab omni solácio terréno lóngius recédit. Tanto étiam áltius ad Deum ascéndit, quanto profúndius in se descéndit et plus sibi ipsi viléscit.}
\Normal{\Verse{2.} Qui autem sibi áliquid boni attríbuit, grátiam Dei in se veníre ímpedit, quia grátia Spíritus Sancti cor húmile semper quærit. Si scires te perfécte adnihiláre atque ab omni creáto amóre evacuáre, tunc debérem in te cum magna grátia emanáre. Quando tu réspicis ad creatúras, subtráhitur tibi aspéctus Creatóris. Disce te in ómnibus propter Creatórem víncere, tunc ad divínam valébis cognitiónem pertíngere. Quantumcúmque módicum sit, si inordináte dilígitur et respícitur, retárdat a summo et vítiat.}
\markright{XLIII}
%LIBER III
\TitreC{CAPUT XLIII}
\TitreC{Contra vanam et sæculárem sciéntiam}
\Normal{\Verse{1.} Fili, non te móveant pulchra et subtília hóminum dicta. Non enim est regnum Dei in sermóne, sed in virtúte. Atténde verba mea, quæ corda accéndunt et mentes illúminant, indúcunt compunctiónem et váriam íngerunt consolatiónem. Numquam ad hoc legas verbum, ut dóctior aut sapiéntior possis vidéri. Stude mortificatióni vitiórum, quia hoc ámplius tibi próderit, quam notítia multárum difficílium quæstiónum.}
\Normal{\Verse{2.} Cum multa légeris et cognóveris, ad unum semper opórtet redíre princípium. Ego sum, qui dóceo hóminem sciéntiam, et clariórem intellegéntiam párvulis tríbuo, quam ab hómine possit docéri. Cui ego loquor, cito sápiens erit et multum in spíritu profíciet. Væ eis, qui multa curiósa ab homínibus inquírunt, et de via mihi serviéndi parum curant. Véniet tempus, quando apparébit magíster magistrórum Christus, dóminus angelórum, cunctórum auditúrus lectiónes, hoc est singulórum examinatúrus consciéntias. Et tunc scrutábitur Ierúsalem in lucérnis, et manifésta erunt abscóndita tenebrárum, tacebúntque arguménta linguárum.}
\Normal{\Verse{3.} Ego sum, qui húmilem in puncto élevo mentem, ut plures ætérnæ veritátis cápiat ratiónes, quam si quis decem annis studuísset in scholis. Ego dóceo sine strépitu verbórum, sine confusióne opiniónum, sine fastu honóris, sine pugnatione argumentórum. Ego sum, qui dóceo terréna despícere, præséntia fastidíre, ætérna quǽrere, ætérna sápere, honóres fúgere, scándala sufférre, omnem spem in me pónere, extra me nil cúpere, et super ómnia me ardénter amáre.}
\Normal{\Verse{4.} Nam quidam, amándo me íntime, dídicit divína et loquebátur mirabília. Plus profécit in relinquéndo ómnia, quam in studéndo subtília. Sed áliis loquor commúnia, áliis speciália~; alíquibus in signis et figúris dúlciter appáreo, quibúsdam vero in multo lúmine revélo mystéria. Una vox librórum, sed non omnes æque infórmat~: quia intus sum doctor veritátis, scrutátor cordis, cogitatiónum intelléctor, actiónum promótor, distríbuens síngulis, sicut dignum iudicávero.}
\markright{XLIV}
%LIBER III
\TitreC{CAPUT XLIV}
\TitreC{De non attrahéndo sibi res exterióres}
\Normal{\Verse{1.} Fili, in multis opórtet te esse ínscium, et æstimáre te tamquam mórtuum super terram, et cui totus mundus crucifíxus sit. Multa étiam opórtet surda aure pertransíre, et quæ tuæ pacis sunt, magis cogitáre. Utílius est óculos a rebus displicéntibus avértere et unicuíque suum sentíre relínquere, quam contentiósis sermónibus deservíre. Si bene stéteris cum Deo et eius iudícium aspéxeris, facílius te victum portábis.}
\Normal{\Verse{2.} O Dómine, quoúsque vénimus~? Ecce, damnum deflétur temporále, pro módico quæstu laborátur et cúrritur, et spirituále detriméntum in obliviónem transit, et vix sero redítur. Quod parum vel nihil prodest, atténditur~; et quod summe necessárium est, neglegénter præterítur~: quia totus homo ad extérna défluit~; et nisi cito resipíscat, libens in exterióribus iacet.}
\markright{XLV}
%LIBER III
\TitreC{CAPUT XLV}
\TitreC{Quod ómnibus non est credéndum, et de fácili lapsu verbórum}
\Normal{\Verse{1.} Da mihi auxílium, Dómine, de tribulatióne, quia vana salus hóminis. Quam sæpe ibi non invéni fidem, ubi me habére putávi~? Quótiens étiam ibi répperi, ubi minus præsúmpsi~? Vana ergo spes in homínibus, salus autem iustórum in te, Deus. Benedíctus sis, Dómine Deus meus, in ómnibus, quæ áccidunt nobis. Infírmi sumus et instábiles, cito fállimur et permutámur.}
\Normal{\Verse{2.} Quis est homo, qui ita caute et circumspécte in ómnibus se custodíre valet, ut aliquándo in áliquam deceptiónem vel perplexitátem non véniat~? Sed qui in te, Dómine, confídit, ac símplici ex corde quærit, non tam fácile lábitur. Et si incíderit áliquam tribulatiónem, quocúmque modo fúerit étiam implicátus, cítius per te eruétur, aut a te consolábitur~: quia tu non déseres in te sperántem usque in finem. Rarus fidus amícus, in cunctis amíci persevérans pressúris. Tu, Dómine, tu solus es fidelíssimus in ómnibus, et præter te non est alter talis.}
\Normal{\Verse{3.} O quam bene sápuit sancta illa ánima, quæ dixit~: Mens mea solidáta est et in Christo fundáta. Si ita mecum foret, non tam fácile timor humánus me sollicitáret, nec verbórum iácula movérent. Quis ómnia prævidére, quis præcavére futúra mala súfficit~? Si prævísa sæpe étiam lædunt, quid improvísa nisi gráviter fériunt~? Sed quare mihi mísero non mélius provídi~? Cur étiam tam fácile áliis crédidi~? Sed hómines sumus, nec áliud quam frágiles hómines sumus, étiam si ángeli a multis æstimámur et dícimur. Cui credam, Dómine~? cui, nisi tibi~? Véritas es, quæ non fallis, nec falli potes. Et rursum~: Omnis homo mendax, infírmus, instábilis et lábilis máxime in verbis, ita ut statim vix credi débeat, quod rectum in fácie sonáre vidétur.}
\Normal{\Verse{4.} Quam prudénter præmonuísti, cavéndum ab homínibus, et quia inimíci hóminis doméstici eius~; nec credéndum, si quis díxerit~: Ecce hic, aut~: Ecce illic. Doctus sum damno meo, et útinam ad cautélam maiórem, et non ad insipiéntiam mihi. Cautus esto, ait quidam, cautus esto, serva apud te, quod dico. Et dum ego síleo et abscónditum credo, nec ille silére potest, quod siléndum pétiit, sed statim prodit me et se, et ábiit. Ab huiúsmodi fábulis et incáutis homínibus prótege me, Dómine, ne in manus eórum íncidam, nec umquam tália commíttam. Verbum verum et stábile da in os meum, et linguam cállidam longe fac a me. Quod pati nolo, omnímode cavére débeo.}
\Normal{\Verse{5.} O quam bonum et pacíficum de áliis silére, nec indifferénter ómnia crédere, neque de fácili ultérius effári, paucis seípsum reveláre, te semper inspectórem cordis quǽrere, nec omni vento verbórum circumférri, sed ómnia íntima et extérna secúndum plácitum tuæ voluntátis optáre pérfici~! Quam tutum pro conservatióne cæléstis grátiæ humánam fúgere apparéntiam, nec appétere, quæ foris admiratiónem vidéntur præbére, sed ea tota sedulitáte sectári, quæ vitæ emendatiónem dant et fervórem~! Quam multis nócuit virtus scita ac præprópere laudáta~! Quam sane prófuit grátia siléntio serváta in hac frágili vita, quæ tota temptátio fertur et milítia!}
\markright{XLVI}
%LIBER III
\TitreC{CAPUT XLVI}
\TitreC{De confidéntia in Deo habénda, quando insúrgunt verbórum iácula}
\Normal{\Verse{1.} Fili, sta fírmiter et spera in me. Quid enim sunt verba, nisi verba~? Per aërem volant, sed lápidem non lædunt. Si reus es, cógita, quod libénter emendáre te velis~; si nihil tibi cónscius es, pensa, quod velis libénter pro Deo hoc sustinére. Parum satis est, ut vel verba intérdum sustíneas, qui necdum fórtia vérbera toleráre vales. Et quare tam parva tibi ad cor tránseunt, nisi quia adhuc carnális es, et hómines magis, quam opórtet, atténdis~? Nam quia déspici métuis, reprehéndi pro excéssibus non vis, et excusatiónum umbrácula quæris.}
\Normal{\Verse{2.} Sed ínspice te mélius, et agnósces, quia vivit adhuc in te mundus et vanus amor placéndi homínibus. Cum enim bassári réfugis et confúndi pro deféctibus, constat útique, quod nec verus húmilis sis, nec vere mundo mórtuus, nec mundus tibi crucifíxus. Sed audi verbum meum, et non curábis decem mília verba hóminum. Ecce, si cuncta contra te diceréntur, quæ fingi malitiosíssime possent~; quid tibi nocérent, si omníno transíre permítteres, nec plus quam festúcam perpénderes~? Numquid vel unum capíllum tibi extráhere possent?}
\Normal{\Verse{3.} Sed qui cor intus non habet, nec Deum præ óculis, facíliter verbo movétur vituperatiónis. Qui autem in me confídit, nec próprio iudício stare áppetit, absque humáno terróre erit. Ego enim sum iudex, et cógnitor ómnium secretórum~; ego scio, quáliter res acta est~; ego iniuriántem novi et sustinéntem. A me éxiit verbum istud, me permitténte hoc áccidit, ut reveléntur ex multis córdibus cogitatiónes. Ego reum et innocéntem iudicábo, sed occúlto iudício utrúmque ante probáre vólui.}
\Normal{\Verse{4.} Testimónium hóminum sæpe fallit~; meum iudícium verum est, stabit, et non subvertétur. Latet plerúmque et paucis ad síngula patet~; numquam tamen errat, nec erráre potest, étiam si óculis insipiéntium non rectum videátur. Ad me ergo curréndum est in omni iudício, nec próprio inniténdum arbítrio. Iustus enim non conturbábitur, quidquid a Deo ei accíderit. Etiam si iniúste áliquid contra eum prolátum fúerit, non multum curábit. Sed nec vane exsultábit, si per álios rationabíliter excusétur. Pensat namque, quia ego sum scrutans corda et renes, qui non iúdico secúndum fáciem et humánam apparéntiam. Nam sæpe in óculis meis reperítur culpábile, quod hóminum iudício créditur laudábile.}
\Normal{\Verse{5.} Dómine Deus, iudex iuste, fortis et pátiens, qui hóminum nosti fragilitátem et pravitátem, esto robur meum et tota fidúcia mea~: non enim mihi súfficit consciéntia mea. Tu nosti, quod ego non novi~; et ídeo in omni reprehensióne humiliáre me débui et mansuéte sustinére. Ignósce quoque mihi propítius, quótiens sic non egi, et dona íterum grátiam amplióris sufferéntiæ. Mélior est enim mihi tua copiósa misericórdia, ad consecutiónem indulgéntiæ, quam mea opináta iustítia, pro defensióne laténtis consciéntiæ. Et si nihil mihi cónscius sum, tamen in hoc iustificáre me non possum~: quia remóta misericórdia tua, non iustificábitur in conspéctu tuo omnis vivens.}
\markright{XLVII}
%LIBER III
\TitreC{CAPUT XLVII}
\TitreC{Quod ómnia grávia pro ætérna vita sunt toleránda}
\Normal{\Verse{1.} Fili, non te frangant labóres, quos assumpsísti propter me, nec tribulatiónes te deíciant usquequáque~; sed mea promíssio in omni evéntu te róboret et consolétur. Ego suffíciens sum ad reddéndum supra omnem modum et mensúram. Non diu hic laborábis, nec semper graváberis dolóribus. Exspécta paulísper, et vidébis célerem finem malórum. Véniet una hora, quando cessábit omnis labor et tumúltus. Módicum est et breve omne, quod transit cum témpore.}
\Normal{\Verse{2.} Age, quod agis~; fidéliter labóra in vínea mea, ego ero merces tua. Scribe, lege, canta, geme, tace, ora, sústine viríliter contrária~: digna est his ómnibus et maióribus prœ́liis vita ætérna. Véniet pax in die una, quæ nota est Dómino~; et erit non dies neque nox huius scílicet témporis, sed lux perpétua, cláritas infiníta, pax firma, et réquies secúra. Non dices tunc~: Quis me liberábit de córpore mortis huius~? Neque clamábis~: Heu mihi, quia incolátus meus prolongátus est~! quóniam præcipitábitur mors, et salus erit indefectíva, anxíetas nulla, iucúnditas beáta, socíetas dulcis et decóra.}
\Normal{\Verse{3.} O si vidísses sanctórum in cælo corónas perpétuas, quanta quoque nunc exsúltant glória, qui huic mundo olim contemptíbiles et quasi vita ipsa indígni putabántur~; profécto te statim humiliáres usque ad terram, et affectáres pótius ómnibus subésse, quam uni præésse~; nec huius vitæ lætos dies concupísceres, sed magis pro Deo tribulári gaudéres, et pro níhilo inter hómines computári, máximum lucrum dúceres.}
\Normal{\Verse{4.} O si tibi hæc sáperent, et profúnde ad cor transírent, quómodo audéres vel semel cónqueri~? Nonne pro vita ætérna cuncta laboriósa sunt toleránda~? Non est parvum quid, pérdere aut lucrári regnum Dei. Leva ígitur fáciem tuam in cælo. Ecce, ego et omnes sancti mei mecum, qui in hoc sǽculo magnum habuére certámen, modo gaudent, modo consolántur, modo secúri sunt, modo requiéscunt, et sine fine mecum in regno patris mei permanébunt.}
\markright{XLVIII}
%LIBER III
\TitreC{CAPUT XLVIII}
\TitreC{De die æternitátis et huius vitæ angústiis}
\Normal{\Verse{1.} O supérnæ civitátis mánsio beatíssima~! O dies æternitátis claríssima, quam nox non obscúrat, sed summa véritas semper irrádiat~; dies semper læta, semper secúra, et numquam statum mutans in contrária~! O útinam dies illa illuxísset, et cuncta hæc temporália finem accepíssent~! Lucet quidem sanctis perpétua claritáte spléndida, sed non nisi a longe et per spéculum peregrinántibus in terra.}
\Normal{\Verse{2.} Norunt cæli cives, quam gaudiósa sit illa~; gemunt éxsules fílii Evæ, quod amára et tædiósa sit ista. Dies huius témporis parvi et mali, pleni dolóribus et angústiis~: ubi homo multis peccátis inquinátur, multis passiónibus irretítur, multis timóribus stríngitur, multis curis disténditur, multis curiositátibus distráhitur, multis vanitátibus implicátur, multis erróribus circumfúnditur, multis labóribus attéritur, temptatiónibus gravátur, delíciis enervátur, egestáte cruciátur.}
\Normal{\Verse{3.} O quando finis horum malórum~? quando liberábor a mísera servitúte vitiórum~? quando memorábor, Dómine, tui solíus~? quando ad plenum lætábor in te~? Quando ero sine omni impediménto in vera libertáte, sine omni gravámine mentis et córporis~? Quando erit pax sólida, pax imperturbábilis et secúra, pax intus et foris, pax ex omni parte firma~? Iesu bone, quando stabo ad vidéndum te~? quando contemplábor glóriam regni tui~? quando eris mihi ómnia in ómnibus~? O quando ero tecum in regno tuo, quod præparásti diléctis tuis ab ætérno~! Relíctus sum pauper et exsul in terra hostíli, ubi bella cotidiána et infortúnia máxima.}
\Normal{\Verse{4.} Consoláre exílium meum, mítiga dolórem meum, quia ad te suspírat omne desidérium meum. Nam onus mihi totum est, quidquid hic mundus offert ad solácium. Desídero te íntime frui, sed néqueo apprehéndere. Opto inhærére cæléstibus, sed déprimunt res temporáles et immortificátæ passiónes. Mente ómnibus rebus superésse volo, carne autem invíte subésse cogor. Sic ego homo infélix mecum pugno, et factus sum mihimetípsi gravis, dum spíritus sursum et caro quærit esse deórsum.}
\Normal{\Verse{5.} O quid intus pátior, dum mente cæléstia tracto, et mox carnálium turba occúrrit oránti~! Deus meus, ne elongéris a me, neque declínes in ira a servo tuo. Fúlgura coruscatiónem tuam et díssipa eas~; emítte sagíttas tuas, et conturbéntur omnes phantásiæ inimíci. Recóllige sensus meos ad te~; fac me oblivísci ómnium mundanórum~; da cito abícere et contémnere phantásmata vitiórum. Succúrre mihi, ætérna véritas, ut nulla me móveat vánitas. Adveni, cæléstis suávitas, et fúgiat a fácie tua omnis impúritas. Ignósce quoque mihi et misericórditer indúlge, quótiens præter te áliud in oratióne revólvo. Confíteor étenim vere, quia valde distrácte me habére consuévi. Nam ibi multótiens non sum, ubi corporáliter sto aut sédeo~; sed ibi magis sum, quo cogitatiónibus feror. Ibi sum, ubi cogitátio mea est. Ibi est frequénter cogitátio mea, ubi est, quod amo. Hoc mihi cito occúrrit, quod naturáliter deléctat aut ex usu placet.}
\Normal{\Verse{6.} Unde tu, véritas, apérte dixísti~: Ubi enim est thesáurus tuus, ibi est et cor tuum. Si cælum díligo, libénter de cæléstibus penso. Si mundum amo, mundi felicitátibus congáudeo et de adversitátibus eius tristor. Si carnem díligo, quæ carnis sunt, sæpe imáginor. Si spíritum amo, de spirituálibus cogitáre deléctor. Quæcúmque enim díligo, de his libénter loquor et áudio, atque tálium imágines mecum ad domum repórto. Sed beátus ille homo, qui propter te, Dómine, ómnibus creatúris licéntiam abeúndi tríbuit, qui natúræ vim facit et concupiscéntias carnis fervóre spíritus crucifígit, ut serenáta consciéntia puram tibi oratiónem ófferat, dignúsque sit angélicis interésse choris, ómnibus terrénis foris et intus exclúsis.}
\markright{XLIX}
%LIBER III
\TitreC{CAPUT XLIX}
\TitreC{De desidério ætérnæ vitæ, et quanta sint certántibus bona promíssa}
\Normal{\Verse{1.} Fili, cum tibi desidérium ætérnæ beatitúdinis désuper infúndi sentis, et de tabernáculo córporis exíre concupíscis, ut claritátem meam sine vicissitúdinis umbra contemplári possis~: diláta cor tuum et omni desidério hanc sanctam inspiratiónem súscipe. Redde amplíssimas supérnæ bonitáti grátias, quæ tecum sic dignánter agit, cleménter vísitat, ardénter éxcitat, poténter súblevat, ne próprio póndere ad terréna labáris. Neque enim hoc cogitátu tuo aut conátu áccipis, sed sola dignatióne supérnæ grátiæ et divíni respéctus~: quátenus in virtútibus et maióri humilitáte profícias, et ad futúra certámina te prǽpares, mihíque toto cordis afféctu adhærére ac fervénti voluntáte stúdeas deservíre.}
\Normal{\Verse{2.} Fili, sæpe ignis ardet, sed sine fumo flamma non ascéndit. Sic et aliquórum desidéria ad cæléstia flagrant, et tamen a temptatióne carnális afféctus líberi non sunt. Idcírco nec omníno pure pro honóre Dei agunt, quod tam desideránter ab eo petunt. Tale est et tuum sæpe desidérium, quod insinuásti fore tam importúnum. Non enim est hoc purum et perféctum, quod própria commoditáte est inféctum.}
\Normal{\Verse{3.} Pete, non quod tibi est delectábile et cómmodum, sed quod mihi est acceptábile atque honoríficum~: quia, si recte iúdicas, meam ordinatiónem tuo desidério et omni desideráto præférre debes ac sequi. Novi desidérium tuum, et frequéntes gémitus audívi. Iam velles esse in libertáte glóriæ filiórum Dei~; iam te deléctat domus ætérna et cæléstis pátria gáudio plena, sed nondum venit hora ista~: sed est adhuc áliud tempus, scílicet tempus belli, tempus labóris et probatiónis. Optas summo repléri bono, sed non potes hoc ássequi modo. Ego sum~: exspécta me, dicit Dóminus, donec véniat regnum Dei.}
\Normal{\Verse{4.} Probándus es adhuc in terris et in multis exercitándus. Consolátio tibi intérdum dábitur, sed copiósa satíetas non concéditur. Confortáre ígitur, et esto robústus, tam in agéndo quam in patiéndo natúræ contrária. Opórtet te novum indúere hóminem et in álterum virum mutári. Opórtet te sæpe ágere, quod non vis~; et quod vis, opórtet relínquere. Quod áliis placet, procéssum habébit~; quod tibi placet, ultra non profíciet. Quod álii dicunt, audiétur~; quod tu dicis, pro níhilo computábitur. Petent álii, et accípient~; tu petes, nec impetrábis.}
\Normal{\Verse{5.} Erunt álii magni in ore hóminum, de te autem tacébitur. Aliis hoc vel illud committétur, tu autem ad nihil útilis iudicáberis. Propter hoc natúra quandóque contristábitur~; et magnum, si silens portáveris. In his et simílibus multis probári solet fidélis Dómini servus, quáliter se abnegáre, et in ómnibus frángere quíverit. Vix est áliquid tale, in quo tantúndem mori índiges, sicut vidére et pati, quæ voluntáti tuæ advérsa sunt~; máxime autem, cum disconveniéntia et quæ minus utília tibi appárent, fíeri iubéntur. Et quia non audes resístere altióri potestáti, sub domínio constitútus~: ídeo durum tibi vidétur ad nutum altérius ambuláre et omne próprium sentíre omíttere.}
\Normal{\Verse{6.} Sed pensa, fili, horum fructum labórum, célerem finem atque prǽmium nimis magnum~; et non habébis inde gravámen, sed fortíssimum patiéntiæ tuæ solámen. Nam et pro módica hac voluntáte, quam nunc sponte déseris, habébis semper voluntátem tuam in cælis. Ibi quippe invénies omne, quod volúeris, omne, quod desideráre póteris. Ibi áderit tibi totíus facúltas boni sine timóre amitténdi. Ibi volúntas tua una semper mecum, nil cúpiet extráneum vel privátum. Ibi nullus resístet tibi, nemo de te conquerétur, nemo impédiet, nihil obviábit~; sed cuncta desideráta simul erunt præséntia, totúmque afféctum tuum refícient et adimplébunt usque ad summum. Ibi reddam glóriam pro contumélia perpéssa, pállium laudis pro mæróre, pro loco novíssimo sedem regni in sǽcula. Ibi apparébit fructus obœdiéntiæ, gaudébit labor pæniténtiæ, et húmilis subiéctio coronábitur glorióse.}
\Normal{\Verse{7.} Nunc ergo inclína te humíliter sub ómnium mánibus~; nec sit tibi curæ, quis hoc díxerit vel iússerit. Sed hoc magnópere curáto, ut sive prælátus seu iúnior aut æquális áliquid a te expóscerit vel innúerit, pro bono totum accípias, et sincéra voluntáte stúdeas adimplére. Quærat álius hoc, álius illud, gloriétur ille in illo, et iste in isto, laudetúrque míllies mille~: tu autem nec in isto, nec in illo, sed in tui ipsíus gaude contémptu, et in mei solíus beneplácito ac honóre. Hoc optándum est tibi, ut sive per vitam, sive per mortem Deus semper in te glorificétur.}
\markright{L}
%LIBER III
\TitreC{CAPUT L}
\TitreC{Quáliter homo desolátus se debet in manus Dei offérre}
\Normal{\Verse{1.} Dómine Deus, sancte Pater, sis nunc et in ætérnum benedíctus, quia, sicut vis, sic factum est, et quod facis, bonum est. Lætétur in te servus tuus, non in se, nec in áliquo álio~; quia tu solus lætítia vera, tu spes mea et coróna mea, tu gáudium meum et honor meus, Dómine. Quid habet servus tuus, nisi quod a te accépit, étiam sine mérito suo~? Tua sunt ómnia, quæ dedísti et quæ fecísti. Pauper sum et in labóribus meis a iuventúte mea~: et contristátur ánima mea nonnúmquam usque ad lácrimas, quandóque étiam conturbátur ad se propter imminéntes passiónes.}
\Normal{\Verse{2.} Desídero pacis gáudium, pacem filiórum tuórum flágito, qui in lúmine consolatiónis a te pascúntur. Si das pacem, si gáudium sanctum infúndis, erit ánima servi tui plena modulatióne, et devóta in laude tua. Sed si te subtráxeris, sicut sæpíssime soles, non póterit cúrrere viam mandatórum tuórum, sed magis ad tundéndum pectus génua eius incurvántur~: quia non est illi sicut heri et núdius tértius, quando splendébat lucérna tua super caput eius, et sub umbra alárum tuárum protegebátur a temptatiónibus irruéntibus.}
\Normal{\Verse{3.} Pater iuste et semper laudánde, venit hora, ut probétur servus tuus. Pater amánde, dignum est, ut hac hora patiátur pro te áliquid servus tuus. Pater perpétuo veneránde, venit hora, quam ab ætérno præsciébas affutúram, ut ad módicum tempus succúmbat foris servus tuus, vivat vero semper apud te intus. Páululum vilipendátur, humiliétur et defíciat coram homínibus, passiónibus conterátur et languóribus~: ut íterum tecum in auróra novæ lucis resúrgat et in cæléstibus clarificétur. Pater sancte, tu sic ordinásti et sic voluísti~; et hoc factum est, quod ipse præcepísti.}
\Normal{\Verse{4.} Hæc est enim grátia ad amícum tuum, pati et tribulári in mundo pro amóre tuo, quotiescúmque et a quocúmque id permíseris fíeri. Sine consílio et providéntia tua et sine causa nihil fit in terra. Bonum mihi, Dómine, quod humiliásti me, ut discam iustificatiónes tuas, et omnes elatiónes cordis atque præsumptiónes abíciam. Utile mihi, quod confúsio coopéruit fáciem meam~; ut te pótius, quam hómines ad consolándum requíram. Dídici étiam ex hoc inscrutábile iudícium tuum expavéscere~: qui afflígis iustum cum ímpio, sed non sine æquitáte et iustítia.}
\Normal{\Verse{5.} Grátias tibi ago, quia non pepercísti malis meis, sed attrivísti me verbéribus amáris, inflígens dolóres et immíttens angústias foris et intus. Non est, qui me consolétur ex ómnibus, quæ sub cælo sunt, nisi tu, Dómine Deus meus, cæléstis médicus animárum~: qui pércutis et sanas, dedúcis ad ínferos et redúcis. Disciplína tua super me, et virga tua ipsa me docébit.}
\Normal{\Verse{6.} Ecce, Pater dilécte, in mánibus tuis sum ego, sub virga correctiónis tuæ me inclíno. Pércute dorsum meum et collum meum, ut incúrvem ad voluntátem tuam tortuositátem meam. Fac me pium et húmilem discípulum, sicut bene fácere consuevísti, ut ámbulem ad omnem nutum tuum. Tibi me et ómnia mea ad corrigéndum comméndo~; mélius est hic córripi, quam in futúro. Tu scis ómnia et síngula, et nil te latet in humána consciéntia. Antequam fiant, nosti ventúra~: et non opus est tibi, ut quis te dóceat aut admóneat de his, quæ gerúntur in terra. Tu scis, quid éxpedit ad proféctum meum, et quantum desérvit tribulátio ad rubíginem vitiórum purgándam. Fac mecum desiderátum beneplácitum tuum, et ne despícias peccaminósam vitam meam, nulli mélius nec clárius, quam tibi soli, notam.}
\Normal{\Verse{7.} Da mihi, Dómine, scire, quod sciéndum est, hoc amáre, quod amándum est, hoc laudáre, quod tibi summe placet~; hoc reputáre, quod tibi pretiósum appáret, hoc vituperáre, quod óculis tuis sordéscit. Non me sinas secúndum visiónem oculórum exteriórum iudicáre, neque secúndum audítum áurium hóminum imperitórum sententiáre~; sed in iudício vero de visibílibus et spirituálibus discérnere, atque super ómnia voluntátem benepláciti tui semper inquírere.}
\Normal{\Verse{8.} Fallúntur sæpe hóminum sensus in iudicándo~; fallúntur et amatóres sǽculi, visibília tantúmmodo amándo. Quid est homo inde mélior, quia reputátur ab hómine maior~? Fallax fallácem, vanus vanum, cæcus cæcum, infírmus infírmum décipit, dum exáltat~; et veráciter magis confúndit, dum inániter laudat. Nam quantum unusquísque est in óculis tuis, tantum est, et non ámplius, ait húmilis sanctus Francíscus.}
\markright{LI}
%LIBER III
\TitreC{CAPUT LI}
\TitreC{Quod humílibus insisténdum est opéribus, cum defícitur a summis}
\Normal{\Verse{1.} Fili, non vales semper in ferventióri desidério virtútum stare, nec in altióri gradu contemplatiónis consístere~: sed necésse habes intérdum ob originálem corruptélam ad inferióra descéndere, et onus corruptíbilis vitæ étiam invíte et cum tǽdio portáre. Quámdiu mortále corpus geris, tǽdium sénties et gravámen cordis. Opórtet ergo sæpe in carne de carnis ónere gémere, eo quod non vales spirituálibus stúdiis et divínæ contemplatióni indesinénter inhærére.}
\Normal{\Verse{2.} Tunc éxpedit tibi ad humília et exterióra ópera confúgere, et in bonis áctibus te recreáre, advéntum meum et supérnam visitatiónem firma confidéntia exspectáre, exílium tuum et ariditátem mentis patiénter sufférre, donec íterum a me visitéris et ab ómnibus anxietátibus liberéris. Nam fáciam te labórum oblivísci et intérna quiéte pérfrui. Expándam coram te prata scripturárum, ut, dilatáto corde, cúrrere incípias viam mandatórum meórum. Et dices~: Non sunt condígnæ passiónes huius témporis ad futúram glóriam, quæ revelábitur in nobis.}
\markright{LII}
%LIBER III
\TitreC{CAPUT LII}
\TitreC{Quod homo non réputet se consolatióne dignum, sed magis verbéribus reum}
\Normal{\Verse{1.} Dómine, non sum dignus consolatióne tua nec áliqua spirituáli visitatióne~; et ídeo iuste mecum agis, quando me ínopem et desolátum relínquis. Si enim ad instar maris lácrimas fúndere possem, adhuc consolatióne tua dignus non essem. Unde nihil dignus sum, quam flagellári et puníri, quia gráviter et sæpe te offéndi et in multis valde delíqui. Ergo vera pensáta ratióne, nec mínima sum dignus consolatióne. Sed tu, clemens et miséricors Deus, qui non vis períre ópera tua, ad ostendéndum divítias bonitátis tuæ in vasa misericórdiæ, étiam præter omne próprium méritum dignáris consolári servum tuum supra humánum modum. Tuæ enim consolatiónes non sunt sicut humánæ confabulatiónes.}
\Normal{\Verse{2.} Quid egi, Dómine, ut mihi conférres áliquam cæléstem consolatiónem~? Ego nihil boni me egísse recólo, sed semper ad vítia pronum et ad emendatiónem pigrum fuísse. Verum est, et negáre non possum. Si áliter dícerem, tu stares contra me, et non esset, qui defénderet. Quid mérui pro peccátis meis, nisi inférnum et ignem ætérnum~? In veritáte confíteor, quóniam dignus sum omni ludíbrio et contémptu, nec decet me inter tuos devótos commemorári. Et licet hoc ægre áudiam, tamen advérsum me pro veritáte peccáta mea árguam, ut facílius misericórdiam tuam mérear impetráre.}
\Normal{\Verse{3.} Quid dicam reus et omni confusióne plenus~? Non hábeo os loquéndi, nisi hoc tantum verbum~: Peccávi, Dómine, peccávi~; miserére mei, ignósce mihi. Sine me páululum, ut plangam dolórem meum, ántequam vadam ad terram tenebrósam et opértam mortis calígine. Quid tam máxime a reo et mísero peccatóre requíris, nisi ut conterátur et humíliet se pro delíctis suis~? In vera contritióne et cordis humiliatióne náscitur spes véniæ, reconciliátur perturbáta consciéntia, reparátur grátia pérdita, tuétur homo a futúra ira, et occúrrunt sibi mútuo, in ósculo sancto, Deus et pǽnitens ánima.}
\Normal{\Verse{4.} Húmilis peccatórum contrítio acceptábile tibi est, Dómine, sacrifícium, longe suávius odórans in conspéctu tuo, quam thuris incénsum. Hæc est gratum étiam unguéntum, quod sacris pédibus tuis infúndi voluísti~: quia cor contrítum et humiliátum numquam despexísti. Ibi est locus refúgii a fácie iræ inimíci. Ibi emendátur et ablúitur, quidquid aliúnde contráctum est et inquinátum.}
\markright{LIII}
%LIBER III
\TitreC{CAPUT LIII}
\TitreC{Quod grátia Dei non miscétur terréna sapiéntibus}
\Normal{\Verse{1.} Fili, pretiósa est grátia mea, non pátitur se miscéri extráneis rebus nec consolatiónibus terrénis. Abícere ergo opórtet ómnia impediménta grátiæ, si optas eius infusiónem suscípere. Pete secrétum tibi, ama solus habitáre tecum, nullíus requíre confabulatiónem~: sed magis ad Deum devótam effúnde precem, ut compúnctam téneas mentem et puram consciéntiam. Totum mundum nihil ǽstima~; Dei vacatiónem ómnibus exterióribus antepóne. Non enim póteris mihi vacáre et in transitóriis páriter delectári. A notis et a caris opórtet elongári, et ab omni temporáli solácio mentem tenére privátam. Sic óbsecrat beátus apóstolus Petrus, ut tamquam ádvenas et peregrínos in hoc mundo se contíneant Christi fidéles.}
\Normal{\Verse{2.} O quanta fidúcia erit moritúro, quem nullíus rei afféctus détinet in mundo. Sed sic segregátum cor habére ab ómnibus, æger necdum capit ánimus, nec animális homo novit intérni hóminis libertátem. Attamen si vere velit esse spirituális, opórtet eum renuntiáre tam remótis quam propínquis, et a némine magis cavére, quam a se ipso. Si temetípsum perfécte víceris, cétera facílius subiugábis. Perfécta victória est de semetípso triumpháre. Qui enim semetípsum subiéctum tenet, ut sensuálitas ratióni, et rátio in cunctis obœ́diat mihi, hic vere victor est sui et dóminus mundi.}
\Normal{\Verse{3.} Si ad hunc ápicem scándere gliscis, opórtet viríliter incípere, et secúrim ad radícem pónere, ut evéllas et déstruas occúltam inordinátam inclinatiónem ad teípsum et ad omne privátum et materiále bonum. Ex hoc vítio, quod homo semetípsum nimis inordináte díligit, pene totum pendet, quidquid radicáliter vincéndum est~: quo devícto et subácto malo pax magna et tranquíllitas erit contínuo. Sed quia pauci sibi ipsis perfécte mori labórant, nec plene extra se tendunt, proptérea in se implicáti rémanent, nec supra se in spíritu elevári possunt. Qui autem líbere mecum ambuláre desíderat, necésse est, ut omnes pravas et inordinátas affectiónes suas mortíficet, atque nulli creatúræ priváto amóre concupiscenter inhǽreat.}
\markright{LIV}
%LIBER III
\TitreC{CAPUT LIV}
\TitreC{De divérsis mótibus natúræ et grátiæ}
\Normal{\Verse{1.} Fili, diligénter advérte motus natúræ et grátiæ, quia valde contrárie et subtíliter movéntur, et vix, nisi a spirituáli et íntimo illumináto hómine, discernúntur. Omnes quidem bonum áppetunt, et áliquid boni in suis dictis vel factis præténdunt~; ídeo sub spécie boni multi fallúntur.}
\Normal{\Verse{2.} Natúra cállida est, et multos trahit, illáqueat et décipit, et se semper pro fine habet~: sed grátia simplíciter ámbulat, ab omni spécie mala declínat, fallácias non præténdit, et ómnia pure propter Deum agit, in quo et fináliter requiéscit.}
\Normal{\Verse{3.} Natúra invíte vult mori, nec premi, nec superári, nec subésse, nec sponte subiugári~: grátia vero studet mortificatióni própriæ, resístit sensualitáti, quærit súbici, áppetit vinci, nec própria vult libertáte fungi~; sub disciplína amat tenéri, nec alícui cupit dominári, sed sub Deo semper vívere, stare et esse~; atque propter Deum omni humánæ creatúræ humíliter paráta est inclinári.}
\Normal{\Verse{4.} Natúra pro suo cómmodo labórat, et quid lucri ex álio sibi provéniat, atténdit~: grátia autem, non quid sibi útile et commodósum sit, sed quod multis profíciat, magis consíderat.}
\Normal{\Verse{5.} Natúra libénter honórem et reveréntiam áccipit~: grátia vero omnem honórem et glóriam Deo fidéliter attríbuit.}
\Normal{\Verse{6.} Natúra confusiónem timet et contémptum~: grátia autem gaudet pro nómine Iesu pati contuméliam.}
\Normal{\Verse{7.} Natúra ótium amat et quiétem corporálem~: grátia vero vácua esse non potest, sed libénter ampléctitur labórem.}
\Normal{\Verse{8.} Natúra quærit habére curiósa et pulchra, abhórret vília et grossa~: grátia vero simplícibus delectátur et humílibus, áspera non aspernátur, nec vetústis réfugit índui pannis.}
\Normal{\Verse{9.} Natúra réspicit temporália, gaudet ad lucra terréna, tristátur de damno, irritátur levi iniúriæ verbo~: sed grátia atténdit ætérna, non inhǽret temporálibus, nec in perditióne rerum turbátur, neque verbis durióribus acerbátur~; quia thesáurum suum et gáudium in cælo, ubi nil perit, constítuit.}
\Normal{\Verse{10.} Natúra cúpida est, et libéntius áccipit quam donat, amat própria et priváta~: grátia autem pia est et commúnis, vitat singulária, contentátur paucis, beátius dare iúdicat quam accípere.}
\Normal{\Verse{11.} Natúra inclínat ad creatúras, ad carnem própriam, ad vanitátes et discúrsus~: sed grátia trahit ad Deum et ad virtútes, renúntiat creatúris, fugit mundum, odit carnis desidéria, restríngit evagatiónes, erubéscit in público apparére.}
\Normal{\Verse{12.} Natúra libénter áliquod solácium habet extérnum, in quo delectétur ad sensum~: sed grátia in solo Deo quærit consolári, et in summo bono super ómnia visibília delectári.}
\Normal{\Verse{13.} Natúra totum agit propter lucrum et cómmodum próprium, nihil gratis fácere potest, sed aut æquále, aut mélius, aut laudem vel favórem pro benefáctis cónsequi sperat, et multum ponderári sua gesta et dona concupíscit~: grátia vero nil temporále quærit, nec áliud prǽmium, quam Deum solum pro mercéde póstulat~; nec ámplius de temporálibus necessáriis desíderat, nisi quantum hæc sibi ad assecutiónem æternórum váleant deservíre.}
\Normal{\Verse{14.} Natúra gaudet de amícis multis et propínquis, gloriátur de nóbili loco et ortu géneris, arrídet poténtibus, blandítur divítibus, appláudit sibi simílibus~: grátia autem et inimícos díligit, nec de amicórum turba extóllitur, nec locum nec ortum natálium réputat, nisi virtus maior ibi fúerit~; favet magis páuperi quam díviti, compátitur plus innocénti quam poténti~; congáudet veráci, non falláci~; exhortátur semper bonos melióra charísmata æmulári, et Fílio Dei per virtútes assimilári.}
\Normal{\Verse{15.} Natúra de deféctu et moléstia cito conquéritur~: grátia constánter fert inópiam.}
\Normal{\Verse{16.} Natúra ómnia ad se refléctit, pro se certat et árguit~: grátia autem ad Deum cuncta redúcit, unde origináliter emánant~; nihil boni sibi ascríbit, nec arrogánter præsúmit~; non conténdit, nec suam senténtiam áliis præfert~; sed in omni sensu et intelléctu ætérnæ sapiéntiæ ac divíno exámini se submíttit. Natúra áppetit scire secréta et nova audíre~; vult extérius apparére et multa per sensus experíri~; desíderat agnósci et ágere, unde laus et admirátio procédit~: sed grátia non curat nova nec curiósa percípere~; quia totum hoc de vetustáte corruptiónis est ortum, cum nihil novum et durábile sit super terram. Docet ítaque sensus restríngere, vanam complacéntiam et ostentatiónem devitáre, laudánda et digne miránda humíliter abscóndere, et de omni re et in omni sciéntia utilitátis fructum atque Dei laudem et honórem quǽrere. Non vult se nec sua prædicári, sed Deum in donis suis optat benedíci, qui cuncta ex mera caritáte largítur.}
\Normal{\Verse{17.} Hæc grátia supernaturále lumen et quoddam Dei speciále donum est, et próprie electórum signáculum et pignus salútis ætérnæ~: quæ hóminem de terrénis ad cæléstia amánda sustóllit, et de carnáli spirituálem éfficit. Quanto ígitur natúra ámplius prémitur et víncitur, tanto maior grátia infúnditur~; et cotídie novis visitatiónibus intérior homo secúndum imáginem Dei reformátur.}
\markright{LV}
%LIBER III
\TitreC{CAPUT LV}
\TitreC{De corruptióne natúræ et efficácia grátiæ divínæ}
\Normal{\Verse{1.} Dómine Deus meus, qui me creásti ad imáginem et similitúdinem tuam, concéde mihi hanc grátiam, quam ostendísti tam magnam et necessáriam ad salútem, ut vincam péssimam natúram meam, trahéntem ad peccáta et in perditiónem. Séntio enim in carne mea legem peccáti, contradicéntem legi mentis meæ, et captívum me ducéntem ad obœdiéndum sensualitáti in multis~; nec possum resístere passiónibus eius, nisi assístat tua sanctíssima grátia, cordi meo ardénter infúsa.}
\Normal{\Verse{2.} Opus est grátia tua, et magna grátia, ut vincátur natúra ad malum semper prona ab adulescéntia sua. Nam per primum hóminem Adam lapsa, et vitiáta per peccátum, in omnes hómines pœna huius máculæ descéndit~: ut ipsa natúra, quæ bene et recta a te cóndita fuit, pro vítio iam et infirmitáte corrúptæ natúræ ponátur, eo quod motus eius, sibi relíctus, ad malum et inferióra trahit. Nam módica vis, quæ remánsit, est tamquam scintílla quædam latens in cínere. Hæc est ipsa rátio naturális, circumfúsa magna calígine, adhuc iudícium habens boni et mali, veri falsíque distántiam~; licet ímpotens sit adimplére omne, quod ápprobat, nec pleno iam lúmine veritátis, nec sanitáte affectiónum suárum potiátur.}
\Normal{\Verse{3.} Hinc est, Deus meus, quod condeléctor legi tuæ secúndum interiórem hóminem, sciens, mandátum tuum fore bonum, iustum et sanctum, árguens étiam omne malum, et peccátum fugiéndum. Carne autem sérvio legi peccáti, dum magis sensualitáti obœ́dio, quam ratióni. Hinc est, quod velle bonum mihi ádiacet, perfícere autem non invénio. Hinc sæpe multa bona propóno, sed quia grátia deest ad iuvándum infirmitátem meam, ex levi resisténtia resílio et defício. Hinc áccidit, quod viam perfectiónis agnósco, et quáliter ágere débeam, clare satis vídeo~: sed própriæ corruptiónis póndere pressus ad perfectióra non assúrgo.}
\Normal{\Verse{4.} O quam máxime est mihi necessária, Dómine, tua grátia, ad inchoándum bonum, ad proficiéndum et ad perficiéndum~! Nam sine ea nihil possum fácere~: ómnia autem possum in te, confortánte me grátia. O vere cæléstis grátia, sine qua nulla sunt própria mérita, nulla quoque dona natúræ ponderánda~! Nihil artes, nihil divítiæ, nihil pulchritúdo vel fortitúdo, nihil ingénium vel eloquéntia valent apud te, Dómine, sine grátia. Nam dona natúræ bonis et malis sunt commúnia~: electórum autem próprium donum est grátia sive diléctio~; qua insigníti digni habéntur vita ætérna. Tantum éminet hæc grátia, ut nec donum prophetíæ, nec signórum operátio, nec quantálibet alta speculátio áliquid æstiméntur sine ea. Sed neque fides, neque spes, neque áliæ virtútes tibi accéptæ sunt sine caritáte et grátia.}
\Normal{\Verse{5.} O beatíssima grátia, quæ páuperem spíritu virtútibus dívitem facis, et dívitem multis bonis húmilem corde reddis~! Veni, descénde ad me, reple me mane consolatióne tua, ne defíciat præ lassitúdine et ariditáte mentis ánima mea. Obsecro, Dómine, ut invéniam grátiam in óculis tuis~: súfficit enim mihi grátia tua, céteris non obténtis, quæ desíderat natúra. Si fúero temptátus et vexátus tribulatiónibus multis, non timébo mala, dum mecum fúerit grátia tua. Ipsa fortitúdo mea, ipsa consílium confert et auxílium. Cunctis hóstibus poténtior est, et sapiéntior univérsis sapiéntibus.}
\Normal{\Verse{6.} Magístra est veritátis, doctrix disciplínæ, lumen cordis, solámen pressúræ, fugátrix tristítiæ, ablátrix timóris, nutrix devotiónis, prodúctrix lacrimárum. Quid sum sine ea, nisi áridum lignum, et stipes inútilis ad eiciéndum~? Tua ergo me, Dómine, grátia semper et prævéniat et sequátur, ac bonis opéribus iúgiter præstet esse inténtum, per Iesum Christum, fílium tuum. Amen.}
\markright{LVI}
%LIBER III
\TitreC{CAPUT LVI}
\TitreC{Quod nos ipsos abnegáre et Christum imitári debémus per crucem}
\Normal{\Verse{1.} Fili, quantum a te vales exíre, tantum in me póteris transíre. Sicut nihil foris concupíscere intérnam pacem facit, sic se intérius relínquere Deo coniúngit. Volo te addíscere perféctam abnegatiónem tui in voluntáte mea sine contradictióne et queréla. Séquere me~: Ego sum via, véritas et vita. Sine via non itur, sine veritáte non cognóscitur, sine vita non vívitur. Ego sum via, quam sequi debes~; véritas, cui crédere debes~; vita, quam speráre debes. Ego sum via inviolábilis, véritas infallíbilis, vita interminábilis. Ego sum via rectíssima, véritas supréma, vita vera, vita beáta, vita increáta. Si mánseris in via mea, cognósces veritátem, et véritas liberábit te, et apprehéndes vitam ætérnam.}
\Normal{\Verse{2.} Si vis ad vitam íngredi, serva mandáta. Si vis veritátem cognóscere, crede mihi. Si vis perféctus esse, vende ómnia. Si vis esse discípulus meus, ábnega temetípsum. Si vis beátam vitam possidére, præséntem vitam contémne. Si vis exaltári in cælo, humília te in mundo. Si vis regnáre mecum, porta crucem mecum. Soli enim servi crucis invéniunt viam beatitúdinis et veræ lucis.}
\Normal{\Verse{3.} Dómine Iesu, quia arcta erat vita tua et mundo despécta, dona mihi te cum mundi despéctu imitári. Non enim maior est servus dómino suo, nec discípulus supra magístrum. Exerceátur servus tuus in vita tua, quia ibi est salus mea et sánctitas vera. Quidquid extra eam lego vel áudio, non me récreat nec deléctat plene.}
\Normal{\Verse{4.} Fili, quia hæc scis et legísti ómnia, beátus eris, si féceris ea. Qui habet mandáta mea et servat ea, ipse est, qui díligit me~: et ego díligam eum et manifestábo ei meípsum, et fáciam eum consedére mecum in regno Patris mei.}
\Normal{\Verse{5.} Dómine Iesu, sicut dixísti et promisísti, sic útique fiat, et mihi promeréri contíngat. Suscépi, suscépi de manu tua crucem~; portábo, et portábo eam usque ad mortem, sicut imposuísti mihi. Vere vita boni mónachi crux est, sed dux paradísi. Incéptum est, retro abíre non licet, nec relínquere opórtet.}
\Normal{\Verse{6.} Eia fratres, pergámus simul, Iesus erit nobíscum. Propter Iesum suscépimus hanc crucem~: propter Iesum perseverémus in cruce. Erit adiútor noster, qui est dux noster et præcéssor. En, rex noster ingréditur ante nos, qui pugnábit pro nobis. Sequámur viríliter, nemo métuat terróres~; simus paráti mori fórtiter in bello, nec inferámus crimen glóriæ nostræ, ut fugiámus a cruce.}
\markright{LVII}
%LIBER III
\TitreC{CAPUT LVII}
\TitreC{Quod homo non sit nimis dejéctus, quando in áliquos lábitur deféctus}
\Normal{\Verse{1.} Fili, magis placent mihi patiéntia et humílitas in advérsis, quam multa consolátio et devótio in prósperis. Ut quid te contrístat parvum factum contra te dictum~? Si ámplius fuísset, commovéri non debuísses. Sed nunc permítte transíre~; non est primum, nec novum, nec últimum erit, si diu víxeris. Satis virílis es, quámdiu nil óbviat advérsi. Bene étiam cónsulis, et álios nosti roboráre verbis~; sed cum ad iánuam tuam venit repentína tribulátio, déficis consílio et róbore. Atténde magnam fragilitátem tuam, quam sǽpius experíris in módicis obiéctis~; tamen pro salúte tua ista fiunt, cum hæc et simília contíngunt.}
\Normal{\Verse{2.} Pone, ut mélius nosti, ex corde~; et si te tétigit, non tamen deíciat nec diu ímplicet. Ad minus sústine patiénter, si non potes gaudénter. Etiam si minus libénter audis, et indignatiónem sentis~: réprime te, nec patiáris áliquid inordinátum ex ore tuo exíre, unde párvuli scandalizéntur. Cito conquiéscet commótio excitáta, et dolor intérnus reverténte dulcorábitur grátia. Adhuc vivo ego, dicit Dóminus, iuváre te parátus, et sólito ámplius consolári, si confísus fúeris mihi, et devóte invocáveris.}
\Normal{\Verse{3.} Animǽquior esto, et ad maiórem sustinéntiam accíngere. Non est totum frustrátum, si te sǽpius pércipis tribulátum vel gráviter temptátum. Homo es, et non Deus~; caro es, non ángelus. Quómodo tu posses semper in eódem statu virtútis permanére, quando hoc défuit ángelo in cælo, et primo hómini in paradíso~? Ego sum, qui mæréntes érigo sospitáte, et suam cognoscéntes infirmitátem ad meam próveho divinitátem.}
\Normal{\Verse{4.} Dómine, benedíctum sit verbum tuum, dulce super mel et favum ori meo. Quid fácerem in tantis tribulatiónibus et angústiis meis, nisi me confortáres tuis sanctis sermónibus~? Dúmmodo tandem ad portum salútis pervéniam, quid curæ est, quæ et quanta passus fúero~? Da finem bonum, da felícem ex hoc mundo tránsitum. Meménto mei, Deus meus, et dírige me recto itínere in regnum tuum. Amen.}
\markright{LVIII}
%LIBER III
\TitreC{CAPUT LVIII}
\TitreC{De altióribus rebus et occúltis iudíciis Dei non scrutándis}
\Normal{\Verse{1.} Fili, cáveas disputáre de altis matériis et de occúltis Dei iudíciis~: cur iste sic relínquitur et ille ad tantam grátiam assúmitur, cur étiam iste tantum afflígitur et ille tam exímie exaltátur. Ista omnem humánam facultátem excédunt, nec ad investigándum iudícium divínum ulla rátio prǽvalet, vel disputátio. Quando ergo hæc tibi súggerit inimícus, vel étiam quidam curiósi inquírunt hómines, respónde illud prophétæ~: Iustus es, Dómine, et rectum iudícium tuum. Et illud~: Iudícia Dómini vera, iustificáta in semetípsa. Iudícia mea metuénda sunt, non discutiénda~; quia humáno intelléctui sunt incomprehensibília.}
\Normal{\Verse{2.} Noli étiam inquírere nec disputáre de méritis sanctórum, quis álio sit sánctior, aut quis maior fúerit in regno cælórum. Tália génerant sæpe lites et contentiónes inútiles, nútriunt quoque supérbiam et vanam glóriam~: unde oriúntur invídiæ et dissensiónes, dum iste illum sanctum, et álius álium conátur supérbe præférre. Tália autem velle scire et investigáre nullum fructum áfferunt, sed magis sanctis dísplicent~: quia non sum Deus dissensiónis, sed pacis~; quæ pax magis in humilitáte vera, quam in própria exaltatióne consístit.}
\Normal{\Verse{3.} Quidam zelo dilectiónis trahúntur ad hos vel ad illos amplióri afféctu, sed humáno pótius quam divíno. Ego sum, qui cunctos cóndidi sanctos~; ego donávi grátiam, ego prǽstiti glóriam. Ego novi singulórum mérita~; ego prævéni eos in benedictiónibus dulcédinis meæ. Ego præscívi diléctos ante sǽcula~; ego eos elégi de mundo, non ipsi me præelegérunt. Ego vocávi per grátiam, attráxi per misericórdiam~; ego perdúxi eos per temptatiónes várias. Ego infúdi consolatiónes magníficas, ego dedi perseverántiam, ego coronávi eórum patiéntiam.}
\Normal{\Verse{4.} Ego primum et novíssimum agnósco, ego omnes inæstimábili dilectióne ampléctor. Ego laudándus sum in ómnibus sanctis meis~; ego super ómnia benedicéndus sum, et honorándus in síngulis, quos sic glorióse magnificávi et prædestinávi, sine ullis præcedéntibus própriis méritis. Qui ergo unum de mínimis meis contémpserit, nec magnum honórat~; quia pusíllum et magnum ego feci. Et qui dérogat alícui sanctórum, dérogat et mihi, et céteris ómnibus in regno cælórum. Omnes unum sunt per caritátis vínculum~; idem séntiunt, idem volunt, et omnes in unum se díligunt.}
\Normal{\Verse{5.} Adhuc autem, quod multo áltius est, plus me, quam se et sua mérita díligunt. Nam supra se rapti, et extra própriam dilectiónem tracti, toti in amórem mei pergunt, in quo fruitíve quiéscunt. Nihil est, quod eos avértere possit aut deprímere~: quippe qui ætérna veritáte pleni, igne ardéscunt inextinguíbilis caritátis. Táceant ígitur carnáles et animáles hómines de sanctórum statu dissérere, qui non norunt nisi priváta gáudia dilígere. Demunt et addunt pro sua inclinatióne, non prout placet ætérnæ veritáti.}
\Normal{\Verse{6.} In multis est ignorántia, eórum máxime, qui parum illumináti, raro áliquem perfécta dilectióne spirituáli dilígere norunt. Multum adhuc naturáli afféctu et humána amicítia ad hos vel ad illos trahúntur, et sicut in inferióribus se habent, ita et de cæléstibus imaginántur. Sed est distántia incomparábilis, quæ imperfécti cógitant, et quæ illumináti viri per revelatiónem supérnam speculántur.}
\Normal{\Verse{7.} Cave ergo, fili, de istis curióse tractáre, quæ tuam sciéntiam excédunt~; sed hoc magis sátage et inténde, ut vel mínimus in regno Dei queas inveníri. Et si quispiam sciret, quis álio sánctior esset, vel maior haberétur in regno cælórum~: quid ei hæc notítia prodésset, nisi se ex hac cognitióne coram me humiliáret, et in maiórem nóminis mei laudem exúrgeret~? Multo accéptius Deo facit, qui de peccatórum suórum magnitúdine et virtútum suárum parvitáte cógitat, et quam longe a perfectióne sanctórum distat, quam is, qui de eórum maioritáte vel parvitáte dísputat. Mélius est sanctos devótis précibus et lácrimis exoráre, et eórum gloriósa suffrágia húmili mente imploráre, quam eórum secréta vana inquisitióne perscrutári.}
\Normal{\Verse{8.} Illi bene et óptime contentántur, si hómines scirent contentári et vanilóquia sua compéscere. Non gloriántur de própriis méritis, quippe qui sibi nihil bonitátis ascríbunt, sed totum mihi, quóniam ipsis cuncta ex infiníta caritáte mea donávi. Tanto amóre divinitátis et gáudio supereffluénti repléntur, ut nihil eis desit glóriæ, nihílque possit deésse felicitátis. Omnes sancti, quanto altióres in glória, tanto humilióres in se ipsis, et mihi vicinióres et dilectióres exsístunt. Ideóque habes scriptum, quia mittébant corónas suas ante Deum, et cecidérunt in fácies suas coram Agno, et adoravérunt Vivéntem in sǽcula sæculórum.}
\Normal{\Verse{9.} Multi quærunt, quis maior sit in regno Dei~; qui ignórant, an cum mínimis erunt digni computári. Magnum est vel mínimum esse in cælo, ubi omnes magni sunt~: quia omnes fílii Dei vocabúntur et erunt. Mínimus erit in mille, et peccátor centum annórum moriétur. Cum enim quǽrerent discípuli, quis maior esset in regno cælórum, tale audiérunt respónsum~: Nisi convérsi fuéritis et efficiámini sicut párvuli, non intrábitis in regnum cælórum. Quicúmque ergo humiliáverit se, sicut párvulus iste, hic maior est in regno cælórum.}
\Normal{\Verse{10.} Væ eis, qui cum párvulis humiliáre se sponte dedignántur, quóniam húmilis iánua regni cæléstis eos non admíttet intráre. Væ étiam divítibus, qui habent hic consolatiónes suas, quia paupéribus intrántibus in regnum Dei ipsi stabunt foris eiulántes. Gaudéte húmiles, et exsultáte páuperes, quia vestrum est regnum Dei, si tamen in veritáte ambulátis.}
\markright{LIX}
%LIBER III
\TitreC{CAPUT LIX}
\TitreC{Quod omnis spes et fidúcia in solo Deo est figénda}
\Normal{\Verse{1.} Dómine, quæ est fidúcia mea, quam in hac vita hábeo~? aut quod maius solácium meum ex ómnibus apparéntibus sub cælo~? Nonne tu, Dómine Deus meus, cuius misericórdiæ non est númerus~? Ubi mihi bene fuit sine te~? Aut quando male esse pótuit præsénte te~? Malo pauper esse propter te, quam dives sine te. Eligo pótius tecum in terra peregrinári, quam sine te cælum possidére. Ubi tu, ibi cælum~; atque ibi mors et inférnus, ubi tu non es. Tu mihi in desidério es~; et ídeo post te gémere, clamáre et exoráre necésse est. In nullo dénique possum plene confídere, qui in necessitátibus auxiliétur opportúnis, nisi in te solo Deo meo. Tu es spes mea, tu fidúcia mea, tu consolátor meus et fidelíssimus in ómnibus.}
\Normal{\Verse{2.} Omnes, quæ sua sunt, quærunt~: tu salútem meam et proféctum meum solúmmodo præténdis, et ómnia in bonum mihi convértis. Etiam si váriis temptatiónibus et adversitátibus expónas me, hoc totum ad utilitátem meam órdinas, qui mille modis diléctos tuos probáre consuevísti. In qua probatióne non minus díligi debes et laudári, quam si cæléstibus consolatiónibus me repléres.}
\Normal{\Verse{3.} In te ergo, Dómine Deus, pono totam spem meam et refúgium, in te omnem tribulatiónem et angústiam meam constítuo~; quia totum infírmum et instábile invénio, quidquid extra te conspício. Non enim próderunt multi amíci, neque fortes auxiliárii adiuváre póterunt, nec prudéntes consiliárii respónsum útile dare, neque libri doctórum consolári, nec áliqua pretiósa substántia liberáre, nec locus áliquis secrétus et amœ́nus contutári~: si tu ipse non assístas, iuves, confórtes, consoléris, ínstruas et custódias.}
\Normal{\Verse{4.} Omnia namque, quæ ad pacem vidéntur esse et felicitátem habéndam, te absénte nihil sunt, nihílque felicitátis in veritáte cónferunt. Finis ergo ómnium bonórum et altitúdo vitæ et profúnditas eloquiórum tu es~; et in te super ómnia speráre fortíssimum solácium servórum tuórum. Ad te sunt óculi mei, in te confído, Deus meus, misericordiárum Pater. Bénedic et sanctífica ánimam meam benedictióne cælésti, ut fiat habitátio sancta tua, et sedes ætérnæ glóriæ tuæ, nihílque in templo tuæ dignitátis inveniátur, quod óculos tuæ maiestátis offéndat. Secúndum magnitúdinem bonitátis tuæ et multitúdinem miseratiónum tuárum réspice in me, et exáudi oratiónem páuperis servi tui, longe exsulántis in regióne umbræ mortis. Prótege et consérva ánimam sérvuli tui inter tot discrímina vitæ corruptíbilis, ac comitánte grátia tua, dírige per viam pacis ad pátriam perpétuæ claritátis. Amen.}
\markboth{IV}{}
LIBER IV
DE SACRAMENTO
Devóta exhortátio ad sacram communiónem
\TitreD{Vox Christi.}
Veníte ad me, omnes, qui laborátis et oneráti estis, et ego refíciam vos, dicit Dóminus. Panis, quem ego dabo, caro mea est, pro mundi vita. Accípite et comédite, hoc est Corpus meum, quod pro vobis tradétur~: hoc fácite in meam commemoratiónem. Qui mandúcat meam carnem, et bibit meum sánguinem, in me manet, et ego in illo. Verba, quæ ego locútus sum vobis, spíritus et vita sunt.
\markright{I}
%LIBER IV
\TitreC{CAPUT I}
\TitreC{Cum quanta reveréntia Christus sit suscipiéndus}
\TitreD{Vox discípuli.}
\Normal{\Verse{1.} Hæc sunt verba tua, Christe, véritas ætérna, quamvis non uno témpore proláta, nec uno in loco conscrípta. Quia ergo tua sunt et vera, gratánter mihi et fidéliter cuncta sunt accipiénda. Tua sunt, et tu ea protulísti~; et mea quoque sunt, quia pro salúte mea ea edidísti. Libénter suscípio ea ex ore tuo, ut árctius inserántur cordi meo. Excitant me verba tantæ pietátis, plena dulcédinis et dilectiónis~; sed terrent me delícta própria, et ad capiénda tanta mystéria me revérberat impúra consciéntia. Próvocat me dulcédo verbórum tuórum, sed ónerat multitúdo vitiórum meórum.}
\Normal{\Verse{2.} Iubes, ut fiduciáliter ad te accédam, si tecum velim habére partem, et ut immortalitátis accípiam alimóniam, si ætérnam cúpiam obtinére vitam et glóriam. Veníte, inquis, ad me omnes, qui laborátis et oneráti estis, et ego refíciam vos. O dulce et amicábile verbum in aure peccatóris, quod tu, Dómine Deus meus, egénum et páuperem invítas ad communiónem tui sanctíssimi Córporis. Sed quis ego sum, Dómine, ut ad te præsúmam accédere~? Ecce, cæli cælórum te non cápiunt~; et tu dicis~: Veníte ad me omnes.}
\Normal{\Verse{3.} Quid sibi vult ista piíssima dignátio et tam amicábilis invitátio~? Quómodo ausus ero veníre, qui nihil boni mihi cónscius sum, unde possim præsúmere~? Quómodo te introdúcam in domum meam, qui sǽpius offéndi benigníssimam fáciem tuam~? Reveréntur ángeli et archángeli, métuunt sancti et iusti~; et tu dicis~: Veníte ad me omnes~? Nisi tu, Dómine, hoc díceres, quis verum esse créderet~? Et nisi tu iubéres, quis accédere attemptáret?}
\Normal{\Verse{4.} Ecce, Noë, vir iustus, in arcæ fábrica centum annis laborávit, ut cum paucis salvarétur~: et ego quómodo me pótero una hora præparáre, ut mundi fabricatórem cum reveréntia sumam~? Móyses, fámulus tuus magnus et speciális amícus tuus, arcam ex lignis imputribílibus fecit, quam et mundíssimo vestívit auro, ut tábulas legis in ea repóneret~: et ego, pútrida creatúra, audébo te, conditórem legis ac vitæ datórem, tam fácile suscípere~? Sálomon, sapientíssimus regum Israel, templum magníficum septem annis in laudem nóminis tui ædificávit, et octo diébus festum dedicatiónis eius celebrávit, mille hóstias pacíficas óbtulit et arcam fœ́deris in clangóre búccinæ et iúbilo in locum sibi præparátum sollémniter collocávit. Et ego infélix et paupérrimus hóminum, quómodo te in domum meam introdúcam, qui vix médiam expéndere devóte novi horam, et útinam vel semel digne fere médiam?}
\Normal{\Verse{5.} O mi Deus, quantum illi ad placéndum tibi ágere studuérunt~! Heu, quam pusíllum est, quod ago~: quam breve éxpleo tempus, cum me ad communicándum dispóno~! Raro totus colléctus, raríssime ab omni distractióne purgátus. Et certe in tua salutári deitátis præséntia nulla debéret occúrrere índecens cogitátio, nulla étiam occupáre creatúra~: quia non ángelum, sed angelórum Dóminum susceptúrus sum hospítio.}
\Normal{\Verse{6.} Est tamen valde magna distántia inter arcam fœ́deris cum suis relíquiis et mundíssimum corpus tuum cum suis ineffabílibus virtútibus~; inter legália illa sacrifícia futurórum præfiguratíva et veram córporis tui hóstiam, ómnium antiquórum sacrificiórum completívam.}
\Normal{\Verse{7.} Quare ígitur non magis ad tuam venerábilem inardésco præséntiam~? Cur non maióri me prǽparo sollicitúdine ad tua sancta suménda, quando illi antíqui sancti patriárchæ et prophétæ, reges quoque et príncipes, cum univérso pópulo, tantum devotiónis demonstrárunt afféctum erga cultum divínum?}
\Normal{\Verse{8.} Saltávit devotíssimus rex David coram arca Dei totis víribus, recólens benefícia olim indúlta pátribus~; fecit divérsi géneris órgana, psalmos édidit, et cantári instítuit cum lætítia, cécinit et ipse frequénter in cíthara, Spíritus Sancti afflátus grátia~; dócuit pópulum Israel toto corde Deum laudáre, et ore cónsono diébus síngulis benedícere et prædicáre. Si tanta agebátur tunc devótio, ac divínæ laudis éxtitit recordátio coram arca testaménti~: quanta nunc mihi et omni pópulo christiáno habénda est reveréntia et devótio in præséntia sacraménti, in sumptióne excellentíssimi córporis Christi?}
\Normal{\Verse{9.} Currunt multi ad divérsa loca pro visitándis relíquiis sanctórum, et mirántur audítis gestis eórum, ampla ædifícia templórum inspíciunt, et osculántur séricis et auro involúta sacra ossa ipsórum. Et ecce, tu præsens es hic apud me in altári, Deus meus, Sanctus sanctórum, Creátor hóminum, et Dóminus angelórum. Sæpe in tálibus vidéndis curiósitas est hóminum et nóvitas invisórum, et módicus reportátur emendatiónis fructus, máxime ubi est tam levis sine vera contritióne discúrsus. Hic autem in sacraménto altáris totus præsens es, Deus meus, homo Christus Iesus~: ubi et copiósus percípitur ætérnæ salútis fructus, quotiescúmque fúeris digne ac devóte suscéptus. Ad istud vero non trahit lévitas áliqua, nec curiósitas aut sensuálitas, sed firma fides, devóta spes et sincéra cáritas.}
\Normal{\Verse{10.} O invisíbilis condítor mundi Deus, quam mirabíliter agis nobíscum~; quam suáviter et gratióse cum eléctis tuis dispónis, quibus temetípsum in sacraménto suméndum propónis~! Hoc namque omnem intelléctum súperat~; hoc speciáliter devotórum corda trahit et accéndit afféctum. Ipsi enim veri fidéles tui, qui totam vitam suam ad emendatiónem dispónunt, ex hoc digníssimo sacraménto magnam devotiónis grátiam et virtútis amórem frequénter recípiunt.}
\Normal{\Verse{11.} O admirábilis et abscóndita grátia sacraménti, quam norunt tantum Christi fidéles, infidéles autem et peccátis serviéntes experíri non possunt~! In hoc sacraménto confértur spirituális grátia, et reparátur in ánima virtus amíssa, et per peccátum deformáta redit pulchritúdo. Tanta est aliquándo hæc grátia, ut ex plenitúdine collátæ devotiónis non tantum mens, sed et débile corpus vires sibi prǽstitas séntiat amplióres.}
\Normal{\Verse{12.} Doléndum tamen valde et miserándum super tepiditáte et neglegéntia nostra, quod non maióri affectióne tráhimur ad Christum suméndum, in quo tota spes salvandórum consístit et méritum. Ipse enim est sanctificátio nostra et redémptio~; ipse consolátio viatórum, et sanctórum ætérna fruítio. Doléndum ítaque valde, quod multi tam parum hoc salutáre mystérium advértunt, quod cælum lætíficat, et mundum consérvat univérsum. Heu cǽcitas et durítia cordis humáni, tam ineffábile donum non magis atténdere, et ex cotidiáno usu étiam ad inadverténtiam deflúere!}
\Normal{\Verse{13.} Si enim hoc sanctíssimum sacraméntum in uno tantum celebrarétur loco, et ab uno tantum consecrarétur sacerdóte in mundo~: quanto putas desidério ad illum locum et ad talem Dei sacerdótem hómines afficeréntur, ut divína mystéria celebrári vidérent~? Nunc autem multi facti sunt sacerdótes, et in multis locis offértur Christus, ut tanto maior appáreat grátia et diléctio Dei ad hóminem, quanto látius est sacra commúnio diffúsa per orbem. Grátias tibi, Iesu bone, pastor ætérne, qui nos páuperes et éxsules dignátus es pretióso córpore et sánguine tuo refícere, et ad hæc mystéria percipiénda étiam próprii oris tui allóquio invitáre, dicéndo~: Veníte ad me omnes, qui laborátis et oneráti estis, et ego refíciam vos.}
\markright{II}
%LIBER IV
\TitreC{CAPUT II}
\TitreC{Quod magna bónitas et cáritas Dei in sacraménto hómini exhibétur}
\TitreD{Vox discípuli.}
\Normal{\Verse{1.} Super bonitáte tua et magna misericórdia tua, Dómine, confísus, accédo æger ad Salvatórem, esúriens et sítiens ad fontem vitæ, egénus ad Regem cæli, servus ad Dóminum, creatúra ad Creatórem, desolátus ad meum pium Consolatórem. Sed unde hoc mihi, ut vénias ad me~? Quis ego sum, ut præstes mihi teípsum~? Quómodo audet peccátor coram te apparére~? et tu quómodo dignáris ad peccatórem veníre~? Tu nosti servum tuum, et scis, quia nil boni in se habet, unde hoc illi præstes. Confíteor ígitur vilitátem meam, agnósco tuam bonitátem, laudo pietátem, et grátias ago propter nímiam caritátem. Propter temetípsum enim hoc facis, non propter mea mérita~; ut bónitas tua mihi magis innotéscat, cáritas ámplior ingerátur, et humílitas perféctius commendétur. Quia ergo tibi hoc placet, et tu sic fíeri iussísti, placet et mihi dignátio tua~; et útinam iníquitas mea non obsístat!}
\Normal{\Verse{2.} O dulcíssime et benigníssime Iesu, quanta tibi reveréntia et gratiárum áctio cum perpétua laude pro susceptióne sacri córporis tui debétur, cuius dignitátem nullus hóminum explicáre potens invenítur~! Sed quid cogitábo in hac communióne, in accéssu ad Dóminum meum, quem débite venerári néqueo, et tamen devóte suscípere desídero~? Quid cogitábo mélius et salúbrius, nisi meípsum totáliter humiliándo coram te, et tuam infinítam bonitátem exaltándo supra me~? Laudo te, Deus meus, et exálto in ætérnum. Despício me, et subício tibi in profúndum vilitátis meæ.}
\Normal{\Verse{3.} Ecce, tu Sanctus sanctórum, et ego sordes peccatórum. Ecce, tu inclínas te ad me, qui non sum dignus ad te respícere. Ecce, tu venis ad me, tu vis esse mecum, tu invítas ad convívium tuum. Tu mihi dare vis cæléstem cibum et panem angelórum ad manducándum~: non álium sane quam teípsum, panem vivum, qui de cælo descendísti, et das vitam mundo.}
\Normal{\Verse{4.} Ecce, unde diléctio procédit, qualis dignátio illucéscit~! quam magnæ gratiárum actiónes et laudes tibi pro his debéntur~! O quam salutáre et útile consílium tuum, cum istud instituísti~! quam suáve et iucúndum convívium, cum teípsum in cibum donásti~! O quam admirábilis operátio tua, Dómine~! quam potens virtus tua~! quam infallíbilis véritas tua~! Dixísti enim, et facta sunt ómnia~; et hoc factum est, quod ipse iussísti.}
\Normal{\Verse{5.} Mira res, et fide digna, ac humánum vincens intelléctum~: quod tu, Dómine Deus meus, verus Deus et homo, sub módica spécie panis et vini ínteger continéris, et sine consumptióne a suménte manducáris. Tu, Dómine universórum, qui nullíus habes indigéntiam, voluísti per sacraméntum tuum habitáre in nobis~; consérva cor meum et corpus immaculátum~; ut læta et pura consciéntia sǽpius tua váleam celebráre mystéria, et ad meam perpétuam accípere salútem, quæ ad tuum præcípue honórem et memoriále perénne sanxísti et instituísti.}
\Normal{\Verse{6.} Lætáre, ánima mea, et grátias age Deo pro tam nóbili múnere et solácio singulári in hac lacrimárum valle tibi relícto. Nam quótiens hoc mystérium recólis et Christi Corpus áccipis, tótiens tuæ redemptiónis opus agis et párticeps ómnium meritórum Christi effíceris. Cáritas étenim Christi numquam minúitur, et magnitúdo propitiatiónis eius numquam exhaurítur. Ideo nova semper mentis renovatióne ad hoc dispónere te debes, et magnum salútis mystérium atténta consideratióne pensáre. Ita magnum, novum et iucúndum tibi vidéri debet, cum célebras aut missam audis, ac si eódem die Christus primum in úterum vírginis descéndens homo factus esset, aut in cruce pendens pro salúte hóminum paterétur et morerétur.}
\markright{III}
%LIBER IV
\TitreC{CAPUT III}
\TitreC{Quod útile sit sæpe communicáre}
\TitreD{Vox discípuli.}
\Normal{\Verse{1.} Ecce, ego ad te vénio, Dómine, ut bene mihi sit ex múnere tuo, et lætíficer in convívio sancto tuo, quod parásti in dulcédine tua páuperi, Deus. Ecce, in te est totum, quod desideráre possum et débeo~; tu salus mea et redémptio, spes et fortitúdo, decus et glória. Lætífica ergo hódie ánimam servi tui, quóniam ad te, Dómine Iesu, ánimam meam levávi. Desídero te nunc devóte ac reverénter suscípere~; cúpio te in domum meam indúcere, quátenus cum Zachǽo mérear a te benedíci ac inter fílios Abrahæ computári. Anima mea corpus tuum concupíscit, cor meum tecum uníri desíderat.}
\Normal{\Verse{2.} Trade te mihi, et súfficit. Nam præter te nulla consolátio valet. Sine te esse néqueo, et sine visitatióne tua vívere non váleo. Ideóque opórtet me frequénter ad te accédere, et in remédium salútis meæ recípere~; ne forte defíciam in via, si fúero cælésti fraudátus alimónia. Sic enim tu, misericordíssime Iesu, prǽdicans pópulis et vários curans languóres, aliquándo dixísti~: Nolo eos ieiúnos dimíttere in domum suam, ne defíciant in via. Age ígitur hoc mecum modo, qui te pro fidélium consolatióne in Sacraménto reliquísti. Tu es enim suávis reféctio ánimæ~: et qui te digne manducáverit, párticeps et heres erit ætérnæ glóriæ. Necessárium quidem mihi est, qui tam sæpe labor et pecco, tam cito torpésco et defício, ut per frequéntes oratiónes et confessiónes ac sacram córporis tui perceptiónem me rénovem, mundem et accéndam~; ne forte diútius abstinéndo a sancto propósito défluam.}
\Normal{\Verse{3.} Proni enim sunt sensus hóminis ad malum ab adulescéntia sua~; et nisi succúrrat divína medicína, lábitur homo mox ad peióra. Rétrahit ergo sancta commúnio a malo et confórtat in bono. Si enim modo tam sæpe néglegens sum et tépidus, quando commúnico aut célebro~; quid fíeret, si medélam non súmerem, et tam grande iuvámen non quǽrerem~? Et licet omni die non sim aptus, nec ad celebrándum bene dispósitus~; dabo tamen óperam, cóngruis tempóribus divína percípere mystéria, ac tantæ grátiæ partícipem me præbére. Nam hæc est una principális fidélis ánimæ consolátio, quámdiu peregrinátur a te in mortáli córpore, ut sǽpius memor Dei sui diléctum suum devóta suscípiat mente.}
\Normal{\Verse{4.} O mira circa nos tuæ pietátis dignátio, quod tu, Dómine Deus, Creátor et Vivificátor ómnium spirítuum, ad paupérculam dignáris veníre ánimam, et cum tota divinitáte tua ac humanitáte eius impinguáre esúriem~! O felix mens, et beáta ánima, quæ te, Dóminum Deum suum, merétur devóte suscípere, et in tua susceptióne spiritáli gáudio repléri~! O quam magnum súscipit Dóminum, quam diléctum indúcit hóspitem, quam iucúndum récipit sócium, quam fidélem accéptat amícum, quam speciósum et nóbilem ampléctitur sponsum præ ómnibus diléctis, et super ómnia desiderabília amándum~! Síleant a fácie tua, dulcíssime dilécte meus, cælum et terra et omnis ornátus eórum~; quóniam quidquid laudis habent ac decóris, ex dignatióne tuæ est largitátis, nec ad decórem tui pervénient nóminis, cuius sapiéntiæ non est númerus.}
\markright{IV}
%LIBER IV
\TitreC{CAPUT IV}
\TitreC{Quod multa bona præstántur devóte communicántibus}
\TitreD{Vox discípuli.}
\Normal{\Verse{1.} Dómine Deus meus, prǽveni servum tuum in benedictiónibus dulcédinis tuæ, ut ad tuum magníficum sacraméntum digne ac devóte mérear accédere. Excita cor meum in te, et a gravi torpóre éxue me. Vísita me in salutári tuo ad gustándam in spíritu suavitátem tuam, quæ in hoc sacraménto tamquam in fonte plenárie latet. Illúmina quoque óculos meos ad intuéndum tantum mystérium, et ad credéndum illud indubitáta fide me róbora. Est enim operátio tua, non humána poténtia~; tua sacra institútio, non hóminis adinvéntio. Non enim ad hæc capiénda et intellegénda áliquis idóneus per se reperítur, quæ angélicam étiam subtilitátem transcéndunt. Quid ergo ego peccátor indígnus, terra et cinis, de tam alto sacro secréto pótero investigáre et cápere?}
\Normal{\Verse{2.} Dómine, in simplicitáte cordis mei, in bona firma fide et in tua iussióne ad te cum spe ac reveréntia accédo~; et vere credo, quia tu præsens es hic in sacraménto, Deus et homo. Vis ergo, ut te suscípiam, et meípsum tibi in caritáte úniam. Unde tuam precor cleméntiam, et speciálem ad hoc implóro mihi donári grátiam, ut totus in te liquefíam et amóre peréffluam, atque de nulla aliéna consolatióne ámplius me intromíttam. Est enim hoc altíssimum et digníssimum sacraméntum, salus ánimæ et córporis, medicína omnis spiritális languóris, in quo vítia mea curántur, passiónes frenántur, temptatiónes vincúntur aut minuúntur, grátia maior infúnditur, virtus incépta augétur, firmátur fides, spes roborátur, et cáritas ignéscit ac dilatátur.}
\Normal{\Verse{3.} Multa namque bona largítus es, et adhuc sǽpius largíris in sacraménto diléctis tuis devóte communicántibus, Deus meus, suscéptor ánimæ meæ, reparátor infirmitátis humánæ, et totíus dator consolatiónis intérnæ. Nam multam ipsis consolatiónem advérsus váriam tribulatiónem infúndis, et de imo deiectiónis própriæ ad spem tuæ protectiónis érigis, atque nova quadam grátia eos intus récreas et illústras, ut, qui ánxii primum, et sine affectióne se ante communiónem sénserant, póstea refécti cibo potúque cælésti in mélius se mutátos invéniant. Quod idcírco cum eléctis tuis dispensánter agis, ut veráciter agnóscant et paténter experiántur, quantum infirmitátis ex se ipsis hábeant, et quid bonitátis ac grátiæ ex te consequántur. Quia ex semetípsis frígidi, duri et indevóti, ex te autem fervéntes, álacres et devóti esse meréntur. Quis enim ad fontem suavitátis humíliter accédens, non módicum suavitátis inde repórtat~? Aut quis iuxta copiósum ignem stans, non parum calóris inde pércipit~? Et tu fons es semper plenus et superabúndans, ignis iúgiter ardens et numquam defíciens.}
\Normal{\Verse{4.} Unde si mihi non licet hauríre de plenitúdine fontis, nec usque ad satietátem potáre, appónam tamen os meum ad forámen cæléstis fístulæ, ut saltem módicam inde gúttulam cápiam ad refocillándam sitim meam, et non pénitus exaréscam. Et si necdum totus cæléstis et tam ignítus, ut Chérubim et Séraphim, esse possum, conábor tamen devotióni insístere et cor meum præparáre, ut vel módicam divíni incéndii flammam ex húmili sumptióne vivífici sacraménti conquíram. Quidquid autem mihi deest, Iesu bone, salvátor sanctíssime, tu pro me supple benígne ac gratióse, qui omnes ad te dignátus es vocáre, dicens~: Veníte ad me omnes, qui laborátis et oneráti estis, et ego refíciam vos.}
\Normal{\Verse{5.} Ego quidem labóro in sudóre vultus mei, dolóre cordis tórqueor, peccátis óneror, temptatiónibus inquiétor, multis malis passiónibus ímplicor et premor~; et non est qui ádiuvet, non est qui líberet et salvum fáciat, nisi tu, Dómine Deus, salvátor meus~: cui commítto me et ómnia mea, ut me custódias et perdúcas in vitam ætérnam. Súscipe me in laudem et glóriam nóminis tui, qui corpus tuum et sánguinem in cibum et potum mihi parásti. Præsta, Dómine Deus salutáris meus, ut cum frequentatióne mystérii tui crescat meæ devotiónis afféctus.}
\markright{V}
%LIBER IV
\TitreC{CAPUT V}
\TitreC{De dignitáte sacraménti et statu sacerdotáli}
\TitreD{Vox discípuli.}
\Normal{\Verse{1.} Si habéres angélicam puritátem et sancti Iohánnis Baptístæ sanctitátem, non esses dignus hoc sacraméntum accípere nec tractáre. Non enim hoc méritis debétur hóminum, quod homo cónsecret et tractet Christi sacraméntum, et sumat in cibum panem angelórum. Grande ministérium et magna dígnitas sacerdótum, quibus datum est, quod ángelis non est concéssum. Soli namque sacerdótes rite in Ecclésia ordináti potestátem habent celebrándi et Corpus Christi consecrándi. Sacérdos quidem miníster est Dei, utens verbo Dei, per iussiónem et institutiónem Dei~; Deus autem ibi principális est auctor et invisíbilis operátor, cui subest omne, quod volúerit, et paret omne, quod iússerit.}
\Normal{\Verse{2.} Plus ergo crédere debes Deo omnipoténti in hoc excellentíssimo sacraménto, quam próprio sénsui aut alícui signo visíbili. Ideóque cum timóre et reveréntia ad hoc opus est accedéndum. Atténde tibi et vide, cuius ministérium tibi tráditum est per impositiónem manus epíscopi. Ecce, sacérdos factus es, et ad celebrándum consecrátus~; vide nunc, ut fidéliter et devóte in suo témpore Deo sacrifícium ófferas, et teípsum irreprehensíbilem exhíbeas. Non alleviásti onus tuum, sed arctióri iam alligátus es vínculo disciplínæ, et ad maiórem tenéris perfectiónem sanctitátis. Sacérdos ómnibus virtútibus debet esse ornátus, et áliis bonæ vitæ exémplum præbére. Eius conversátio non cum populáribus et commúnibus hóminum viis, sed cum ángelis in cælo aut cum perféctis viris in terra.}
\Normal{\Verse{3.} Sacérdos sacris véstibus indútus Christi vices gerit, ut Deum pro se et pro omni pópulo supplíciter et humíliter roget. Habet ante se et retro Domínicæ crucis signum, ad memorándam iúgiter Christi passiónem. Ante se crucem in cásula portat, ut Christi vestígia diligénter inspíciat, et sequi fervénter stúdeat. Post se cruce signátus est, ut advérsa quǽlibet ab áliis illáta cleménter pro Deo tóleret. Ante se crucem gerit, ut própria peccáta lúgeat~; post se, ut aliórum étiam commíssa per compassiónem défleat, et se médium inter Deum et peccatórem constitútum esse sciat, nec ab oratióne et oblatióne sancta torpéscat, donec grátiam et misericórdiam impetráre mereátur. Quando sacérdos célebrat, Deum honórat, ángelos lætíficat, Ecclésiam ædíficat, vivos ádiuvat, defúnctis réquiem præstat, et sese ómnium bonórum partícipem éfficit.}
\markright{VI}
%LIBER IV
\TitreC{CAPUT VI}
\TitreC{Interrogátio de exercítio ante communiónem}
\TitreD{Vox discípuli.}
\Normal{\Verse{1.} Cum tuam dignitátem, Dómine, et meam vilitátem penso, valde contremísco et in me ipso confúndor. Si enim non accédo, vitam fúgio~; et si indígne me ingéssero, offénsam incúrro. Quid ergo fáciam, Deus meus, auxiliátor meus et consiliátor in necessitátibus?}
\Normal{\Verse{2.} Tu doce me viam rectam~; propóne breve áliquod exercítium sacræ communióni cóngruum. Utile est enim scire, quáliter scílicet devóte ac reverénter tibi præparáre débeam cor meum, ad recipiéndum salúbriter tuum sacraméntum, seu étiam celebrándum tam magnum et divínum sacrifícium.}
\markright{VII}
%LIBER IV
\TitreC{CAPUT VII}
\TitreC{De discussióne própriæ consciéntiæ et emendatiónis propósito}
\TitreD{Vox dilécti.}
\Normal{\Verse{1.} Super ómnia cum summa humilitáte cordis et súpplici reveréntia, cum plena fide et pia intentióne honóris Dei, ad hoc sacraméntum celebrándum, tractándum et suméndum opórtet Dei accédere sacerdótem. Diligénter exámina consciéntiam tuam et pro posse tuo vera contritióne et húmili confessióne eam munda et clarífica~: ita ut nil grave hábeas aut scias, quod te remórdeat et líberum accéssum impédiat. Hábeas displicéntiam ómnium peccatórum tuórum in generáli, et pro cotidiánis excéssibus magis in speciáli dóleas et gemas. Et si tempus pátitur, Deo in secréto cordis cunctas confitére passiónum tuárum misérias.}
\Normal{\Verse{2.} Ingemísce et dole, quod adhuc ita carnális sis et mundánus~; tam immortificátus a passiónibus, tam plenus concupiscentiárum mótibus~; tam incustodítus in sénsibus exterióribus, tam sæpe multis vanis phantásiis implicátus~; tam multum inclinátus ad exterióra, tam néglegens ad interióra~; tam levis ad risum et dissolutiónem, tam durus ad fletum et compunctiónem~; tam promptus ad laxióra et carnis cómmoda, tam segnis ad rigórem et fervórem~; tam curiósus ad nova audiénda et pulchra intuénda, tam remíssus ad humília et abiécta amplecténda~; tam cúpidus ad multa habénda, tam parcus ad dandum, tam tenax ad retinéndum~; tam inconsiderátus in loquéndo, tam incóntinens in tacéndo~; tam incompósitus in móribus, tam importúnus in áctibus~; tam effúsus super cibum, tam surdus ad Dei verbum~; tam velox ad quiétem, tam tardus ad labórem~; tam vígilans ad fábulas, tam somnoléntus ad vigílias sacras~; tam festínus ad finem, tam vagus ad attendéndum~; tam néglegens in horis persolvéndis, tam tépidus in celebrándo, tam áridus in communicándo~; tam cito distráctus, tam raro plene tibi colléctus~; tam súbito commótus ad iram, tam fácilis ad altérius displicéntiam~; tam pronus ad iudicándum, tam rígidus ad arguéndum~; tam lætus ad próspera, tam débilis in advérsis~; tam sæpe multa bona propónens, et módicum ad efféctum perdúcens.}
\Normal{\Verse{3.} His et áliis deféctibus tuis, cum dolóre et magna displicéntia própriæ infirmitátis conféssis ac deplorátis, firmum státue propósitum semper emendándi vitam tuam et in mélius proficiéndi. Deínde cum plena resignatióne et íntegra voluntáte offer teípsum in honórem nóminis mei in ara cordis tui holocáustum perpétuum, corpus tuum scílicet et ánimam mihi fidéliter committéndo~: quátenus et sic digne mereáris ad offeréndum Deo sacrifícium accédere, et sacraméntum córporis mei salúbriter suscípere.}
\Normal{\Verse{4.} Non est enim oblátio dígnior et satisfáctio maior pro peccátis diluéndis, quam seípsum pure et íntegre cum oblatióne córporis Christi in missa et in communióne Deo offérre. Si fécerit homo, quod in se est, et vere pænitúerit, quotiescúmque pro vénia et grátia ad me accésserit~: Vivo ego, dicit Dóminus, qui nolo mortem peccatóris, sed magis, ut convertátur et vivat~: quóniam peccatórum suórum non recordábor ámplius, sed cuncta sibi indúlta erunt.}
\markright{VIII}
%LIBER IV
\TitreC{CAPUT VIII}
\TitreC{De oblatióne Christi in cruce et própria resignatióne}
\TitreD{Vox dilécti.}
\Normal{\Verse{1.} Sicut ego meípsum, expánsis in cruce mánibus et nudo córpore, pro peccátis tuis Deo Patri sponte óbtuli, ita ut nihil in me remanéret, quin totum in sacrifícium divínæ placatiónis transíret~: ita debes et tu temetípsum mihi voluntárie in oblatiónem puram et sanctam, cotídie in missa, cum ómnibus víribus et afféctibus tuis, quanto intímius vales, offérre. Quid magis a te requíro, quam ut te stúdeas mihi ex íntegro resignáre~? Quidquid præter teípsum das, nihil curo~: quia non quæro datum tuum, sed te.}
\Normal{\Verse{2.} Sicut non suffíceret tibi ómnibus hábitis, præter me~: ita nec mihi placére póterit, quidquid déderis, te non obláto. Offer te mihi, et da te totum pro Deo, et erit accépta oblátio. Ecce, ego me totum óbtuli Patri pro te~; dedi étiam totum corpus meum et sánguinem in cibum, ut totus tuus essem, et tu meus permanéres. Si autem in te ipso stéteris, nec sponte te ad voluntátem meam obtúleris, non est plena oblátio, nec íntegra erit inter nos únio. Igitur ómnia ópera tua præcédere debet spontánea tui ipsíus in manus Dei oblátio, si libertátem cónsequi vis et grátiam. Ideo enim tam pauci illumináti et líberi intus efficiúntur, quia seípsos ex toto abnegáre nésciunt. Est firma senténtia mea~: Nisi quis renuntiáverit ómnibus, non potest meus esse discípulus. Tu ergo si optas meus esse discípulus, offer teípsum mihi cum ómnibus afféctibus tuis.}
\markright{IX}
%LIBER IV
\TitreC{CAPUT IX}
\TitreC{Quod nos et ómnia nostra Deo debémus offérre et pro ómnibus oráre}
\TitreD{Vox discípuli.}
\Normal{\Verse{1.} Dómine, ómnia tua sunt, quæ in cælo sunt, et quæ in terra. Desídero meípsum tibi in spontáneam oblatiónem offérre, et tuus perpétue permanére. Dómine, in simplicitáte cordis mei óffero meípsum tibi hódie in servum sempitérnum, in obséquium et in sacrifícium laudis perpétuæ. Súscipe me cum hac sancta oblatióne tui pretiósi córporis~; quam tibi hódie in præséntia angelórum, invisibíliter assisténtium, óffero, ut sit pro me et pro cuncto pópulo tuo in salútem.}
\Normal{\Verse{2.} Dómine, óffero tibi ómnia peccáta et delícta mea, quæ commísi coram te et sanctis ángelis tuis a die, quo primum peccáre pótui, usque ad horam hanc, super placábili altári tuo~: ut tu ómnia páriter incéndas et combúras igne caritátis tuæ, et déleas univérsas máculas peccatórum meórum, et consciéntiam meam ab omni delícto emúndes, et restítuas mihi grátiam tuam, quam peccándo amísi, ómnia mihi plene indulgéndo et in ósculum pacis me misericórditer assuméndo.}
\Normal{\Verse{3.} Quid possum ágere pro peccátis meis, nisi humíliter ea confiténdo et lamentándo, et tuam propitiatiónem incessánter deprecándo~? Déprecor te, exáudi me propítius, ubi asto coram te, Deus meus. Omnia peccáta mea mihi máxime dísplicent, nolo ea umquam ámplius perpetráre~; sed pro eis dóleo et dolébo, quámdiu víxero, parátus pæniténtiam ágere et pro posse satisfácere. Dimítte mihi, Deus, dimítte mihi peccáta mea, propter nomen sanctum tuum, salva ánimam meam, quam pretióso sánguine tuo redemísti. Ecce commítto me misericórdiæ tuæ, resígno me mánibus tuis. Age mecum secúndum bonitátem tuam, non secúndum meam malítiam et iniquitátem.}
\Normal{\Verse{4.} Offero étiam tibi ómnia bona mea, quamvis valde pauca et imperfécta~; ut tu ea eméndes et sanctífices~; ut ea grata hábeas et accépta tibi fácias, et semper ad melióra trahas~; nec non ad beátum ac laudábilem finem me pigrum et inútilem homunciónem perdúcas.}
\Normal{\Verse{5.} Offero quoque tibi ómnia pia desidéria devotórum, necessitátes paréntum, amicórum, fratrum, sorórum, omniúmque carórum meórum, et eórum, qui mihi vel áliis propter amórem tuum benefecérunt, et qui oratiónes et missas pro se suísque ómnibus dici a me desideravérunt et petiérunt~; sive in carne adhuc vivant, sive iam sǽculo defúncti sint~; ut omnes sibi auxílium grátiæ tuæ, opem consolatiónis, protectiónem a perículis, liberatiónem a pœnis adveníre séntiant, et ut ab ómnibus malis erépti, grátias tibi magníficas læti persólvant.}
\Normal{\Verse{6.} Offero étiam tibi preces et hóstias placatiónis, pro illis speciáliter, qui me in áliquo læsérunt, contristavérunt, aut vituperavérunt, vel áliquod damnum vel gravámen intulérunt~; pro his quoque ómnibus, quos aliquándo contristávi, conturbávi, gravávi et scandalizávi, verbis, factis, sciénter vel ignoránter, ut nobis ómnibus páriter indúlgeas peccáta nostra et mútuas offensiónes. Aufer, Dómine, a córdibus nostris omnem suspiciónem, indignatiónem, iram et disceptatiónem, et quidquid potest caritátem lǽdere et fratérnam dilectiónem minúere. Miserére, miserére, Dómine, misericórdiam tuam poscéntibus, da grátiam indigéntibus~; et fac nos tales exsístere, ut simus digni grátia tua pérfrui, et ad vitam proficiámus ætérnam. Amen.}
\markright{X}
%LIBER IV
\TitreC{CAPUT X}
\TitreC{Quod sacra commúnio de fácili non est relinquénda}
\TitreD{Vox dilécti.}
\Normal{\Verse{1.} Frequénter recurréndum est ad fontem grátiæ et divínæ misericórdiæ, ad fontem bonitátis et totíus puritátis~: quátenus a passiónibus tuis et vítiis curári váleas, et contra univérsas temptatiónes et fallácias diáboli fórtior atque vigilántior éffici mereáris. Inimícus sciens fructum et remédium máximum in sacra communióne pósitum, omni modo et occasióne nítitur fidéles et devótos, quantum prǽvalet, retráhere et impedíre.}
\Normal{\Verse{2.} Cum enim quidam sacræ communióni se aptáre dispónunt, peióres sátanæ immissiónes patiúntur. Ipse nequam spíritus, ut in Iob scríbitur, venit inter fílios Dei, ut sólita illos nequítia sua pertúrbet, aut tímidos nímium reddat et perpléxos~: quátenus afféctum eórum mínuat vel fidem impugnando áuferat~: si forte aut omníno communiónem relínquant, aut cum tepóre accédant. Sed non est quidquam curándum de versútiis et phantásiis illíus, quantúmlibet túrpibus et hórridis~; sed cuncta phantásmata in caput eius sunt retorquénda. Contemnéndus est miser et deridéndus~; nec propter insúltus eius et commotiónes, quas súscitat, sacra est omitténda commúnio.}
\Normal{\Verse{3.} Sæpe étiam ímpedit nímia sollicitúdo pro devotióne habénda, et anxíetas quædam de confessióne faciénda. Age secúndum consílium sapiéntum, et depóne anxietátem et scrúpulum~; quia grátiam Dei ímpedit et devotiónem mentis déstruit. Propter áliquam parvam turbatiónem vel gravitátem sacram ne dimíttas communiónem~; sed vade cítius confitéri, et omnes offensiónes áliis libénter indúlge. Si vero tu áliquem offendísti, véniam humíliter precáre, et Deus libénter indulgébit tibi.}
\Normal{\Verse{4.} Quid prodest diu tardáre confessiónem, aut sacram différre communiónem~? Expúrga te cumprímis, éxspue velóciter venénum, festína accípere remédium, et sénties mélius, quam si diu distúleris. Si hódie propter istud dimíttis, cras fórsitan áliud maius evéniet~: et sic diu posses a communióne impedíri et magis inéptus fíeri. Quanto cítius vales, a præsénti gravitáte et inértia te excútias~: quia nihil impórtat diu anxiári, diu cum turbatióne transíre, et ob cotidiána obstácula se a divínis sequestráre. Immo plúrimum nocet diu communiónem proteláre~; nam et gravem torpórem consuévit indúcere. Proh dolor~! quidam tépidi et dissolúti moras confiténdi libénter accípiunt, et communiónem sacram idcírco différri cúpiunt, ne ad maiórem sui custódiam se dare teneántur.}
\Normal{\Verse{5.} Heu, quam módicam caritátem et débilem devotiónem habent, qui sacram communiónem tam facíliter postpónunt~! Quam felix ille et Deo accéptus habétur, qui sic vivit, et in tali puritáte consciéntiam suam custódit, ut étiam omni die communicáre parátus, et bene affectátus esset, si ei licéret, et sine nota ágere posset. Si quis intérdum ábstinet humilitátis grátia, aut legítima impediénte causa, laudándus est de reveréntia. Si autem torpor obrépserit, seípsum excitáre debet, et fácere, quod in se est~; et Dóminus áderit desidério suo pro bona voluntáte, quam speciáliter réspicit.}
\Normal{\Verse{6.} Cum vero legítime præpedítus est, habébit semper bonam voluntátem, et piam intentiónem communicándi, et sic non carébit fructu sacraménti. Potest enim quílibet devótus, omni die et omni hora, ad spirituálem Christi communiónem salúbriter et sine prohibitióne accédere. Et tamen certis diébus et statúto témpore corpus sui Redemptóris cum affectuósa reveréntia sacramentáliter debet suscípere, et magis laudem Dei et honórem præténdere, quam suam consolatiónem quǽrere. Nam tótiens mýstice commúnicat, et invisibíliter refícitur, quótiens incarnatiónis Christi mystérium passionémque devóte recólit, et in amóre eius accénditur.}
\Normal{\Verse{7.} Qui áliter se non prǽparat, nisi instánte festo, vel consuetúdine compellénte, sǽpius imparátus erit. Beátus, qui se Dómino in holocáustum offert, quótiens célebrat aut commúnicat. Non sis in celebrándo nimis prolíxus aut festínus, sed serva bonum commúnem modum, cum quibus vivis. Non debes áliis generáre moléstiam et tǽdium, sed commúnem serváre viam secúndum maiórum institutiónem, et pótius aliórum servíre utilitáti, quam própriæ devotióni vel afféctui.}
\markright{XI}
%LIBER IV
\TitreC{CAPUT XI}
\TitreC{Quod corpus Christi et sacra scriptúra máxime sint ánimæ fidéli necessária}
\TitreD{Vox discípuli.}
\Normal{\Verse{1.} O dulcíssime Dómine Iesu, quanta est dulcédo devótæ ánimæ, tecum epulántis in convívio tuo, ubi ei non álius cibus manducándus propónitur, nisi tu, únicus diléctus eius, super ómnia desidéria cordis eius desiderábilis. Et mihi quidem dulce foret in præséntia tua ex íntimo afféctu lácrimas fúndere, et cum pia Magdaléna pedes tuos lácrimis irrigáre. Sed ubi est hæc devótio~? ubi lacrimárum sanctárum copiósa effúsio~? Certe in conspéctu tuo et sanctórum angelórum tuórum totum cor meum ardére debéret et ex gáudio flere. Hábeo enim te in sacraménto vere præséntem, quamvis aliéna spécie occultátum.}
\Normal{\Verse{2.} Nam in própria et divína claritáte te conspícere, óculi mei ferre non possent, sed neque totus mundus in fulgóre glóriæ maiestátis tuæ subsísteret. In hoc ergo imbecillitáti meæ cónsulis, quod te sub sacraménto abscóndis. Hábeo vere et adóro, quem ángeli adórant in cælo~; sed ego adhuc ínterim in fide, illi autem in spécie et sine velámine. Me opórtet conténtum esse in lúmine veræ fídei, et in ea ambuláre, donec aspíret dies ætérnæ claritátis, et umbræ figurárum inclinéntur. Cum autem vénerit, quod perféctum est, cessábit usus sacramentórum~; quia beáti in glória cælésti non egent medicámine sacramentáli~; gaudent enim sine fine in præséntia Dei, fácie ad fáciem glóriam eius speculántes~; et de claritáte in claritátem abyssális Deitátis transformáti, gustant verbum Dei caro factum, sicut fuit ab inítio et manet in ætérnum.}
\Normal{\Verse{3.} Memor horum mirabílium, grave mihi fit tǽdium étiam quódlibet spirituále solácium~: quia quámdiu Dóminum meum apérte in sua glória non vídeo, pro níhilo duco omne, quod in mundo conspício et áudio. Testis es tu mihi, Deus, quod nulla res me potest consolári, nulla creatúra quietáre, nisi tu, Deus meus, quem desídero æternáliter contemplári. Sed non est hoc possíbile, duránte me in hac mortalitáte. Ideo opórtet, ut me ponam ad magnam patiéntiam, et meípsum in omni desidério tibi submíttam. Nam et sancti tui, Dómine, qui tecum iam in regno cælórum exsúltant, in fide et patiéntia magna, dum víverent, advéntum glóriæ tuæ exspectábant. Quod illi credidérunt, ego credo~; quod illi speravérunt, ego spero~; quo illi pervenérunt, per grátiam tuam me ventúrum confído. Ambulábo ínterim in fide, exémplis confortátus sanctórum. Habébo étiam libros sanctos pro solácio et vitæ spéculo~; atque super hæc ómnia sanctíssimum corpus tuum pro singulári remédio et refúgio.}
\Normal{\Verse{4.} Duo namque mihi necessária permáxime séntio in hac vita, sine quibus mihi importábilis foret ista miserábilis vita. In cárcere córporis huius deténtus, duóbus me egére fáteor, cibo scílicet et lúmine. Dedísti ítaque mihi infírmo sacrum corpus tuum ad refectiónem mentis et córporis, et posuísti lucérnam pédibus meis verbum tuum. Sine his duóbus bene vívere non possem~; nam verbum Dei lux ánimæ meæ, et sacraméntum tuum panis vitæ. Hæc possunt étiam dici mensæ duæ, hinc et inde in gazophylácio sanctæ Ecclésiæ pósitæ. Una mensa est sacri altáris, habens panem sanctum, id est, corpus Christi pretiósum~; áltera est divínæ legis, cóntinens doctrínam sanctam, erúdiens fidem rectam, et fírmiter usque ad interióra veláminis, ubi sunt sancta sanctórum, perdúcens.}
\Normal{\Verse{5.} Grátias tibi, Dómine Iesu, lux lucis ætérnæ, pro doctrínæ sacræ mensa, quam nobis per servos tuos prophétas et apóstolos aliósque doctóres ministrásti. Grátias tibi, Creátor ac Redémptor hóminum, qui ad declarándam toti mundo caritátem tuam, cenam parásti magnam, in qua non agnum týpicum, sed tuum sanctíssimum corpus et sánguinem proposuísti manducándum~: lætíficans omnes fidéles convívio sacro, et cálice inébrians salutári, in quo sunt omnes delíciæ paradísi, et epulántur nobíscum ángeli sancti, sed suavitáte felicióri.}
\Normal{\Verse{6.} O quam magnum et honorábile est offícium sacerdótum, quibus datum est Dóminum maiestátis verbis sacris consecráre, lábiis benedícere, mánibus tenére, ore próprio súmere, et céteris ministráre~! O quam mundæ debent esse manus illæ, quam purum os, quam sanctum corpus, quam immaculátum cor erit sacerdótis, ad quem tótiens ingréditur auctor puritátis~! Ex ore sacerdótis nihil nisi sanctum, nihil nisi honéstum et útile procédere debet verbum, qui tam sæpe Christi áccipit sacraméntum.}
\Normal{\Verse{7.} Oculi eius símplices et pudíci, qui Christi corpus solent intuéri. Manus puræ et in cælum elevátæ, quæ Creatórem cæli et terræ solent contrectáre. Sacerdótibus speciáliter in lege dícitur~: Sancti estóte, quóniam ego sanctus sum, Dóminus Deus vester.}
\Normal{\Verse{8.} Adiuvet nos grátia tua, omnípotens Deus, ut, qui offícium sacerdotále suscépimus, digne ac devóte tibi in omni puritáte et consciéntia bona famulári valeámus. Et si non póssumus in tanta innocéntia vitæ conversári, ut debémus~: concéde nobis tamen digne flere mala, quæ géssimus, et in spíritu humilitátis ac bonæ voluntátis propósito tibi fervéntius de cétero deservíre.}
\markright{XII}
%LIBER IV
\TitreC{CAPUT XII}
\TitreC{Quod magna diligéntia se débeat communicatúrus Christo præparáre}
\TitreD{Vox dilécti.}
\Normal{\Verse{1.} Ego sum puritátis amátor et dator omnis sanctitátis. Ego cor purum quæro, et ibi est locus requietiónis meæ. Para mihi cenáculum grande stratum, et fáciam apud te pascha cum discípulis meis. Si vis, ut véniam ad te et apud te máneam~: expúrga vetus ferméntum, et munda cordis tui habitáculum. Exclúde totum sǽculum et omnem vitiórum tumúltum~; sede tamquam passer solitárius in tecto, et cógita excéssus tuos in amaritúdine ánimæ tuæ. Omnis namque amans suo dilécto amatóri óptimum et pulchérrimum prǽparat locum, quia in hoc cognóscitur afféctus suscipiéntis diléctum.}
\Normal{\Verse{2.} Scito tamen te non posse satisfácere huic præparatióni ex mérito tuæ actiónis, étiam si per íntegrum annum te præparáres et nihil áliud in mente habéres. Sed ex sola pietáte et grátia mea permítteris ad mensam meam accédere~: ac si mendícus ad prándium vocarétur dívitis, et ille nihil áliud hábeat ad retribuéndum benefíciis eius, nisi se humiliándo et ei regratiándo. Fac, quod in te est, et diligénter fácito, non ex consuetúdine, non ex necessitáte, sed cum timóre et reveréntia et afféctu áccipe corpus dilécti Dómini Dei tui, dignántis ad te veníre. Ego sum, qui vocávi, ego iussi fíeri~; ego supplébo, quod tibi deest~: veni, et súscipe me.}
\Normal{\Verse{3.} Cum grátiam devotiónis tríbuo, grátias age Deo tuo, non quia dignus es, sed quia tui misértus sum. Si non habes, sed magis áridum te sentis, insíste oratióni, ingemísce et pulsa~; nec desístas, donec mereáris micam aut guttam grátiæ salutáris accípere. Tu mei índiges, non ego tui indígeo. Nec tu me sanctificáre venis, sed ego te sanctificáre et melioráre vénio. Tu venis, ut ex me sanctificéris et mihi uniáris~; ut novam grátiam recípias et de novo ad emendatiónem accendáris. Noli neglégere hanc grátiam, sed prǽpara cum omni diligéntia cor tuum, et intróduc ad te diléctum tuum.}
\Normal{\Verse{4.} Opórtet autem, ut non solum te prǽpares ad devotiónem ante communiónem, sed ut étiam te sollícite consérves in ea post sacraménti perceptiónem. Nec minor custódia post exígitur, quam devóta præparátio prius. Nam bona póstmodum custódia óptima íterum est præparátio ad maiórem grátiam consequéndam. Ex eo quippe valde indispósitus quis rédditur, si statim fúerit nimis effúsus ad exterióra solácia. Cave a multilóquio, mane in secréto, et frúere Deo tuo~: ipsum enim habes, quem totus mundus tibi auférre non potest. Ego sum, cui te totum dare debes~: ita ut iam ultra non in te, sed in me absque omni sollicitúdine vivas.}
\markright{XIII}
%LIBER IV
\TitreC{CAPUT XIII}
\TitreC{Quod toto corde ánima devóta Christi uniónem in sacraménto affectáre debet}
\TitreD{Vox discípuli.}
\Normal{\Verse{1.} Quis mihi det, Dómine, ut invéniam te solum, et apériam tibi totum cor meum, et fruar te, sicut desíderat ánima mea, et iam me nemo despíciat, nec ulla creatúra me móveat vel respíciat, sed tu solus mihi loquáris, et ego tibi, sicut solet diléctus ad diléctum loqui, et amícus cum amíco convivári~? Hoc oro, hoc desídero, ut tibi totus úniar, et cor meum ab ómnibus creátis rebus ábstraham, magísque per sacram communiónem ac frequéntem celebratiónem cæléstia et ætérna sápere discam. Ah, Dómine Deus, quando ero tecum totus unítus et absórptus, meíque totáliter oblítus~? Tu in me, et ego in te~: et sic nos páriter in unum manére concéde.}
\Normal{\Verse{2.} Vere tu es diléctus meus, eléctus ex míllibus, in quo complácuit ánimæ meæ habitáre ómnibus diébus vitæ suæ. Vere tu pacíficus meus, in quo pax summa et réquies vera, extra quem labor et dolor et infiníta miséria. Vere tu es Deus abscónditus~; et consílium tuum non est cum ímpiis, sed cum humílibus et simplícibus sermo tuus. O quam suávis est, Dómine, spíritus tuus, qui ut dulcédinem tuam in fílios demonstráres, pane suavíssimo, de cælo descendénte, illos refícere dignáris~! Vere non est ália nátio tam grandis, quæ hábeat deos appropinquántes sibi, sicut tu, Deus noster, ades univérsis fidélibus tuis~; quibus ob cotidiánum solácium et cor erigéndum in cælum te tríbuis ad edéndum et fruéndum.}
\Normal{\Verse{3.} Quæ est enim ália gens tam ínclita, sicut plebs christiána~? Aut quæ creatúra sub cælo tam dilécta, ut ánima devóta, ad quam ingréditur Deus, ut pascat eam carne sua gloriósa~? O ineffábilis grátia~! o admirábilis dignátio~! o amor imménsus, hómini singuláriter impénsus~! Sed quid retríbuam Dómino pro grátia ista, pro caritáte tam exímia~? Non est áliud, quod grátius donáre queam, quam ut cor meum Deo meo totáliter tríbuam et íntime coniúngam. Tunc exsultábunt ómnia interióra mea, cum perfécte fúerit uníta Deo ánima mea. Tunc dicet mihi~: Si tu vis esse mecum, ego volo esse tecum. Et ego respondébo illi~: Dignáre, Dómine, manére mecum, ego volo libénter esse tecum. Hoc est totum desidérium meum, ut cor meum tibi sit unítum.}
\markright{XIV}
%LIBER IV
\TitreC{CAPUT XIV}
\TitreC{De quorundam devotórum ardénti desidério ad Corpus Christi}
\TitreD{Vox discípuli.}
\Normal{\Verse{1.} O quam magna multitúdo dulcédinis tuæ, Dómine, quam abscondísti timéntibus te~! Quando recórdor devotórum aliquórum ad sacraméntum tuum, Dómine, cum máxima devotióne et afféctu accedéntium, tunc sǽpius in me ipso confúndor et erubésco, quod ad altáre tuum et sacræ communiónis mensam tam tépide et frígide accédo, quod ita áridus et sine affectióne cordis máneo, quod non sum totáliter accénsus coram te, Deo meo, nec ita veheménter attráctus et afféctus, sicut multi devóti fuérunt, qui præ nímio desidério communiónis et sensíbili cordis amóre a fletu se non potuérunt continére~: sed ore cordis et córporis páriter ad te, Deum, fontem vivum, medúllitus inhiábant, suam esúriem non valéntes áliter temperáre nec satiáre, nisi corpus tuum cum omni iucunditáte et spirituáli aviditáte accepíssent.}
\Normal{\Verse{2.} O vera ardens fides eórum, probábile exsístens arguméntum sacræ præséntiæ tuæ~! Isti enim veráciter cognóscunt Dóminum suum in fractióne panis, quorum cor tam válide ardet in eis de Iesu ambulánte cum eis. Longe est a me sæpe talis afféctus et devótio, tam véhemens amor et ardor. Esto mihi propítius, Iesu bone, dulcis et benígne, et concéde páuperi mendíco tuo, vel intérdum módicum de cordiali afféctu amóris tui in sacra communióne sentíre, ut fides mea magis convaléscat, spes in bonitáte tua profíciat, et cáritas semel perfécte accénsa et cæléste manna expérta numquam defíciat.}
\Normal{\Verse{3.} Potens est autem misericórdia tua, étiam grátiam desiderátam mihi præstáre, et in spíritu ardóris, cum dies benepláciti tui vénerit, me clementíssime visitáre. Etenim licet tanto desidério tam speciálium devotórum tuórum non árdeo, tamen de grátia tua, illíus magni inflammáti desidérii desidérium hábeo, orans et desíderans, ómnium tálium fervidórum amatórum tuórum partícipem me fíeri, ac eórum sancto consórtio annumerári.}
\markright{XV}
%LIBER IV
\TitreC{CAPUT XV}
\TitreC{Quod grátia devotiónis humilitáte et sui ipsíus abnegatióne acquíritur}
\TitreD{Vox dilécti.}
\Normal{\Verse{1.} Opórtet te devotiónis grátiam instánter quǽrere, desideránter pétere, patiénter et fiduciáliter exspectáre, gratánter recípere, humíliter conserváre, studióse cum ea operári, ac Deo términum et modum supérnæ visitatiónis, donec véniat, commíttere. Humiliáre præcípue te debes, cum parum aut nihil devotiónis intérius sentis~; sed non nímium déici, nec inordináte contristári. Dat sæpe Deus in uno brevi moménto, quod longo negávit témpore~; dat quandóque in fine, quod in princípio oratiónis dístulit dare.}
\Normal{\Verse{2.} Si semper cito grátia darétur, et pro voto adésset, non esset infírmo hómini bene portábile. Proptérea in bona spe et húmili patiéntia exspectánda est devotiónis grátia. Tibi tamen et peccátis tuis ímputa, cum non datur vel étiam occúlte tóllitur. Módicum quandóque est, quod grátiam ímpedit et abscóndit~; si tamen módicum, et non pótius grande dici débeat, quod tantum bonum próhibet. Et si hoc ipsum módicum vel grande amóveris, et perfécte víceris, erit, quod petísti.}
\Normal{\Verse{3.} Statim namque, ut te Deo ex toto corde tradíderis, nec hoc vel illud pro tuo líbitu seu velle quæsíeris, sed íntegre te in ipso posúeris, unítum te invénies et pacátum~; quia nil ita bene sápiet et placébit, sicut beneplácitum divínæ voluntátis. Quisquis ergo intentiónem suam símplici corde sursum ad Deum leváverit, seque ab omni inordináto amóre seu displicéntia cuiúslibet rei creátæ evacuáverit, aptíssimus grátiæ percipiéndæ ac dignus devotiónis múnere erit. Dat enim Dóminus ibi benedictiónem suam, ubi vasa vácua invénerit. Et quanto perféctius ínfimis quis renúntiat, et magis sibi ipsi per contémptum sui móritur, tanto grátia celérius venit, copiósius intrat et áltius líberum cor élevat.}
\Normal{\Verse{4.} Tunc vidébit, et áffluet, et mirábitur, et dilatábitur cor eius in ipso, quia manus Dómini cum eo, et ipse se pósuit totáliter in manu eius usque in sǽculum. Ecce, sic benedicétur homo, qui quærit Deum in toto corde suo, nec in vanum áccipit ánimam suam. Hic in accipiéndo sacram eucharístiam magnam promerétur divínæ uniónis grátiam, quia non réspicit ad própriam devotiónem et consolatiónem, sed super omnem devotiónem et consolatiónem ad Dei glóriam et honórem.}
\markright{XVI}
%LIBER IV
\TitreC{CAPUT XVI}
\TitreC{Quod necessitátes nostras Christo aperíre et eius grátiam postuláre debémus}
\TitreD{Vox discípuli.}
\Normal{\Verse{1.} O dulcíssime atque amantíssime Dómine, quem nunc devóte desídero suscípere, tu scis infirmitátem meam et necessitátem, quam pátior~; in quantis malis et vítiis iáceo~; quam sæpe sum gravátus, temptátus, turbátus et inquinátus. Pro remédio ad te vénio, pro consolatióne et sublevámine te déprecor. Ad ómnia sciéntem loquor, cui manifésta sunt ómnia interióra mea, et qui solus potes me perfécte consolári et adiuváre. Tu scis, quibus bonis indígeo præ ómnibus, et quam pauper sum in virtútibus.}
\Normal{\Verse{2.} Ecce, sto ante te pauper et nudus, grátiam póstulans et misericórdiam implórans. Réfice esuriéntem mendícum tuum, accénde frigiditátem meam igne amóris tui, illúmina cæcitátem meam claritáte præséntiæ tuæ. Verte mihi ómnia terréna in amaritúdinem, ómnia grávia et contrária in patiéntiam, ómnia ínfima et creáta in contémptum et obliviónem. Erige cor meum ad te in cælum, et ne dimíttas me vagári super terram. Tu solus mihi ex hoc iam dulcéscas usque in sǽculum~; quia tu solus cibus et potus meus, amor meus et gáudium meum, dulcédo mea et totum bonum meum.}
\Normal{\Verse{3.} Utinam me totáliter ex tua præséntia accéndas, combúras et in te transmútes, ut unus tecum effíciar spíritus, per grátiam intérnæ uniónis et liquefactiónem ardéntis amóris~! Ne patiáris, me ieiúnum et áridum a te recédere, sed operáre mecum misericórditer, sicut sǽpius operátus es cum sanctis tuis mirabíliter. Quid mirum, si totus ex te ignéscerem, et in me ipso defícerem~; cum tu sis ignis semper ardens et numquam defíciens, amor corda puríficans et intelléctum illúminans?}
\markright{XVII}
%LIBER IV
\TitreC{CAPUT XVII}
\TitreC{De ardénti amóre et veheménti afféctu suscipiéndi Christum}
\TitreD{Vox discípuli.}
\Normal{\Verse{1.} Cum summa devotióne et ardénti amóre, cum toto cordis afféctu et fervóre, desídero te, Dómine, suscípere, quemádmodum multi sancti et devótæ persónæ in communicándo te desideravérunt, qui tibi máxime in sanctitáte vitæ placuérunt et in ardentíssima devotióne fuérunt. O Deus meus, amor ætérnus, totum bonum meum, felícitas interminábilis, cúpio te suscípere cum vehementíssimo desidério et digníssima reveréntia, quam áliquis sanctórum umquam hábuit et sentíre pótuit.}
\Normal{\Verse{2.} Et licet indígnus sum ómnia illa sentiménta devotiónis habére, tamen óffero tibi totum cordis mei afféctum, ac si ómnia illa gratíssima inflammáta desidéria solus habérem. Sed et quæcúmque potest pia mens concípere et desideráre, hæc ómnia tibi cum summa veneratióne et íntimo favóre prǽbeo et óffero. Nihil opto mihi reserváre, sed me et ómnia mea tibi sponte et libentíssime immoláre. Dómine Deus meus, Creátor meus et Redémptor meus, cum tali afféctu, reveréntia, laude et honóre, cum tali gratitúdine, dignitáte et amóre, cum tali fide, spe et puritáte te affécto hódie suscípere, sicut te suscépit et desiderávit sanctíssima mater tua, gloriósa virgo María, quando ángelo evangelizánti sibi incarnatiónis mystérium, humíliter ac devóte respóndit~: Ecce ancílla Dómini, fiat mihi secúndum verbum tuum.}
\Normal{\Verse{3.} Et sicut beátus præcúrsor tuus, excellentíssimus sanctórum, Johánnes baptísta, in præséntia tua lætabúndus exsultávit in gáudio Spíritus Sancti, dum adhuc matérnis clauderétur viscéribus, et póstmodum cernens inter hómines Iesum ambulántem, valde se humílians, devóto cum afféctu dicébat~: Amícus autem sponsi, qui stat et audit eum, gáudio gaudet propter vocem sponsi~: sic et ego magnis et sacris desidériis opto inflammári, et tibi ex toto corde meípsum præsentáre. Unde et ómnium devotórum córdium iubilatiónes, ardéntes afféctus, mentáles excéssus, ac supernaturáles illuminatiónes, et cǽlicas visiónes tibi óffero et exhíbeo, cum ómnibus virtútibus et láudibus, ab omni creatúra in cælo et in terra celebrátis et celebrándis, pro me et ómnibus mihi in oratióne commendátis, quátenus ab ómnibus digne laudéris, et in perpétuum glorificéris.}
\Normal{\Verse{4.} Accipe vota mea, Dómine Deus meus, et desidéria infinítæ laudatiónis ac imménsæ benedictiónis, quæ tibi secúndum multitúdinem ineffábilis magnitúdinis tuæ iure debéntur. Hæc tibi reddo et réddere desídero per síngulos dies et moménta témporum, atque ad reddéndum mecum tibi grátias et laudes omnes cæléstes spíritus et cunctos fidéles tuos précibus et afféctibus invíto et exóro.}
\Normal{\Verse{5.} Laudent te univérsi pópuli, tribus et linguæ, et sanctum ac mellífluum nomen tuum cum summa iubilatióne et ardénti devotióne magníficent. Et quicúmque reverénter ac devóte altíssimum sacraméntum tuum célebrant, et plena fide recípiunt, grátiam et misericórdiam apud te inveníre mereántur, et pro me peccatóre supplíciter exórent. Cumque optáta devotióne ac fruíbili unióne potíti fúerint, et bene consoláti ac mirífice refécti, de sacra mensa cælésti abscésserint, mei páuperis recordári dignéntur.}
\markright{XVIII}
%LIBER IV
\TitreC{CAPUT XVIII}
\TitreC{Quod homo non sit curiósus scrutátor sacraménti, sed húmilis imitátor Christi, subdéndo sensum suum sacræ fídei}
\TitreD{Vox dilécti.}
\Normal{\Verse{1.} Cavéndum est tibi a curiósa et inútili perscrutatióne huius profundíssimi sacraménti, si non vis in dubitatiónis profúndum submérgi. Qui scrutátor est maiestátis, opprimétur a glória. Plus valet Deus operári, quam homo intellégere potest. Tolerábilis est pia et húmilis inquisítio veritátis, paráta semper docéri, et per sanas patrum senténtias studens ambuláre.}
\Normal{\Verse{2.} Beáta simplícitas, quæ diffíciles quæstiónum relínquit vias, et plana ac firma pergit sémita mandatórum Dei. Multi devotiónem perdidérunt, dum altióra scrutári voluérunt. Fides a te exígitur et sincéra vita, non altitúdo intelléctus, neque profúnditas mysteriórum Dei. Si non intéllegis, nec capis, quæ infra te sunt, quómodo comprehéndes, quæ supra te sunt~? Súbdere Deo, et humília sensum tuum fídei, et dábitur tibi sciéntiæ lumen, prout tibi fúerit útile ac necessárium.}
\Normal{\Verse{3.} Quidam gráviter temptántur de fide et sacraménto~; sed non est hoc ipsis imputándum, sed pótius inimíco. Noli curáre, noli disputáre cum cogitatiónibus tuis, nec ad immíssas a diábolo dubitatiónes respónde~; sed crede verbis Dei, crede sanctis eius et prophétis, et fúgiet a te nequam inimícus. Sæpe multum prodest, quod tália sústinet Dei servus. Nam infidéles et peccatóres non temptat, quos secúre iam póssidet. fidéles autem devótos váriis modis temptat et vexat.}
\Normal{\Verse{4.} Perge ergo cum símplici et indubitáta fide, et cum súpplici reveréntia ad sacraméntum accéde. Et quidquid intellégere non vales, Deo omnipoténti secúre commítte. Non fallit te Deus~; fállitur, qui sibi ipsi nímium credit. Gráditur Deus cum simplícibus, revélat se humílibus, dat intelléctum párvulis, áperit sensum puris méntibus, et abscóndit grátiam curiósis et supérbis. Rátio humána débilis est et falli potest~; fides autem vera falli non potest.}
\Normal{\Verse{5.} Omnis rátio et naturális investigátio fidem sequi debet, non præcédere nec infríngere. Nam fides et amor ibi máxime præcéllunt, et occúltis modis in hoc sanctíssimo et superexcellentíssimo sacraménto operántur. Deus ætérnus et imménsus, infinitǽque poténtiæ, facit magna et inscrutabília in cælo et in terra, nec est investigátio mirabílium óperum eius. Si tália essent ópera Dei, ut fácile ab humána ratióne caperéntur, non essent mirabília nec ineffabília dicénda.}

\vspace{1cm}
\TitreD{FINIS}

\end{document}
